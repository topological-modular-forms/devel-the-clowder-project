\usepackage{tikz}
\usetikzlibrary{arrows.meta,calc,positioning,decorations.markings,fit,patterns,decorations.pathreplacing,3d,decorations.pathmorphing,intersections}
\pgfdeclarelayer{background}
\pgfsetlayers{background,main}
\usepackage{tikz-cd}
\newlength{\DL}
\setlength{\DL}{0.9em}% 0.9 / 1.1625
\tikzset{double line with arrow/.style args={#1,#2}{decorate,decoration={markings,mark=at position 0 with {\coordinate (ta-base-1) at (0,1pt);\coordinate (ta-base-2) at (0,-1pt);},mark=at position 1 with {\draw[#1] (ta-base-1) -- (0,1pt);\draw[#2] (ta-base-2) -- (0,-1pt);}}}}
\tikzset{Equals/.style={-,double line with arrow={-,-}}}
\tikzcdset{
    arrow style=tikz,
    %diagrams={>={Straight Barb[scale=0.75]}}
    diagrams={>={Stealth[round,length=4pt,width=4.95pt,inset=2.75pt]}}
}
\tikzcdset{%
    bigisoarrow/.style={%
        "\scalebox{1.75}{$\unsim$}"{sloped}, dash pattern=on 1.65pt off 1.65pt, outer sep=-1.5pt%
        %dash pattern=on 1.65pt off 1.65pt,%
    }%
}%
\tikzcdset{%
    bigisoarrowprime/.style={%
        "\scalebox{1.75}{$\unsim$}"'{sloped}, dash pattern=on 1.65pt off 1.65pt, outer sep=+0.1pt%
        %dash pattern=on 1.65pt off 1.65pt,%
    }%
}%
\tikzcdset{%
    isoarrow/.style={%
        "\scalebox{1.25}{$\unsim$}"{sloped,pos=0.45}, dash pattern=on 1.65pt off 1.65pt, outer sep=-2.5pt%
        %dash pattern=on 1.65pt off 1.65pt,%
    }%
}%
\tikzcdset{%
    isoarrowprime/.style={%
        "\scalebox{1.25}{$\unsim$}"'{sloped}, dash pattern=on 1.65pt off 1.65pt, outer sep=-0.75pt%
        %dash pattern=on 1.65pt off 1.65pt,%
    }%
}%
\ifplastex
\else
    \tikzset{mid vert/.style={/utils/exec=\tikzset{every node/.append style={outer sep=0.4*\the\DL}},postaction=decorate,decoration={markings,mark=at position #1 with {\draw[solid,-] (0,0.75ex) -- (0,-0.75ex);}}},mid vert/.default=0.5}
    \tikzset{
        densely dashed for mid vert/.style={
        decoration={
        markings,
        mark=at position 0 with {
          \pgfmathsetmacro{\myOn}{4pt}
          \pgfmathsetmacro{\myOff}{2pt}
          \pgfmathsetmacro{\mystretch}{((\pgfdecoratedpathlength-\myOn)/(\myOn+\myOff))/int((\pgfdecoratedpathlength-\myOn)/(\myOn+\myOff))}
          \pgfmathsetmacro{\myon}{\myOn*\mystretch}
          \xdef\myon{\myon}
          \pgfmathsetmacro{\myoff}{\myOff*\mystretch}
          \xdef\myoff{\myoff}
          \pgfmathsetmacro{\mypathlength}{\pgfdecoratedpathlength}
          \xdef\mypathlength{\mypathlength}
        },
      },
      preaction={decorate},draw=none,
      postaction={draw,dash pattern=on \myon pt off \myoff pt,dash phase=0.5*\myon-0.5*\mypathlength}
      },
    }
    \tikzset{lddr_to_path/.style={to path={-| ([xshift=-5ex]\tikztotarget.west) \ifx\relax#1\relax \else node[near end,left]{$\scriptstyle#1$} \fi |- (\tikztotarget)}}}
    \tikzset{rddl_to_path/.style={to path={-| ([xshift=5ex]\tikztotarget.east) \ifx\relax#1\relax \else node[near end,right]{$\scriptstyle#1$} \fi -- (\tikztotarget)}}}
    \tikzset{lddr_to_path_large/.style={to path={-| ([xshift=-10ex]\tikztotarget.west) \ifx\relax#1\relax \else node[near end,left]{$\scriptstyle#1$} \fi |- (\tikztotarget)}}}
    \tikzset{rddl_to_path_large/.style={to path={-| ([xshift=10ex]\tikztotarget.east) \ifx\relax#1\relax \else node[near end,right]{$\scriptstyle#1$} \fi -- (\tikztotarget)}}}
    \tikzset{ldddr_to_path/.style={to path={-| ([xshift=-5ex]\tikztotarget.west) \ifx\relax#1\relax \else node[near end,left]{$\scriptstyle#1$} \fi |- (\tikztotarget)}}}
    \tikzset{rdddl_to_path/.style={to path={-| ([xshift=5ex]\tikztotarget.east) \ifx\relax#1\relax \else node[near end,right]{$\scriptstyle#1$} \fi -- (\tikztotarget)}}}
    \tikzset{ldddr_to_path_large/.style={to path={-| ([xshift=-10ex]\tikztotarget.west) \ifx\relax#1\relax \else node[near end,left]{$\scriptstyle#1$} \fi |- (\tikztotarget)}}}
    \tikzset{rdddl_to_path_large/.style={to path={-| ([xshift=10ex]\tikztotarget.east) \ifx\relax#1\relax \else node[near end,right]{$\scriptstyle#1$} \fi -- (\tikztotarget)}}}
\fi
\tikzcdset{
    productArrows/.style args={#1#2#3}{
    execute at end picture={
        % FIRST ARROW
        % Step 1: Draw arrow body
        \begin{scope}
            \clip (\tikzcdmatrixname-1-2.east) -- (\tikzcdmatrixname-2-2.center) -- (\tikzcdmatrixname-2-3.north) -- (\tikzcdmatrixname-1-3.center) -- cycle;
            \path[draw,line width=rule_thickness] (\tikzcdmatrixname-1-2) arc[start angle=90,end angle=0,radius=#1];
        \end{scope}
        % Step 2: Draw arrow head
        % Step 2.1: Find the point at which to place the arrowhead
        \path[name path=curve-1-a] (\tikzcdmatrixname-1-2.east) -- (\tikzcdmatrixname-2-2.center) -- (\tikzcdmatrixname-2-3.north) -- (\tikzcdmatrixname-1-3.center) -- cycle;
        \path[name path=curve-1-b] (\tikzcdmatrixname-1-2) arc[start angle=90,end angle=0,radius=#1];
        \fill [name intersections={of=curve-1-a and curve-1-b}] (intersection-2);
        % Step 2.2: Find the angle at which to place the arrowhead
        \coordinate (arc-start) at (\tikzcdmatrixname-1-2.east);
        \coordinate (arc-center) at (\tikzcdmatrixname-2-2.center);
        \draw let
            \p1 = ($(intersection-2) - (arc-center)$), % \p1 is the vector from the arc's centre to the intersection point (we use i-2 for the 2nd intersection)
            \n1 = {atan2(\y1, \x1)}, % \n1 is the angle of that vector in degrees
            \n2 = {\n1 - 90} % \n2 is the angle of the tangent (90 degrees from the radius vector for a circle)
          in [->] (intersection-2) -- ++(\n2:0.1pt);
        % SECOND ARROW
        % Step 1: Draw arrow body
        \begin{scope}
            \clip (\tikzcdmatrixname-1-2.west) -- (\tikzcdmatrixname-2-2.center) -- (\tikzcdmatrixname-2-1.north) -- (\tikzcdmatrixname-1-1.center) -- cycle;
            \path[draw,line width=rule_thickness] (\tikzcdmatrixname-1-2) arc[start angle=90,end angle=180,radius=#1];
        \end{scope}
        % Step 2: Draw arrow head
        % Step 2.1: Find the point at which to place the arrowhead
        \path[name path=curve-2-a] (\tikzcdmatrixname-1-2.west) -- (\tikzcdmatrixname-2-2.center) -- (\tikzcdmatrixname-2-1.north) -- (\tikzcdmatrixname-1-1.center) -- cycle;
        \path[name path=curve-2-b] (\tikzcdmatrixname-1-2) arc[start angle=90,end angle=180,radius=#1];
        \fill [name intersections={of=curve-2-a and curve-2-b}] (intersection-2);
        % Step 2.2: Find the angle at which to place the arrowhead
        \coordinate (arc-start) at (\tikzcdmatrixname-1-2.west);
        \coordinate (arc-center) at (\tikzcdmatrixname-2-2.center);
        \draw let
            \p1 = ($(intersection-2) - (arc-center)$), % \p1 is the vector from the arc's centre to the intersection point (we use i-2 for the 2nd intersection)
            \n1 = {atan2(\y1, \x1)}, % \n1 is the angle of that vector in degrees
            \n2 = {\n1 - 90} % \n2 is the angle of the tangent (90 degrees from the radius vector for a circle)
          in [<-] (intersection-2) -- ++(\n2:0.1pt);
          % Labels
          \path (\tikzcdmatrixname-1-2) arc[start angle=90,end angle=180,radius=#1] node[above left,pos=0.5] {$\scriptstyle #2$};
          \path (\tikzcdmatrixname-1-2) arc[start angle=90,end angle=0,radius=#1] node[above right,pos=0.5] {$\scriptstyle #3$};
    }
  }
}
\tikzcdset{
    coproductArrows/.style args={#1#2#3}{
    execute at end picture={
        % FIRST ARROW
        % Step 1: Draw arrow body
        \begin{scope}
            \clip (\tikzcdmatrixname-1-2.east) -- (\tikzcdmatrixname-2-2.center) -- (\tikzcdmatrixname-2-3.north) -- (\tikzcdmatrixname-1-3.center) -- cycle;
            \path[draw,line width=rule_thickness] (\tikzcdmatrixname-1-2) arc[start angle=90,end angle=0,radius=#1];
        \end{scope}
        % Step 2: Draw arrow head
        % Step 2.1: Find the point at which to place the arrowhead
        \path[name path=curve-1-a] (\tikzcdmatrixname-1-2.east) -- (\tikzcdmatrixname-2-2.center) -- (\tikzcdmatrixname-2-3.north) -- (\tikzcdmatrixname-1-3.center) -- cycle;
        \path[name path=curve-1-b] (\tikzcdmatrixname-1-2) arc[start angle=90,end angle=0,radius=#1];
        \fill [name intersections={of=curve-1-a and curve-1-b}] (intersection-1);
        % Step 2.2: Find the angle at which to place the arrowhead
        \coordinate (arc-start) at (\tikzcdmatrixname-1-2.east);
        \coordinate (arc-center) at (\tikzcdmatrixname-2-2.center);
        \draw let
            \p1 = ($(intersection-1) - (arc-center)$), % \p1 is the vector from the arc's centre to the intersection point (we use i-2 for the 2nd intersection)
            \n1 = {atan2(\y1, \x1)}, % \n1 is the angle of that vector in degrees
            \n2 = {\n1 - 90} % \n2 is the angle of the tangent (90 degrees from the radius vector for a circle)
          in [<-] (intersection-1) -- ++(\n2:0.1pt);
        % SECOND ARROW
        % Step 1: Draw arrow body
        \begin{scope}
            \clip (\tikzcdmatrixname-1-2.west) -- (\tikzcdmatrixname-2-2.center) -- (\tikzcdmatrixname-2-1.north) -- (\tikzcdmatrixname-1-1.center) -- cycle;
            \path[draw,line width=rule_thickness] (\tikzcdmatrixname-1-2) arc[start angle=90,end angle=180,radius=#1];
        \end{scope}
        % Step 2: Draw arrow head
        % Step 2.1: Find the point at which to place the arrowhead
        \path[name path=curve-2-a] (\tikzcdmatrixname-1-2.west) -- (\tikzcdmatrixname-2-2.center) -- (\tikzcdmatrixname-2-1.north) -- (\tikzcdmatrixname-1-1.center) -- cycle;
        \path[name path=curve-2-b] (\tikzcdmatrixname-1-2) arc[start angle=90,end angle=180,radius=#1];
        \fill [name intersections={of=curve-2-a and curve-2-b}] (intersection-1);
        % Step 2.2: Find the angle at which to place the arrowhead
        \coordinate (arc-start) at (\tikzcdmatrixname-1-2.west);
        \coordinate (arc-center) at (\tikzcdmatrixname-2-2.center);
        \draw let
            \p1 = ($(intersection-1) - (arc-center)$), % \p1 is the vector from the arc's centre to the intersection point (we use i-2 for the 2nd intersection)
            \n1 = {atan2(\y1, \x1)}, % \n1 is the angle of that vector in degrees
            \n2 = {\n1 - 90} % \n2 is the angle of the tangent (90 degrees from the radius vector for a circle)
          in [->] (intersection-1) -- ++(\n2:0.1pt);
          % Labels
          \path (\tikzcdmatrixname-1-2) arc[start angle=90,end angle=180,radius=#1] node[above left,pos=0.5] {$\scriptstyle #2$};
          \path (\tikzcdmatrixname-1-2) arc[start angle=90,end angle=0,radius=#1] node[above right,pos=0.5] {$\scriptstyle #3$};
    }
  }
}
\tikzset{
    partial ellipse/.style args={#1:#2:#3}{
        insert path={+ (#1:#3) arc (#1:#2:#3)}
    }
}
