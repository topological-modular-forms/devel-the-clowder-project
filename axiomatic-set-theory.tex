\input{preamble}

% OK, start here.
%
\input{chapter_modifications.tex}
\begin{document}

\title{Axiomatic Set Theory}

\maketitle

\phantomsection
\label{section-phantom}

This chapter (will eventually) contain material on axiomatic set theory, as well as a couple other things.

\ChapterTableOfContents

TODO:
\begin{itemize}
    \item SEAR, \url{https://ncatlab.org/nlab/show/structural+set+theory#sets_elements_and_relations_sear}
\end{itemize}

\section{Sets and Functions}\label{section-sets-and-functions}
\subsection{Functions}\label{subsection-sets-and-functions-functions}
\begin{definition}{Functions}{functions}%
    A \index[set-theory]{function}\textbf{function} is a functional and total relation.
\end{definition}
\section{Operations With Sets}\label{section-operations-with-sets}
\subsection{The Empty Set}\label{subsection-the-empty-set}
\begin{definition}{The Empty Set}{the-empty-set}%
     The \index[set-theory]{empty set}\textbf{empty set} is the set \index[notation]{emptyset@$\emptyset$}$\emptyset$ defined by
    \[
        \emptyset
        \defeq
        \{%
            x\in X%
            \ \middle|\ %
            x\neq x%
        \},%
    \]%
    where $X$ is the set in the set existence axiom, \cref{zermelo-fraenkel-set-theory-the-axiom-of-set-existence} of \cref{zermelo-fraenkel-set-theory}.
\end{definition}
\subsection{Singleton Sets}\label{subsection-singleton-sets}
Let $X$ be a set.
\begin{definition}{Singleton Sets}{singleton-sets}%
    The \index[set-theory]{singleton set}\textbf{singleton set containing $X$} is the set \index[notation]{X@$\{X\}$}$\{X\}$ defined by
    \[
        \{X\}
        \defeq
        \{X,X\},
    \]%
    where $\{X,X\}$ is the pairing of $X$ with itself of \cref{pairings-of-sets}.
\end{definition}
\subsection{Pairings of Sets}\label{subsection-pairings-of-sets}
Let $X$ and $Y$ be sets.
\begin{definition}{Pairings of Sets}{pairings-of-sets}%
    The \index[set-theory]{pairing of two sets}\textbf{pairing of $X$ and $Y$} is the set \index[notation]{XY@$\{X,Y\}$}$\{X,Y\}$ defined by
    \[
        \{X,Y\}
        \defeq
        \{%
            x\in A%
            \ \middle|\ %
            \text{%
                $x=X$ or $x=Y$%
            }%
        \},
    \]%
    where $A$ is the set in the axiom of pairing, \cref{zermelo-fraenkel-set-theory-the-axiom-of-pairing} of \cref{zermelo-fraenkel-set-theory}.
\end{definition}
\subsection{Ordered Pairs}\label{subsection-ordered-pairs}
Let $A$ and $B$ be sets.%
\begin{definition}{Ordered Pairs}{ordered-pairs}%
    The \index[set-theory]{ordered pairing}\textbf{ordered pair associated to $A$ and $B$} is the set \index[notation]{AB@$(A,B)$}$(A,B)$ defined by%
    \[
        (A,B)%
        \defeq%
        \{\{A\},\{A,B\}\}.%
    \]%
\end{definition}
\begin{proposition}{Properties of Ordered Pairs}{properties-of-ordered-pairs}%
    Let $A$ and $B$ be sets.%
    \begin{enumerate}
        \item\label{properties-of-ordered-pairs-uniqueness}\SloganFont{Uniqueness. }Let $A$, $B$, $C$, and $D$ be sets. The following conditions are equivalent:
            \begin{enumerate}
                \item\label{properties-of-ordered-pairs-uniqueness-1}We have $(A,B)=(C,D)$.
                \item\label{properties-of-ordered-pairs-uniqueness-2}We have $A=C$ and $B=D$.
            \end{enumerate}
        %\item\label{properties-of-ordered-pairs-}\SloganFont{. }
    \end{enumerate}
\end{proposition}
\begin{Proof}{Proof of \cref{properties-of-ordered-pairs}}%
    \FirstProofBox{\cref{properties-of-ordered-pairs-uniqueness}: Uniqueness}%
    See \cite[Theorem 1.2.3]{ciesielski1997set}.
\end{Proof}
\begin{appendices}
\input{ABSOLUTEPATH/chapters2.tex}
\end{appendices}
\end{document}
