\input{preamble}

% OK, start here.
%
\input{chapter_modifications.tex}
\begin{document}

\title{Axiomatic Set Theory}

\maketitle

\phantomsection
\label{section-phantom}

This chapter (will eventually) contain material on axiomatic set theory, as well as a couple other things.

\ChapterTableOfContents

\section{Sets and Functions}\label{section-sets-and-functions}
\subsection{Functions}\label{subsection-sets-and-functions-functions}
\begin{definition}{Functions}{functions}%
    A \index[set-theory]{function}\textbf{function} is a functional and total relation.
\end{definition}
\begin{notation}{Additional Notation for Functions}{additional-notation-for-functions}%
    Throughout this work, we will sometimes denote a function $f\colon X\to Y$ by
    \[
        f%
        \defeq%
        \llbracket x\mapsto f(x)\rrbracket.%
    \]%
    \begin{enumerate}
        \item\label{additional-notation-for-functions-1}For example, given a function%
            \[
                \Phi%
                \colon%
                \Hom_{\Sets}(X,Y)%
                \to%
                K%
            \]%
            taking values on a set of functions such as $\Hom_{\Sets}(X,Y)$, we will sometimes also write
            \[
                \Phi(f)%
                \defeq%
                \Phi(\llbracket x\mapsto f(x)\rrbracket).%
            \]%
        \item\label{additional-notation-for-functions-2}This notational choice is based on the lambda notation
            \[
                f%
                \defeq%
                (\lambda x.\ f(x)),%
            \]%
            but uses a \say{$\mathord{\mapsto}$} symbol for better spacing and double brackets instead of either:
            \begin{enumerate}
                \item\label{additional-notation-for-functions-2-a}Square brackets $[x\mapsto f(x)]$;
                \item\label{additional-notation-for-functions-2-b}Parentheses $(x\mapsto f(x))$;
            \end{enumerate}
            hoping to improve readability when dealing with e.g.:
            \begin{enumerate}
                \item\label{additional-notation-for-functions-2-c}Equivalence classes, cf.:
                    \begin{enumerate}
                        \item\label{additional-notation-for-functions-2-c-i}$\llbracket[x]\mapsto f([x])\rrbracket$%
                        \item\label{additional-notation-for-functions-2-c-ii}$[[x]\mapsto f([x])]$%
                        \item\label{additional-notation-for-functions-2-c-iii}$(\lambda[x].\ f([x]))$%
                    \end{enumerate}
                \item\label{additional-notation-for-functions-2-d}Function evaluations, cf.:
                    \begin{enumerate}
                        \item\label{additional-notation-for-functions-2-d-i}$\Phi(\llbracket x\mapsto f(x)\rrbracket)$%
                        \item\label{additional-notation-for-functions-2-d-ii}$\Phi((x\mapsto f(x)))$%
                        \item\label{additional-notation-for-functions-2-d-iii}$\Phi((\lambda x.\ f(x)))$%
                    \end{enumerate}
            \end{enumerate}
        \item\label{additional-notation-for-functions-3}We will also sometimes write $-$, $-_{1}$, $-_{2}$, etc.\ for the arguments of a function. Some examples include:
            \begin{enumerate}
                \item\label{additional-notation-for-functions-3-a}Writing $f(-_{1})$ for a function $f\colon A\to B$.
                \item\label{additional-notation-for-functions-3-b}Writing $f(-_{1},-_{2})$ for a function $f\colon A\times B\to C$.
                \item\label{additional-notation-for-functions-3-c}Given a function $f\colon A\times B\to C$, writing
                    \[
                        f(a,-)%
                        \colon%
                        B%
                        \to%
                        C%
                    \]%
                    for the function $\llbracket b\mapsto f(a,b)\rrbracket$.
                \item\label{additional-notation-for-functions-3-d}Denoting a composition of the form%
                    \[
                        A\times B%
                        \xlongrightarrow{\phi\times\id_{B}}%
                        A'\times B%
                        \xlongrightarrow{f}%
                        C%
                    \]%
                    by $f(\phi(-_{1}),-_{2})$.
            \end{enumerate}
        \item\label{additional-notation-for-functions-4}Finally, given a function $f\colon A\to B$, we will sometimes write
            \[
                \ev_{a}(f)%
                \defeq%
                f(a)%
            \]%
            for the value of $f$ at some $a\in A$.
    \end{enumerate}
    For an example of the above notations being used in practice, see the proof of the adjunction
    \begin{webcompile}
        \AdjunctionShort#A\times -#{\Hom_{\Sets}(A,-)}#\Sets#\Sets,#
    \end{webcompile}
    stated in \ChapterRef{\ChapterConstructionsWithSets, \cref{constructions-with-sets:properties-of-products-of-sets-adjointness-1} of \cref{constructions-with-sets:properties-of-products-of-sets}}{\cref{properties-of-products-of-sets-adjointness-1} of \cref{properties-of-products-of-sets}}.
\end{notation}
\section{Operations With Sets}\label{section-operations-with-sets}
\subsection{The Empty Set}\label{subsection-the-empty-set}
\begin{definition}{The Empty Set}{the-empty-set}%
     The \index[set-theory]{empty set}\textbf{empty set} is the set \index[notation]{emptyset@$\emptyset$}$\emptyset$ defined by
    \[
        \emptyset
        \defeq
        \{%
            x\in X%
            \ \middle|\ %
            x\neq x%
        \},%
    \]%
    where $X$ is the set in the set existence axiom, \cref{zermelo-fraenkel-set-theory-the-axiom-of-set-existence} of \cref{zermelo-fraenkel-set-theory}.
\end{definition}
\subsection{Singleton Sets}\label{subsection-singleton-sets}
Let $X$ be a set.
\begin{definition}{Singleton Sets}{singleton-sets}%
    The \index[set-theory]{singleton set}\textbf{singleton set containing $X$} is the set \index[notation]{X@$\{X\}$}$\{X\}$ defined by
    \[
        \{X\}
        \defeq
        \{X,X\},
    \]%
    where $\{X,X\}$ is the pairing of $X$ with itself of \cref{pairings-of-sets}.
\end{definition}
\subsection{Pairings of Sets}\label{subsection-pairings-of-sets}
Let $X$ and $Y$ be sets.
\begin{definition}{Pairings of Sets}{pairings-of-sets}%
    The \index[set-theory]{pairing of two sets}\textbf{pairing of $X$ and $Y$} is the set \index[notation]{XY@$\{X,Y\}$}$\{X,Y\}$ defined by
    \[
        \{X,Y\}
        \defeq
        \{%
            x\in A%
            \ \middle|\ %
            \text{%
                $x=X$ or $x=Y$%
            }%
        \},
    \]%
    where $A$ is the set in the axiom of pairing, \cref{zermelo-fraenkel-set-theory-the-axiom-of-pairing} of \cref{zermelo-fraenkel-set-theory}.
\end{definition}
\subsection{Ordered Pairs}\label{subsection-ordered-pairs}
Let $A$ and $B$ be sets.%
\begin{definition}{Ordered Pairs}{ordered-pairs}%
    The \index[set-theory]{ordered pairing}\textbf{ordered pair associated to $A$ and $B$} is the set \index[notation]{AB@$(A,B)$}$(A,B)$ defined by%
    \[
        (A,B)%
        \defeq%
        \{\{A\},\{A,B\}\}.%
    \]%
\end{definition}
\begin{proposition}{Properties of Ordered Pairs}{properties-of-ordered-pairs}%
    Let $A$ and $B$ be sets.%
    \begin{enumerate}
        \item\label{properties-of-ordered-pairs-uniqueness}\SloganFont{Uniqueness. }Let $A$, $B$, $C$, and $D$ be sets. The following conditions are equivalent:
            \begin{enumerate}
                \item\label{properties-of-ordered-pairs-uniqueness-1}We have $(A,B)=(C,D)$.
                \item\label{properties-of-ordered-pairs-uniqueness-2}We have $A=C$ and $B=D$.
            \end{enumerate}
        %\item\label{properties-of-ordered-pairs-}\SloganFont{. }
    \end{enumerate}
\end{proposition}
\begin{Proof}{Proof of \cref{properties-of-ordered-pairs}}%
    \FirstProofBox{\cref{properties-of-ordered-pairs-uniqueness}: Uniqueness}%
    See \cite[Theorem 1.2.3]{ciesielski1997set}.
\end{Proof}
\begin{appendices}
\input{ABSOLUTEPATH/chapters2.tex}
\end{appendices}
\end{document}
