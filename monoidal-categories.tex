\input{preamble}

% OK, start here.
%
\usepackage{fontspec}
\let\hyperwhite\relax
\let\hyperred\relax
\newcommand{\hyperwhite}{\hypersetup{citecolor=white,filecolor=white,linkcolor=white,urlcolor=white}}
\newcommand{\hyperred}{%
\hypersetup{%
    citecolor=TitlingRed,%
    filecolor=TitlingRed,%
    linkcolor=TitlingRed,%
     urlcolor=TitlingRed%
}}
\let\ChapterRef\relax
\newcommand{\ChapterRef}[2]{#1}
\setcounter{tocdepth}{2}
%▓▓▓▓▓▓▓▓▓▓▓▓▓▓▓▓▓▓▓▓▓▓▓▓▓▓▓▓▓▓▓▓▓
%▓▓ ╔╦╗╦╔╦╗╦  ╔═╗  ╔═╗╔═╗╔╗╔╔╦╗ ▓▓
%▓▓  ║ ║ ║ ║  ║╣   ╠╣ ║ ║║║║ ║  ▓▓
%▓▓  ╩ ╩ ╩ ╩═╝╚═╝  ╚  ╚═╝╝╚╝ ╩  ▓▓
%▓▓▓▓▓▓▓▓▓▓▓▓▓▓▓▓▓▓▓▓▓▓▓▓▓▓▓▓▓▓▓▓▓
%\usepackage{titlesec}
%▓▓▓▓▓▓▓▓▓▓▓▓▓▓▓▓▓▓▓▓▓▓▓▓▓▓▓▓▓▓▓▓▓▓▓▓▓▓▓▓▓▓▓▓▓▓▓▓▓▓▓▓▓▓▓
%▓▓ ╔╦╗╔═╗╔╗ ╦  ╔═╗  ╔═╗╔═╗  ╔═╗╔═╗╔╗╔╔╦╗╔═╗╔╗╔╔╦╗╔═╗ ▓▓
%▓▓  ║ ╠═╣╠╩╗║  ║╣   ║ ║╠╣   ║  ║ ║║║║ ║ ║╣ ║║║ ║ ╚═╗ ▓▓
%▓▓  ╩ ╩ ╩╚═╝╩═╝╚═╝  ╚═╝╚    ╚═╝╚═╝╝╚╝ ╩ ╚═╝╝╚╝ ╩ ╚═╝ ▓▓
%▓▓▓▓▓▓▓▓▓▓▓▓▓▓▓▓▓▓▓▓▓▓▓▓▓▓▓▓▓▓▓▓▓▓▓▓▓▓▓▓▓▓▓▓▓▓▓▓▓▓▓▓▓▓▓
\newcommand{\ChapterTableOfContents}{%
    \begingroup
    \addfontfeature{Numbers={Lining,Monospaced}}
    \hypersetup{hidelinks}\tableofcontents%
    \endgroup
}%

\let\DotFill\relax
\makeatletter
\newcommand \DotFill {\leavevmode \cleaders \hb@xt@ .33em{\hss .\hss }\hfill \kern \z@}
\makeatother

\definecolor{ToCGrey}{rgb}{0.4,0.4,0.4}
\definecolor{mainColor}{rgb}{0.82745098,0.18431373,0.18431373}
\usepackage{titletoc}
\titlecontents{part}
[0.0em]
{\addvspace{1pc}\color{TitlingRed}\large\bfseries\text{Part }}
{\bfseries\textcolor{TitlingRed}{\contentslabel{0.0em}}\hspace*{1.35em}}
{}
{\textcolor{TitlingRed}{{\hfill\bfseries\contentspage\nobreak}}}
[]
\titlecontents{section}
[0.0em]
{\addvspace{1pc}}
{\color{black}\bfseries\textcolor{TitlingRed}{\contentslabel{0.0em}}\hspace*{1.65em}}
{}
{\textcolor{black}{\textbf{\DotFill}{\bfseries\contentspage\nobreak}}}
[]
\titlecontents{subsection}
[0.0em]
{}
{\hspace*{1.65em}\color{ToCGrey}{\contentslabel{0.0em}}\hspace*{2.5em}}
{}
{{\textcolor{ToCGrey}\DotFill}\textcolor{ToCGrey}{\contentspage}\nobreak}
[]
\usepackage{marginnote}
\renewcommand*{\marginfont}{\normalfont}
\usepackage{inconsolata}
\setmonofont{inconsolata}%
\let\ChapterRef\relax
\newcommand{\ChapterRef}[2]{#1}
\AtBeginEnvironment{subappendices}{%%
    \section*{\huge Appendices}%
}%

\begin{document}

\title{Monoidal Categories}

\maketitle

\phantomsection
\label{section-phantom}

This chapter contains some material on monoidal categories.

\ChapterTableOfContents

TODO:
\begin{itemize}
    \item Or, equivalently, a one-object bicategory.
    \item $\Hom_{\CatFont{C}}(A\otimes B,C)\cong\Nat(h_{A}\boxtimes h_{B},h_{C}\circ\mathord{\otimes})$, as in \url{https://mathoverflow.net/questions/268193/generalized-elements-in-monoidal-categories}
        \begin{itemize}
            \item There should be a monoidal enhancement of this, when $A$, $B$, and $C$ are co/monoids
        \end{itemize}
\end{itemize}

\section{Monoidal Categories}\label{section-monoidal-categories}
\subsection{Foundations}\label{subsection-monoidal-categories-foundations}
\begin{definition}{Monoidal Categories}{monoidal-categories}%
    A \index[categories]{monoidal category}\textbf{monoidal category} is a pseudomonoid $\smash{\big(\CatFont{C},\otimes,\Unit,\alpha,\LUnitor,\RUnitor\big)}$ in $(\TwoCategoryOfCategories,\times,\PunctualCategory)$.%
\end{definition}
\begin{remark}{Unwinding \cref{monoidal-categories}}{unwinding-monoidal-categories}%
    In detail, a \textbf{monoidal category} $\smash{\big(\CatFont{C},\otimes,\Unit,\alpha,\LUnitor,\RUnitor\big)}$ consists of:%
    %--- Begin Footnote ---%
    \footnote{%
        \SloganFont{Further Notation: }When working with more than one monoidal category, we will use subscripts and superscripts in the notation, writing e.g.\ $\smash{\big(\CatFont{C},\otimes_{\CatFont{C}},\Unit,\alpha,\LUnitor,\RUnitor\big)}$.
    }%
    %---  End Footnote  ---%
    \begin{itemize}
        \item\SloganFont{The Underlying Category. }A category $\CatFont{C}$.
        \item\SloganFont{The Monoidal Product. }A functor
            \[
                \otimes
                \colon
                \CatFont{C}\times\CatFont{C}
                \rightarrow
                \CatFont{C},
            \]%
            called the \textbf{monoidal product}%
            %--- Begin Footnote ---%
            \footnote{%
                \SloganFont{Further Terminology: }Also called the \textbf{tensor product} of $\CatFont{C}$.
            } %
            %---  End Footnote  ---%
            of $\CatFont{C}$.
        \item\SloganFont{The Monoidal Unit. }A functor
            \[
                \Unit
                \colon
                \PunctualCategory
                \rightarrow
                \CatFont{C}
            \]
            determining an object $\Unit$ of $\CatFont{C}$, called the \textbf{monoidal unit of $\CatFont{C}$}.
        \item\SloganFont{The Associators. }A natural isomorphism
            \begin{webcompile}
                \alpha
                \colon
                \mathord{\otimes}\circ(\mathord{\otimes}\times\id)
                \Longisorightarrow
                \mathord{\otimes}\circ(\id_{\CatFont{C}}\times\mathord{\otimes}),
                \qquad
                \begin{tikzcd}[row sep={5.0*\the\DL,between origins}, column sep={8.0*\the\DL,between origins}, background color=backgroundColor, ampersand replacement=\&]
                    \CatFont{C}\times\CatFont{C}\times\CatFont{C}
                    \arrow[r,"\id_{\CatFont{C}}\times\mathord{\otimes}"]
                    \arrow[d,"\mathord{\otimes}\times\id_{\CatFont{C}}"']
                    \&
                    \CatFont{C}\times\CatFont{C}
                    \arrow[d,"\otimes"]
                    \\
                    \CatFont{C}\times\CatFont{C}
                    \arrow[r,"\otimes"']
                    \&
                    \CatFont{C}\mrp{,}
                    % 2-arrows
                    \arrow[from=2-1,to=1-2,"\alpha"description,shorten=0.5em,Rightarrow]%
                \end{tikzcd}
            \end{webcompile}
            called the \textbf{associator} of $\CatFont{C}$, with components
            \[
                \alpha_{A,B,C}
                \colon
                (A\otimes B)\otimes C
                \isorightarrow
                A\otimes(B\otimes C).
            \]%
        \item\SloganFont{The Left Unitors. }A natural isomorphism
            \begin{webcompile}
                \LUnitor
                \colon
                \mathord{\otimes}\circ(\Unit\times\id_{\CatFont{C}})
                \Longisorightarrow
                \id_{\CatFont{C}},
                \qquad
                \begin{tikzcd}[row sep={7.0*\the\DL,between origins}, column sep={7.0*\the\DL,between origins}, background color=backgroundColor, ampersand replacement=\&]
                    \PunctualCategory\times\CatFont{C}
                    \arrow[r,  "\Unit\times\id_{\CatFont{C}}"]
                    \arrow[rd, "\id_{\CatFont{C}}"'{name=1},bend right=30]
                    \&
                    \CatFont{C}\times\CatFont{C}
                    \arrow[d, "\otimes"]
                    \\
                    {}
                    \&
                    \CatFont{C}\mathrlap{,}
                    % 2-Arrows
                    \arrow[Rightarrow,from=1-2,to=1,shorten >=1.0em,shorten <=1.0em,"\LUnitor"description]
                \end{tikzcd}
            \end{webcompile}
            called the \textbf{left unitor} of $\CatFont{C}$, with components
            \[
                \LUnitor_{A}
                \colon
                \Unit\otimes A
                \isorightarrow
                A.
            \]%
        \item\SloganFont{The Right Unitors. }A natural isomorphism
            \begin{webcompile}
                \RUnitor
                \colon
                \mathord{\otimes}\circ(\id_{\CatFont{C}}\times\Unit)
                \Longisorightarrow
                \id_{\CatFont{C}},
                \qquad
                \begin{tikzcd}[row sep={7.0*\the\DL,between origins}, column sep={7.0*\the\DL,between origins}, background color=backgroundColor, ampersand replacement=\&]
                    \CatFont{C}\times\PunctualCategory
                    \arrow[r, "\id_{\CatFont{C}}\times\Unit"]
                    \arrow[rd, "\id_{\CatFont{C}}"'{name=1},bend right=30]
                    \&
                    \CatFont{C}\times\CatFont{C}
                    \arrow[d, "\otimes"]
                    \\
                    {}
                    \&
                    \CatFont{C}\mathrlap{,}
                    % 2-Arrows
                    \arrow[Rightarrow,from=1-2,to=1,shorten >=1.0em,shorten <=1.0em,"\RUnitor"description]
                \end{tikzcd}
            \end{webcompile}
            called the \textbf{right unitor} of $\CatFont{C}$, with components
            \[
                \RUnitor_{A}
                \colon
                A\otimes\Unit
                \isorightarrow
                A.
            \]%
    \end{itemize}
    satisfying the following conditions:%
    %--- Begin Footnote ---%
    \footnote{%
        The conditions in \cref{unwinding-monoidal-categories-interaction-of-identities-with-the-monoidal-product,unwinding-monoidal-categories-interaction-of-composition-with-the-monoidal-product} are already captured in the requirement for $\otimes$ to be a functor, in particular preserving identities and composition. A similar remark applies to \cref{unwinding-monoidal-categories-naturality-of-the-associator,unwinding-monoidal-categories-naturality-of-the-left-unitor,unwinding-monoidal-categories-naturality-of-the-right-unitor}. Nevertheless, we spell out these conditions here for convenience.
    }%
    %---  End Footnote  ---%
    \begin{enumerate}
        \item\label{unwinding-monoidal-categories-interaction-of-identities-with-the-monoidal-product}\SloganFont{Interaction of Identities With the Monoidal Product. }For each $A,B\in\Obj(\CatFont{C})$, we have
            \[
                \id_{A\otimes B}
                =
                \id_{A}\otimes\id_{B}.
            \]%
        \item\label{unwinding-monoidal-categories-interaction-of-composition-with-the-monoidal-product}\SloganFont{Interaction of Composition With the Monoidal Product. }For each pair $\smash{A\xlongrightarrow{f}B\xlongrightarrow{g}C}$ and $\smash{X\xlongrightarrow{h}Y\xlongrightarrow{k}Z}$ of composable pairs of morphisms of $\CatFont{C}$, we have%
            %--- Begin Footnote ---%
            \footnote{%
                \SloganFont{Further Terminology: }This is called the \textbf{law of middle-four exchange}.
                \par\vspace*{\TCBBoxCorrection}
            }%
            %---  End Footnote  ---%
            \[
                (g\otimes k)\circ(f\otimes h)
                =
                (g\circ f)\otimes(k\circ h),
            \]%
            i.e.\ the diagram
            \[
                \begin{tikzcd}[row sep={5.0*\the\DL,between origins}, column sep={6.0*\the\DL,between origins}, background color=backgroundColor, ampersand replacement=\&]
                    A\otimes X
                    \arrow[r,"f\otimes h"]
                    \arrow[rd,"{(g\circ f)\otimes(k\circ h)}"']
                    \&
                    \arrow[d,"g\otimes k"]
                    B\otimes Y
                    \\
                    \&
                    C\otimes Z
                \end{tikzcd}
            \]%
            commutes.
        \item\label{unwinding-monoidal-categories-naturality-of-the-associator}\SloganFont{Naturality of the Associator. }Given morphisms $f\colon A\to X$, $g\colon B\to Y$, and $h\colon C\to Z$ of $\CatFont{C}$, the diagram
            \[
                \begin{tikzcd}[row sep={6.0*\the\DL,between origins}, column sep={11.0*\the\DL,between origins}, background color=backgroundColor, ampersand replacement=\&]
                    (A\otimes B)\otimes C
                    \arrow[r,"(f\otimes g)\otimes h"]
                    \arrow[d,"\alpha_{A,B,C}"']
                    \&
                    (X\otimes Y)\otimes Z
                    \arrow[d,"\alpha_{X,Y,Z}"]
                    \\
                    A\otimes(B\otimes C)
                    \arrow[r,"f\otimes(g\otimes h)"']
                    \&
                    X\otimes(Y\otimes Z)
                \end{tikzcd}
            \]%
            commutes.
        \item\label{unwinding-monoidal-categories-naturality-of-the-left-unitor}\SloganFont{Naturality of the Left Unitor. }Given a morphism $f\colon A\to B$ of $\CatFont{C}$, the diagram
            \[
                \begin{tikzcd}[row sep={5.0*\the\DL,between origins}, column sep={6.5*\the\DL,between origins}, background color=backgroundColor, ampersand replacement=\&]
                    A\otimes\Unit
                    \arrow[r,"f\otimes\id_{\Unit}"]
                    \arrow[d,"\LUnitor_{A}"']
                    \&
                    B\otimes\Unit
                    \arrow[d,"\LUnitor_{B}"]
                    \\
                    A
                    \arrow[r,"f"']
                    \&
                    B
                \end{tikzcd}
            \]%
            commutes.
        \item\label{unwinding-monoidal-categories-naturality-of-the-right-unitor}\SloganFont{Naturality of the Right Unitor. }Given a morphism $f\colon A\to B$ of $\CatFont{C}$, the diagram
            \[
                \begin{tikzcd}[row sep={5.0*\the\DL,between origins}, column sep={6.5*\the\DL,between origins}, background color=backgroundColor, ampersand replacement=\&]
                    \Unit\otimes A
                    \arrow[r,"\id_{\Unit}\otimes f"]
                    \arrow[d,"\RUnitor_{A}"']
                    \&
                    \Unit\otimes B
                    \arrow[d,"\RUnitor_{B}"]
                    \\
                    A
                    \arrow[r,"f"']
                    \&
                    B
                \end{tikzcd}
            \]%
            commutes.
        \item\label{unwinding-monoidal-categories-the-pentagon-identity}\SloganFont{The Pentagon Identity. }We have an equality
            \begin{scalemath}
                \begin{tikzcd}[row sep={7.0*\the\DL,between origins}, column sep={7.0*\the\DL,between origins}, background color=backgroundColor, ampersand replacement=\&]
                    \CatFont{C}\times\CatFont{C}\times\CatFont{C}\times\CatFont{C}
                    \arrow[rr, "\id_{\CatFont{C}\times\CatFont{C}}\times\mathord{\otimes}"]
                    \arrow[dd, "\mathord{\otimes}\times\id_{\CatFont{C}\times\CatFont{C}}"']
                    \arrow[rd, "\id_{\CatFont{C}}\times(\mathord{\otimes}\times\id_{\CatFont{C}})"{name=1,description}]
                    \&
                    \&
                    \CatFont{C}\times\CatFont{C}\times\CatFont{C}
                    \arrow[rd, "\id_{\CatFont{C}}\times\mathord{\otimes}"]
                    \&
                    \\
                    \&
                    \CatFont{C}\times\CatFont{C}\times\CatFont{C}
                    \arrow[rr, "\id_{\CatFont{C}}\times\mathord{\otimes}"description]
                    \arrow[dd, "\mathord{\otimes}\times\id_{\CatFont{C}}"description]
                    \&
                    \&
                    \CatFont{C}\times\CatFont{C}
                    \arrow[dd, "\mathord{\otimes}"description]
                    \\
                    \CatFont{C}\times\CatFont{C}\times\CatFont{C}
                    \arrow[rd, "\mathord{\otimes}\times\id_{\CatFont{C}}"'{name=2}]
                    \&
                    \&
                    \&
                    \\
                    \&
                    \CatFont{C}\times\CatFont{C}
                    \arrow[rr, "\mathord{\otimes}"']
                    \&\&
                    \CatFont{C}
                    % 2-Arrows
                    \arrow[from=3-1,to=2-2,"\alpha"description,shorten=0.5em,Rightarrow]%
                    \arrow[from=2-2,to=1-3,"\id_{\id_{\CatFont{C}}}\times\alpha"description,shorten=0.5em,Rightarrow]%
                    \arrow[from=4-2,to=2-4,"\alpha"description,Rightarrow,shorten=3.0em]
                \end{tikzcd}%
                \qquad
                \bigequalssign
                \qquad
                \begin{tikzcd}[row sep={7.0*\the\DL,between origins}, column sep={7.0*\the\DL,between origins}, background color=backgroundColor, ampersand replacement=\&]
                    \CatFont{C}\times\CatFont{C}\times\CatFont{C}\times\CatFont{C}
                    \arrow[rr, "\id_{\CatFont{C}\times\CatFont{C}}\times\mathord{\otimes}"]
                    \arrow[dd, "\mathord{\otimes}\times\id_{\CatFont{C}\times\CatFont{C}}"']
                    \&
                    \&
                    \CatFont{C}\times\CatFont{C}\times\CatFont{C}
                    \arrow[dd, "\mathord{\otimes}\times\id_{\CatFont{C}}"description,dashed]
                    \arrow[rd, "\id_{\CatFont{C}}\times\mathord{\otimes}"]
                    \&
                    \\
                    \&
                    \&
                    \&
                    \CatFont{C}\times\CatFont{C}
                    \arrow[dd, "\mathord{\otimes}"]
                    \\
                    \CatFont{C}\times\CatFont{C}\times\CatFont{C}
                    \arrow[rr, "\id_{\CatFont{C}}\times\mathord{\otimes}"description,dashed]
                    \arrow[rd, "\mathord{\otimes}\times\id_{\CatFont{C}}"'{name=2}]
                    \&
                    \&
                    \CatFont{C}\times\CatFont{C}
                    \arrow[rd, "\mathord{\otimes}"description,dashed]
                    \&
                    \\
                    \&
                    \CatFont{C}\times\CatFont{C}
                    \arrow[rr, "\mathord{\otimes}"']
                    \&\&
                    \CatFont{C}
                    % 2-Arrows
                    \arrow[from=3-3,to=2-4,"\alpha"description,shorten=0.5em,Rightarrow]%
                    \arrow[from=3-1,to=1-3,"\sfid",phantom]%
                    \arrow[from=4-2,to=3-3,"\alpha"description,shorten=0.5em,Rightarrow]%
                \end{tikzcd}
            \end{scalemath}
            of pasting diagrams in $\TwoCategoryOfCategories$, i.e.\ for each $A,B,C,D\in\Obj(\CatFont{C})$, the diagram
            \begin{scalemath}
                \begin{tikzcd}[row sep={0*\the\DL,between origins}, column sep={0*\the\DL,between origins}, background color=backgroundColor, ampersand replacement=\&]
                    \&[0.30901699437\FourCmPlusHalf]
                    \&[0.5\FourCmPlusHalf]
                    ((A\otimes B)\otimes C)\otimes D
                    \arrow[lld, "\alpha_{A,B,C}\otimes\id_{D}"']
                    \arrow[rrd, "\alpha_{A\otimes B,C,D}"]
                    \&[0.5\FourCmPlusHalf]
                    \&[0.30901699437\FourCmPlusHalf]
                    \\[0.58778525229\FourCmPlusHalf]
                    (A\otimes(B\otimes C))\otimes D
                    \arrow[rd, "\alpha_{A,B\otimes C,D}"']
                    \&[0.30901699437\FourCmPlusHalf]
                    \&[0.5\FourCmPlusHalf]
                    \&[0.5\FourCmPlusHalf]
                    \&[0.30901699437\FourCmPlusHalf]
                    (A\otimes B)\otimes(C\otimes D)
                    \arrow[ld, "\alpha_{A,B,C\otimes D}"]
                    \\[0.95105651629\FourCmPlusHalf]
                    \&[0.30901699437\FourCmPlusHalf]
                    A\otimes((B\otimes C)\otimes D)
                    \arrow[rr, "\id_{A}\otimes\alpha_{B,C,D}"']
                    \&[0.5\FourCmPlusHalf]
                    \&[0.5\FourCmPlusHalf]
                    A\otimes(B\otimes(C\otimes D))
                    \&[0.30901699437\FourCmPlusHalf]
                \end{tikzcd}
            \end{scalemath}
            commutes.
        \item\label{unwinding-monoidal-categories-the-triangle-identity}\SloganFont{The Triangle Identity. }We have an equality
            \begin{scalemath}
                \begin{tikzcd}[row sep={8.0*\the\DL,between origins}, column sep={10.0*\the\DL,between origins}, background color=backgroundColor, ampersand replacement=\&]
                    \&
                    \CatFont{C}\times\PunctualCategory\times\CatFont{C}
                    \arrow[ldd,"\id_{\CatFont{C}}\times\id_{\CatFont{C}}"'{name=B}, bend right=20]
                    \arrow[rdd,"\id_{\CatFont{C}}\times\id_{\CatFont{C}}"{name=A}, bend left=20]
                    \arrow[d,  "\id_{\CatFont{C}}\times\Unit\times\id_{\CatFont{C}}"{description}, dashed]
                    \&
                    \\
                    \&
                    \CatFont{C}\times\CatFont{C}\times\CatFont{C}
                    \arrow[ld, "\mathord{\otimes}\times\id_{\CatFont{C}}"{description}, dashed]
                    \&
                    \\
                    \CatFont{C}\times\CatFont{C}
                    \arrow[rd, "\otimes"'{name=f03}, bend right=20]
                    \&\&
                    \CatFont{C}\times\CatFont{C}
                    \arrow[ld, "\otimes", bend left=20]
                    \\
                    \&
                    \CatFont{C}
                    \&
                    % 2-Arrows
                    %\arrow[from=3-1, to=3-3, Rightarrow, shorten=2.0em, "\sfid"{description,yshift=-0.4em},bend right=20,yshift=+0.0em]
                    \arrow[from=3-1,to=3-3,"\sfid",phantom]%
                    \arrow[from=2-2, to=B, Rightarrow, shorten >=0.5em, "\RUnitor\times\sfid"{sloped},pos=0.35]
                \end{tikzcd}%
                \qquad
                \bigequalssign
                \qquad
                \begin{tikzcd}[row sep={8.0*\the\DL,between origins}, column sep={10.0*\the\DL,between origins}, background color=backgroundColor, ampersand replacement=\&]
                    \&
                    \CatFont{C}\times\PunctualCategory\times\CatFont{C}
                    \arrow[rdd,"\id_{\CatFont{C}}\times\id_{\CatFont{C}}"{name=A}, bend left=20]
                    \arrow[d,  "\id_{\CatFont{C}}\times\Unit\times\id_{\CatFont{C}}"{description}]
                    \&
                    \\
                    \&
                    \CatFont{C}\times\CatFont{C}\times\CatFont{C}
                    \arrow[rd, "\id_{\CatFont{C}}\times\mathord{\otimes}"'{description,name=f13}]
                    \arrow[ld, "\mathord{\otimes}\times\id_{\CatFont{C}}"{description}]
                    \&
                    \\
                    \CatFont{C}\times\CatFont{C}
                    \arrow[rd, "\mathord{\otimes}"'{name=f03}, bend right=20]
                    \&\&
                    \CatFont{C}\times\CatFont{C}
                    \arrow[ld, "\mathord{\otimes}", bend left=20]
                    \\
                    \&
                    \CatFont{C}
                    \&
                    % 2-Arrows
                    \arrow[from=3-1, to=3-3, Rightarrow, shorten=4.0em, "\alpha"description]
                    \arrow[from=2-2, to=A, Rightarrow, shorten >=0.5em, "\sfid\times\LUnitor"{sloped},pos=0.4]
                \end{tikzcd}
            \end{scalemath}
            of pasting diagrams in $\TwoCategoryOfCategories$, i.e.\ for each $A,B\in\Obj(\CatFont{C})$, the diagram
            \begin{webcompile}
                \begin{tikzcd}[row sep={6.0*\the\DL,between origins}, column sep={5.0*\the\DL,between origins}, background color=backgroundColor, ampersand replacement=\&]
                    (A\otimes\Unit)\otimes B
                    \arrow[rr, "\alpha_{A,\Unit,B}"]
                    \arrow[rd, "\RUnitor_{A}\otimes\id_{B}"']
                    \&
                    \&
                    A\otimes(\Unit\otimes B)
                    \arrow[ld, "\id_{A}\otimes\LUnitor_{B}"]
                    \\
                    \&
                    A\otimes B
                    \&
                \end{tikzcd}%
            \end{webcompile}
            commutes.
    \end{enumerate}
\end{remark}
\begin{definition}{Strict Monoidal Categories}{strict-monoidal-categories}%
    A monoidal category is \index[categories]{monoidal category!strict}\textbf{strict} if its associators, left unitors, and right unitors are identities.
\end{definition}
\begin{example}{Examples of Monoidal Categories}{examples-of-monoidal-categories}%
    Here are some examples of monoidal categories.
    \begin{enumerate}
        \item\label{examples-of-monoidal-categories-the-integers-with-addition}\SloganFont{The Integers With Addition. }The category $\Z_{\disc}$ has a strict monoidal structure $(\oplus,\Unit_{\Z})$ where
            \begin{itemize}
                \item We have $\Unit_{\Z}\defeq0$;
                \item For each $m,n\in\Z$, we have $m\oplus n\defeq m+n$.
            \end{itemize}
        \item\label{examples-of-monoidal-categories-the-integers-with-multiplication}\SloganFont{The Integers With Multiplication. }The category $\Z_{\disc}$ has a strict monoidal structure $(\otimes,\Unit_{\Z})$ where
            \begin{itemize}
                \item We have $\Unit_{\Z}\defeq1$;
                \item For each $m,n\in\Z$, we have $m\otimes n\defeq mn$.
            \end{itemize}
        \item\label{examples-of-monoidal-categories-monoids}\SloganFont{Monoids. }More generally, every monoid $A$ gives rise to a strict monoidal category $A_{\disc}$; see \cref{TODO}.
        \item\label{examples-of-monoidal-categories-cartesian-monoidal-categories}\SloganFont{Cartesian Monoidal Categories. }Every category $\CatFont{C}$ with binary products and a terminal object $\star$ can be endowed with a monoidal structure $(\times,\star)$; see \cref{sec:cartesian-monoidal-categories}. Particular cases of this include:
            \begin{itemize}
                \item $\Sets$     (\ChapterRef{\ChapterMonoidalStructuresOnTheCategoryOfSets, \cref{monoidal-structures-on-the-category-of-sets:the-monoidal-structure-on-sets-associated-to-the-product}}{\cref{the-monoidal-structure-on-sets-associated-to-the-product}}).
                \item $\Top$      (\cref{TODO}).
                \item $\sSets$    (\cref{TODO}).
                \item $\Cats$     (\cref{TODO}).
                \item $\Grpd$     (\cref{TODO}).
                \item $\Sch_{/S}$ (\cref{TODO}).
            \end{itemize}
        \item\label{examples-of-monoidal-categories-categories-and-joins}\SloganFont{Categories and Joins. }The triple $(\Cats,\star,\emptyset_{\Cats})$, where:
            \begin{itemize}
                \item $\star$ is the join of categories of \cref{TODO};
                \item $\emptyset_{\Cats}$ is the empty category;
            \end{itemize}
            is a monoidal category.
        \item\label{examples-of-monoidal-categories-simplicial-sets}\SloganFont{Simplicial Sets and Joins. }The triple $(\sSets,\star,\emptyset_{\bullet})$, where:
            \begin{itemize}
                \item $\star$ is the join of simplicial sets of \cref{TODO};
                \item $\emptyset_{\bullet}$ is the empty simplicial set;
            \end{itemize}
            is a monoidal category.
        \item\label{examples-of-monoidal-categories-r-modules}\SloganFont{$R$-Modules. }The triple $(\Mod_{R},\otimes_{R},R)$ is a monoidal category.
        \item\label{examples-of-monoidal-categories-abelian-groups}\SloganFont{Abelian Groups. }As a particular example of \cref{examples-of-monoidal-categories-r-modules}, the triple $(\Ab,\otimes_{\Z},\Z)$ is a monoidal category.
        \item\label{examples-of-monoidal-categories-vector-spaces}\SloganFont{Vector Spaces. }As another particular example of \cref{examples-of-monoidal-categories-r-modules}, the triple $(\Vect_{k},\otimes_{k},k)$ is a monoidal category.
        \item\label{examples-of-monoidal-categories-quasicoherent-sheaves}\SloganFont{Quasicoherent Sheaves. }The triple $(\QCoh(X),\otimes_{\SheafFont{O}_{X}},\SheafFont{O}_{X})$ is a monoidal category.
        \item\label{examples-of-monoidal-categories-chain-complexes}\SloganFont{Chain Complexes. }The triple $(\Ch(\Mod_{R}),\otimes^{\Z}_{R},\underline{R}_{0})$ is a monoidal category.
    \end{enumerate}
\end{example}
\begin{proposition}{Properties of Monoidal Categories}{properties-of-monoidal-categories}%
    Let $\smash{\big(\CatFont{C},\otimes,\Unit,\alpha,\LUnitor,\RUnitor\big)}$ be a monoidal category and let $A,B\in\Obj(\CatFont{C})$.
    \begin{enumerate}
        \item\label{properties-of-monoidal-categories-cancellation-of-identities-in-a-monoidal-category}\SloganFont{Cancellation of Identities in Monoidal Categories. }Let $f,g\colon A\rightrightarrows B$ be morphisms of $\CatFont{C}$. The following conditions are equivalent:
            \begin{enumerate}
                \item\label{properties-of-monoidal-categories-cancellation-of-identities-in-a-monoidal-category-a}We have $f=g$.
                \item\label{properties-of-monoidal-categories-cancellation-of-identities-in-a-monoidal-category-b}We have $\id_{\Unit}\otimes f=\id_{\Unit}\otimes g$.
                \item\label{properties-of-monoidal-categories-cancellation-of-identities-in-a-monoidal-category-c}We have $f\otimes\id_{\Unit}=g\otimes\id_{\Unit}$.
            \end{enumerate}
        \item\label{properties-of-monoidal-categories-tensoring-left-unitors-with-identities}\SloganFont{Tensoring Left Unitors With Identities. }We have an equality
            \[
                \LUnitor_{\Unit}\otimes\id_{A}%
                =%
                \LUnitor_{\Unit\otimes A},%
            \]%
            in $\Hom_{\CatFont{C}}(\Unit\otimes(\Unit\otimes A),\Unit\otimes A)$.
        \item\label{properties-of-monoidal-categories-tensoring-right-unitors-with-identities}\SloganFont{Tensoring Right Unitors With Identities. }We have an equality
            \[
                \id_{A}\otimes\RUnitor_{\Unit}%
                =%
                \RUnitor_{A\otimes\Unit},%
            \]%
            in $\Hom_{\CatFont{C}}((A\otimes\Unit)\otimes\Unit,A\otimes\Unit)$.
        \item\label{properties-of-monoidal-categories-tensoring-invertible-morphisms}\SloganFont{Tensoring Invertible Morphisms. }Let $f\colon A\rightarrow B$ and $h\colon X\rightarrow Y$ be morphisms of $\CatFont{C}$.
            \begin{itemize}
                \itemstar If $f$ and $h$ are invertible, then so is $f\otimes g$ and we have
                    \[
                        (f\otimes h)^{-1}
                        =
                        f^{-1}\otimes h^{-1}.
                    \]
            \end{itemize}
        \item\label{properties-of-monoidal-categories-more-triangle-identities-1-left-unitors}\SloganFont{More Triangle Identities \rmI: Left Unitors. }Let $A,B\in\Obj(\CatFont{C})$. The diagram
            \[
                \begin{tikzcd}[row sep={6.0*\the\DL,between origins}, column sep={5.0*\the\DL,between origins}, background color=backgroundColor, ampersand replacement=\&]
                    (\Unit\otimes A)\otimes B
                    \arrow[rr, "\alpha_{\Unit,A,B}"]
                    \arrow[rd, "\LUnitor_{A}\otimes\id_{B}"']
                    \&\&
                    \Unit\otimes(A\otimes B)
                    \arrow[ld, "\LUnitor_{A\otimes B}"]
                    \\
                    \&
                    A\otimes B
                    \&
                \end{tikzcd}
            \]%
        \item\label{properties-of-monoidal-categories-more-triangle-identities-2-right-unitors}\SloganFont{More Triangle Identities \rmII: Right Unitors. }Let $A,B\in\Obj(\CatFont{C})$. The diagram
            \[
                \begin{tikzcd}[row sep={6.0*\the\DL,between origins}, column sep={5.0*\the\DL,between origins}, background color=backgroundColor, ampersand replacement=\&]
                    (A\otimes B)\otimes\Unit
                    \arrow[rr, "\alpha_{A,B,\Unit}"]
                    \arrow[rd, "\RUnitor_{A\otimes B}"']
                    \&\&
                    A\otimes(B\otimes\Unit)
                    \arrow[ld, "\id_{A}\otimes\RUnitor_{B}"]
                    \\\&
                    A\otimes B
                    \&
                \end{tikzcd}
            \]%
        \item\label{properties-of-monoidal-categories-coherence-for-left-and-right-unitors-of-the-monoidal-unit}\SloganFont{Coherence for Left and Right Unitors of the Monoidal Unit. }We have an equality
            \[
                \LUnitor_{\Unit}
                =
                \RUnitor_{\Unit},
            \]
            in $\Hom_{\CatFont{C}}(\Unit\otimes\Unit,\Unit)$.
        \item\label{properties-of-monoidal-categories-interaction-with-representable-functors}\SloganFont{Interaction With Representable Functors. }We have an isomorphism%
            %--- Begin Footnote ---%
            \footnote{%
                See also \cref{TODO} for a monoidal enhancement of this isomorphism.%TODO
            }%
            %---  End Footnote  ---%
            \[
                \Hom_{\CatFont{C}}(A\otimes B,C)%
                \cong
                \Nat(h_{A}\boxtimes_{\PSh(\CatFont{C})}h_{B},h_{C}\circ\mathord{\otimes}),%
            \]%
            natural in $A,B,C\in\Obj(\CatFont{C})$, where:
            \begin{itemize}
                \item $h_{A}\boxtimes_{\PSh(\CatFont{C})}h_{B}$ is the functor of \cref{TODO}, given on objects by $(X,Y)\mapsto h_{A}(X)\times h_{B}(Y)$.
                \item $h_{C}\circ\mathord{\otimes}$ is given on objects by $(X,Y)\mapsto h_{C}(X\otimes Y)$.
            \end{itemize}
        %\item\label{properties-of-monoidal-categories-}\SloganFont{. }
    \end{enumerate}
\end{proposition}
\begin{Proof}{Proof of \cref{properties-of-monoidal-categories}}%
    \FirstProofBox{\cref{properties-of-monoidal-categories-cancellation-of-identities-in-a-monoidal-category}: Cancellation of Identities in Monoidal Categories}%
    We claim that \cref{properties-of-monoidal-categories-cancellation-of-identities-in-a-monoidal-category-a,properties-of-monoidal-categories-cancellation-of-identities-in-a-monoidal-category-b,properties-of-monoidal-categories-cancellation-of-identities-in-a-monoidal-category-c} are equivalent.

    \SubProofBox{\cref{properties-of-monoidal-categories-cancellation-of-identities-in-a-monoidal-category-a}$\implies$\cref{properties-of-monoidal-categories-cancellation-of-identities-in-a-monoidal-category-b,properties-of-monoidal-categories-cancellation-of-identities-in-a-monoidal-category-c}}%
    This follows from the well-definedness of $\otimes$.

    \SubProofBox{\cref{properties-of-monoidal-categories-cancellation-of-identities-in-a-monoidal-category-b}$\implies$\cref{properties-of-monoidal-categories-cancellation-of-identities-in-a-monoidal-category-a}}%
    Consider the diagrams
    \begin{webcompile}
        \begin{tikzcd}[row sep={6.0*\the\DL,between origins}, column sep={6.0*\the\DL,between origins}, background color=backgroundColor, ampersand replacement=\&]
            \Unit\otimes A
            \arrow[r, "\LUnitor_{A}",pos=0.45]
            \arrow[d, "\id_{\Unit}\otimes f"']
            \&
            A
            \arrow[d, "f"]
            \\
            \Unit\otimes B
            \arrow[r, "\LUnitor_{B}"',pos=0.45]
            \&
            B
        \end{tikzcd}
        \qquad
        \begin{tikzcd}[row sep={6.0*\the\DL,between origins}, column sep={6.0*\the\DL,between origins}, background color=backgroundColor, ampersand replacement=\&]
            \Unit\otimes A
            \arrow[r, "\LUnitor_{A}",pos=0.45]
            \arrow[d, "\id_{\Unit}\otimes g"']
            \&
            B
            \arrow[d, "g"]
            \\
            \Unit\otimes B
            \arrow[r, "\LUnitor_{B}"',pos=0.45]
            \&
            B\mrp{,}
        \end{tikzcd}
    \end{webcompile}
    which commute by the naturality of the left unitor of $\CatFont{C}$. Since $\LUnitor_{A}$ and $\LUnitor_{B}$ are invertible and $\id_{\Unit}\otimes f=\id_{\Unit}\otimes g$, it follows that $f=g$.

    \SubProofBox{\cref{properties-of-monoidal-categories-cancellation-of-identities-in-a-monoidal-category-c}$\implies$\cref{properties-of-monoidal-categories-cancellation-of-identities-in-a-monoidal-category-a}}%
    Consider the diagrams
    \begin{webcompile}
        \begin{tikzcd}[row sep={6.0*\the\DL,between origins}, column sep={6.0*\the\DL,between origins}, background color=backgroundColor, ampersand replacement=\&]
            A\otimes\Unit
            \arrow[r, "\RUnitor_{A}",pos=0.45]
            \arrow[d, "f\otimes\Unit"']
            \&
            A
            \arrow[d, "f"]
            \\
            B\otimes\Unit
            \arrow[r, "\RUnitor_{B}"',pos=0.45]
            \&
            B
        \end{tikzcd}
        \qquad
        \begin{tikzcd}[row sep={6.0*\the\DL,between origins}, column sep={6.0*\the\DL,between origins}, background color=backgroundColor, ampersand replacement=\&]
            B\otimes\Unit
            \arrow[r, "\RUnitor_{B}",pos=0.45]
            \arrow[d, "g"description]
            \&
            B
            \arrow[d, "g\otimes\Unit"]
            \\
            B\otimes\Unit
            \arrow[r, "\RUnitor_{B}"',pos=0.45]
            \&
            B\mrp{,}
        \end{tikzcd}
    \end{webcompile}
    which commute by the naturality of the right unitor of $\CatFont{C}$. Since $\RUnitor_{A}$ and $\RUnitor_{B}$ are invertible and $f\otimes\id_{\Unit}=g\otimes\id_{\Unit}$, it follows that $f=g$.

    \ProofBox{\cref{properties-of-monoidal-categories-tensoring-left-unitors-with-identities}: Tensoring Left Unitors With Identities}%
    Consider the diagram
    \[
        \begin{tikzcd}[row sep={5.0*\the\DL,between origins}, column sep={8.0*\the\DL,between origins}, background color=backgroundColor, ampersand replacement=\&]
            {\Unit\otimes(\Unit\otimes A)}
            \arrow[r, "\LUnitor_{\Unit\otimes A}"]
            \arrow[d, "\id_{\Unit}\otimes\LUnitor_{A}"']
            \&
            \Unit\otimes A
            \arrow[d, "\LUnitor_{A}"]
            \\
            \Unit\otimes A
            \arrow[r, "\LUnitor_{A}"']
            \&
            A\mrp{,}
        \end{tikzcd}
    \]%
    which commutes by the naturality of the left unitor of $\CatFont{C}$. Since $\LUnitor_{A}$ is invertible, the equality
    \[
        \LUnitor_{\Unit}\otimes\id_{A}%
        =%
        \LUnitor_{\Unit\otimes A}%
    \]%
    follows.

    \ProofBox{\cref{properties-of-monoidal-categories-tensoring-right-unitors-with-identities}: Tensoring Right Unitors With Identities}%
    Consider the diagram
    \[
        \begin{tikzcd}[row sep={5.0*\the\DL,between origins}, column sep={8.0*\the\DL,between origins}, background color=backgroundColor, ampersand replacement=\&]
            (A\otimes\Unit)\otimes\Unit
            \arrow[r, "\RUnitor_{A\otimes\Unit}"]
            \arrow[d, "\RUnitor_{\Unit}\otimes\id_{\Unit}"']
            \&
            A\otimes\Unit
            \arrow[d, "\RUnitor_{A}"]
            \\
            A\otimes\Unit
            \arrow[r, "\RUnitor_{A}"']
            \&
            A\mrp{.}
        \end{tikzcd}
    \]%
    which commutes by the naturality of the right unitor of $\CatFont{C}$. Since $\RUnitor_{A}$ is invertible, the equality
    \[
        \id_{A}\otimes\RUnitor_{\Unit}%
        =%
        \RUnitor_{A\otimes\Unit}%
    \]%
    follows.

    \ProofBox{\cref{properties-of-monoidal-categories-tensoring-invertible-morphisms}: Tensoring Invertible Morphisms}%
    We have%
    \begin{align*}
        (f^{-1}\otimes g^{-1})\circ(f\otimes g) &= (f^{-1}\circ f)\otimes (g^{-1}\circ g)\\
                                                                          &= \id_{A}\otimes\id_{C}\\
                                                                          &= \id_{A}\otimes B
    \end{align*}
    and
    \begin{align*}
        (f\otimes g)\circ(f^{-1}\otimes g^{-1}) &= (f\circ f^{-1})\otimes (g\circ g^{-1})\\
                                                                          &= \id_{B}\otimes\id_{D}\\
                                                                          &= \id_{B\otimes D},%
    \end{align*}
    where we have used:
    \begin{itemize}
        \item The functoriality of $\otimes$ for the first equality (\cref{unwinding-monoidal-categories-interaction-of-composition-with-the-monoidal-product} of \cref{unwinding-monoidal-categories});
        \item The functoriality of $\otimes$ for the third equality (\cref{unwinding-monoidal-categories-interaction-of-identities-with-the-monoidal-product} of \cref{unwinding-monoidal-categories});
    \end{itemize}
    for both chains of equalities.

    \ProofBox{\cref{properties-of-monoidal-categories-more-triangle-identities-1-left-unitors}: More Triangle Identities \rmI: Left Unitors}%
    Consider the diagram
    \begin{scalemath}
        \begin{tikzcd}[row sep={0.0*\the\DL,between origins}, column sep={0.0*\the\DL,between origins}, background color=backgroundColor, ampersand replacement=\&]
            \&[0.5\ThreeCmPlusAQuarter]
            \&[0.30901699437\ThreeCmPlusAQuarter]
            \&[0.5\ThreeCmPlusAQuarter]
            (\Unit\otimes\Unit)\otimes(A\otimes B)
            \arrow[rrd, "\alpha_{\Unit\otimes\Unit,A,B}"]
            \arrow[rdd, "\RUnitor_{\Unit}\otimes\id_{A\otimes B}"description, bend right=30]
            \&[0.5\ThreeCmPlusAQuarter]
            \&[0.30901699437\ThreeCmPlusAQuarter]
            \&[0.5\ThreeCmPlusAQuarter]
            \\[0.58778525229\ThreeCmPlusAQuarter]
            \&[0.5\ThreeCmPlusAQuarter]
            ((\Unit\otimes\Unit)\otimes A)\otimes B
            \arrow[rru, "\alpha_{\Unit,\Unit,A\otimes B}"]
            \arrow[rd, "(\RUnitor_{\Unit}\otimes\id_{A})\otimes\id_{B}"description]
            \arrow[ldd, "\alpha_{\Unit,\Unit,A}\otimes\id_{B}"', bend right=30]
            \&[0.30901699437\ThreeCmPlusAQuarter]
            \&[0.5\ThreeCmPlusAQuarter]
            \&[0.5\ThreeCmPlusAQuarter]
            \&[0.30901699437\ThreeCmPlusAQuarter]
            \Unit\otimes(\Unit\otimes(A\otimes B))
            \arrow[ld, "\id_{\Unit}\otimes\LUnitor_{A\otimes B}"description]
            \&[0.5\ThreeCmPlusAQuarter]
            \\[0.95105651629\ThreeCmPlusAQuarter]
            \&[0.5\ThreeCmPlusAQuarter]
            \&[0.30901699437\ThreeCmPlusAQuarter]
            (\Unit\otimes A)\otimes B
            \arrow[rr, "\alpha_{\Unit,A,B}"description]
            \&[0.5\ThreeCmPlusAQuarter]
            \&[0.5\ThreeCmPlusAQuarter]
            \Unit\otimes(A\otimes B)
            \&[0.30901699437\ThreeCmPlusAQuarter]
            \&[0.5\ThreeCmPlusAQuarter]
            \\[1.0\ThreeCmPlusAQuarter]
            (\Unit\otimes(\Unit\otimes A))\otimes B
            \arrow[rrrrrr, "\alpha_{\Unit,\Unit\otimes A,B}"',bend right=25]
            \arrow[rru, "(\id_{\Unit}\otimes\LUnitor_{A})\otimes\id_{B}"description]
            \&[0.5\ThreeCmPlusAQuarter]
            \&[0.30901699437\ThreeCmPlusAQuarter]
            \&[0.5\ThreeCmPlusAQuarter]
            \&[0.5\ThreeCmPlusAQuarter]
            \&[0.30901699437\ThreeCmPlusAQuarter]
            \&[0.5\ThreeCmPlusAQuarter]
            \Unit\otimes((\Unit\otimes A)\otimes B)\mrp{,}
            \arrow[llu, "\id_{\Unit}\otimes(\LUnitor_{A}\otimes\id_{B})"description]
            \arrow[luu, "\id_{\Unit}\otimes\alpha_{\Unit,A,B}"', bend right=30]
            % Subdiagrams
            \arrow[from=2-2,to=2-6, phantom, "\scriptstyle(1)", xshift=-0.335\ThreeCmPlusAQuarter,yshift=-0.35\ThreeCmPlusAQuarter]
            \arrow[from=2-2,to=2-6, phantom, "\scriptstyle(2)", xshift=+0.3\ThreeCmPlusAQuarter,  yshift=-0.15\ThreeCmPlusAQuarter]
            \arrow[from=3-3,to=3-5, phantom, "(3)", xshift=-1.0\ThreeCmPlusAQuarter]
            \arrow[from=3-3,to=3-5, phantom, "(4)", xshift=+1.0\ThreeCmPlusAQuarter]
            \arrow[from=3-3,to=3-5, phantom, "(5)", yshift=-0.65\ThreeCmPlusAQuarter]
        \end{tikzcd}
    \end{scalemath}
    where
    \begin{itemize}
        \item The boundary diagram commutes, as it is the pentagon identity of $\CatFont{C}$;
        \item Subdiagram $(1)$ commutes by the naturality of the associator of $\CatFont{C}$ and the functoriality of $\otimes$, which gives $\id_{A\otimes B}=\id_{A}\otimes\id_{B}$;
        \item Subdiagram $(2)$ commutes, as it is the triangle identity of $\CatFont{C}$;
        \item Subdiagram $(3)$ commutes since
            \begin{envsmallsize}
                \begin{align*}
                    ((\id_{\Unit}\otimes\LUnitor_{A})\otimes\id_{B})\circ(\alpha_{\Unit,\Unit,A}\otimes\id_{B}) &= ((\id_{\Unit}\otimes\LUnitor_{A})\circ\alpha_{\Unit,\Unit,A})\otimes(\id_{B}\circ\id_{B})\\
                                                                                                            &= (\RUnitor_{\Unit}\otimes\id_{A})\otimes\id_{B},
                \end{align*}
            \end{envsmallsize}
            where we have used the triangle identity of $\CatFont{C}$;
        \item Subdiagram $(5)$ commutes by the naturality of the associator of $\CatFont{C}$.
    \end{itemize}
    Since every morphism in the above diagram is invertible, it follows that subdiagram $(4)$ commutes. Hence we have
    \begin{align*}
        \id_{\Unit}\otimes(\LUnitor_{A\otimes B}\circ\alpha_{\Unit,A,B}) &= (\id_{\Unit}\circ\id_{\Unit})\otimes(\LUnitor_{A\otimes B}\circ\alpha_{\Unit,A,B})\\
                                                                         &= (\id_{\Unit}\otimes\LUnitor_{A\otimes B})\circ(\id_{\Unit}\otimes\alpha_{\Unit,A,B})\\
                                                                         &= \id_{\Unit}\otimes(\LUnitor_{A}\otimes\id_{B}).
    \end{align*}
    It then follows from \cref{properties-of-monoidal-categories-cancellation-of-identities-in-a-monoidal-category} that we indeed have
    \[
        \LUnitor_{A\otimes B}\circ\alpha_{\Unit,A,B}
        =
        \LUnitor_{A}\otimes\id_{B}.
    \]
    This finishes the proof.

    \ProofBox{\cref{properties-of-monoidal-categories-more-triangle-identities-2-right-unitors}: More Triangle Identities \rmII: Right Unitors}%
    Consider the diagram
    \begin{scalemath}
        \begin{tikzcd}[row sep={0*\the\DL,between origins}, column sep={0*\the\DL,between origins}, background color=backgroundColor, ampersand replacement=\&]
            \&[0.5\ThreeCmPlusHalf]
            \&[0.30901699437\ThreeCmPlusHalf]
            \&[0.5\ThreeCmPlusHalf]
            (A\otimesB)\otimes(\Unit\otimes\Unit)
            \arrow[rrd, "\alpha_{A,B,\Unit\otimes\Unit}"]
            \arrow[ldd, "\id_{A\otimes B}\otimes\LUnitor_{\Unit}"description, bend left=30]
            \&[0.5\ThreeCmPlusHalf]
            \&[0.30901699437\ThreeCmPlusHalf]
            \&[0.5\ThreeCmPlusHalf]
            \\[0.58778525229\ThreeCmPlusHalf]
            \&[0.5\ThreeCmPlusHalf]
            ((A\otimes B)\otimes\Unit)\otimes\Unit
            \arrow[rru, "\alpha_{A\otimes B,\Unit,\Unit}"]
            \arrow[rd, "(\RUnitor_{A\otimes B})\otimes\id_{\Unit}"description]
            \arrow[ldd, "\alpha_{A,B,\Unit}\otimes\id_{\Unit}"', bend right=30]
            \&[0.30901699437\ThreeCmPlusHalf]
            \&[0.5\ThreeCmPlusHalf]
            \&[0.5\ThreeCmPlusHalf]
            \&[0.30901699437\ThreeCmPlusHalf]
            A\otimes(B\otimes(\Unit\otimes\Unit))
            \arrow[ld, "\id_{A}\otimes(\id_{B}\otimes\LUnitor_{\Unit})"description]
            \&[0.5\ThreeCmPlusHalf]
            \\[0.95105651629\ThreeCmPlusHalf]
            \&[0.5\ThreeCmPlusHalf]
            \&[0.30901699437\ThreeCmPlusHalf]
            (A\otimes B)\otimes\Unit
            \arrow[rr, "\alpha_{A,B,\Unit}"description]
            \&[0.5\ThreeCmPlusHalf]
            \&[0.5\ThreeCmPlusHalf]
            A\otimes(B\otimes\Unit)
            \&[0.30901699437\ThreeCmPlusHalf]
            \&[0.5\ThreeCmPlusHalf]
            \\[1.0\ThreeCmPlusHalf]
            (A\otimes(B\otimes\Unit))\otimes\Unit
            \arrow[rrrrrr, "\alpha_{A,B\otimes\Unit,\Unit}"',bend right=25]
            \arrow[rru, "(\id_{A}\otimes\RUnitor_{B})\otimes\id_{\Unit}"description]
            \&[0.5\ThreeCmPlusHalf]
            \&[0.30901699437\ThreeCmPlusHalf]
            \&[0.5\ThreeCmPlusHalf]
            \&[0.5\ThreeCmPlusHalf]
            \&[0.30901699437\ThreeCmPlusHalf]
            \&[0.5\ThreeCmPlusHalf]
            A\otimes((B\otimes\Unit)\otimes\Unit)\mrp{,}
            \arrow[llu, "\id_{A}\otimes(\RUnitor_{B}\otimes\id_{\Unit})"description]
            \arrow[luu, "\id_{A}\otimes\alpha_{B,\Unit,\Unit}"', bend right=30]
            % Subdiagrams
            \arrow[from=2-2,to=2-6, phantom, "\scriptstyle(1)", xshift=-0.335\ThreeCmPlusHalf, yshift=-0.15\ThreeCmPlusHalf]
            \arrow[from=2-2,to=2-6, phantom, "\scriptstyle(2)", xshift=+0.3\ThreeCmPlusHalf, yshift=-0.35\ThreeCmPlusHalf]
            \arrow[from=3-3,to=3-5, phantom, "(3)", xshift=-1.1\ThreeCmPlusHalf]
            \arrow[from=3-3,to=3-5, phantom, "(4)", xshift=+1.1\ThreeCmPlusHalf]
            \arrow[from=3-3,to=3-5, phantom, "(5)", yshift=-0.7\ThreeCmPlusHalf]
        \end{tikzcd}
    \end{scalemath}
    where
    \begin{itemize}
        \item The boundary diagram commutes, as it is the pentagon identity of $\CatFont{C}$;
        \item Subdiagram $(1)$ commutes, as it is the triangle identity of $\CatFont{C}$;
        \item Subdiagram $(2)$ commutes by the naturality of the associator of $\CatFont{C}$ and the fact that horizontal composition preserves identities, i.e.\ that $\id_{A\otimes B}=\id_{A}\otimes\id_{B}$;
        \item Subdiagram $(4)$ commutes since
            \begin{envsmallsize}
                \begin{align*}
                    (\id_{A}\otimes(\id_{B}\otimes\LUnitor_{\Unit}))\circ(\id_{A}\otimes\alpha_{B,\Unit,\Unit}) &= (\id_{A}\circ\id_{B})\otimes(\id_{B}\otimes(\LUnitor_{\Unit}\circ\alpha_{B,\Unit,\Unit}))\\
                                                                                                            &= \id_{A}\otimes(\RUnitor_{B}\otimes\id_{\Unit}),
                \end{align*}
            \end{envsmallsize}
            where we have used the triangle identity of $\CatFont{C}$;
        \item Subdiagram $(5)$ commutes by the naturality of the associator of $\CatFont{C}$.
    \end{itemize}
    Since every morphism in the diagram above is invertible, it follows that subdiagram $(3)$ commutes. Hence we have
    \begin{align*}
        ((\id_{A}\otimes\RUnitor_{B})\circ\alpha_{A,B,\Unit})\otimes\id_{\Unit} &= ((\id_{A}\otimes\RUnitor_{B})\circ\alpha_{A,B,\Unit})\otimes(\id_{\Unit}\circ\id_{\Unit}) \\
                                                &= ((\id_{A}\otimes\RUnitor_{B})\otimes\id_{\Unit})\circ(\alpha_{A,B,\Unit}\otimes\id_{\Unit})\\
                                                &= (\RUnitor_{A\otimes B})\otimes\id_{\Unit}.
    \end{align*}
    It then follows from \cref{properties-of-monoidal-categories-cancellation-of-identities-in-a-monoidal-category} that
    \[(\id_{A}\otimes\RUnitor_{B})\circ\alpha_{A,B,\Unit}=\RUnitor_{A\otimes B}.\]
    This finishes the proof.

    \ProofBox{\cref{properties-of-monoidal-categories-coherence-for-left-and-right-unitors-of-the-monoidal-unit}: Coherence for Left and Right Unitors of the Monoidal Unit}%
    By the triangle identity of $\CatFont{C}$, the diagram
    \[
        \begin{tikzcd}[row sep={7.0*\the\DL,between origins}, column sep={5.0*\the\DL,between origins}, background color=backgroundColor, ampersand replacement=\&]
            {(\Unit\otimes\Unit)\otimes\Unit}
            \arrow[rr, "\alpha_{\Unit,\Unit,\Unit}"]
            \arrow[rd, "\RUnitor_{\Unit}\otimes\id_{\Unit}"']
            \&\&
            {\Unit\otimes(\Unit\otimes\Unit)}
            \arrow[ld, "\id_{\Unit}\otimes\LUnitor_{\Unit}"]
            \\\&
            \Unit\otimes\Unit
            \&
        \end{tikzcd}
    \]%
    %commutes. By \cref{{properties-of-monoidal-categories-more-triangle-identities-2-right-unitors}, the diagram
    %\[
    %    \begin{tikzcd}[row sep={7.0*\the\DL,between origins}, column sep={5.0*\the\DL,between origins}, background color=backgroundColor, ampersand replacement=\&]
    %        {(\Unit\otimes\Unit)\otimes\Unit}
    %        \arrow[rr, "\alpha_{\Unit,\Unit,\Unit}"]
    %        \arrow[rd, "\RUnitor_{\Unit\otimes\Unit}"']
    %        \&\&
    %        {\Unit\otimes(\Unit\otimes\Unit)}
    %        \arrow[ld, "\id_{\Unit}\otimes\RUnitor_{\Unit}"]
    %        \\
    %        \&
    %        \Unit\otimes\Unit
    %        \&
    %    \end{tikzcd}
    %\]%
    %also commutes. Moreover, by \cref{properties-of-monoidal-categories-tensoring-right-unitors-with-identities}, we have
    %\[\RUnitor_{\Unit\otimes\Unit}=\RUnitor_{\Unit}\otimes\id_{\Unit}.\]
    %Since the associator is invertible, it follows that
    %\[\id_{\Unit}\otimes\LUnitor_{\Unit}=\id_{\Unit}\otimes\RUnitor_{\Unit}.\]
    %By \cref{properties-of-monoidal-categories-cancellation-of-identities-in-a-monoidal-category}, we have $\LUnitor_{\Unit}=\RUnitor_{\Unit}$.

    %\ProofBox{\cref{properties-of-monoidal-categories-interaction-with-representable-functors}: Interaction With Representable Functors}%
    %We have
    %\begin{align*}
    %    \Nat(h_{A}\boxtimes_{\PSh(\CatFont{C})}h_{B},h_{C}\circ\mathord{\otimes}) &\cong \int_{(X,Y)\in\CatFont{C}\times\CatFont{C}}\Sets\big(h^{X}_{A}\times h^{Y}_{B},h^{X\otimes Y}_{C}\big)\\
    %                                                                                            &\cong \int_{X\in\CatFont{C}}\int_{Y\in\CatFont{C}}\Sets\big(h^{X}_{A}\times h^{Y}_{B},h^{X\otimes Y}_{C}\big)\\
    %                                                                                            &\cong \int_{Y\in\CatFont{C}}\int_{X\in\CatFont{C}}\Sets\big(h^{X}_{A}\times h^{Y}_{B},h^{X\otimes Y}_{C}\big)\\
    %                                                                                            &\cong \int_{Y\in\CatFont{C}}\int_{X\in\CatFont{C}}\Sets\big(h^{X}_{A},\Sets\big(h^{Y}_{B},h^{X\otimes Y}_{C}\big)\big)\\
    %                                                                                            &\cong \int_{Y\in\CatFont{C}}\Nat\big(h_{A},\Sets\big(h^{Y}_{B},h^{-\otimes Y}_{C}\big)\big)\\
    %                                                                                            &\cong \int_{Y\in\CatFont{C}}\Sets\big(h^{Y}_{B},h^{A\otimes Y}_{C}\big)\\
    %                                                                                            &\cong \Nat\big(h_{B},h^{A\otimes-}_{C}\big)\\
    %                                                                                            &\cong h^{A\otimes B}_{C}\\
    %                                                                                            &=     \Hom_{\CatFont{C}}(A\otimes B,C)
    %\end{align*}
    %where we have used:
    %\begin{itemize}
    %    \item \cref{NATASEND}     for the first natural isomorphism.
    %    \item \cref{FUBINICOENDS} for the second and third natural isomorphisms.
    %    \item \ChapterRef{\ChapterConstructionsWithSets, \cref{constructions-with-sets:properties-of-products-of-sets-adjointness-1} of \cref{constructions-with-sets:properties-of-products-of-sets}}{\cref{properties-of-products-of-sets-adjointness-1} of \cref{properties-of-products-of-sets}} for the fourth natural isomorphism.
    %    \item \cref{NATASEND} again for the fifth natural isomorphism.
    %    \item \ChapterRef{\ChapterPresheavesAndTheYonedaLemma, \cref{presheaves-and-the-yoneda-lemma:the-yoneda-lemma}}{\cref{the-yoneda-lemma}} for the sixth natural isomorphism.
    %    \item \cref{NATASEND} once again for the seventh natural isomorphism.
    %    \item \ChapterRef{\ChapterPresheavesAndTheYonedaLemma, \cref{presheaves-and-the-yoneda-lemma:the-yoneda-lemma}}{\cref{the-yoneda-lemma}} for the last natural isomorphism.
    %\end{itemize}
    %This finishes the proof.
\end{Proof}
\begin{appendices}
\input{ABSOLUTEPATH/chapters2.tex}
\end{appendices}
\end{document}
