\input{preamble}

% OK, start here.
%
\usepackage{fontspec}
\let\hyperwhite\relax
\let\hyperred\relax
\newcommand{\hyperwhite}{\hypersetup{citecolor=white,filecolor=white,linkcolor=white,urlcolor=white}}
\newcommand{\hyperred}{%
\hypersetup{%
    citecolor=TitlingRed,%
    filecolor=TitlingRed,%
    linkcolor=TitlingRed,%
     urlcolor=TitlingRed%
}}
\let\ChapterRef\relax
\newcommand{\ChapterRef}[2]{#1}
\setcounter{tocdepth}{2}
%▓▓▓▓▓▓▓▓▓▓▓▓▓▓▓▓▓▓▓▓▓▓▓▓▓▓▓▓▓▓▓▓▓
%▓▓ ╔╦╗╦╔╦╗╦  ╔═╗  ╔═╗╔═╗╔╗╔╔╦╗ ▓▓
%▓▓  ║ ║ ║ ║  ║╣   ╠╣ ║ ║║║║ ║  ▓▓
%▓▓  ╩ ╩ ╩ ╩═╝╚═╝  ╚  ╚═╝╝╚╝ ╩  ▓▓
%▓▓▓▓▓▓▓▓▓▓▓▓▓▓▓▓▓▓▓▓▓▓▓▓▓▓▓▓▓▓▓▓▓
%\usepackage{titlesec}
%▓▓▓▓▓▓▓▓▓▓▓▓▓▓▓▓▓▓▓▓▓▓▓▓▓▓▓▓▓▓▓▓▓▓▓▓▓▓▓▓▓▓▓▓▓▓▓▓▓▓▓▓▓▓▓
%▓▓ ╔╦╗╔═╗╔╗ ╦  ╔═╗  ╔═╗╔═╗  ╔═╗╔═╗╔╗╔╔╦╗╔═╗╔╗╔╔╦╗╔═╗ ▓▓
%▓▓  ║ ╠═╣╠╩╗║  ║╣   ║ ║╠╣   ║  ║ ║║║║ ║ ║╣ ║║║ ║ ╚═╗ ▓▓
%▓▓  ╩ ╩ ╩╚═╝╩═╝╚═╝  ╚═╝╚    ╚═╝╚═╝╝╚╝ ╩ ╚═╝╝╚╝ ╩ ╚═╝ ▓▓
%▓▓▓▓▓▓▓▓▓▓▓▓▓▓▓▓▓▓▓▓▓▓▓▓▓▓▓▓▓▓▓▓▓▓▓▓▓▓▓▓▓▓▓▓▓▓▓▓▓▓▓▓▓▓▓
\newcommand{\ChapterTableOfContents}{%
    \begingroup
    \addfontfeature{Numbers={Lining,Monospaced}}
    \hypersetup{hidelinks}\tableofcontents%
    \endgroup
}%

\let\DotFill\relax
\makeatletter
\newcommand \DotFill {\leavevmode \cleaders \hb@xt@ .33em{\hss .\hss }\hfill \kern \z@}
\makeatother

\definecolor{ToCGrey}{rgb}{0.4,0.4,0.4}
\definecolor{mainColor}{rgb}{0.82745098,0.18431373,0.18431373}
\usepackage{titletoc}
\titlecontents{part}
[0.0em]
{\addvspace{1pc}\color{TitlingRed}\large\bfseries\text{Part }}
{\bfseries\textcolor{TitlingRed}{\contentslabel{0.0em}}\hspace*{1.35em}}
{}
{\textcolor{TitlingRed}{{\hfill\bfseries\contentspage\nobreak}}}
[]
\titlecontents{section}
[0.0em]
{\addvspace{1pc}}
{\color{black}\bfseries\textcolor{TitlingRed}{\contentslabel{0.0em}}\hspace*{1.65em}}
{}
{\textcolor{black}{\textbf{\DotFill}{\bfseries\contentspage\nobreak}}}
[]
\titlecontents{subsection}
[0.0em]
{}
{\hspace*{1.65em}\color{ToCGrey}{\contentslabel{0.0em}}\hspace*{2.5em}}
{}
{{\textcolor{ToCGrey}\DotFill}\textcolor{ToCGrey}{\contentspage}\nobreak}
[]
\usepackage{marginnote}
\renewcommand*{\marginfont}{\normalfont}
\usepackage{inconsolata}
\setmonofont{inconsolata}%
\let\ChapterRef\relax
\newcommand{\ChapterRef}[2]{#1}
\AtBeginEnvironment{subappendices}{%%
    \section*{\huge Appendices}%
}%

\begin{document}

\title{Constructions With Relations}

\maketitle

\phantomsection
\label{section-phantom}

This chapter contains some material about constructions with relations. Notably, we discuss and explore:
\begin{enumerate}
    \item\label{constructions-with-relations-introduction-item-1}The existence or non-existence of Kan extensions and Kan lifts in the $2$-category $\sfbfRel$ (\cref{section-kan-extensions-and-kan-lifts-in-the-2-category-of-relations}).
    \item\label{constructions-with-relations-introduction-item-2}The various kinds of constructions involving relations, such as graphs, domains, ranges, unions, intersections, products, converse relations, composition of relations, and collages (\cref{section-more-constructions-with-relations}).
\end{enumerate}
This chapter is under revision. TODO:
\begin{enumerate}
    \item Rename range to image
    \item Co/limits in $\sfbfRel$.
\end{enumerate}

\ChapterTableOfContents

\section{Co/Limits in the Category of Relations}\label{section-co-limits-in-the-category-of-relations}
This section is currently just a stub, and will be properly developed later on.
\section{More Constructions With Relations}\label{section-more-constructions-with-relations}
\subsection{The Domain and Range of a Relation}\label{subsection-the-domain-and-range-of-a-relation}
Let $A$ and $B$ be sets.
\begin{definition}{The Domain and Range of a Relation}{the-domain-and-range-of-a-relation}%
    Let $R\colon A\rightproarrow B$ be a relation.%
    %--- Begin Footnote ---%
    \footnote{%
        Following \ChapterRef{\ChapterProfunctors, \cref{profunctors:the-domain-and-range-of-a-profunctor}}{\cref{the-domain-and-range-of-a-profunctor}}, we may compute the (characteristic functions associated to the) domain and range of a relation using the following colimit formulas:
        \begin{align*}
            \chi_{\dom(R)}(a)   &\cong \colim_{b\in B}(R^{b}_{a})\qquad(a\in A)\\
                                &\cong \bigvee_{b\in B}R^{b}_{a},\\
            \chi_{\range(R)}(b) &\cong \colim_{a\in A}(R^{b}_{a})\qquad(b\in B)\\
                                &\cong \bigvee_{a\in A}R^{b}_{a},
        \end{align*}
        where the join $\bigvee$ is taken in the poset $(\TV,\preceq)$ of \ChapterRef{\ChapterConstructionsWithSets, \cref{sets:the-poset-of-truth-values}}{\cref{the-poset-of-truth-values}}.
    }%
    %---  End Footnote  ---%
    %--- Begin Footnote ---%
    \footnote{%
        Viewing $R$ as a function $R\colon A\to\mathcal{P}(B)$, we have
        \begin{align*}
            \dom(R)   &\cong \colim_{y\in Y}(R(y))\\
                      &\cong \bigcup_{y\in Y}R(y),\\
            \range(R) &\cong \colim_{x\in X}(R(x))\\
                      &\cong \bigcup_{x\in X}R(x),
        \end{align*}
        \par\vspace*{\TCBBoxCorrection}
    }%
    %---  End Footnote  ---%
    \begin{enumerate}
        \item\label{the-domain-of-a-relation}The \index[set-theory]{relation!domain of}\textbf{domain of $R$} is the subset \index[notation]{domR@$\dom(R)$}$\dom(R)$ of $A$ defined by
            \[
                \dom(R)%
                \defeq%
                \{%
                    a\in A%
                    \ \middle|\ %
                    \begin{aligned}
                        &\text{there exists some $b\in B$}\\%
                        &\text{such that $a\sim_{R}b$}\\%
                    \end{aligned}
                \}.%
            \]%
        \item\label{the-range-of-a-relation}The \index[set-theory]{relation!range of}\textbf{range of $R$} is the subset \index[notation]{rangeR@$\range(R)$}$\range(R)$ of $B$ defined by
            \[
                \range(R)%
                \defeq%
                \{%
                    b\in B%
                    \ \middle|\ %
                    \begin{aligned}
                        &\text{there exists some $a\in A$}\\%
                        &\text{such that $a\sim_{R}b$}\\%
                    \end{aligned}
                \}.%
            \]%
    \end{enumerate}
\end{definition}
\subsection{Binary Unions of Relations}\label{subsection-binary-unions-of-relations}
Let $A$ and $B$ be sets and let $R$ and $S$ be relations from $A$ to $B$.
\begin{definition}{Binary Unions of Relations}{binary-unions-of-relations}%
    The \index[set-theory]{relation!union of}\textbf{union of $R$ and $S$}%
    %--- Begin Footnote ---%
    \footnote{%
        \SloganFont{Further Terminology: }Also called the \textbf{binary union of $R$ and $S$}, for emphasis.
    } %
    %---  End Footnote  ---%
    is the relation \index[notation]{RunionS@$R\cup S$}$R\cup S$ from $A$ to $B$ defined as follows:%
    \begin{itemize}
        \item Viewing relations from $A$ to $B$ as subsets of $A\times B$, we define%
            %--- Begin Footnote ---%
            \footnote{%
                This is the same as the union of $R$ and $S$ as subsets of $A\times B$.
                \par\vspace*{\TCBBoxCorrection}
            }%
            %---  End Footnote  ---%
            \[
                R\cup S%
                \defeq%
                \{%
                    (a,b)\in B\times A%
                    \ \middle|\ %
                    \text{%
                        we have $a\sim_{R}b$ or $a\sim_{S}b$%
                    }%
                \}.%
            \]%
        \item Viewing relations from $A$ to $B$ as functions $A\to\mathcal{P}(B)$, we define
            \[%
                [R\cup S](a)%
                \defeq%
                R(a)\cup S(a)%
            \]%
            for each $a\in A$.
    \end{itemize}
\end{definition}
\begin{proposition}{Properties of Binary Unions of Relations}{properties-of-binary-unions-of-relations}%
    Let $R$, $S$, $R_{1}$, and $R_{2}$ be relations from $A$ to $B$, and let $S_{1}$ and $S_{2}$ be relations from $B$ to $C$.
    \begin{enumerate}
        \item\label{properties-of-binary-unions-of-relations-interaction-with-converses}\SloganFont{Interaction With Converses. }We have
            \[
                (R\cup S)^{\dagger}
                =
                R^{\dagger}\cup S^{\dagger}.
            \]%
        \item\label{properties-of-binary-unions-of-relations-interaction-with-composition}\SloganFont{Interaction With Composition. }We have
            \[
                (S_{1}\procirc R_{1})
                \cup
                (S_{2}\procirc R_{2})
                \nneq
                (S_{1}\cup S_{2})
                \procirc
                (R_{1}\cup R_{2}).
            \]%
        %\item\label{properties-of-binary-unions-of-relations-}\SloganFont{. }
    \end{enumerate}
\end{proposition}
\begin{Proof}{Proof of \cref{properties-of-binary-unions-of-relations}}%
    \FirstProofBox{\cref{properties-of-binary-unions-of-relations-interaction-with-converses}: Interaction With Converses}%
    Clear.

    \ProofBox{\cref{properties-of-binary-unions-of-relations-interaction-with-composition}: Interaction With Composition}%
    Unwinding the definitions, we see that:
    \begin{itemize}
        \item The condition for $(S_{1}\procirc R_{1})\cup(S_{2}\procirc R_{2})$ is:
            \begin{itemize}
                \item There exists some $b\in B$ such that:
                    \begin{itemize}
                        \item $\color{OIvermillion}{a\sim_{R_{1}}b}$ and $\color{OIblue}{b\sim_{S_{1}}c}$;
                    \end{itemize}
                    or
                    \begin{itemize}\setcounter{enumi}{2}
                        \item $\color{OIvermillion}{a\sim_{R_{2}}b}$ and $\color{OIblue}{b\sim_{S_{2}}c}$;
                    \end{itemize}
            \end{itemize}
        \item The condition for $(S_{1}\cup S_{2})\procirc(R_{1}\cup R_{2})$ is:
            \begin{itemize}
                \item There exists some $b\in B$ such that:
                    \begin{itemize}
                        \item $\color{OIvermillion}{a\sim_{R_{1}}b}$ or $\color{OIvermillion}{a\sim_{R_{2}}b}$;
                    \end{itemize}
                    and
                    \begin{itemize}\setcounter{enumi}{2}
                        \item $\color{OIblue}{b\sim_{S_{1}}c}$ or $\color{OIblue}{b\sim_{S_{2}}c}$.
                    \end{itemize}
            \end{itemize}
    \end{itemize}
    These two conditions may fail to agree (counterexample omitted), and thus the two resulting relations on $A\times C$ may differ.
\end{Proof}
\subsection{Unions of Families of Relations}\label{subsection-unions-of-families-of-relations}
Let $A$ and $B$ be sets and let $\{R_{i}\}_{i\in I}$ be a family of relations from $A$ to $B$.
\begin{definition}{The Union of a Family of Relations}{the-union-of-a-family-of-relations}%
    The \index[set-theory]{relation!union of a family of}\textbf{union of the family $\{R_{i}\}_{i\in I}$} is the relation \index[notation]{unioniiniRi@$\bigcup_{i\in I}R_{i}$}$\bigcup_{i\in I}R_{i}$ from $A$ to $B$ defined as follows:%
    \begin{itemize}
        \item Viewing relations from $A$ to $B$ as subsets of $A\times B$, we define%
            %--- Begin Footnote ---%
            \footnote{%
                This is the same as the union of $\{R_{i}\}_{i\in I}$ as a collection of subsets of $A\times B$.
                \par\vspace*{\TCBBoxCorrection}
            }%
            %---  End Footnote  ---%
            \[%
                \bigcup_{i\in I}R_{i}
                \defeq%
                \{%
                    (a,b)\in(A\times B)^{\times I}%
                    \ \middle|\ %
                    \begin{aligned}
                        &\text{there exists some $i\in I$}\\
                        &\text{such that $a\sim_{R_{i}}b$}
                    \end{aligned}
                \}.%
            \]%
        \item Viewing relations from $A$ to $B$ as functions $A\to\mathcal{P}(B)$, we define
            \[%
                \left[\bigcup_{i\in I}R_{i}\right](a)%
                \defeq%
                \bigcup_{i\in I}R_{i}(a)%
            \]%
            for each $a\in A$.
    \end{itemize}
\end{definition}
\begin{proposition}{Properties of Unions of Families of Relations}{properties-of-unions-of-families-of-relations}%
    Let $A$ and $B$ be sets and let $\{R_{i}\}_{i\in I}$ be a family of relations from $A$ to $B$.
    \begin{enumerate}
        \item\label{properties-of-unions-of-families-of-relations-interaction-with-converses}\SloganFont{Interaction With Converses. }We have
            \[
                (\bigcup_{i\in I}R_{i})^{\dagger}
                =
                \bigcup_{i\in I}R^{\dagger}_{i}.
            \]%
        %\item\label{properties-of-unions-of-families-of-relations-interaction-with-composition}\SloganFont{Interaction With Composition. }We have
        %    \[
        %        \bigcup_{i\in I}(S_{i}\procirc R_{i})
        %        =
        %        (\bigcup_{i\in I}(S_{i})\procirc(\bigcup_{i\in I}(R_{i}).
        %    \]%
        %\item\label{properties-of-unions-of-families-of-relations-interaction-with-intersections}\SloganFont{Interaction With Intersections. }We have
        %    \[
        %        \bigcup_{i\in I}(S_{i}\procirc R_{i})
        %    \]%
        %\item\label{properties-of-unions-of-families-of-relations-}\SloganFont{. }
    \end{enumerate}
\end{proposition}
\begin{Proof}{Proof of \cref{properties-of-unions-of-families-of-relations}}%
    \FirstProofBox{\cref{properties-of-unions-of-families-of-relations-interaction-with-converses}: Interaction With Converses}%
    Clear.
\end{Proof}
\subsection{Binary Intersections of Relations}\label{subsection-binary-intersections-of-relations}
Let $A$ and $B$ be sets and let $R$ and $S$ be relations from $A$ to $B$.
\begin{definition}{Binary Intersections of Relations}{binary-intersections-of-relations}%
    The \index[set-theory]{relation!intersection of}\textbf{intersection of $R$ and $S$}%
    %--- Begin Footnote ---%
    \footnote{%
        \SloganFont{Further Terminology: }Also called the \textbf{binary intersection of $R$ and $S$}, for emphasis.
    } %
    %---  End Footnote  ---%
    is the relation \index[notation]{RintersectionS@$R\cap S$}$R\cap S$ from $A$ to $B$ defined as follows:
    \begin{itemize}
        \item Viewing relations from $A$ to $B$ as subsets of $A\times B$, we define%
            %--- Begin Footnote ---%
            \footnote{%
                This is the same as the intersection of $R$ and $S$ as subsets of $A\times B$.
                \par\vspace*{\TCBBoxCorrection}
            }%
            %---  End Footnote  ---%
            \[
                R\cap S%
                \defeq%
                \{%
                    (a,b)\in B\times A%
                    \ \middle|\ %
                    \text{%
                        we have $a\sim_{R}b$ and $a\sim_{S}b$%
                    }%
                \}.%
            \]%
        \item Viewing relations from $A$ to $B$ as functions $A\to\mathcal{P}(B)$, we define%
            \[%
                [R\cap S](a)%
                \defeq%
                R(a)\cap S(a)%
            \]%
            for each $a\in A$.
    \end{itemize}
\end{definition}
\begin{proposition}{Properties of Binary Intersections of Relations}{properties-of-binary-intersections-of-relations}%
    Let $R$, $S$, $R_{1}$, and $R_{2}$ be relations from $A$ to $B$, and let $S_{1}$ and $S_{2}$ be relations from $B$ to $C$.
    \begin{enumerate}
        \item\label{properties-of-binary-intersections-of-relations-interaction-with-converses}\SloganFont{Interaction With Converses. }We have
            \[
                (R\cap S)^{\dagger}
                =
                R^{\dagger}\cap S^{\dagger}.
            \]%
        \item\label{properties-of-binary-intersections-of-relations-interaction-with-composition}\SloganFont{Interaction With Composition. }We have
            \[
                (S_{1}\procirc R_{1})
                \cap
                (S_{2}\procirc R_{2})
                =
                (S_{1}\cap S_{2})
                \procirc
                (R_{1}\cap R_{2}).
            \]%
        %\item\label{properties-of-binary-intersections-of-relations-}\SloganFont{. }
    \end{enumerate}
\end{proposition}
\begin{Proof}{Proof of \cref{properties-of-binary-intersections-of-relations}}%
    \FirstProofBox{\cref{properties-of-binary-intersections-of-relations-interaction-with-converses}: Interaction With Converses}%
    Clear.

    \ProofBox{\cref{properties-of-binary-intersections-of-relations-interaction-with-composition}: Interaction With Composition}%
    Unwinding the definitions, we see that:
    \begin{itemize}
        \item The condition for $(S_{1}\procirc R_{1})\cap(S_{2}\procirc R_{2})$ is:
            \begin{itemize}
                \item There exists some $b\in B$ such that:
                    \begin{itemize}
                        \item $\color{OIvermillion}{a\sim_{R_{1}}b}$ and $\color{OIblue}{b\sim_{S_{1}}c}$;
                    \end{itemize}
                    and
                    \begin{itemize}\setcounter{enumi}{2}
                        \item $\color{OIvermillion}{a\sim_{R_{2}}b}$ and $\color{OIblue}{b\sim_{S_{2}}c}$;
                    \end{itemize}
            \end{itemize}
        \item The condition for $(S_{1}\cap S_{2})\procirc(R_{1}\cap R_{2})$ is:
            \begin{itemize}
                \item There exists some $b\in B$ such that:
                    \begin{itemize}
                        \item $\color{OIvermillion}{a\sim_{R_{1}}b}$ and $\color{OIvermillion}{a\sim_{R_{2}}b}$;
                    \end{itemize}
                    and
                    \begin{itemize}\setcounter{enumi}{2}
                        \item $\color{OIblue}{b\sim_{S_{1}}c}$ and $\color{OIblue}{b\sim_{S_{2}}c}$.
                    \end{itemize}
            \end{itemize}
    \end{itemize}
    These two conditions agree, and thus so do the two resulting relations on $A\times C$.
\end{Proof}
\subsection{Intersections of Families of Relations}\label{subsection-intersections-of-families-of-relations}
Let $A$ and $B$ be sets and let $\{R_{i}\}_{i\in I}$ be a family of relations from $A$ to $B$.
\begin{definition}{The Intersection of a Family of Relations}{the-intersection-of-a-family-of-relations}%
    The \index[set-theory]{relation!intersection of a family of}\textbf{intersection of the family $\{R_{i}\}_{i\in I}$} is the relation \index[notation]{intersectioniiniRi@$\bigcup_{i\in I}R_{i}$}$\smash{\bigcup_{i\in I}R_{i}}$ defined as follows:
    \begin{itemize}
        \item Viewing relations from $A$ to $B$ as subsets of $A\times B$, we define%
            %--- Begin Footnote ---%
            \footnote{%
                This is the same as the intersection of $\{R_{i}\}_{i\in I}$ as a collection of subsets of $A\times B$.
                \par\vspace*{\TCBBoxCorrection}
            }%
            %---  End Footnote  ---%
            \[%
                \bigcup_{i\in I}R_{i}%
                \defeq%
                \{%
                    (a,b)\in(A\times B)^{\times I}%
                    \ \middle|\ %
                    \begin{aligned}
                        &\text{for each $i\in I$,}\\%
                        &\text{we have $a\sim_{R_{i}}b$}%
                    \end{aligned}
                \}.%
            \]%
        \item Viewing relations from $A$ to $B$ as functions $A\to\mathcal{P}(B)$, we define%
            \[%
                \left[\bigcap_{i\in I}R_{i}\right](a)%
                \defeq%
                \bigcap_{i\in I}R_{i}(a)%
            \]%
            for each $a\in A$.
    \end{itemize}
\end{definition}
\begin{proposition}{Properties of Intersections of Families of Relations}{properties-of-intersections-of-families-of-relations}%
    Let $A$ and $B$ be sets and let $\{R_{i}\}_{i\in I}$ be a family of relations from $A$ to $B$.
    \begin{enumerate}
        \item\label{properties-of-intersections-of-families-of-relations-interaction-with-converses}\SloganFont{Interaction With Converses. }We have
            \[
                (\bigcap_{i\in I}R_{i})^{\dagger}
                =
                \bigcap_{i\in I}R^{\dagger}_{i}.
            \]%
        %\item\label{properties-of-intersections-of-families-of-relations-}\SloganFont{. }
    \end{enumerate}
\end{proposition}
\begin{Proof}{Proof of \cref{properties-of-intersections-of-families-of-relations}}%
    \FirstProofBox{\cref{properties-of-intersections-of-families-of-relations-interaction-with-converses}: Interaction With Converses}%
    Clear.
\end{Proof}
\subsection{Binary Products of Relations}\label{subsection-binary-products-of-relations}
Let $A$, $B$, $X$, and $Y$ be sets, let $R\colon A\rightproarrow B$ be a relation from $A$ to $B$, and let $S\colon X\rightproarrow Y$ be a relation from $X$ to $Y$.
\begin{definition}{Binary Products of Relations}{binary-products-of-relations}%
    The \index[set-theory]{relation!product of}\textbf{product of $R$ and $S$}%
    %--- Begin Footnote ---%
    \footnote{%
        \SloganFont{Further Terminology: }Also called the \textbf{binary product of $R$ and $S$}, for emphasis.
        \par\vspace*{\TCBBoxCorrection}
    } %
    %---  End Footnote  ---%
    is the relation \index[notation]{RtimesS@$R\times S$}$R\times S$ from $A\times X$ to $B\times Y$ defined as follows:
    \begin{itemize}
        \item Viewing relations from $A\times X$ to $B\times Y$ as subsets of $(A\times X)\times(B\times Y)$, we define $R\times S$ as the Cartesian product of $R$ and $S$ as subsets of $A\times X$ and $B\times Y$.%
            %--- Begin Footnote ---%
            \footnote{%
                That is, $R\times S$ is the relation given by declaring $(a,x)\sim_{R\times S}(b,y)$ \textiff $a\sim_{R}b$ and $x\sim_{S}y$.
                \par\vspace*{\TCBBoxCorrection}
            }%
            %---  End Footnote  ---%
        \item Viewing relations from $A\times X$ to $B\times Y$ as functions $A\times X\to\mathcal{P}(B\times Y)$, we define $R\times S$ as the composition
            \[
                A\times X
                \xlongrightarrow{R\times S}
                \mathcal{P}(B)\times\mathcal{P}(Y)
                \xlonghookrightarrow{\mathcal{P}^{\otimes}_{B,Y}}
                \mathcal{P}(B\times Y)
            \]%
            in $\Sets$, i.e.\ by
            \[
                [R\times S](a,x)
                \defeq
                R(a)\times S(x)
            \]%
            for each $(a,x)\in A\times X$.
    \end{itemize}
\end{definition}
\begin{proposition}{Properties of Binary Products of Relations}{properties-of-binary-products-of-relations}%
    Let $A$, $B$, $X$, and $Y$ be sets.
    \begin{enumerate}
        \item\label{properties-of-binary-products-of-relations-interaction-with-converses}\SloganFont{Interaction With Converses. }Let
            \begin{align*}
                R &\colon A\rightproarrow A,\\
                S &\colon X\rightproarrow X
            \end{align*}
            We have
            \[
                (R\times S)^{\dagger}
                =
                R^{\dagger}\times S^{\dagger}.
            \]%
        \item\label{properties-of-binary-products-of-relations-interaction-with-composition}\SloganFont{Interaction With Composition. }Let
            \begin{align*}
                R_{1} &\colon A\rightproarrow B,\\
                S_{1} &\colon B\rightproarrow C,\\
                R_{2} &\colon X\rightproarrow Y,\\
                S_{2} &\colon Y\rightproarrow Z
            \end{align*}
            be relations. We have
            \[
                (S_{1}\procirc R_{1})\times(S_{2}\procirc R_{2})%
                =%
                (S_{1}\times S_{2})\procirc(R_{1}\times R_{2}).%
            \]%
        %\item\label{properties-of-binary-products-of-relations-}\SloganFont{. }
    \end{enumerate}
\end{proposition}
\begin{Proof}{Proof of \cref{properties-of-binary-products-of-relations}}%
    \FirstProofBox{\cref{properties-of-binary-products-of-relations-interaction-with-converses}: Interaction With Converses}%
    Unwinding the definitions, we see that:
    \begin{itemize}
        \item We have $(a,x)\sim_{(R\times S)^{\dagger}}(b,y)$ \textiff:
            \begin{itemize}
                \item We have $(b,y)\sim_{R\times S}(a,x)$, i.e.\ \textiff:
                    \begin{itemize}
                        \item We have $b\sim_{R}a$;
                        \item We have $y\sim_{S}x$;
                    \end{itemize}
            \end{itemize}
        \item We have $(a,x)\sim_{R^{\dagger}\times S^{\dagger}}(b,y)$ \textiff:
            \begin{itemize}
                \item We have $a\sim_{R^{\dagger}}b$ and $x\sim_{S^{\dagger}}y$, i.e.\ \textiff:
                    \begin{itemize}
                        \item We have $b\sim_{R}a$;
                        \item We have $y\sim_{S}x$.
                    \end{itemize}
            \end{itemize}
    \end{itemize}
    These two conditions agree, and thus the two resulting relations on $A\times X$ are equal.

    \ProofBox{\cref{properties-of-binary-products-of-relations-interaction-with-composition}: Interaction With Composition}%
    Unwinding the definitions, we see that:
    \begin{itemize}
        \item We have $(a,x)\sim_{(S_{1}\procirc R_{1})\times(S_{2}\procirc R_{2})}(c,z)$ \textiff:
            \begin{itemize}
                \item We have $a\sim_{S_{1}\procirc R_{1}}c$ and $x\sim_{S_{2}\procirc R_{2}}z$, i.e.\ \textiff:
                    \begin{itemize}
                        \item There exists some $b\in B$ such that $a\sim_{R_{1}}b$ and $b\sim_{S_{1}}c$;
                        \item There exists some $y\in Y$ such that $x\sim_{R_{2}}y$ and $y\sim_{S_{2}}z$;
                    \end{itemize}
            \end{itemize}
        \item We have $(a,x)\sim_{(S_{1}\times S_{2})\procirc(R_{1}\times R_{2})}(c,z)$ \textiff:
            \begin{itemize}
                \item There exists some $(b,y)\in B\times Y$ such that $(a,x)\sim_{R_{1}\times R_{2}}(b,y)$ and $(b,y)\sim_{S_{1}\times S_{2}}(c,z)$, i.e.\ such that:
                    \begin{itemize}
                        \item We have $a\sim_{R_{1}}b$ and $x\sim_{R_{2}}y$;
                        \item We have $b\sim_{S_{1}}c$ and $y\sim_{S_{2}}z$.
                    \end{itemize}
            \end{itemize}
    \end{itemize}
    These two conditions agree, and thus the two resulting relations from $A\times X$ to $C\times Z$ are equal.
\end{Proof}
\subsection{Products of Families of Relations}\label{subsection-products-of-families-of-relations}
Let $\{A_{i}\}_{i\in I}$ and $\{B_{i}\}_{i\in I}$ be families of sets, and let $\{R_{i}\colon A_{i}\rightproarrow B_{i}\}_{i\in I}$ be a family of relations.
\begin{definition}{The Product of a Family of Relations}{the-product-of-a-family-of-relations}%
    The \index[set-theory]{relation!product of a family of}\textbf{product of the family $\{R_{i}\}_{i\in I}$} is the relation \index[notation]{timesiiniRi@$\prod_{i\in I}R_{i}$}$\smash{\prod_{i\in I}R_{i}}$ from $\smash{\prod_{i\in I}A_{i}}$ to $\smash{\prod_{i\in I}B_{i}}$ defined as follows:
    \begin{itemize}
        \item Viewing relations as subsets, we define $\smash{\prod_{i\in I}R_{i}}$ as its product as a family of sets, i.e.\ we have
            \[%
                \prod_{i\in I}R_{i}
                \defeq%
                \{%
                    (a_{i},b_{i})_{i\in I}\in\prod_{i\in I}(A_{i}\times B_{i})%
                    \ \middle|\ %
                    \begin{aligned}
                        &\text{for each $i\in I$,}\\%
                        &\text{we have $a_{i}\sim_{R_{i}}b_{i}$}%
                    \end{aligned}
                \}.%
            \]%
        \item Viewing relations as functions to powersets, we define
            \[%
                \left[\prod_{i\in I}R_{i}\right]((a_{i})_{i\in I})
                \defeq%
                \prod_{i\in I}R_{i}(a_{i})
            \]%
            for each $(a_{i})_{i\in I}\in\prod_{i\in I}R_{i}$.
    \end{itemize}
\end{definition}
\subsection{The Collage of a Relation}\label{subsection-the-collage-of-a-relation}
Let $A$ and $B$ be sets and let $R\colon A\rightproarrow B$ be a relation from $A$ to $B$.
\begin{definition}{The Collage of a Relation}{the-collage-of-a-relation}%
    The \index[set-theory]{relation!collage of}\textbf{collage of $R$}%
    %--- Begin Footnote ---%
    \footnote{%
        \SloganFont{Further Terminology: }Also called the \textbf{cograph of $R$}.
        \par\vspace*{\TCBBoxCorrection}
    } %
    %---  End Footnote  ---%
    is the poset \index[notation]{CollR@$\eColl(R)$}$\smash{\eColl(R)\defeq(\rmColl(R),\preceq_{\eColl(R)})}$ consisting of:
    \begin{itemize}
        \item\SloganFont{The Underlying Set. }The set \index[notation]{CollR@$\rmColl(R)$}$\rmColl(R)$ defined by%
            \[%
                \rmColl(R)%
                \defeq%
                A\icoprod B.%
            \]%
        \item\SloganFont{The Partial Order. }The partial order
            \[
                \preceq_{\eColl(R)}%
                \colon
                \rmColl(R)\times\rmColl(R)%
                \to%
                \TV%
            \]%
            on $\rmColl(R)$ defined by
            \[
                \mathord{\preceq}(a,b)%
                \defeq%
                \begin{cases}
                    \true  &\text{if $a=b$ or $a\sim_{R}b$,}\\%
                    \false &\text{otherwise.}%
                \end{cases}
            \]%
    \end{itemize}
\end{definition}
\begin{notation}{Notation: $\Pos_{/\Delta^{1}}(A,B)$}{notation-pos-over-delta-1-a-b}%
    We write $\Pos_{/\Delta^{1}}(A,B)$ for the category defined as the pullback
    \[
        \Pos_{/\Delta^{1}}(A,B)%
        \defeq%
        \PunctualCategory\ttimes_{[A],\Pos,\ev_{0}}\Pos_{/\Delta^{1}}\ttimes_{\ev_{1},\Pos,[B]}\PunctualCategory,%
    \]%
    as in the diagram
    \begingroup\footnotesize% PDF ONLY, so we use \begingroup\footnotesize instead of \begin{envfootnotesize}
    \[
        \begin{tikzcd}[row sep={4.0*\the\DL,between origins}, column sep={4.0*\the\DL,between origins}, background color=backgroundColor, ampersand replacement=\&]
            \&
            \&
            \ePos_{/\Delta^{1}}(A,B)
            \&
            \&
            \\
            \&
            \displaystyle\ePos_{/\Delta^{1}}\ttimes_{\ePos}\PunctualCategory
            \&
            \&
            \displaystyle\PunctualCategory\ttimes_{\ePos}\ePos_{/\Delta^{1}}
            \&
            \\
            \PunctualCategory
            \&
            \&
            \ePos_{/\Delta^{1}}
            \&
            \&
            \PunctualCategory\mrp{.}
            \\
            \&
            \ePos
            \&
            \&
            \ePos
            \&
            % 1-Arrows
            % Row 1 to Row 2
            \arrow[from=1-3,to=2-2]%
            \arrow[from=1-3,to=2-4]%
            % Row 2 to Row 3
            \arrow[from=2-2,to=3-1]%
            \arrow[from=2-2,to=3-3]%
            \arrow[from=2-4,to=3-3]%
            \arrow[from=2-4,to=3-5]%
            % Row 3 to Row 4
            \arrow[from=3-1,to=4-2,"{[A]}"']%
            \arrow[from=3-3,to=4-2,"\ev_{[0]}"]%
            \arrow[from=3-3,to=4-4,"\ev_{[1]}"']%
            \arrow[from=3-5,to=4-4,"{[B]}"]%
            % Pullback Symbols
            \arrow[from=2-2,to=4-2,very near start,phantom,"\lrcorner"{rotate=-45}]
            \arrow[from=1-3,to=3-3,very near start,phantom,"\lrcorner"{rotate=-45}]
            \arrow[from=2-4,to=4-4,very near start,phantom,"\lrcorner"{rotate=-45}]
        \end{tikzcd}
    \]%
    \endgroup
\end{notation}
\begin{remark}{Unwinding \cref{notation-pos-over-delta-1-a-b}}{unwinding-notation-pos-over-delta-1-a-b}%
    In detail, $\Pos_{/\Delta^{1}}(A,B)$ is the category where:
    \begin{itemize}
        \item\SloganFont{Objects. }An object of $\Pos_{/\Delta^{1}}(A,B)$ is a pair $(X,\phi_{X})$ consisting of
            \begin{itemize}
                \item A poset $X$;
                \item A morphism $\phi_{X}\colon X\to\Delta^{1}$;
            \end{itemize}
            such that we have
            \begin{align*}
                \phi^{-1}_{X}(0) &= A,\\
                \phi^{-1}_{X}(1) &= B.
            \end{align*}
        \item\SloganFont{Morphisms. }A morphism of $\Pos_{/\Delta^{1}}(A,B)$ from $(X,\phi_{X})$ to $(Y,\phi_{Y})$ is a morphism of posets $f\colon X\to Y$ making the diagram
            \[
                \begin{tikzcd}[row sep={4.0*\the\DL,between origins}, column sep={3.0*\the\DL,between origins}, background color=backgroundColor, ampersand replacement=\&]
                    X
                    \arrow[rr,"f"]
                    \arrow[rd,"\phi_{X}"']
                    \&
                    \&
                    Y
                    \arrow[ld,"\phi_{Y}"]
                    \\
                    \&
                    \Delta^{1}
                    \&
                \end{tikzcd}
            \]%
            commute.
    \end{itemize}
\end{remark}
\begin{proposition}{Properties of Collages of Relations}{properties-of-collages-of-relations}%
    Let $A$ and $B$ be sets and let $R\colon A\rightproarrow B$ be a relation from $A$ to $B$.
    \begin{enumerate}
        \item\label{properties-of-collages-of-relations-functoriality}\SloganFont{Functoriality. }The assignment $R\mapsto\eColl(R)$ defines a functor%
            \[
                \eColl%
                \colon%
                \eRel(A,B)%
                \to%
                \Pos_{/\Delta^{1}}(A,B),%
            \]%
            where
            \begin{itemize}
                \item\SloganFont{Action on Objects. }For each $R\in\Obj(\eRel(A,B))$, we have
                    \[
                        [\eColl](R)
                        \defeq
                        (\eColl(R),\phi_{R})%
                    \]%
                    for each $R\in\eRel(A,B)$, where
                    \begin{itemize}
                        \item The poset $\eColl(R)$ is the collage of $R$ of \cref{the-collage-of-a-relation}.
                        \item The morphism $\phi_{R}\colon\eColl(R)\to\Delta^{1}$ is given by
                            \[
                                \phi_{R}(x)%
                                \defeq
                                \begin{cases}
                                    0 &\text{if $x\in A$,}\\
                                    1 &\text{if $x\in B$}
                                \end{cases}
                            \]%
                            for each $x\in\eColl(R)$.
                    \end{itemize}
                \item\SloganFont{Action on Morphisms. }For each $R,S\in\Obj(\eRel(A,B))$, the action on $\Hom$-sets
                    \[
                        \eColl_{R,S}%
                        \colon%
                        \Hom_{\eRel(A,B)}(R,S)
                        \to
                        \Pos(\eColl(R),\eColl(S))
                    \]%
                    of $\eColl$ at $(R,S)$ is given by sending an inclusion
                    \[
                        \iota%
                        \colon%
                        R%
                        \subset%
                        S%
                    \]%
                    to the morphism
                    \[
                        \eColl(\iota)%
                        \colon%
                        \eColl(R)%
                        \to%
                        \eColl(S)%
                    \]%
                    of posets over $\Delta^{1}$ defined by
                    \[
                        [\eColl(\iota)](x)%
                        \defeq%
                        x%
                    \]%
                    for each $x\in\eColl(R)$.%
                    %--- Begin Footnote ---%
                    \footnote{%
                        Note that this is indeed a morphism of posets: if $x\preceq_{\eColl(R)}y$, then $x=y$ or $x\sim_{R}y$, so we have either $x=y$ or $x\sim_{S}y$ (as $R\subset S$), and thus $x\preceq_{\eColl(S)}y$.
                        \par\vspace*{\TCBBoxCorrection}
                    }%
                    %---  End Footnote  ---%
            \end{itemize}
        \item\label{properties-of-collages-of-relations-equivalence}\SloganFont{Equivalence. }The functor of \cref{properties-of-collages-of-relations-functoriality} is an equivalence of categories.
        %\item\label{properties-of-collages-of-relations-}\SloganFont{. }
    \end{enumerate}
\end{proposition}
\begin{Proof}{Proof of \cref{properties-of-collages-of-relations}}%
    \FirstProofBox{\cref{properties-of-collages-of-relations-functoriality}: Functoriality}%
    Clear.

    \ProofBox{\cref{properties-of-collages-of-relations-equivalence}: Equivalence}%
    Omitted.
\end{Proof}
\begin{appendices}
\input{ABSOLUTEPATH/chapters2.tex}
\end{appendices}
\end{document}
