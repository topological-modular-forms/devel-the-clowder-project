\input{preamble}

% OK, start here.
%
\input{chapter_modifications.tex}
\begin{document}

\title{Constructions With Sets}

\maketitle

\phantomsection
\label{section-phantom}

This chapter develops some material relating to constructions with sets with an eye towards its categorical and higher-categorical counterparts to be introduced later in this work. Of particular interest are perhaps the following:
\begin{enumerate}
    \item\label{constructions-with-sets-introduction-item-1}Explicit descriptions of the major types of co/limits in $\Sets$, including in particular explicit descriptions of pushouts and coequalisers (see \cref{pushouts-of-sets,unwinding-pushouts-of-sets,coequalisers-of-sets,unwinding-coequalisers-of-sets}).
    \item\label{constructions-with-sets-introduction-item-2}A discussion of powersets as decategorifications of categories of presheaves, including in particular results such as:
        \begin{enumerate}
            \item\label{constructions-with-sets-introduction-item-2a}A discussion of the internal Hom of a powerset (\cref{subsection-the-internal-hom-of-a-powerset}).
            \item\label{constructions-with-sets-introduction-item-2b}A 0-categorical version of the Yoneda lemma (\ChapterRef{\ChapterPresheavesAndTheYonedaLemma, \cref{presheaves-and-the-yoneda-lemma:the-yoneda-lemma}}{\cref{the-yoneda-lemma}}), which we term the \textit{Yoneda lemma for sets} (\cref{the-yoneda-lemma-for-sets}).
            \item\label{constructions-with-sets-introduction-item-2c}A characterisation of powersets as free cocompletions (\cref{subsection-powersets-as-free-cocompletions}), mimicking the corresponding statement for categories of presheaves (\cref{TODO1}).
            \item\label{constructions-with-sets-introduction-item-2d}A characterisation of powersets as free completions (\cref{subsection-powersets-as-free-completions}), mimicking the corresponding statement for categories of copresheaves (\cref{TODO2}).
            \item\label{constructions-with-sets-introduction-item-2e}A $(-1)$-categorical version of un/straightening (\cref{properties-of-characteristic-functions-of-subsets-bijectivity} of \cref{properties-of-characteristic-functions-of-subsets} and \cref{powersets-as-sets-of-functions-and-un-straightening}).
            \item\label{constructions-with-sets-introduction-item-2f}A 0-categorical form of Isbell duality internal to powersets (\cref{subsection-isbell-duality-for-sets}).
        \end{enumerate}
    \item\label{constructions-with-sets-introduction-item-3}A lengthy discussion of the adjoint triple%
        \[
            f_{!}\dashv f^{-1}\dashv f_{*}%
            \colon%
            \mathcal{P}(A)%
            \rightleftrightarrows%
            \mathcal{P}(B)%
        \]%
        of functors (i.e.\ morphisms of posets) between $\mathcal{P}(A)$ and $\mathcal{P}(B)$ induced by a map of sets $f\colon A\to B$, including in particular:
        \begin{enumerate}
            \item\label{constructions-with-sets-introduction-item-3a}How $f^{-1}$ can be described as a precomposition while $f_{!}$ and $f_{*}$ can be described as Kan extensions (\cref{unwinding-the-direct-image-function-associated-to-a-function,unwinding-the-inverse-image-function-associated-to-a-function,unwinding-the-codirect-image-function-associated-to-a-function}).
            \item\label{constructions-with-sets-introduction-item-3b}An extensive list of the properties of $f_{!}$, $f^{-1}$, and $f_{*}$ (\cref{properties-of-direct-images-i,properties-of-direct-images-ii,properties-of-inverse-images-i,properties-of-inverse-images-ii,properties-of-codirect-images-i,properties-of-codirect-images-ii}).
            \item\label{constructions-with-sets-introduction-item-3c}How the functors $f_{!}$, $f^{-1}$, $f_{*}$, along with the functors
                \begin{align*}
                    -_{1}\cap-_{2}    &\colon \mathcal{P}(X)\times\mathcal{P}(X)       \to \mathcal{P}(X),\\
                    [-_{1},-_{2}]_{X} &\colon \mathcal{P}(X)^{\op}\times\mathcal{P}(X) \to \mathcal{P}(X)
                \end{align*}
                may be viewed as a six-functor formalism with the empty set $\emptyset$ as the dualising object (\cref{subsection-a-six-functor-formalism-for-sets}).
        \end{enumerate}
    \item\label{constructions-with-sets-introduction-item-4}A discussion of injective, surjective, and bijective functions, including several characterisations phrased using the adjoint triple $f_{!}\dashv f^{-1}\dashv f_{*}$ (\cref{section-constructions-with-sets-injective-surjective-and-bijective-functions}).
\end{enumerate}

\ChapterTableOfContents

\section{Limits of Sets}\label{section-limits-of-sets}
\subsection{The Terminal Set}\label{subsection-the-terminal-set}
\begin{definition}{The Terminal Set}{the-terminal-set}%
    The \index[set-theory]{terminal set}\textbf{terminal set} is the terminal object of $\Sets$ as in \ChapterRef{\ChapterLimitsAndColimits, \cref{limits-and-colimits:terminal-objects}}{\cref{terminal-objects}}.
\end{definition}
\begin{construction}{Construction of the Terminal Set}{construction-of-the-terminal-set}%
    Concretely, the terminal set is the pair \index[notation]{pt@$\pt$}$\smash{(\pt,\{!_{A}\}_{A\in\Obj(\Sets)})}$ consisting of:
    \begin{enumerate}
        \item\label{construction-of-the-terminal-set-the-limit}\SloganFont{The Limit. }The punctual set $\pt\defeq\{\point\}$.
        \item\label{construction-of-the-terminal-set-the-cone}\SloganFont{The Cone. }The collection of maps
            \[
                \{%
                    !_{A}%
                    \colon%
                    A%
                    \to%
                    \pt%
                \}_{A\in\Obj(\Sets)}%
            \]%
            defined by
            \[
                !_{A}(a)%
                \defeq%
                \point%
            \]%
            for each $a\in A$ and each $A\in\Obj(\Sets)$.
    \end{enumerate}
\end{construction}
\begin{Proof}{Proof of \cref{construction-of-the-terminal-set}}%
    We claim that $\pt$ is the terminal object of $\Sets$. Indeed, suppose we have a diagram of the form
    \[
        \begin{tikzcd}[row sep={3.0*\the\DL,between origins}, column sep={4.0*\the\DL,between origins}, background color=backgroundColor, ampersand replacement=\&]
            A
            \&
            \pt
        \end{tikzcd}
    \]%
    in $\Sets$. Then there exists a unique map $\phi\colon A\to\pt$ making the diagram
    \[
        \begin{tikzcd}[row sep={3.0*\the\DL,between origins}, column sep={4.0*\the\DL,between origins}, background color=backgroundColor, ampersand replacement=\&]
            A
            \arrow[r,"\phi"{pos=0.475},"\exists!"'{pos=0.475}, dashed]
            \&
            \pt
        \end{tikzcd}
    \]%
    commute, namely $!_{A}$.
\end{Proof}
\subsection{Products of Families of Sets}\label{subsection-products-of-families-of-sets}
Let $\{A_{i}\}_{i\in I}$ be a family of sets.%
\begin{definition}{The Product of a Family of Sets}{the-product-of-a-family-of-sets}%
    The \index[set-theory]{product of a family of sets}\textbf{product}%
    %--- Begin Footnote ---%
    \footnote{%
        \SloganFont{Further Terminology: }Also called the \index[set-theory]{Cartesian product}\textbf{Cartesian product of $\{A_{i}\}_{i\in I}$}.
        \par\vspace*{\TCBBoxCorrection}
    } %
    %---  End Footnote  ---%
    \textbf{of $\{A_{i}\}_{i\in I}$} is the product of $\{A_{i}\}_{i\in I}$ in $\Sets$ as in \ChapterRef{\ChapterLimitsAndColimits, \cref{limits-and-colimits:the-product-of-a-family-of-objects}}{\cref{the-product-of-a-family-of-objects}}.
\end{definition}
\begin{construction}{Construction of the Product of a Family of Sets}{construction-of-the-product-of-a-family-of-sets}%
    Concretely, the product of $\{A_{i}\}_{i\in I}$ is the pair \index[notation]{prodiiniai@$\prod_{i\in I}A_{i}$}$(\prod_{i\in I}A_{i},\{\pr_{i}\}_{i\in I})$ consisting of:
    \begin{enumerate}
        \item\label{construction-of-the-product-of-a-family-of-sets-the-limit}\SloganFont{The Limit. }The set $\prod_{i\in I}A_{i}$ given by%
            \[
                \prod_{i\in I}A_{i}
                =%
                \{%
                    f\in\Sets\left(I,\bigcup_{i\in I}A_{i}\right)%
                    \ \middle|\ %
                    \begin{aligned}
                        &\text{for each $i\in I$, we}\\
                        &\text{have $f(i)\in A_{i}$}%
                    \end{aligned}
                \}.%
            \]%
        \item\label{construction-of-the-product-of-a-family-of-sets-the-cone}\SloganFont{The Cone. }The collection
            \[
                \{%
                    \pr_{i}
                    \colon%
                    \prod_{i\in I}A_{i}%
                    \to%
                    A_{i}%
                \}_{i\in I}%
            \]%
            of maps defined by%
            \[
                \pr_{i}(f)%
                \defeq%
                f(i)%
            \]%
            for each $f\in\prod_{i\in I}A_{i}$ and each $i\in I$.
    \end{enumerate}
\end{construction}
\begin{Proof}{Proof of \cref{construction-of-the-product-of-a-family-of-sets}}%
    We claim that $\prod_{i\in I}A_{i}$ is the categorical product of $\{A_{i}\}_{i\in I}$ in $\Sets$. Indeed, suppose we have, for each $i\in I$, a diagram of the form
    \[
        \begin{tikzcd}[row sep={5.0*\the\DL,between origins}, column sep={5.0*\the\DL,between origins}, background color=backgroundColor, ampersand replacement=\&]
            P
            \arrow[rd,"p_{i}"]
            \&
            \\
            {\displaystyle\prod_{i\in I}A_{i}}
            \arrow[r,"\pr_{i}"']
            \&
            A_{i}
        \end{tikzcd}
    \]%
    in $\Sets$. Then there exists a unique map $\phi\colon P\to\prod_{i\in I}A_{i}$ making the diagram
    \[
        \begin{tikzcd}[row sep={5.0*\the\DL,between origins}, column sep={5.0*\the\DL,between origins}, background color=backgroundColor, ampersand replacement=\&]
            P
            \arrow[rd,"p_{i}"]
            \arrow[d,"\phi"'{pos=0.45},"\exists!"{pos=0.45},dashed]
            \&
            \\
            {\displaystyle\prod_{i\in I}A_{i}}
            \arrow[r,"\pr_{i}"']
            \&
            A_{i}
        \end{tikzcd}
    \]%
    commute, being uniquely determined by the condition $\pr_{i}\circ\phi=p_{i}$ for each $i\in I$ via
    \[
        \phi(x)%
        =%
        (p_{i}(x))_{i\in I}
    \]%
    for each $x\in P$.
\end{Proof}
\begin{remark}{Unwinding \cref{construction-of-the-product-of-a-family-of-sets}}{unwinding-the-product-of-a-family-of-sets}%
    Less formally, we may think of Cartesian products and projection maps as follows:
    \begin{enumerate}
        \item\label{unwinding-the-product-of-a-family-of-sets-cartesian-product}We think of $\prod_{i\in I}A_{i}$ as the set whose elements are $I$-indexed collections $(a_{i})_{i\in I}$ with $a_{i}\in A_{i}$ for each $i\in I$.
        \item\label{unwinding-the-product-of-a-family-of-sets-projection-maps}We view the projection maps
            \[
                \{%
                    \pr_{i}
                    \colon%
                    \prod_{i\in I}A_{i}%
                    \to%
                    A_{i}%
                \}_{i\in I}%
            \]%
            as being given by
            \[
                \pr_{i}((a_{j})_{j\in I})%
                =%
                a_{i}%
            \]%
            for each $(a_{j})_{j\in I}\in\prod_{i\in I}A_{i}$ and each $i\in I$.
    \end{enumerate}
\end{remark}
\begin{proposition}{Properties of Products of Families of Sets}{properties-of-products-of-families-of-sets}%
    Let $\{A_{i}\}_{i\in I}$ be a family of sets.%
    \begin{enumerate}
        \item\label{properties-of-products-of-families-of-sets-functoriality}\SloganFont{Functoriality. }The assignment $\{A_{i}\}_{i\in I}\mapsto\prod_{i\in I}A_{i}$ defines a functor
            \[
                \prod_{i\in I}%
                \colon%
                \Fun(I_{\disc},\Sets)%
                \to%
                \Sets%
            \]%
            where
            \begin{itemize}
                \item\SloganFont{Action on Objects. }For each $(A_{i})_{i\in I}\in\Obj(\Fun(I_{\disc},\Sets))$, we have
                    \[
                        \left[\prod_{i\in I}\right]((A_{i})_{i\in I})%
                        \defeq%
                        \prod_{i\in I}A_{i}%
                    \]%
                \item\SloganFont{Action on Morphisms. }For each $(A_{i})_{i\in I},(B_{i})_{i\in I}\in\Obj(\Fun(I_{\disc},\Sets))$, the action on $\Hom$-sets
                    \[
                        \left(\prod_{i\in I}\right)_{(A_{i})_{i\in I},(B_{i})_{i\in I}}
                        \colon
                        \Nat((A_{i})_{i\in I},(B_{i})_{i\in I})%
                        \to%
                        \Sets\left(\prod_{i\in I}A_{i},\prod_{i\in I}B_{i}\right)%
                    \]%
                    of $\prod_{i\in I}$ at $((A_{i})_{i\in I},(B_{i})_{i\in I})$ is defined by sending a map
                    \[
                        \{%
                            f_{i}%
                            \colon%
                            A_{i}%
                            \to%
                            B_{i}
                        \}_{i\in I}%
                        %
                    \]%
                    in $\Nat((A_{i})_{i\in I},(B_{i})_{i\in I})$ to the map of sets
                    \[
                        \prod_{i\in I}f_{i}%
                        \colon
                        \prod_{i\in I}A_{i}%
                        \to%
                        \prod_{i\in I}B_{i}%
                    \]%
                    defined by
                    \[
                        \left[\prod_{i\in I}f_{i}\right]((a_{i})_{i\in I})
                        \defeq%
                        (f_{i}(a_{i}))_{i\in I}
                    \]%
                    for each $(a_{i})_{i\in I}\in\prod_{i\in I}A_{i}$.
            \end{itemize}
        %\item\label{properties-of-products-of-families-of-sets-}\SloganFont{. }
    \end{enumerate}
\end{proposition}
\begin{Proof}{Proof of \cref{properties-of-products-of-families-of-sets}}%
    \FirstProofBox{\cref{properties-of-products-of-families-of-sets-functoriality}: Functoriality}%
    This follows from \ChapterRef{\ChapterLimitsAndColimits, \cref{limits-and-colimits:properties-of-co-limits-functoriality} of \cref{limits-and-colimits:properties-of-co-limits}}{\cref{properties-of-co-limits-functoriality} of \cref{properties-of-co-limits}}.
\end{Proof}
\subsection{Binary Products of Sets}\label{subsection-binary-products-of-sets}
Let $A$ and $B$ be sets.%
\begin{definition}{Binary Products of Sets}{binary-products-of-sets}%
    The \index[set-theory]{product of sets}\textbf{product of $A$ and $B$}%
    %--- Begin Footnote ---%
    \footnote{%
        \SloganFont{Further Terminology: }Also called the \index[set-theory]{Cartesian product}\textbf{Cartesian product of $A$ and $B$}.
        \par\vspace*{\TCBBoxCorrection}
    } %
    %---  End Footnote  ---%
    is the product of $A$ and $B$ in $\Sets$ as in \ChapterRef{\ChapterLimitsAndColimits, \cref{limits-and-colimits:binary-products}}{\cref{binary-products}}.
\end{definition}
\begin{construction}{Construction of Binary Products of Sets}{construction-of-binary-products-of-sets}%
    Concretely, the product of $A$ and $B$ is the pair \index[notation]{AtimesB@$A\times B$}$\smash{(A\times B,\{\pr_{1},\pr_{2}\})}$ consisting of:
    \begin{enumerate}
        \item\label{construction-of-binary-products-of-sets-the-limit}\SloganFont{The Limit. }The set $A\times B$ given by%
            \begin{align*}
                A\times B &=      \prod_{z\in\{A,B\}}z\\
                          &\eqdef \{%
                                      f\in\Sets(\{0,1\},A\cup B)%
                                      \ \middle|\ %
                                      \text{%
                                          we have $f(0)\in A$ and $f(1)\in B$%
                                      }%
                                  \}\\%
                          &\cong \{\{\{a\},\{a,b\}\}\in\mathcal{P}(\mathcal{P}(A\cup B))\ \middle|\ \text{we have $a\in A$ and $b\in B$}\}\\%
                          &\cong \{%
                                     \begin{aligned}
                                         &\text{ordered pairs $(a,b)$ with}\\%
                                         &\text{$a\in A$ and $b\in B$}%
                                     \end{aligned}
                                 \}.%
            \end{align*}
        \item\label{construction-of-binary-products-of-sets-the-cone}\SloganFont{The Cone. }The maps
            \begin{align*}
                \pr_{1} &\colon A\times B\to A,\\
                \pr_{2} &\colon A\times B\to B
            \end{align*}
            defined by
            \begin{align*}
                \pr_{1}(a,b) &\defeq a,\\
                \pr_{2}(a,b) &\defeq b
            \end{align*}
            for each $(a,b)\in A\times B$.
    \end{enumerate}
\end{construction}
\begin{Proof}{Proof of \cref{construction-of-binary-products-of-sets}}%
    We claim that $A\times B$ is the categorical product of $A$ and $B$ in the category of sets. Indeed, suppose we have a diagram of the form
    \[
        \begin{tikzcd}[row sep={4.5*\the\DL,between origins}, column sep={4.5*\the\DL,between origins}, background color=backgroundColor, ampersand replacement=\&,productArrows={4.5*\the\DL}{p_{1}}{p_{2}}]
            {}%
            \&
            P
            \&
            {}%
            \\
            A
            \&
            A\times B
            \arrow[l,"\pr_{1}"{pos=0.425},two heads]
            \arrow[r,"\pr_{2}"'{pos=0.425},two heads]
            \&
            B
        \end{tikzcd}
    \]%
    in $\Sets$. Then there exists a unique map $\phi\colon P\to A\times B$ making the diagram
    \[
        \begin{tikzcd}[row sep={4.5*\the\DL,between origins}, column sep={4.5*\the\DL,between origins}, background color=backgroundColor, ampersand replacement=\&,productArrows={4.5*\the\DL}{p_{1}}{p_{2}}]
            {}%
            \&
            P
            \arrow[d,"\phi"'{pos=0.475},"\exists!"{pos=0.475}, dashed]
            \&
            {}%
            \\
            A
            \&
            A\times B
            \arrow[l,"\pr_{1}"{pos=0.425},two heads]
            \arrow[r,"\pr_{2}"'{pos=0.425},two heads]
            \&
            B
        \end{tikzcd}
    \]%
    commute, being uniquely determined by the conditions
    \begin{align*}
        \pr_{1}\circ\phi &= p_{1},\\
        \pr_{2}\circ\phi &= p_{2}
    \end{align*}
    via
    \[
        \phi(x)%
        =%
        (p_{1}(x),p_{2}(x))%
    \]%
    for each $x\in P$.
\end{Proof}
\begin{proposition}{Properties of Products of Sets}{properties-of-products-of-sets}%
    Let $A$, $B$, $C$, and $X$ be sets.
    \begin{enumerate}
        \item\label{properties-of-products-of-sets-functoriality}\SloganFont{Functoriality. }The assignments $A,B,(A,B)\mapsto A\times B$ define functors
            \[
                \Bifunctoriality{A\times-}{-\times B}{-_{1}\times-_{2}}{\Sets}{\Sets}{\Sets\times\Sets}{\Sets}%
            \]%
            where $-_{1}\times-_{2}$ is the functor where
            \begin{itemize}
                \item\SloganFont{Action on Objects. }For each $(A,B)\in\Obj(\Sets\times\Sets)$, we have
                    \[
                        [-_{1}\times-_{2}](A,B)%
                        \defeq
                        A\times B.%
                    \]%
                \item\SloganFont{Action on Morphisms. }For each $(A,B),(X,Y)\in\Obj(\Sets)$, the action on $\Hom$-sets
                    \[
                        \mathord{\times}_{(A,B),(X,Y)}
                        \colon
                        \Sets(A,X)\times\Sets(B,Y)%
                        \to%
                        \Sets(A\times B,X\times Y)%
                    \]%
                    of $\times$ at $((A,B),(X,Y))$ is defined by sending $(f,g)$ to the function
                    \[
                        f\times g%
                        \colon%
                        A\times B%
                        \to%
                        X\times Y%
                    \]%
                    defined by
                    \[
                        [f\times g](a,b)%
                        \defeq%
                        (f(a),g(b))
                    \]%
                    for each $(a,b)\in A\times B$.
            \end{itemize}
            and where $A\times-$ and $-\times B$ are the partial functors of $-_{1}\times-_{2}$ at $A,B\in\Obj(\Sets)$.
        \item\label{properties-of-products-of-sets-adjointness-1}\SloganFont{Adjointness \rmI. }We have adjunctions
            \begin{webcompile}
                \begin{gathered}
                    \AdjunctionShort#A\times -#{\Sets(A,-)}#\Sets#\Sets,#\\
                    \AdjunctionShort#-\times B#{\Sets(B,-)}#\Sets#\Sets,#
                \end{gathered}
            \end{webcompile}%
            witnessed by bijections
            \begin{align*}
                \Sets(A\times B,C) &\cong \Sets(A,\Sets(B,C)),\\
                \Sets(A\times B,C) &\cong \Sets(B,\Sets(A,C)),
            \end{align*}
            natural in $A,B,C\in\Obj(\Sets)$.
        \item\label{properties-of-products-of-sets-adjointness-2}\SloganFont{Adjointness \rmII. }We have an adjunction
            \begin{webcompile}
                \Adjunction#{\Delta_{\Sets}}#-_{1}\times-_{2}#\Sets#\Sets\times\Sets,#\\
            \end{webcompile}%
            witnessed by a bijection
            \[
                \Hom_{\Sets\times\Sets}((A,A),(B,C))%
                \cong%
                \Sets(A,B\times C),%
            \]%
            natural in $A\in\Obj(\Sets)$ and in $(B,C)\in\Obj(\Sets\times\Sets)$.
        \item\label{properties-of-products-of-sets-associativity}\SloganFont{Associativity. }We have an isomorphism of sets
            \[
                \alpha^{\Sets}_{A,B,C}%
                \colon%
                (A\times B)\times C%
                \isorightarrow%
                A\times(B\times C),%
            \]%
            natural in $A,B,C\in\Obj(\Sets)$.
        \item\label{properties-of-products-of-sets-unitality}\SloganFont{Unitality. }We have isomorphisms of sets
            \begin{align*}
                \LUnitor^{\Sets}_{A} &\colon \pt\times A \isorightarrow A,\\
                \RUnitor^{\Sets}_{A} &\colon A\times\pt  \isorightarrow A,
            \end{align*}
            natural in $A\in\Obj(\Sets)$.
        \item\label{properties-of-products-of-sets-commutativity}\SloganFont{Commutativity. }We have an isomorphism of sets
            \[
                \sigma^{\Sets}_{A,B}%
                \colon%
                A\times B
                \isorightarrow%
                B\times A,
            \]%
            natural in $A,B\in\Obj(\Sets)$.
        \item\label{properties-of-products-of-sets-distributivity-over-coproducts}\SloganFont{Distributivity Over Coproducts. }We have isomorphisms of sets
            \begin{align*}
                \delta^{\Sets}_{\ell} &\colon A\times(B\icoprod C)  \isorightarrow (A\times B)\icoprod(A\times C),\\
                \delta^{\Sets}_{r}    &\colon (A\icoprod B)\times C \isorightarrow (A\times C)\icoprod(B\times C),
            \end{align*}
            natural in $A,B,C\in\Obj(\Sets)$.
        \item\label{properties-of-products-of-sets-annihilation-with-the-empty-set}\SloganFont{Annihilation With the Empty Set. }We have isomorphisms of sets
            \begin{align*}
                \zeta^{\Sets}_{\ell} &\colon \emptyset\times A \isorightarrow \emptyset,\\
                \zeta^{\Sets}_{r}    &\colon A\times\emptyset  \isorightarrow \emptyset,
            \end{align*}
            natural in $A\in\Obj(\Sets)$.
        \item\label{properties-of-products-of-sets-distributivity-over-unions}\SloganFont{Distributivity Over Unions. }Let $X$ be a set. For each $U,V,W\in\mathcal{P}(X)$, we have equalities
            \begin{align*}
                U\times(V\cup W)  &= (U\times V)\cup(U\times W),\\%
                (U\cup V)\times W &= (U\times W)\cup(V\times W)
            \end{align*}
            of subsets of $\mathcal{P}(X\times X)$.
        \item\label{properties-of-products-of-sets-distributivity-over-intersections}\SloganFont{Distributivity Over Intersections. }Let $X$ be a set. For each $U,V,W\in\mathcal{P}(X)$, we have equalities
            \begin{align*}
                U\times(V\cap W)  &= (U\times V)\cap(U\times W),\\%
                (U\cap V)\times W &= (U\times W)\cap(V\times W)
            \end{align*}
            of subsets of $\mathcal{P}(X\times X)$.
        \item\label{properties-of-products-of-sets-distributivity-over-differences}\SloganFont{Distributivity Over Differences. }Let $X$ be a set. For each $U,V,W\in\mathcal{P}(X)$, we have equalities
            \begin{align*}
                U\times(V\setminus W)  &= (U\times V)\setminus(U\times W),\\%
                (U\setminus V)\times W &= (U\times W)\setminus(V\times W)
            \end{align*}
            of subsets of $\mathcal{P}(X\times X)$.
        \item\label{properties-of-products-of-sets-distributivity-over-symmetric-differences}\SloganFont{Distributivity Over Symmetric Differences. }Let $X$ be a set. For each $U,V,W\in\mathcal{P}(X)$, we have equalities
            \begin{align*}
                U\times(V\sdiff W)  &= (U\times V)\sdiff(U\times W),\\%
                (U\sdiff V)\times W &= (U\times W)\sdiff(V\times W)
            \end{align*}
            of subsets of $\mathcal{P}(X\times X)$.
        \item\label{properties-of-products-of-sets-middle-four-exchange-with-respect-to-intersections}\SloganFont{Middle-Four Exchange with Respect to Intersections. }The diagram
            \[
                \begin{tikzcd}[row sep={5.0*\the\DL,between origins}, column sep={14.5*\the\DL,between origins}, background color=backgroundColor, ampersand replacement=\&]
                    {(\mathcal{P}(X)\times\mathcal{P}(X))\times(\mathcal{P}(X)\times\mathcal{P}(X))}
                    \arrow[r,"\mathord{\cap}\times\mathord{\cap}"]
                    \arrow[d,"\mathcal{P}^{\times}_{X,X}\times\mathcal{P}^{\times}_{X,X}"']
                    \&
                    {\mathcal{P}(X)\times\mathcal{P}(X)}
                    \arrow[d,"\mathcal{P}^{\times}_{X,X}"]
                    \\
                    {\mathcal{P}(X\times X)\times\mathcal{P}(X\times X)}
                    \arrow[r,"\cap"']
                    \&
                    {\mathcal{P}(X\times X)}
                \end{tikzcd}
            \]%
            commutes, i.e.\ we have
            \[
                (U\times V)\cap(W\times T)%
                =%
                (U\cap V)\times(W\cap T).%
            \]%
            for each $U,V,W,T\in\mathcal{P}(X)$.
        \item\label{properties-of-products-of-sets-symmetric-monoidality}\SloganFont{Symmetric Monoidality. }The 8-tuple $\left(\phantom{\mrp{\alpha^{\Sets}}}\Sets\right.$, $\times$, $\pt$, $\Sets(-_{1},-_{2})$, $\alpha^{\Sets}$, $\LUnitor^{\Sets}$, $\RUnitor^{\Sets}$, $\left.\sigma^{\Sets}\right)$ is a closed symmetric monoidal category.
        \item\label{properties-of-products-of-sets-symmetric-bimonoidality}\SloganFont{Symmetric Bimonoidality. }The 18-tuple
            \[
                \begin{aligned}
                    &\left(\Sets,\icoprod,\times,\emptyset,\pt,\Sets(-_{1},-_{2}),\alpha^{\Sets},\LUnitor^{\Sets},\RUnitor^{\Sets},\sigma^{\Sets},\right.\\%
                    &\left.\alpha^{\Sets,\icoprod},\LUnitor^{\Sets,\icoprod},\RUnitor^{\Sets,\icoprod},\sigma^{\Sets,\icoprod},\delta^{\Sets}_{\ell},\delta^{\Sets}_{r},\zeta^{\Sets}_{\ell},\zeta^{\Sets}_{r}\right),%
                \end{aligned}
            \]%
            is a symmetric closed bimonoidal category, where $\alpha^{\Sets,\icoprod}$, $\LUnitor^{\Sets,\icoprod}$, $\RUnitor^{\Sets,\icoprod}$, and $\sigma^{\Sets,\icoprod}$ are the natural transformations from \cref{properties-of-coproducts-of-sets-associativity,properties-of-coproducts-of-sets-unitality,properties-of-coproducts-of-sets-commutativity} of \cref{properties-of-coproducts-of-sets}.
        %\item\label{properties-of-products-of-sets-}\SloganFont{. }
    \end{enumerate}
\end{proposition}
\begin{Proof}{Proof of \cref{properties-of-products-of-sets}}%
    \FirstProofBox{\cref{properties-of-products-of-sets-functoriality}: Functoriality}%
    This follows from \ChapterRef{\ChapterLimitsAndColimits, \cref{limits-and-colimits:properties-of-co-limits-functoriality} of \cref{limits-and-colimits:properties-of-co-limits}}{\cref{properties-of-co-limits-functoriality} of \cref{properties-of-co-limits}}.

    \ProofBox{\cref{properties-of-products-of-sets-adjointness-1}: Adjointness}%
    We prove only that there's an adjunction $-\times B\dashv\Sets(B,-)$, witnessed by a bijection
    \[
        \Sets(A\times B,C)%
        \cong%
        \Sets(A,\Sets(B,C)),%
    \]%
    natural in $B,C\in\Obj(\Sets)$, as the proof of the existence of the adjunction $A\times-\dashv\Sets(A,-)$ follows almost exactly in the same way.%
    \begin{itemize}
        \item\label{proof-of-properties-of-products-of-sets-adjointness-1-1}\SloganFont{Map \rmI. }We define a map
            \[
                \Phi_{B,C}%
                \colon%
                \Sets(A\times B,C)%
                \to%
                \Sets(A,\Sets(B,C)),%
            \]%
            by sending a function
            \[
                \xi%
                \colon%
                A\times B%
                \to%
                C%
            \]%
            to the function%
            \begin{webcompile}
                \phantom{\xi^{\dagger}\colon}
                \begin{tikzcd}[row sep=0.0*\the\DL, column sep=1.0*\the\DL, background color=backgroundColor, ampersand replacement=\&]
                    \mathllap{\xi^{\dagger}\colon}A%
                    \arrow[r]
                    \&
                    \Sets(B,C)\mrp{,}%
                    \\
                    a
                    \arrow[r, mapsto]
                    \&
                    {(\xi^{\dagger}_{a}\colon B\to C)\mrp{,}}
                \end{tikzcd}
            \end{webcompile}
            where we define
            \[
                \xi^{\dagger}_{a}(b)%
                \defeq
                \xi(a,b)%
            \]%
            for each $b\in B$. In terms of the $\llbracket a\mapsto f(a)\rrbracket$ notation of \ChapterRef{\ChapterSets, \cref{sets:additional-notation-for-functions}}{\cref{additional-notation-for-functions}}, we have
            \[
                \xi^{\dagger}%
                \defeq%
                \llbracket a\mapsto\llbracket b\mapsto\xi(a,b)\rrbracket\rrbracket.%
            \]%
        \item\label{proof-of-properties-of-products-of-sets-adjointness-1-2}\SloganFont{Map \rmII. }We define a map
            \[
                \Psi_{B,C}%
                \colon%
                \Sets(A,\Sets(B,C)),%
                \to%
                \Sets(A\times B,C)%
            \]%
            given by sending a function
            \begin{webcompile}
                \phantom{\xi\colon}
                \begin{tikzcd}[row sep=0.0*\the\DL, column sep=1.0*\the\DL, background color=backgroundColor, ampersand replacement=\&]
                    \mathllap{\xi\colon}A%
                    \arrow[r]
                    \&
                    \Sets(B,C)\mrp{,}%
                    \\
                    a
                    \arrow[r, mapsto]
                    \&
                    {(\xi_{a}\colon B\to C)\mrp{,}}
                \end{tikzcd}
            \end{webcompile}
            to the function
            \[
                \xi^{\dagger}%
                \colon%
                A\times B%
                \to
                C
            \]%
            defined by
            \begin{align*}
                \xi^{\dagger}(a,b) &\defeq \ev_{b}(\ev_{a}(\xi))\\%
                                   &\eqdef \ev_{b}(\xi_{a})\\%
                                   &\eqdef \xi_{a}(b)%
            \end{align*}
            for each $(a,b)\in A\times B$.
        \item\label{proof-of-properties-of-products-of-sets-adjointness-1-3}\SloganFont{Invertibility \rmI. }We claim that
            \[
                \Psi_{A,B}\circ\Phi_{A,B}%
                =%
                \id_{\Sets(A\times B,C)}.%
            \]%
            Indeed, given a function $\xi\colon A\times B\to C$, we have
            \begin{align*}
                [\Psi_{A,B}\circ\Phi_{A,B}](\xi) &= \Psi_{A,B}(\Phi_{A,B}(\xi))\\%
                                                 &= \Psi_{A,B}(\Phi_{A,B}(\llbracket(a,b)\mapsto\xi(a,b)\rrbracket))\\%
                                                 &= \Psi_{A,B}(\llbracket a\mapsto\llbracket b\mapsto\xi(a,b)\rrbracket\rrbracket)\\%
                                                 &= \Psi_{A,B}(\llbracket a'\mapsto\llbracket b'\mapsto\xi(a',b')\rrbracket\rrbracket)\\%
                                                 &= \llbracket(a,b)\mapsto\ev_{b}(\ev_{a}(\llbracket a'\mapsto\llbracket b'\mapsto\xi(a',b')\rrbracket\rrbracket))\rrbracket\\%
                                                 &= \llbracket(a,b)\mapsto\ev_{b}(\llbracket b'\mapsto\xi(a,b')\rrbracket)\rrbracket\\%
                                                 &= \llbracket(a,b)\mapsto\xi(a,b)\rrbracket\\%
                                                 &= \xi.
            \end{align*}
        \item\label{proof-of-properties-of-products-of-sets-adjointness-1-4}\SloganFont{Invertibility \rmII. }We claim that
            \[
                \Phi_{A,B}\circ\Psi_{A,B}%
                =%
                \id_{\Sets(A,\Sets(B,C))}.%
            \]%
            Indeed, given a function
            \begin{webcompile}
                \phantom{\xi\colon}
                \begin{tikzcd}[row sep=0.0*\the\DL, column sep=1.0*\the\DL, background color=backgroundColor, ampersand replacement=\&]
                    \mathllap{\xi\colon}A%
                    \arrow[r]
                    \&
                    \Sets(B,C)\mrp{,}%
                    \\
                    a
                    \arrow[r, mapsto]
                    \&
                    {(\xi_{a}\colon B\to C)\mrp{,}}
                \end{tikzcd}
            \end{webcompile}
            we have
            \begin{align*}
                [\Phi_{A,B}\circ\Psi_{A,B}](\xi) &\eqdef \Phi_{A,B}(\Psi_{A,B}(\xi))\\%
                                                 &\eqdef \Phi_{A,B}(\llbracket(a,b)\mapsto\xi_{a}(b)\rrbracket)\\%
                                                 &\eqdef \Phi_{A,B}(\llbracket(a',b')\mapsto\xi_{a'}(b')\rrbracket)\\%
                                                 &\eqdef \llbracket a\mapsto\llbracket b\mapsto\ev_{(a,b)}(\llbracket(a',b')\mapsto\xi_{a'}(b')\rrbracket)\rrbracket\rrbracket\\%
                                                 &\eqdef \llbracket a\mapsto\llbracket b\mapsto\xi_{a}(b)\rrbracket\rrbracket\\%
                                                 &\eqdef \llbracket a\mapsto\xi_{a}\rrbracket\\%
                                                 &\eqdef \xi.%
            \end{align*}
        \item\label{proof-of-properties-of-products-of-sets-adjointness-1-5}\SloganFont{Naturality for $\Phi$, Part \rmI. }We need to show that, given a function $g\colon B\to B'$, the diagram
            \[
                \begin{tikzcd}[row sep={5.0*\the\DL,between origins}, column sep={12.0*\the\DL,between origins}, background color=backgroundColor, ampersand replacement=\&]
                    \Sets(A\times B',C)%
                    \arrow[r,"\Phi_{B',C}"]
                    \arrow[d,"{\id_{A}\times g^{*}}"']
                    \&
                    \Sets(A,\Sets(B',C)),%
                    \arrow[d,"{(g^{*})_{!}}"]
                    \\
                    \Sets(A\times B,C)%
                    \arrow[r,"\Phi_{B,C}"']
                    \&
                    \Sets(A,\Sets(B,C))%
                \end{tikzcd}
            \]%
            commutes. Indeed, given a function
            \[
                \xi%
                \colon%
                A\times B'%
                \to%
                C,%
            \]%
            we have
            \begin{align*}
                [\Phi_{B,C}\circ(\id_{A}\times g^{*})](\xi) &= \Phi_{B,C}([\id_{A}\times g^{*}](\xi))\\
                                                            &= \Phi_{B,C}(\xi(-_{1},g(-_{2})))\\
                                                            &= [\xi(-_{1},g(-_{2}))]^{\dagger}\\
                                                            &= \xi^{\dagger}_{-_{1}}(g(-_{2}))\\
                                                            &= (g^{*})_{!}(\xi^{\dagger})\\
                                                            &= (g^{*})_{!}(\Phi_{B',C}(\xi))\\
                                                            &= [(g^{*})_{!}\circ\Phi_{B',C}](\xi).
            \end{align*}
            Alternatively, using the $\llbracket a\mapsto f(a)\rrbracket$ notation of \ChapterRef{\ChapterSets, \cref{sets:additional-notation-for-functions}}{\cref{additional-notation-for-functions}}, we have
            \begin{align*}
                [\Phi_{B,C}\circ(\id_{A}\times g^{*})](\xi) &= \Phi_{B,C}([\id_{A}\times g^{*}](\xi))\\
                                                            &= \Phi_{B,C}([\id_{A}\times g^{*}](\llbracket(a,b')\mapsto\xi(a,b')\rrbracket))\\
                                                            &= \Phi_{B,C}(\llbracket(a,b)\mapsto\xi(a,g(b))\rrbracket)\\
                                                            &= \llbracket a\mapsto\llbracket b\mapsto\xi(a,g(b))\rrbracket\rrbracket\\
                                                            &= \llbracket a\mapsto g^{*}(\llbracket b'\mapsto\xi(a,b')\rrbracket)\rrbracket\\
                                                            &= (g^{*})_{!}(\llbracket a\mapsto\llbracket b'\mapsto\xi(a,b')\rrbracket\rrbracket)\\
                                                            &= (g^{*})_{!}(\Phi_{B',C}(\llbracket(a,b')\mapsto\xi(a,b')\rrbracket))\\
                                                            &= (g^{*})_{!}(\Phi_{B',C}(\xi))\\
                                                            &= [(g^{*})_{!}\circ\Phi_{B',C}](\xi).
            \end{align*}
        \item\label{proof-of-properties-of-products-of-sets-adjointness-1-6}\SloganFont{Naturality for $\Phi$, Part \rmII. }We need to show that, given a function $h\colon C\to C'$, the diagram
            \[
                \begin{tikzcd}[row sep={5.0*\the\DL,between origins}, column sep={12.0*\the\DL,between origins}, background color=backgroundColor, ampersand replacement=\&]
                    \Sets(A\times B,C)%
                    \arrow[r,"\Phi_{B,C}"]
                    \arrow[d,"h_{!}"']
                    \&
                    \Sets(A,\Sets(B,C)),%
                    \arrow[d,"{(h_{!})_{!}}"]
                    \\
                    \Sets(A\times B,C')%
                    \arrow[r,"\Phi_{B,C'}"']
                    \&
                    \Sets(A,\Sets(B,C'))%
                \end{tikzcd}
            \]%
            commutes. Indeed, given a function
            \[
                \xi%
                \colon%
                A\times B%
                \to%
                C,%
            \]%
            we have
            \begin{align*}
                [\Phi_{B,C}\circ h_{!}](\xi) &= \Phi_{B,C}(h_{!}(\xi))\\
                                             &= \Phi_{B,C}(h_{!}(\llbracket(a,b)\mapsto\xi(a,b)\rrbracket))\\
                                             &= \Phi_{B,C}(\llbracket(a,b)\mapsto h(\xi(a,b))\rrbracket)\\
                                             &= \llbracket a\mapsto\llbracket b\mapsto h(\xi(a,b))\rrbracket\rrbracket\\
                                             &= \llbracket a\mapsto h_{!}(\llbracket b\mapsto\xi(a,b)\rrbracket\rrbracket)\\
                                             &= (h_{!})_{!}(\llbracket a\mapsto\llbracket b\mapsto\xi(a,b)\rrbracket\rrbracket)\\
                                             &= (h_{!})_{!}(\Phi_{B,C}(\llbracket(a,b)\mapsto\xi(a,b)\rrbracket))\\
                                             &= (h_{!})_{!}(\Phi_{B,C}(\xi))\\
                                             &= [(h_{!})_{!}\circ\Phi_{B,C}](\xi).
            \end{align*}
        \item\label{proof-of-properties-of-products-of-sets-adjointness-1-7}\SloganFont{Naturality for $\Psi$. }Since $\Phi$ is natural in each argument and $\Phi$ is a componentwise inverse to $\Psi$ in each argument, it follows from \ChapterRef{\ChapterCategories, \cref{categories:properties-of-natural-isomorphisms-componentwise-inverses-of-natural-transformations-assemble-into-natural-transformations} of \cref{categories:properties-of-natural-isomorphisms}}{\cref{properties-of-natural-isomorphisms-componentwise-inverses-of-natural-transformations-assemble-into-natural-transformations} of \cref{properties-of-natural-isomorphisms}} that $\Psi$ is also natural in each argument.
    \end{itemize}
    This finishes the proof.

    \ProofBox{\cref{properties-of-products-of-sets-adjointness-2}: Adjointness \rmII}%
    This follows from the universal property of the product.

    \ProofBox{\cref{properties-of-products-of-sets-associativity}: Associativity}%
    This is proved in the proof of \ChapterRef{\ChapterMonoidalStructuresOnTheCategoryOfSets, \cref{monoidal-structures-on-the-category-of-sets:the-associator-of-the-product-of-sets}}{\cref{the-associator-of-the-product-of-sets}}.

    \ProofBox{\cref{properties-of-products-of-sets-unitality}: Unitality}%
    This is proved in the proof of \ChapterRef{\ChapterMonoidalStructuresOnTheCategoryOfSets, \cref{monoidal-structures-on-the-category-of-sets:the-left-unitor-of-the-product-of-sets,monoidal-structures-on-the-category-of-sets:the-right-unitor-of-the-product-of-sets}}{\cref{the-left-unitor-of-the-product-of-sets,the-right-unitor-of-the-product-of-sets}}.

    \ProofBox{\cref{properties-of-products-of-sets-commutativity}: Commutativity}%
    This is proved in the proof of \ChapterRef{\ChapterMonoidalStructuresOnTheCategoryOfSets, \cref{monoidal-structures-on-the-category-of-sets:the-symmetry-of-the-product-of-sets}}{\cref{the-symmetry-of-the-product-of-sets}}.

    \ProofBox{\cref{properties-of-products-of-sets-distributivity-over-coproducts}: Distributivity Over Coproducts}%
    This is proved in the proof of \ChapterRef{\ChapterMonoidalStructuresOnTheCategoryOfSets, \cref{monoidal-structures-on-the-category-of-sets:the-left-distributor-of-the-product-of-sets-over-the-coproduct-of-sets,monoidal-structures-on-the-category-of-sets:the-right-distributor-of-the-product-of-sets-over-the-coproduct-of-sets}}{\cref{the-left-distributor-of-the-product-of-sets-over-the-coproduct-of-sets,the-right-distributor-of-the-product-of-sets-over-the-coproduct-of-sets}}.

    \ProofBox{\cref{properties-of-products-of-sets-annihilation-with-the-empty-set}: Annihilation With the Empty Set}%
    This is proved in the proof of \ChapterRef{\ChapterMonoidalStructuresOnTheCategoryOfSets, \cref{monoidal-structures-on-the-category-of-sets:the-left-annihilator-of-the-product-of-sets,monoidal-structures-on-the-category-of-sets:the-right-annihilator-of-the-product-of-sets}}{\cref{the-left-annihilator-of-the-product-of-sets,the-right-annihilator-of-the-product-of-sets}}.

    \ProofBox{\cref{properties-of-products-of-sets-distributivity-over-unions}: Distributivity Over Unions}%
    See \cite{proof-wiki:cartesian-product-distributes-over-union}.

    \ProofBox{\cref{properties-of-products-of-sets-distributivity-over-intersections}: Distributivity Over Intersections}%
    See \cite[Corollary 1]{proof-wiki:cartesian-product-of-intersections}.

    \ProofBox{\cref{properties-of-products-of-sets-distributivity-over-differences}: Distributivity Over Differences}%
    See \cite{proof-wiki:cartesian-product-distributes-over-set-difference}.

    \ProofBox{\cref{properties-of-products-of-sets-distributivity-over-symmetric-differences}: Distributivity Over Symmetric Differences}%
    See \cite{proof-wiki:cartesian-product-distributes-over-symmetric-difference}.

    \ProofBox{\cref{properties-of-products-of-sets-middle-four-exchange-with-respect-to-intersections}: Middle-Four Exchange With Respect to Intersections}%
    See \cite[Corollary 1]{proof-wiki:cartesian-product-of-intersections}.

    \ProofBox{\cref{properties-of-products-of-sets-symmetric-monoidality}: Symmetric Monoidality}%
    This is a repetition of \ChapterRef{\ChapterMonoidalStructuresOnTheCategoryOfSets, \cref{monoidal-structures-on-the-category-of-sets:the-monoidal-structure-on-sets-associated-to-the-product}}{\cref{the-monoidal-structure-on-sets-associated-to-the-product}}, and is proved there.

    \ProofBox{\cref{properties-of-products-of-sets-symmetric-bimonoidality}: Symmetric Bimonoidality}%
    This is a repetition of \ChapterRef{\ChapterMonoidalStructuresOnTheCategoryOfSets, \cref{monoidal-structures-on-the-category-of-sets:the-bimonoidal-structure-on-sets-associated-to-the-product-and-the-coproduct}}{\cref{the-bimonoidal-structure-on-sets-associated-to-the-product-and-the-coproduct}}, and is proved there.
\end{Proof}
\begin{remark}{The Cartesian Product of Sets as an $(\E_{k},\E_{\ell})$-Tensor Product}{the-cartesian-product-of-sets-as-an-e-k-e-ell-tensor-product}%
    As shown in \cref{properties-of-products-of-sets-functoriality} of \cref{properties-of-products-of-sets}, the Cartesian product of sets defines a functor
    \[
        -_{1}\times-_{2}%
        \colon%
        \Sets\times\Sets%
        \to%
        \Sets.%
    \]%
    This functor is the $(k,\ell)=(-1,-1)$ case of a family of functors
    \[
        \otimes_{k,\ell}%
        \colon%
        \Mon_{\E_{k}}(\Sets)%
        \times%
        \Mon_{\E_{\ell}}(\Sets)%
        \to%
        \Mon_{\E_{k+\ell}}(\Sets)%
    \]%
    of tensor products of $\E_{k}$-monoid objects on $\Sets$ with $\E_{\ell}$-monoid objects on $\Sets$; see \cref{TODO3}.
\end{remark}
\begin{remark}{Diagrams for \cref{properties-of-products-of-sets-distributivity-over-unions,properties-of-products-of-sets-distributivity-over-intersections,properties-of-products-of-sets-distributivity-over-differences,properties-of-products-of-sets-distributivity-over-symmetric-differences} of \cref{properties-of-products-of-sets}}{diagrams-for-some-items-in-properties-of-products-of-sets}%
    We may state the equalities in \cref{properties-of-products-of-sets-distributivity-over-unions,properties-of-products-of-sets-distributivity-over-intersections,properties-of-products-of-sets-distributivity-over-differences,properties-of-products-of-sets-distributivity-over-symmetric-differences} of \cref{properties-of-products-of-sets} as the commutativity of the following diagrams:%
    \begin{scalemath}
        \begin{tikzcd}[row sep={0.0*\the\DL,between origins}, column sep={0.0*\the\DL,between origins}, background color=backgroundColor, ampersand replacement=\&]
            \&[0.86602540378\ThreeCm]
            (\mathcal{P}(X)\times\mathcal{P}(X))\times(\mathcal{P}(X)\times\mathcal{P}(X))
            \&[0.86602540378\ThreeCm]
            \\[0.5\ThreeCm]
            \mathcal{P}(X)\times(\mathcal{P}(X)\times\mathcal{P}(X))
            \&[0.86602540378\ThreeCm]
            \&[0.86602540378\ThreeCm]
            (\mathcal{P}(X)\times\mathcal{P}(X))\times(\mathcal{P}(X)\times\mathcal{P}(X))
            \\[\ThreeCm]
            \mathcal{P}(X)\times\mathcal{P}(X)
            \&[0.86602540378\ThreeCm]
            \&[0.86602540378\ThreeCm]
            \mathcal{P}(X\times X)\times\mathcal{P}(X\times X)
            \\[0.5\ThreeCm]
            \&[0.86602540378\ThreeCm]
            \mathcal{P}(X\times X)
            \&[0.86602540378\ThreeCm]
            % 1-Arrows
            % Left
            \arrow[from=2-1,to=3-1,"\id_{\mathcal{P}(X)}\times\mathord{\cup}"']%
            \arrow[from=3-1,to=4-2,"\mathcal{P}^{\times}_{X,X}"']%
            % Right
            \arrow[from=2-1,to=1-2,"\Delta_{\mathcal{P}(X)}\times\id_{\mathcal{P}(X)}\times\id_{\mathcal{P}(X)}"{pos=0.35}]%
            \arrow[from=1-2,to=2-3,"\mu^{\Sets}_{4}"]%
            \arrow[from=2-3,to=3-3,"\mathcal{P}^{\times}_{X,X}\times\mathcal{P}^{\times}_{X,X}"]%
            \arrow[from=3-3,to=4-2,"\cup"]%
        \end{tikzcd}
        \quad
        \begin{tikzcd}[row sep={0.0*\the\DL,between origins}, column sep={0.0*\the\DL,between origins}, background color=backgroundColor, ampersand replacement=\&]
            \&[0.86602540378\ThreeCm]
            (\mathcal{P}(X)\times\mathcal{P}(X))\times(\mathcal{P}(X)\times\mathcal{P}(X))
            \&[0.86602540378\ThreeCm]
            \\[0.5\ThreeCm]
            (\mathcal{P}(X)\times\mathcal{P}(X))\times\mathcal{P}(X)
            \&[0.86602540378\ThreeCm]
            \&[0.86602540378\ThreeCm]
            (\mathcal{P}(X)\times\mathcal{P}(X))\times(\mathcal{P}(X)\times\mathcal{P}(X))
            \\[\ThreeCm]
            \mathcal{P}(X)\times\mathcal{P}(X)
            \&[0.86602540378\ThreeCm]
            \&[0.86602540378\ThreeCm]
            \mathcal{P}(X\times X)\times\mathcal{P}(X\times X)
            \\[0.5\ThreeCm]
            \&[0.86602540378\ThreeCm]
            \mathcal{P}(X\times X)
            \&[0.86602540378\ThreeCm]
            % 1-Arrows
            % Left
            \arrow[from=2-1,to=3-1,"\mathord{\cup}\times\id_{\mathcal{P}(X)}"']%
            \arrow[from=3-1,to=4-2,"\mathcal{P}^{\times}_{X,X}"']%
            % Right
            \arrow[from=2-1,to=1-2,"\id_{\mathcal{P}(X)}\times\id_{\mathcal{P}(X)}\times\Delta_{\mathcal{P}(X)}"{pos=0.35}]%
            \arrow[from=1-2,to=2-3,"\mu^{\Sets}_{4}"]%
            \arrow[from=2-3,to=3-3,"\mathcal{P}^{\times}_{X,X}\times\mathcal{P}^{\times}_{X,X}"]%
            \arrow[from=3-3,to=4-2,"\cup"]%
        \end{tikzcd}
    \end{scalemath}
    \begin{scalemath}
        \begin{tikzcd}[row sep={0.0*\the\DL,between origins}, column sep={0.0*\the\DL,between origins}, background color=backgroundColor, ampersand replacement=\&]
            \&[0.86602540378\ThreeCm]
            (\mathcal{P}(X)\times\mathcal{P}(X))\times(\mathcal{P}(X)\times\mathcal{P}(X))
            \&[0.86602540378\ThreeCm]
            \\[0.5\ThreeCm]
            \mathcal{P}(X)\times(\mathcal{P}(X)\times\mathcal{P}(X))
            \&[0.86602540378\ThreeCm]
            \&[0.86602540378\ThreeCm]
            (\mathcal{P}(X)\times\mathcal{P}(X))\times(\mathcal{P}(X)\times\mathcal{P}(X))
            \\[\ThreeCm]
            \mathcal{P}(X)\times\mathcal{P}(X)
            \&[0.86602540378\ThreeCm]
            \&[0.86602540378\ThreeCm]
            \mathcal{P}(X\times X)\times\mathcal{P}(X\times X)
            \\[0.5\ThreeCm]
            \&[0.86602540378\ThreeCm]
            \mathcal{P}(X\times X)
            \&[0.86602540378\ThreeCm]
            % 1-Arrows
            % Left
            \arrow[from=2-1,to=3-1,"\id_{\mathcal{P}(X)}\times\mathord{\cap}"']%
            \arrow[from=3-1,to=4-2,"\mathcal{P}^{\times}_{X,X}"']%
            % Right
            \arrow[from=2-1,to=1-2,"\Delta_{\mathcal{P}(X)}\times\id_{\mathcal{P}(X)}\times\id_{\mathcal{P}(X)}"{pos=0.35}]%
            \arrow[from=1-2,to=2-3,"\mu^{\Sets}_{4}"]%
            \arrow[from=2-3,to=3-3,"\mathcal{P}^{\times}_{X,X}\times\mathcal{P}^{\times}_{X,X}"]%
            \arrow[from=3-3,to=4-2,"\cap"]%
        \end{tikzcd}
        \quad
        \begin{tikzcd}[row sep={0.0*\the\DL,between origins}, column sep={0.0*\the\DL,between origins}, background color=backgroundColor, ampersand replacement=\&]
            \&[0.86602540378\ThreeCm]
            (\mathcal{P}(X)\times\mathcal{P}(X))\times(\mathcal{P}(X)\times\mathcal{P}(X))
            \&[0.86602540378\ThreeCm]
            \\[0.5\ThreeCm]
            (\mathcal{P}(X)\times\mathcal{P}(X))\times\mathcal{P}(X)
            \&[0.86602540378\ThreeCm]
            \&[0.86602540378\ThreeCm]
            (\mathcal{P}(X)\times\mathcal{P}(X))\times(\mathcal{P}(X)\times\mathcal{P}(X))
            \\[\ThreeCm]
            \mathcal{P}(X)\times\mathcal{P}(X)
            \&[0.86602540378\ThreeCm]
            \&[0.86602540378\ThreeCm]
            \mathcal{P}(X\times X)\times\mathcal{P}(X\times X)
            \\[0.5\ThreeCm]
            \&[0.86602540378\ThreeCm]
            \mathcal{P}(X\times X)
            \&[0.86602540378\ThreeCm]
            % 1-Arrows
            % Left
            \arrow[from=2-1,to=3-1,"\mathord{\cap}\times\id_{\mathcal{P}(X)}"']%
            \arrow[from=3-1,to=4-2,"\mathcal{P}^{\times}_{X,X}"']%
            % Right
            \arrow[from=2-1,to=1-2,"\id_{\mathcal{P}(X)}\times\id_{\mathcal{P}(X)}\times\Delta_{\mathcal{P}(X)}"{pos=0.35}]%
            \arrow[from=1-2,to=2-3,"\mu^{\Sets}_{4}"]%
            \arrow[from=2-3,to=3-3,"\mathcal{P}^{\times}_{X,X}\times\mathcal{P}^{\times}_{X,X}"]%
            \arrow[from=3-3,to=4-2,"\cap"]%
        \end{tikzcd}
    \end{scalemath}
    \begin{scalemath}
        \begin{tikzcd}[row sep={0.0*\the\DL,between origins}, column sep={0.0*\the\DL,between origins}, background color=backgroundColor, ampersand replacement=\&]
            \&[0.86602540378\ThreeCm]
            (\mathcal{P}(X)\times\mathcal{P}(X))\times(\mathcal{P}(X)\times\mathcal{P}(X))
            \&[0.86602540378\ThreeCm]
            \\[0.5\ThreeCm]
            \mathcal{P}(X)\times(\mathcal{P}(X)\times\mathcal{P}(X))
            \&[0.86602540378\ThreeCm]
            \&[0.86602540378\ThreeCm]
            (\mathcal{P}(X)\times\mathcal{P}(X))\times(\mathcal{P}(X)\times\mathcal{P}(X))
            \\[\ThreeCm]
            \mathcal{P}(X)\times\mathcal{P}(X)
            \&[0.86602540378\ThreeCm]
            \&[0.86602540378\ThreeCm]
            \mathcal{P}(X\times X)\times\mathcal{P}(X\times X)
            \\[0.5\ThreeCm]
            \&[0.86602540378\ThreeCm]
            \mathcal{P}(X\times X)
            \&[0.86602540378\ThreeCm]
            % 1-Arrows
            % Left
            \arrow[from=2-1,to=3-1,"\id_{\mathcal{P}(X)}\times\mathord{\setminus}"']%
            \arrow[from=3-1,to=4-2,"\mathcal{P}^{\times}_{X,X}"']%
            % Right
            \arrow[from=2-1,to=1-2,"\Delta_{\mathcal{P}(X)}\times\id_{\mathcal{P}(X)}\times\id_{\mathcal{P}(X)}"{pos=0.35}]%
            \arrow[from=1-2,to=2-3,"\mu^{\Sets}_{4}"]%
            \arrow[from=2-3,to=3-3,"\mathcal{P}^{\times}_{X,X}\times\mathcal{P}^{\times}_{X,X}"]%
            \arrow[from=3-3,to=4-2,"\setminus"]%
        \end{tikzcd}
        \quad
        \begin{tikzcd}[row sep={0.0*\the\DL,between origins}, column sep={0.0*\the\DL,between origins}, background color=backgroundColor, ampersand replacement=\&]
            \&[0.86602540378\ThreeCm]
            (\mathcal{P}(X)\times\mathcal{P}(X))\times(\mathcal{P}(X)\times\mathcal{P}(X))
            \&[0.86602540378\ThreeCm]
            \\[0.5\ThreeCm]
            (\mathcal{P}(X)\times\mathcal{P}(X))\times\mathcal{P}(X)
            \&[0.86602540378\ThreeCm]
            \&[0.86602540378\ThreeCm]
            (\mathcal{P}(X)\times\mathcal{P}(X))\times(\mathcal{P}(X)\times\mathcal{P}(X))
            \\[\ThreeCm]
            \mathcal{P}(X)\times\mathcal{P}(X)
            \&[0.86602540378\ThreeCm]
            \&[0.86602540378\ThreeCm]
            \mathcal{P}(X\times X)\times\mathcal{P}(X\times X)
            \\[0.5\ThreeCm]
            \&[0.86602540378\ThreeCm]
            \mathcal{P}(X\times X)
            \&[0.86602540378\ThreeCm]
            % 1-Arrows
            % Left
            \arrow[from=2-1,to=3-1,"\mathord{\setminus}\times\id_{\mathcal{P}(X)}"']%
            \arrow[from=3-1,to=4-2,"\mathcal{P}^{\times}_{X,X}"']%
            % Right
            \arrow[from=2-1,to=1-2,"\id_{\mathcal{P}(X)}\times\id_{\mathcal{P}(X)}\times\Delta_{\mathcal{P}(X)}"{pos=0.35}]%
            \arrow[from=1-2,to=2-3,"\mu^{\Sets}_{4}"]%
            \arrow[from=2-3,to=3-3,"\mathcal{P}^{\times}_{X,X}\times\mathcal{P}^{\times}_{X,X}"]%
            \arrow[from=3-3,to=4-2,"\setminus"]%
        \end{tikzcd}
    \end{scalemath}
    \begin{scalemath}
        \begin{tikzcd}[row sep={0.0*\the\DL,between origins}, column sep={0.0*\the\DL,between origins}, background color=backgroundColor, ampersand replacement=\&]
            \&[0.86602540378\ThreeCm]
            (\mathcal{P}(X)\times\mathcal{P}(X))\times(\mathcal{P}(X)\times\mathcal{P}(X))
            \&[0.86602540378\ThreeCm]
            \\[0.5\ThreeCm]
            \mathcal{P}(X)\times(\mathcal{P}(X)\times\mathcal{P}(X))
            \&[0.86602540378\ThreeCm]
            \&[0.86602540378\ThreeCm]
            (\mathcal{P}(X)\times\mathcal{P}(X))\times(\mathcal{P}(X)\times\mathcal{P}(X))
            \\[\ThreeCm]
            \mathcal{P}(X)\times\mathcal{P}(X)
            \&[0.86602540378\ThreeCm]
            \&[0.86602540378\ThreeCm]
            \mathcal{P}(X\times X)\times\mathcal{P}(X\times X)
            \\[0.5\ThreeCm]
            \&[0.86602540378\ThreeCm]
            \mathcal{P}(X\times X)
            \&[0.86602540378\ThreeCm]
            % 1-Arrows
            % Left
            \arrow[from=2-1,to=3-1,"\id_{\mathcal{P}(X)}\times\mathord{\sdiff}"']%
            \arrow[from=3-1,to=4-2,"\mathcal{P}^{\times}_{X,X}"']%
            % Right
            \arrow[from=2-1,to=1-2,"\Delta_{\mathcal{P}(X)}\times\id_{\mathcal{P}(X)}\times\id_{\mathcal{P}(X)}"{pos=0.35}]%
            \arrow[from=1-2,to=2-3,"\mu^{\Sets}_{4}"]%
            \arrow[from=2-3,to=3-3,"\mathcal{P}^{\times}_{X,X}\times\mathcal{P}^{\times}_{X,X}"]%
            \arrow[from=3-3,to=4-2,"\sdiff"]%
        \end{tikzcd}
        \quad
        \begin{tikzcd}[row sep={0.0*\the\DL,between origins}, column sep={0.0*\the\DL,between origins}, background color=backgroundColor, ampersand replacement=\&]
            \&[0.86602540378\ThreeCm]
            (\mathcal{P}(X)\times\mathcal{P}(X))\times(\mathcal{P}(X)\times\mathcal{P}(X))
            \&[0.86602540378\ThreeCm]
            \\[0.5\ThreeCm]
            (\mathcal{P}(X)\times\mathcal{P}(X))\times\mathcal{P}(X)
            \&[0.86602540378\ThreeCm]
            \&[0.86602540378\ThreeCm]
            (\mathcal{P}(X)\times\mathcal{P}(X))\times(\mathcal{P}(X)\times\mathcal{P}(X))
            \\[\ThreeCm]
            \mathcal{P}(X)\times\mathcal{P}(X)
            \&[0.86602540378\ThreeCm]
            \&[0.86602540378\ThreeCm]
            \mathcal{P}(X\times X)\times\mathcal{P}(X\times X)
            \\[0.5\ThreeCm]
            \&[0.86602540378\ThreeCm]
            \mathcal{P}(X\times X)
            \&[0.86602540378\ThreeCm]
            % 1-Arrows
            % Left
            \arrow[from=2-1,to=3-1,"\mathord{\sdiff}\times\id_{\mathcal{P}(X)}"']%
            \arrow[from=3-1,to=4-2,"\mathcal{P}^{\times}_{X,X}"']%
            % Right
            \arrow[from=2-1,to=1-2,"\id_{\mathcal{P}(X)}\times\id_{\mathcal{P}(X)}\times\Delta_{\mathcal{P}(X)}"{pos=0.35}]%
            \arrow[from=1-2,to=2-3,"\mu^{\Sets}_{4}"]%
            \arrow[from=2-3,to=3-3,"\mathcal{P}^{\times}_{X,X}\times\mathcal{P}^{\times}_{X,X}"]%
            \arrow[from=3-3,to=4-2,"\sdiff"]%
        \end{tikzcd}
    \end{scalemath}
\end{remark}
\subsection{Pullbacks}\label{subsection-limits-of-sets-pullbacks}
Let $A$, $B$, and $C$ be sets and let $f\colon A\to C$ and $g\colon B\to C$ be functions.
\begin{definition}{Pullbacks of Sets}{pullbacks-of-sets}%
    The \index[set-theory]{pullback of sets}\textbf{pullback of $A$ and $B$ over $C$ along $f$ and $g$}%
    %--- Begin Footnote ---%
    \footnote{%
        \SloganFont{Further Terminology: }Also called the \index[set-theory]{fibre product of sets}\textbf{fibre product of $A$ and $B$ over $C$ along $f$ and $g$}.
        \par\vspace*{\TCBBoxCorrection}
    } %
    %---  End Footnote  ---%
    is the pullback of $A$ and $B$ over $C$ along $f$ and $g$ in $\Sets$ as in \ChapterRef{\ChapterLimitsAndColimits, \cref{limits-and-colimits:pullbacks}}{\cref{pullbacks}}.
\end{definition}
\begin{construction}{Construction of Pullbacks of Sets}{construction-of-pullbacks-of-sets}%
    Concretely, the pullback of $A$ and $B$ over $C$ along $f$ and $g$ is the pair \index[notation]{AtimesCB@$A\times_{C}B$}$(A\times_{C}B,\{\pr_{1},\pr_{2}\})$ consisting of:
    \begin{enumerate}
        \item\label{construction-of-pullbacks-of-sets-the-limit}\SloganFont{The Limit. }The set $A\times_{C}B$ given by%
            \[
                A\times_{C}B%
                =%
                \{(a,b)\in A\times B\ \middle|\ f(a)=g(b)\}.
            \]%
        \item\label{construction-of-pullbacks-of-sets-the-cone}\SloganFont{The Cone. }The maps%
            %--- Begin Footnote ---%
            \footnote{%
                \SloganFont{Further Notation: }Also written \index[notation]{prAtimesCB1@$\pr^{A\times_{C}B}_{1}$}$\pr^{A\times_{C}B}_{1}$ and \index[notation]{prAtimesCB2@$\pr^{A\times_{C}B}_{2}$}$\pr^{A\times_{C}B}_{2}$.
                \par\vspace*{\TCBBoxCorrection}
            }%
            %---  End Footnote  ---%
            \begin{align*}
                \pr_{1} &\colon A\times_{C}B\to A,\\
                \pr_{2} &\colon A\times_{C}B\to B
            \end{align*}
            defined by
            \begin{align*}
                \pr_{1}(a,b) &\defeq a,\\
                \pr_{2}(a,b) &\defeq b
            \end{align*}
            for each $(a,b)\in A\times_{C}B$.
    \end{enumerate}
\end{construction}
\begin{Proof}{Proof of \cref{construction-of-pullbacks-of-sets}}%
    We claim that $A\times_{C}B$ is the categorical pullback of $A$ and $B$ over $C$ with respect to $(f,g)$ in $\Sets$. First we need to check that the relevant pullback diagram commutes, i.e.\ that we have
    \begin{webcompile}
        f\circ\pr_{1}%
        =%
        g\circ\pr_{2},%
        \qquad
        \begin{tikzcd}[row sep={5.0*\the\DL,between origins}, column sep={5.0*\the\DL,between origins}, background color=backgroundColor, ampersand replacement=\&]
            A\times_{C}B
            \arrow[r,"\pr_{2}",two heads]
            \arrow[d,"\pr_{1}"',two heads]
            \&
            B
            \arrow[d,"g"]
            \\
            A
            \arrow[r,"f"']
            \&
            C\mrp{.}
        \end{tikzcd}
    \end{webcompile}
    Indeed, given $(a,b)\in A\times_{C}B$, we have
    \begin{align*}
        [f\circ\pr_{1}](a,b) &= f(\pr_{1}(a,b))\\%
                             &= f(a)\\%
                             &= g(b)\\%
                             &= g(\pr_{2}(a,b))\\%
                             &= [g\circ\pr_{2}](a,b),%
    \end{align*}
    where $f(a)=g(b)$ since $(a,b)\in A\times_{C}B$. Next, we prove that $A\times_{C}B$ satisfies the universal property of the pullback. Suppose we have a diagram of the form
    \[
        \begin{tikzcd}[row sep={6.0*\the\DL,between origins}, column sep={6.0*\the\DL,between origins}, background color=backgroundColor, ampersand replacement=\&]
            P
            \arrow[rrd, "p_{2}",  bend left =25]
            \arrow[rdd, "p_{1}"', bend right=27.5]
            \&[-2.0*\the\DL]
            \&
            \\[-2.0*\the\DL]
            \&[-2.0*\the\DL]
            A\times_{C}B
            \arrow[rd, phantom, "\lrcorner", very near start]
            \arrow[r, "\pr_{2}"'description,two heads]
            \arrow[d, "\pr_{1}"description,two heads]
            \&
            B
            \arrow[d, "g"]
            \\
            \&[-2.0*\the\DL]
            A
            \arrow[r, "f"']
            \&
            C
        \end{tikzcd}
    \]%
    in $\Sets$. Then there exists a unique map $\phi\colon P\to A\times_{C}B$ making the diagram
    \[
        \begin{tikzcd}[row sep={6.0*\the\DL,between origins}, column sep={6.0*\the\DL,between origins}, background color=backgroundColor, ampersand replacement=\&]
            P
            \arrow[rrd, "p_{2}",  bend left =25]
            \arrow[rdd, "p_{1}"', bend right=27.5]
            \arrow[rd,  "\phi","\exists!"', dashed]
            \&[-2.0*\the\DL]
            \&
            \\[-2.0*\the\DL]
            \&[-2.0*\the\DL]
            A\times_{C}B
            \arrow[rd, phantom, "\lrcorner", very near start]
            \arrow[r, "\pr_{2}"'description,two heads]
            \arrow[d, "\pr_{1}"description,two heads]
            \&
            B
            \arrow[d, "g"]
            \\
            \&[-2.0*\the\DL]
            A
            \arrow[r, "f"']
            \&
            C
        \end{tikzcd}
    \]%
    commute, being uniquely determined by the conditions%
    \begin{align*}
        \pr_{1}\circ\phi &= p_{1},\\%
        \pr_{2}\circ\phi &= p_{2}%
    \end{align*}
    via
    \[
        \phi(x)%
        =%
        (p_{1}(x),p_{2}(x))%
    \]%
    for each $x\in P$, where we note that $(p_{1}(x),p_{2}(x))\in A\times B$ indeed lies in $A\times_{C}B$ by the condition
    \[
        f\circ p_{1}%
        =%
        g\circ p_{2},%
    \]%
    which gives
    \[
        f(p_{1}(x))%
        =%
        g(p_{2}(x))%
    \]%
    for each $x\in P$, so that $(p_{1}(x),p_{2}(x))\in A\times_{C}B$.
\end{Proof}
\begin{remark}{Pullbacks of Sets Depend on the Maps}{pullbacks-of-sets-depend-on-the-maps}%
    It is common practice to write $A\times_{C}B$ for the pullback of $A$ and $B$ over $C$ along $f$ and $g$, omitting the maps $f$ and $g$ from the notation and instead leaving them implicit, to be understood from the context.

    \indent However, the set $A\times_{C}B$ depends very much on the maps $f$ and $g$, and sometimes it is necessary or useful to note this dependence explicitly. In such situations, we will write \index[notation]{AtimesfCgB@$A\times_{f,C,g}B$}$A\times_{f,C,g}B$ or \index[notation]{AtimesfgCB@$A\times^{f,g}_{C}B$}$A\times^{f,g}_{C}B$ for $A\times_{C}B$.
\end{remark}
\begin{example}{Examples of Pullbacks of Sets}{examples-of-pullbacks-of-sets}%
    Here are some examples of pullbacks of sets.
    \begin{enumerate}
        \item\label{examples-of-pullbacks-of-sets-unions-via-intersections}\SloganFont{Unions via Intersections. }Let $X$ be a set. We have
            \begin{webcompile}
                A\cap B%
                \cong%
                A\times_{A\cup B}B,%
                \quad
                \begin{tikzcd}[row sep={5.0*\the\DL,between origins}, column sep={5.0*\the\DL,between origins}, background color=backgroundColor, ampersand replacement=\&]
                    A\cap B
                    \arrow[r,two heads]
                    \arrow[d,two heads]
                    \arrow[rd,very near start,phantom,"\lrcorner"]
                    \&
                    B
                    \arrow[d,"\iota_{B}",hook']
                    \\
                    A
                    \arrow[r,"\iota_{A}"',hook]
                    \&
                    A\cup B
                \end{tikzcd}
            \end{webcompile}
            for each $A,B\in\mathcal{P}(X)$.
    \end{enumerate}
\end{example}
\begin{Proof}{Proof of \cref{examples-of-pullbacks-of-sets}}%
    \FirstProofBox{\cref{examples-of-pullbacks-of-sets-unions-via-intersections}: Unions via Intersections}%
    Indeed, we have
    \begin{align*}
        A\times_{A\cup B}B &\cong \{(x,y)\in A\times B\ \middle|\ x=y\}\\
                           &\cong A\cap B.
    \end{align*}
    This finishes the proof.
\end{Proof}
\begin{proposition}{Properties of Pullbacks of Sets}{properties-of-pullbacks-of-sets}%
    Let $A$, $B$, $C$, and $X$ be sets.
    \begin{enumerate}
        \item\label{properties-of-pullbacks-of-sets-functoriality}\SloganFont{Functoriality. }The assignment $(A,B,C,f,g)\mapsto A\times_{f,C,g}B$ defines a functor
            \[
                -_{1}\times_{-_{3}}-_{1}%
                \colon%
                \Fun(\CatFont{P},\Sets)%
                \to%
                \Sets,%
            \]%
            where $\CatFont{P}$ is the category that looks like this:
            \[
                \begin{tikzcd}[row sep={3.0*\the\DL,between origins}, column sep={3.0*\the\DL,between origins}, background color=backgroundColor, ampersand replacement=\&]
                    \&
                    \bullet
                    \arrow[d]
                    \\
                    \bullet
                    \arrow[r]
                    \&
                    \bullet\mrp{.}
                \end{tikzcd}
            \]%
            In particular, the action on morphisms of $-_{1}\times_{-_{3}}-_{1}$ is given by sending a morphism
            \[
                \begin{tikzcd}[row sep={4.0*\the\DL,between origins}, column sep={4.0*\the\DL,between origins}, background color=backgroundColor, ampersand replacement=\&]
                    A\times_{C}B
                    \arrow[rr]
                    \arrow[dd]
                    %\arrow[rd, dashed]
                    \arrow[rd,very near start,phantom,"\lrcorner"{xshift=0.675em}]
                    \&
                    \&
                    B
                    \arrow[dd, "g"{description,pos=0.25}]
                    \arrow[rd, "\psi"]
                    \&
                    \\
                    \&
                    A'\times_{C'}B'
                    \arrow[rr,crossing over]
                    \arrow[rd,very near start,phantom,"\lrcorner"]
                    \&
                    \&
                    B'
                    \arrow[dd, "g'"]
                    \\
                    A
                    \arrow[rr, "f"{pos=0.25}]
                    \arrow[rd, "\phi"']
                    \&
                    \&
                    C
                    \arrow[rd,"\chi"description]
                    \&
                    \\
                    \&
                    A'
                    \arrow[rr, "f'"']
                    \arrow[from=uu, "", crossing over]
                    \&\&
                    C'
                \end{tikzcd}
            \]%
            in $\Fun(\CatFont{P},\Sets)$ to the map $\xi\colon A\times_{C}B\uearrow A'\times_{C'}B'$ given by
            \[
                \xi(a,b)%
                \defeq%
                (\phi(a),\psi(b))%
            \]%
            for each $(a,b)\in A\times_{C}B$, which is the unique map making the diagram
            \[
                \begin{tikzcd}[row sep={4.0*\the\DL,between origins}, column sep={4.0*\the\DL,between origins}, background color=backgroundColor, ampersand replacement=\&]
                    A\times_{C}B
                    \arrow[rr]
                    \arrow[dd]
                    \arrow[rd, dashed]
                    \arrow[rd,very near start,phantom,"\lrcorner"{xshift=0.675em}]
                    \&
                    \&
                    B
                    \arrow[dd, "g"{description,pos=0.25}]
                    \arrow[rd, "\psi"]
                    \&
                    \\
                    \&
                    A'\times_{C'}B'
                    \arrow[rr,crossing over]
                    \arrow[rd,very near start,phantom,"\lrcorner"]
                    \&
                    \&
                    B'
                    \arrow[dd, "g'"]
                    \\
                    A
                    \arrow[rr, "f"{pos=0.25}]
                    \arrow[rd, "\phi"']
                    \&
                    \&
                    C
                    \arrow[rd,"\chi"description]
                    \&
                    \\
                    \&
                    A'
                    \arrow[rr, "f'"']
                    \arrow[from=uu, "", crossing over]
                    \&\&
                    C'
                \end{tikzcd}
            \]%
            commute.
        \item\label{properties-of-pullbacks-of-sets-adjointness-1}\SloganFont{Adjointness \rmI. }We have adjunctions
            \begin{webcompile}
                \begin{gathered}
                    \Adjunction#A\times_{X}-#{\eSets_{/X}(A,-)}#\Sets_{/X}#\Sets_{/X},#\\
                    \Adjunction#-\times_{X}B#{\eSets_{/X}(B,-)}#\Sets_{/X}#\Sets_{/X},#
                \end{gathered}
            \end{webcompile}%
            witnessed by bijections
            \begin{align*}
                \Sets_{/X}(A\times_{X}B,C) &\cong \Sets_{/X}(A,\eSets_{/X}(B,C)),\\
                \Sets_{/X}(A\times_{X}B,C) &\cong \Sets_{/X}(B,\eSets_{/X}(A,C)),
            \end{align*}
            natural in $(A,\phi_{A}),(B,\phi_{B}),(C,\phi_{C})\in\Obj(\Sets_{/X})$, where $\eSets_{/X}(A,B)$ is the object of $\Sets_{/X}$ consisting of (see \ChapterRef{\ChapterFibredSets, \cref{fibred-sets:the-internal-hom-of-fibred-sets}}{\cref{the-internal-hom-of-fibred-sets}}):
            \begin{itemize}
                \item\SloganFont{The Set. }The set $\eSets_{/X}(A,B)$ defined by
                    \[
                        \eSets_{/X}(A,B)%
                        \defeq%
                        \coprod_{x\in X}\Sets(\phi^{-1}_{A}(x),\phi^{-1}_{Y}(x))%
                    \]%
                \item\SloganFont{The Map to $X$. }The map
                    \[
                        \phi_{\eSets_{/X}(A,B)}%
                        \colon%
                        \eSets_{/X}(A,B)%
                        \to%
                        X%
                    \]%
                    defined by
                    \[
                        \phi_{\eSets_{/X}(A,B)}(x,f)%
                        \defeq%
                        x
                    \]%
                    for each $(x,f)\in\eSets_{/X}(A,B)$.
            \end{itemize}
        \item\label{properties-of-pullbacks-of-sets-adjointness-2}\SloganFont{Adjointness \rmII. }We have an adjunction
            \begin{webcompile}
                \Adjunction#{\Delta_{\Sets_{/X}}}#-_{1}\times-_{2}#\Sets_{/X}#\Sets_{/X}\times\Sets_{/X},#\\
            \end{webcompile}%
            witnessed by a bijection
            \[
                \Hom_{\Sets_{/X}\times\Sets_{/X}}((A,A),(B,C))%
                \cong%
                \Sets_{/X}(A,B\times_{X}C),%
            \]%
            natural in $A\in\Obj(\Sets_{/X})$ and in $(B,C)\in\Obj(\Sets_{/X}\times\Sets_{/X})$.
        \item\label{properties-of-pullbacks-of-sets-associativity}\SloganFont{Associativity. }Given a diagram
            \[
                \begin{tikzcd}[row sep={3.5*\the\DL,between origins}, column sep={3.5*\the\DL,between origins}, background color=backgroundColor, ampersand replacement=\&]
                    A
                    \&
                    \&
                    B
                    \&
                    \&
                    C
                    \\
                    \&
                    X
                    \&
                    \&
                    Y
                    \&
                    % 1-Arrows
                    \arrow[from=1-1,to=2-2,"f"']%
                    \arrow[from=1-3,to=2-2,"g"]%
                    %
                    \arrow[from=1-3,to=2-4,"h"']%
                    \arrow[from=1-5,to=2-4,"k"]%
                \end{tikzcd}
            \]%
            in $\Sets$, we have isomorphisms of sets
            \[
                (A\times_{X}B)\times_{Y}C%
                \cong
                (A\times_{X}B)\times_{B}(B\times_{Y}C)
                \cong
                A\times_{X}(B\times_{Y}C),%
            \]%
            where these pullbacks are built as in the diagrams
            \begin{webcompile}
                \resizebox{\textwidth}{!}{$%
                    \begin{tikzcd}[row sep={3.5*\the\DL,between origins}, column sep={3.5*\the\DL,between origins}, background color=backgroundColor, ampersand replacement=\&]
                        \&
                        \&
                        (A\times_{X}B)\times_{Y}C
                        \&
                        \&
                        \\
                        \&
                        A\times_{X}B
                        \&
                        \&
                        \&
                        \\
                        A
                        \&
                        \&
                        B
                        \&
                        \&
                        C\mrp{,}
                        \\
                        \&
                        X
                        \&
                        \&
                        Y
                        \&
                        % 1-Arrows
                        \arrow[from=1-3,to=2-2]%
                        \arrow[from=1-3,to=3-5]%
                        %
                        \arrow[from=2-2,to=3-1]%
                        \arrow[from=2-2,to=3-3]%
                        %
                        \arrow[from=3-1,to=4-2,"f"']%
                        \arrow[from=3-3,to=4-2,"g"]%
                        \arrow[from=3-3,to=4-4,"h"']%
                        \arrow[from=3-5,to=4-4,"k"]%
                        %
                        \arrow[from=1-3,to=3-3,very near start,phantom,"\lrcorner"{rotate=-45}]
                        \arrow[from=2-2,to=4-2,very near start,phantom,"\lrcorner"{rotate=-45}]
                    \end{tikzcd}
                    \qquad
                    \begin{tikzcd}[row sep={3.5*\the\DL,between origins}, column sep={3.5*\the\DL,between origins}, background color=backgroundColor, ampersand replacement=\&]
                        \&
                        \&
                        (A\times_{X}B)\times_{B}(B\times_{Y}C)
                        \&
                        \&
                        \\
                        \&
                        A\times_{X}B
                        \&
                        \&
                        B\times_{Y}C
                        \&
                        \\
                        A
                        \&
                        \&
                        B
                        \&
                        \&
                        C\mrp{,}
                        \\
                        \&
                        X
                        \&
                        \&
                        Y
                        \&
                        % 1-Arrows
                        \arrow[from=1-3,to=2-2]%
                        \arrow[from=1-3,to=2-4]%
                        %
                        \arrow[from=2-2,to=3-1]%
                        \arrow[from=2-2,to=3-3]%
                        \arrow[from=2-4,to=3-3]%
                        \arrow[from=2-4,to=3-5]%
                        %
                        \arrow[from=3-1,to=4-2,"f"']%
                        \arrow[from=3-3,to=4-2,"g"]%
                        \arrow[from=3-3,to=4-4,"h"']%
                        \arrow[from=3-5,to=4-4,"k"]%
                        %
                        \arrow[from=1-3,to=3-3,very near start,phantom,"\lrcorner"{rotate=-45}]
                        \arrow[from=2-2,to=4-2,very near start,phantom,"\lrcorner"{rotate=-45}]
                        \arrow[from=2-4,to=4-4,very near start,phantom,"\lrcorner"{rotate=-45}]
                    \end{tikzcd}
                    \qquad
                    \begin{tikzcd}[row sep={3.5*\the\DL,between origins}, column sep={3.5*\the\DL,between origins}, background color=backgroundColor, ampersand replacement=\&]
                        \&
                        \&
                        A\times_{X}(B\times_{Y}C)
                        \&
                        \&
                        \\
                        \&
                        \&
                        \&
                        B\times_{Y}C
                        \&
                        \\
                        A
                        \&
                        \&
                        B
                        \&
                        \&
                        C\mrp{.}
                        \\
                        \&
                        X
                        \&
                        \&
                        Y
                        \&
                        % 1-Arrows
                        \arrow[from=1-3,to=3-1]%
                        \arrow[from=1-3,to=2-4]%
                        %
                        \arrow[from=2-4,to=3-3]%
                        \arrow[from=2-4,to=3-5]%
                        %
                        \arrow[from=3-1,to=4-2,"f"']%
                        \arrow[from=3-3,to=4-2,"g"]%
                        \arrow[from=3-3,to=4-4,"h"']%
                        \arrow[from=3-5,to=4-4,"k"]%
                        %
                        \arrow[from=1-3,to=3-3,very near start,phantom,"\lrcorner"{rotate=-45}]
                        \arrow[from=2-4,to=4-4,very near start,phantom,"\lrcorner"{rotate=-45}]
                    \end{tikzcd}
                $}%
            \end{webcompile}
        \item\label{properties-of-pullbacks-of-sets-interaction-with-composition}\SloganFont{Interaction With Composition. }Given a diagram
            \[
                \begin{tikzcd}[row sep={3.5*\the\DL,between origins}, column sep={3.5*\the\DL,between origins}, background color=backgroundColor, ampersand replacement=\&]
                    X
                    \&
                    \&
                    \&
                    \&
                    Y
                    \\
                    \&
                    A
                    \&
                    \&
                    B
                    \&
                    \\
                    \&
                    \&
                    K
                    \&
                    \&
                    % 1-Arrows
                    \arrow[from=1-1,to=2-2,"\phi"']%
                    \arrow[from=2-2,to=3-3,"f"']%
                    %
                    \arrow[from=1-5,to=2-4,"\psi"]%
                    \arrow[from=2-4,to=3-3,"g"]%
                \end{tikzcd}
            \]%
            in $\Sets$, we have isomorphisms of sets
            \begin{align*}
                X\times^{f\circ\phi,g\circ\psi}_{K}Y &\cong (X\times^{\phi,q_{1}}_{A}(A\times^{f,g}_{K}B))\times^{p_{2},p_{1}}_{A\times^{f,g}_{K}B}((A\times^{f,g}_{K}B)\times^{q_{2},\psi}_{B}Y)\\
                                                     &\cong X\times^{\phi,p}_{A}((A\times^{f,g}_{K}B)\times^{q_{2},\psi}_{B}Y)\\
                                                     &\cong (X\times^{\phi,q_{1}}_{A}(A\times^{f,g}_{K}B))\times^{q,\psi}_{B}Y
            \end{align*}
            where
            \[
                \begin{aligned}
                    q_{1} &= \pr^{A\times^{f,g}_{K}B}_{1},\\
                    p_{1} &= \pr^{(A\times^{f,g}_{K}B)\times^{q_{2},\psi}_{Y}}_{1},\\
                    p     &= q_{1}\circ\pr^{(A\times^{f,g}_{K}B)\times^{q_{2},\psi}_{B}Y}_{1},
                \end{aligned}
                \qquad
                \begin{aligned}
                    q_{2} &= \pr^{A\times^{f,g}_{K}B}_{2},\\
                    p_{2} &= \pr^{X\times^{\phi,q_{1}}_{A\times^{f,g}_{K}B}(A\times^{f,g}_{K}B)}_{2},\\
                    q     &= q_{2}\circ\pr^{X\times^{\phi,q_{1}}_{A}(A\times^{f,g}_{K}B)}_{2},
                \end{aligned}
            \]%
            and where these pullbacks are built as in the following diagrams:
            \begin{scalemath}
                \begin{tikzcd}[row sep={5.0*\the\DL,between origins}, column sep={5.0*\the\DL,between origins}, background color=backgroundColor, ampersand replacement=\&]
                    \&
                    \&
                    (X\times_{A}(A\times_{K}B))\times_{A\times_{K}B}((A\times_{K}B)\times_{B}Y)
                    \&
                    \&
                    \\
                    \&
                    X\times_{A}(A\times_{K}B)
                    \&
                    \&
                    (A\times_{K}B)\times_{B}Y
                    \&
                    \\
                    X
                    \&
                    \&
                    A\times_{K}B
                    \&
                    \&
                    Y\mrp{,}
                    \\
                    \&
                    A
                    \&
                    \&
                    B
                    \&
                    \\
                    \&
                    \&
                    K
                    \&
                    \&
                    % 1-Arrows
                    %
                    \arrow[from=1-3,to=2-2,two heads]%
                    \arrow[from=1-3,to=2-4,two heads]%
                    %
                    \arrow[from=2-2,to=3-1,two heads]%
                    \arrow[from=2-2,to=3-3,two heads]%
                    \arrow[from=2-4,to=3-3,two heads]%
                    \arrow[from=2-4,to=3-5,two heads]%
                    %
                    \arrow[from=3-1,to=4-2,"\phi"']%
                    \arrow[from=4-2,to=5-3,"f"']%
                    %
                    \arrow[from=3-3,to=4-2,two heads]%
                    \arrow[from=3-3,to=4-4,two heads]%
                    %
                    \arrow[from=3-5,to=4-4,"\psi"]%
                    \arrow[from=4-4,to=5-3,"g"]%
                    %
                    \arrow[from=1-3,to=3-3,very near start,phantom,"\lrcorner"{rotate=-45}]
                    \arrow[from=2-2,to=4-2,very near start,phantom,"\lrcorner"{rotate=-45}]
                    \arrow[from=2-4,to=4-4,very near start,phantom,"\lrcorner"{rotate=-45}]
                    \arrow[from=3-3,to=5-3,very near start,phantom,"\lrcorner"{rotate=-45}]
                \end{tikzcd}
                \quad
                \begin{tikzcd}[row sep={5.0*\the\DL,between origins}, column sep={5.0*\the\DL,between origins}, background color=backgroundColor, ampersand replacement=\&]
                    \&
                    \&
                    X\times_{A}((A\times_{K}B)\times_{B}Y)
                    \&
                    \&
                    \\
                    \&
                    \&
                    \&
                    (A\times_{K}B)\times_{B}Y
                    \&
                    \\
                    X
                    \&
                    \&
                    A\times_{K}B
                    \&
                    \&
                    Y\mrp{,}
                    \\
                    \&
                    A
                    \&
                    \&
                    B
                    \&
                    \\
                    \&
                    \&
                    K
                    \&
                    \&
                    % 1-Arrows
                    %
                    \arrow[from=1-3,to=2-4,two heads]%
                    \arrow[from=1-3,to=3-1,two heads]%
                    %
                    \arrow[from=2-4,to=3-3,two heads]%
                    \arrow[from=2-4,to=3-5,two heads]%
                    %
                    \arrow[from=3-1,to=4-2,"\phi"']%
                    \arrow[from=4-2,to=5-3,"f"']%
                    %
                    \arrow[from=3-3,to=4-2,two heads]%
                    \arrow[from=3-3,to=4-4,two heads]%
                    %
                    \arrow[from=3-5,to=4-4,"\psi"]%
                    \arrow[from=4-4,to=5-3,"g"]%
                    %
                    \arrow[from=1-3,to=3-3,very near start,phantom,"\lrcorner"{rotate=-45}]
                    \arrow[from=2-4,to=4-4,very near start,phantom,"\lrcorner"{rotate=-45}]
                    \arrow[from=3-3,to=5-3,very near start,phantom,"\lrcorner"{rotate=-45}]
                \end{tikzcd}
            \end{scalemath}
            \begin{scalemath}
                \begin{tikzcd}[row sep={5.0*\the\DL,between origins}, column sep={5.0*\the\DL,between origins}, background color=backgroundColor, ampersand replacement=\&]
                    \&
                    \&
                    (X\times_{A}(A\times_{K}B))\times_{B}Y
                    \&
                    \&
                    \\
                    \&
                    X\times_{A}(A\times_{K}B)
                    \&
                    \&
                    \&
                    \\
                    X
                    \&
                    \&
                    A\times_{K}B
                    \&
                    \&
                    Y\mrp{,}
                    \\
                    \&
                    A
                    \&
                    \&
                    B
                    \&
                    \\
                    \&
                    \&
                    K
                    \&
                    \&
                    % 1-Arrows
                    %
                    \arrow[from=1-3,to=2-2,two heads]%
                    \arrow[from=1-3,to=3-5,two heads]%
                    %
                    \arrow[from=2-2,to=3-1,two heads]%
                    \arrow[from=2-2,to=3-3,two heads]%
                    %
                    \arrow[from=3-1,to=4-2,"\phi"']%
                    \arrow[from=4-2,to=5-3,"f"']%
                    %
                    \arrow[from=3-3,to=4-2,two heads]%
                    \arrow[from=3-3,to=4-4,two heads]%
                    %
                    \arrow[from=3-5,to=4-4,"\psi"]%
                    \arrow[from=4-4,to=5-3,"g"]%
                    %
                    \arrow[from=1-3,to=3-3,very near start,phantom,"\lrcorner"{rotate=-45}]
                    \arrow[from=2-2,to=4-2,very near start,phantom,"\lrcorner"{rotate=-45}]
                    \arrow[from=3-3,to=5-3,very near start,phantom,"\lrcorner"{rotate=-45}]
                \end{tikzcd}
                \quad
                \begin{tikzcd}[row sep={5.0*\the\DL,between origins}, column sep={5.0*\the\DL,between origins}, background color=backgroundColor, ampersand replacement=\&]
                    \&
                    \&
                    X\times_{K}Y
                    \&
                    \&
                    \\
                    \&
                    \&
                    \&
                    \&
                    \\
                    X
                    \&
                    \&
                    {}
                    \&
                    \&
                    Y\mrp{.}
                    \\
                    \&
                    A
                    \&
                    \&
                    B
                    \&
                    \\
                    \&
                    \&
                    K
                    \&
                    \&
                    % 1-Arrows
                    %
                    \arrow[from=1-3,to=3-1,two heads]%
                    \arrow[from=1-3,to=3-5,two heads]%
                    %
                    \arrow[from=3-1,to=4-2,"\phi"']%
                    \arrow[from=4-2,to=5-3,"f"']%
                    %
                    \arrow[from=3-5,to=4-4,"\psi"]%
                    \arrow[from=4-4,to=5-3,"g"]%
                    %
                    \arrow[from=1-3,to=3-3,very near start,phantom,"\lrcorner"{rotate=-45}]
                \end{tikzcd}
            \end{scalemath}
        \item\label{properties-of-pullbacks-of-sets-unitality}\SloganFont{Unitality. }We have isomorphisms of sets
            \begin{webcompile}
                \begin{tikzcd}[row sep={5.0*\the\DL,between origins}, column sep={5.0*\the\DL,between origins}, background color=backgroundColor, ampersand replacement=\&]
                    A
                    \arrow[r,Equals]
                    \arrow[d,"f"']
                    \arrow[rd,very near start,phantom,"\lrcorner"]
                    \&
                    A
                    \arrow[d,"f"]
                    \\
                    X
                    \arrow[r,Equals]
                    \&
                    X
                \end{tikzcd}
                \qquad
                \begin{aligned}
                    \LUnitor^{\Sets_{/X}}_{A} &\colon X\times_{X}A \isorightarrow A,\\
                    \RUnitor^{\Sets_{/X}}_{A} &\colon A\times_{X}X \isorightarrow A,
                \end{aligned}
                \qquad
                \begin{tikzcd}[row sep={5.0*\the\DL,between origins}, column sep={5.0*\the\DL,between origins}, background color=backgroundColor, ampersand replacement=\&]
                    A
                    \arrow[r,"f"]
                    \arrow[d,Equals]
                    \arrow[rd,very near start,phantom,"\lrcorner"]
                    \&
                    X
                    \arrow[d,Equals]
                    \\
                    X
                    \arrow[r,"f"']
                    \&
                    X\mrp{,}
                \end{tikzcd}
            \end{webcompile}
            natural in $(A,f)\in\Obj(\Sets_{/X})$.
        \item\label{properties-of-pullbacks-of-sets-commutativity}\SloganFont{Commutativity. }We have an isomorphism of sets
            \begin{webcompile}
                \begin{tikzcd}[row sep={5.0*\the\DL,between origins}, column sep={5.0*\the\DL,between origins}, background color=backgroundColor, ampersand replacement=\&]
                    A\times_{X}B
                    \arrow[r]
                    \arrow[d]
                    \arrow[rd,very near start,phantom,"\lrcorner"]
                    \&
                    B
                    \arrow[d,"g"]
                    \\
                    A
                    \arrow[r,"f"']
                    \&
                    X\mrp{,}
                \end{tikzcd}
                \quad
                \sigma^{\Sets_{/X}}_{A,B}%
                \colon%
                A\times_{X}B%
                \isorightarrow%
                B\times_{X}A%
                \quad
                \begin{tikzcd}[row sep={5.0*\the\DL,between origins}, column sep={5.0*\the\DL,between origins}, background color=backgroundColor, ampersand replacement=\&]
                    B\times_{X}A
                    \arrow[r]
                    \arrow[d]
                    \arrow[rd,very near start,phantom,"\lrcorner"]
                    \&
                    A
                    \arrow[d,"f"]
                    \\
                    B
                    \arrow[r,"g"']
                    \&
                    X\mrp{,}
                \end{tikzcd}
            \end{webcompile}
            natural in $(A,f),(B,g)\in\Obj(\Sets_{/X})$.
        \item\label{properties-of-pullbacks-of-sets-distributivity-over-coproducts}\SloganFont{Distributivity Over Coproducts. }Let $A$, $B$, and $C$ be sets and let $\phi_{A}\colon A\to X$, $\phi_{B}\colon B\to X$, and $\phi_{C}\colon C\to X$ be morphisms of sets. We have isomorphisms of sets
            \begin{align*}
                \delta^{\Sets_{/X}}_{\ell} &\colon A\times_{X}(B\icoprod C) \isorightarrow (A\times_{X}B)\icoprod(A\times_{X}C),\\
                \delta^{\Sets_{/X}}_{r}    &\colon (A\icoprod B)\times_{X}C \isorightarrow (A\times_{X}C)\icoprod(B\times_{X}C),
            \end{align*}
            as in the diagrams
            \begin{scalemath}
                \begin{tikzcd}[row sep={5.0*\the\DL,between origins}, column sep={9.0*\the\DL,between origins}, background color=backgroundColor, ampersand replacement=\&]
                    (A\times_{X}B)\icoprod(A\times_{X}C)
                    \arrow[r,two heads]
                    \arrow[d,two heads]
                    \arrow[rd,very near start,phantom,"\lrcorner"]
                    \&
                    B\icoprod C
                    \arrow[d,"\phi_{B}\icoprod\phi_{C}"]
                    \\
                    A
                    \arrow[r,"\phi_{A}"']
                    \&
                    X
                \end{tikzcd}
                \begin{tikzcd}[row sep={5.0*\the\DL,between origins}, column sep={7.5*\the\DL,between origins}, background color=backgroundColor, ampersand replacement=\&]
                    (A\times_{X}C)\icoprod(B\times_{X}C)
                    \arrow[r,two heads]
                    \arrow[d,two heads]
                    \arrow[rd,very near start,phantom,"\lrcorner"]
                    \&
                    C
                    \arrow[d,"\phi_{C}"]
                    \\
                    A\icoprod B
                    \arrow[r,"\phi_{A}\icoprod\phi_{B}"']
                    \&
                    X
                \end{tikzcd}
            \end{scalemath}
            natural in $A,B,C\in\Obj(\Sets_{/X})$.
        \item\label{properties-of-pullbacks-of-sets-annihilation-with-the-empty-set}\SloganFont{Annihilation With the Empty Set. }We have isomorphisms of sets
            \begin{webcompile}
                \begin{tikzcd}[row sep={5.0*\the\DL,between origins}, column sep={5.0*\the\DL,between origins}, background color=backgroundColor, ampersand replacement=\&]
                    \emptyset
                    \arrow[r]
                    \arrow[d]
                    \arrow[rd,very near start,phantom,"\lrcorner"]
                    \&
                    \emptyset
                    \arrow[d]
                    \\
                    A
                    \arrow[r,"f"']
                    \&
                    X\mrp{,}
                \end{tikzcd}
                \qquad
                \begin{aligned}
                    \zeta^{\Sets_{/X}}_{\ell} &\colon A\times_{X}\emptyset \isorightarrow \emptyset,\\
                    \zeta^{\Sets_{/X}}_{r}    &\colon \emptyset\times_{X}A \isorightarrow \emptyset,
                \end{aligned}
                \qquad
                \begin{tikzcd}[row sep={5.0*\the\DL,between origins}, column sep={5.0*\the\DL,between origins}, background color=backgroundColor, ampersand replacement=\&]
                    \emptyset
                    \arrow[r]
                    \arrow[d]
                    \arrow[rd,very near start,phantom,"\lrcorner"]
                    \&
                    A
                    \arrow[d,"f"]
                    \\
                    \emptyset
                    \arrow[r]
                    \&
                    X\mrp{,}
                \end{tikzcd}
            \end{webcompile}
            natural in $(A,f)\in\Obj(\Sets_{/X})$.
        \item\label{properties-of-pullbacks-of-sets-interaction-with-products}\SloganFont{Interaction With Products. }We have an isomorphism of sets
            \begin{webcompile}
                A\times_{\pt}B%
                \cong%
                A\times B,%
                \quad
                \begin{tikzcd}[row sep={5.0*\the\DL,between origins}, column sep={5.0*\the\DL,between origins}, background color=backgroundColor, ampersand replacement=\&]
                    A\times B
                    \arrow[r]
                    \arrow[d]
                    \arrow[rd,very near start,phantom,"\lrcorner"]
                    \&
                    B
                    \arrow[d,"!_{B}"]
                    \\
                    A
                    \arrow[r,"!_{A}"']
                    \&
                    \pt\mrp{.}
                \end{tikzcd}
            \end{webcompile}
        \item\label{properties-of-pullbacks-of-sets-symmetric-monoidality}\SloganFont{Symmetric Monoidality. }The 8-tuple $\left(\mrp{\phantom{\LUnitor^{\Sets_{/X}}}}\Sets_{/X}\right.$, $\times_{X}$, $X$, $\eSets_{/X}$, $\alpha^{\Sets_{/X}}$, $\LUnitor^{\Sets_{/X}}$, $\RUnitor^{\Sets_{/X}}$, $\left.\sigma^{\Sets_{/X}}\right)$ is a symmetric closed monoidal category.
        %\item\label{properties-of-pullbacks-of-sets-}\SloganFont{. }
    \end{enumerate}
\end{proposition}
\begin{Proof}{Proof of \cref{properties-of-pullbacks-of-sets}}%
    \FirstProofBox{\cref{properties-of-pullbacks-of-sets-functoriality}: Functoriality}%
    This is a special case of functoriality of co/limits, \ChapterRef{\ChapterLimitsAndColimits, \cref{limits-and-colimits:properties-of-co-limits-functoriality} of \cref{limits-and-colimits:properties-of-co-limits}}{\cref{properties-of-co-limits-functoriality} of \cref{properties-of-co-limits}}, with the explicit expression for $\xi$ following from the commutativity of the cube pullback diagram.

    \ProofBox{\cref{properties-of-pullbacks-of-sets-adjointness-1}: Adjointness \rmI}%
    This is a repetition of \ChapterRef{\ChapterFibredSets, \cref{fibred-sets:properties-of-the-internal-hom-of-fibred-sets-adjointness} of \cref{fibred-sets:properties-of-the-internal-hom-of-fibred-sets}}{\cref{properties-of-the-internal-hom-of-fibred-sets-adjointness} of \cref{properties-of-the-internal-hom-of-fibred-sets}}, and is proved there.

    \ProofBox{\cref{properties-of-pullbacks-of-sets-adjointness-2}: Adjointness \rmII}%
    This follows from the universal property of the product (pullbacks are products in $\Sets_{/X}$).

    \ProofBox{\cref{properties-of-pullbacks-of-sets-associativity}: Associativity}%
    We have
    \begin{envsmallsize}
        \begin{align*}
            (A\times_{X}B)\times_{Y}C &\cong \{((a,b),c)\in(A\times_{X}B)\times C\ \middle|\ h(b)=k(c)\}\\
                                      &\cong \{((a,b),c)\in(A\times B)\times C\ \middle|\ \text{$f(a)=g(b)$ and $h(b)=k(c)$}\}\\
                                      &\cong \{(a,(b,c))\in A\times(B\times C)\ \middle|\ \text{$f(a)=g(b)$ and $h(b)=k(c)$}\}\\
                                      &\cong \{(a,(b,c))\in A\times(B\times_{Y}C)\ \middle|\ \text{$f(a)=g(b)$}\}\\
                                      &\cong A\times_{X}(B\times_{Y}C)
        \end{align*}
    \end{envsmallsize}
    and
    \begin{envscriptsize}
        \begin{align*}
            (A\times_{X}B)\times_{B}(B\times_{Y}C) &\cong \{((a,b),(b',c))\in(A\times_{X}B)\times(B\times_{Y}C)\ \middle|\ b=b'\}\\
                                                   &\cong \{((a,b),(b',c))\in(A\times B)\times(B\times C)\ \middle|\ \begin{aligned}&\text{$f(a)=g(b)$, $b=b'$,}\\&\text{and $h(b')=k(c)$}\end{aligned}\}\\
                                                   &\cong \{(a,(b,(b',c)))\in A\times(B\times(B\times C))\ \middle|\ \begin{aligned}&\text{$f(a)=g(b)$, $b=b'$,}\\&\text{and $h(b')=k(c)$}\end{aligned}\}\\
                                                   &\cong \{(a,((b,b'),c))\in A\times((B\times B)\times C)\ \middle|\ \begin{aligned}&\text{$f(a)=g(b)$, $b=b'$,}\\&\text{and $h(b')=k(c)$}\end{aligned}\}\\
                                                   &\cong \{(a,((b,b'),c))\in A\times((B\times_{B}B)\times C)\ \middle|\ \begin{aligned}&\text{$f(a)=g(b)$ and}\\&\text{$h(b')=k(c)$}\end{aligned}\}\\
                                                   &\cong \{(a,(b,c))\in A\times(B\times C)\ \middle|\ \text{$f(a)=g(b)$ and $h(b)=k(c)$}\}\\
                                                   &\cong A\times_{X}(B\times_{Y}C),
        \end{align*}
    \end{envscriptsize}
    where we have used \cref{properties-of-pullbacks-of-sets-unitality} for the isomorphism $B\times_{B}B\cong B$.

    \ProofBox{\cref{properties-of-pullbacks-of-sets-interaction-with-composition}: Interaction With Composition}%
    By \cref{properties-of-pullbacks-of-sets-associativity}, it suffices to construct only the isomorphism
    \[
        X\times^{f\circ\phi,g\circ\psi}_{K}Y%
        \cong%
        (X\times^{\phi,q_{1}}_{A}(A\times^{f,g}_{K}B))\times^{p_{2},p_{1}}_{A\times^{f,g}_{K}B}((A\times^{f,g}_{K}B)\times^{q_{2},\psi}_{B}Y).%
    \]%
    We have
    \begin{envfootnotesize}
        \begin{align*}
            (X\times^{\phi,q_{1}}_{A}(A\times^{f,g}_{K}B)) &\eqdef \{(x,(a,b))\in X\times(A\times^{f,g}_{K}B)\ \middle|\ \phi(x)=q_{1}(a,b)\}\\
                                                           &\eqdef \{(x,(a,b))\in X\times(A\times^{f,g}_{K}B)\ \middle|\ \phi(x)=a\}\\
                                                           &\cong  \{(x,(a,b))\in X\times(A\times B)\ \middle|\ \text{$\phi(x)=a$ and $f(a)=g(b)$}\},\\
                                                           %&\cong  \{(x,b)\in X\times B\ \middle|\ f(\phi(x))=g(b)\},\\
            ((A\times^{f,g}_{K}B)\times^{q_{2},\psi}_{B}Y) &\eqdef \{((a,b),y)\in(A\times^{f,g}_{K}B)\times Y\ \middle|\ q_{2}(a,b)=\psi(y)\}\\
                                                           &\eqdef \{((a,b),y)\in(A\times^{f,g}_{K}B)\times Y\ \middle|\ b=\psi(y)\}\\
                                                           &\cong  \{((a,b),y)\in(A\times B)\times Y\ \middle|\ \text{$b=\psi(y)$ and $f(a)=g(b)$}\},%\\
                                                           %&\cong  \{(a,y)\in A\times Y\ \middle|\ f(a)=g(\psi(y))\},
        \end{align*}
    \end{envfootnotesize}
    so writing
    \begin{align*}
        S  &= (X\times^{\phi,q_{1}}_{A}(A\times^{f,g}_{K}B))\\
        S' &= ((A\times^{f,g}_{K}B)\times^{q_{2},\psi}_{B}Y),
    \end{align*}
    we have
    \begin{envfootnotesize}
        \begin{align*}
            S\times^{p_{2},p_{1}}_{A\times^{f,g}_{K}B}S' &\eqdef \{((x,(a,b)),((a',b'),y))\in S\times S'\ \middle|\ p_{1}(x,(a,b))=p_{2}((a',b'),y)\}\\
                                                         &\eqdef \{((x,(a,b)),((a',b'),y))\in S\times S'\ \middle|\ (a,b)=(a',b')\}\\
                                                         &\cong  \{((x,a,b,y))\in X\times A\times B\times Y\ \middle|\ \text{$\phi(x)=a$, $\psi(y)=b$, and $f(a)=g(b)$}\}\\
                                                         &\cong  \{((x,a,b,y))\in X\times A\times B\times Y\ \middle|\ f(\phi(x))=g(\psi(y))\}\\
                                                         &\eqdef X\times_{K}Y.
        \end{align*}
    \end{envfootnotesize}
    This finishes the proof.

    \ProofBox{\cref{properties-of-pullbacks-of-sets-unitality}: Unitality}%
    We have
    \begin{align*}
        X\times_{X}A &\cong \{(x,a)\in X\times A\ \middle|\ f(a)=x\},\\
        A\times_{X}X &\cong \{(a,x)\in X\times A\ \middle|\ f(a)=x\},
    \end{align*}
    which are isomorphic to $A$ via the maps $(x,a)\mapsto a$ and $(a,x)\mapsto a$. The proof of the naturality of $\LUnitor^{\Sets_{/X}}$ and $\RUnitor^{\Sets_{/X}}$ is omitted.

    \ProofBox{\cref{properties-of-pullbacks-of-sets-commutativity}: Commutativity}%
    We have
    \begin{align*}
        A\times_{C}B &\eqdef \{(a,b)\in A\times B\ \middle|\ f(a)=g(b)\}\\
                     &=      \{(a,b)\in A\times B\ \middle|\ g(b)=f(a)\}\\
                     &\cong  \{(b,a)\in B\times A\ \middle|\ g(b)=f(a)\}\\
                     &\eqdef B\times_{C}A.
    \end{align*}
    The proof of the naturality of $\sigma^{\Sets_{/X}}$ is omitted.

    \ProofBox{\cref{properties-of-pullbacks-of-sets-distributivity-over-coproducts}: Distributivity Over Coproducts}%
    We have
    \begin{envsmallsize}
        \begin{align*}
            A\times_{X}(B\icoprod C) &\eqdef        \{(a,z)\in A\times(B\icoprod C)\ \middle|\ \phi_{A}(a)=\phi_{B\icoprod C}(z)\}\\
                                     &=             \{(a,z)\in A\times(B\icoprod C)\ \middle|\ \text{$z=(0,b)$ and $\phi_{A}(a)=\phi_{B\icoprod C}(z)$}\}\\
                                     &\phantom{={}} \mkern4mu\cup\{(a,z)\in A\times(B\icoprod C)\ \middle|\ \text{$z=(1,c)$ and $\phi_{A}(a)=\phi_{B\icoprod C}(z)$}\}\\
                                     &=             \{(a,z)\in A\times(B\icoprod C)\ \middle|\ \text{$z=(0,b)$ and $\phi_{A}(a)=\phi_{B}(b)$}\}\\
                                     &\phantom{={}} \mkern4mu\cup\{(a,z)\in A\times(B\icoprod C)\ \middle|\ \text{$z=(1,c)$ and $\phi_{A}(a)=\phi_{C}(c)$}\}\\
                                     &\cong         \{(a,b)\in A\times B\ \middle|\ \phi_{A}(a)=\phi_{B}(b)\}\\
                                     &\phantom{={}} \mkern4mu\cup\{(a,c)\in A\times C\ \middle|\ \phi_{A}(a)=\phi_{C}(c)\}\\
                                     &\eqdef        (A\times_{X}B)\cup(A\times_{X}C)\\
                                     &\cong         (A\times_{X}B)\icoprod(A\times_{X}C),
        \end{align*}
    \end{envsmallsize}
    with the construction of the isomorphism
    \[
        \delta^{\Sets_{/X}}_{r}
        \colon
        (A\icoprod B)\times_{X}C
        \isorightarrow
        (A\times_{X}C)\icoprod(B\times_{X}C)
    \]%
    being similar. The proof of the naturality of $\delta^{\Sets_{/X}}_{\ell}$ and $\delta^{\Sets_{/X}}_{r}$ is omitted.

    \ProofBox{\cref{properties-of-pullbacks-of-sets-annihilation-with-the-empty-set}: Annihilation With the Empty Set}%
    We have
    \begin{align*}
        A\times_{X}\emptyset &\eqdef \{(a,b)\in A\times\emptyset\ \middle|\ f(a)=g(b)\}\\
                             &=      \{k\in\emptyset\ \middle|\ f(a)=g(b)\}\\
                             &=      \emptyset,
    \end{align*}
    and similarly for $\emptyset\times_{X}A$, where we have used \cref{properties-of-products-of-sets-annihilation-with-the-empty-set} of \cref{properties-of-products-of-sets}. The proof of the naturality of $\zeta^{\Sets_{/X}}_{\ell}$ and $\zeta^{\Sets_{/X}}_{r}$ is omitted.

    \ProofBox{\cref{properties-of-pullbacks-of-sets-interaction-with-products}: Interaction With Products}%
    We have
    \begin{align*}
        A\times_{\pt}B &\eqdef \{(a,b)\in A\times B\ \middle|\ \mathord{!}_{A}(a)\mathbin{=}\mathord{!}_{B}(b)\}\\
                       &\eqdef \{(a,b)\in A\times B\ \middle|\ \point=\point\}\\
                       &=      \{(a,b)\in A\times B\}\\
                       &=      A\times B.
    \end{align*}

    \ProofBox{\cref{properties-of-pullbacks-of-sets-symmetric-monoidality}: Symmetric Monoidality}%
    Omitted.
\end{Proof}
\subsection{Equalisers}\label{subsection-limits-of-sets-equalisers}
Let $A$ and $B$ be sets and let $f,g\colon A\rightrightarrows B$ be functions.
\begin{definition}{Equalisers of Sets}{equalisers-of-sets}%
    The \index[set-theory]{equaliser of sets}\textbf{equaliser of $f$ and $g$} is the equaliser of $f$ and $g$ in $\Sets$ as in \ChapterRef{\ChapterLimitsAndColimits, \cref{limits-and-colimits:equalisers}}{\cref{equalisers}}.
\end{definition}
\begin{construction}{Construction of Equalisers of Sets}{construction-of-equalisers-of-sets}%
    Concretely, the equaliser of $f$ and $g$ is the pair \index[notation]{Eqfg@$\Eq(f,g)$}\index[notation]{eqfg@$\eq(f,g)$}$(\Eq(f,g),\eq(f,g))$ consisting of:
    \begin{enumerate}
        \item\label{construction-of-equalisers-of-sets-the-limit}\SloganFont{The Limit. }The set $\Eq(f,g)$ given by
            \[
                \Eq(f,g)%
                =%
                \{a\in A\ \middle|\ f(a)=g(a)\}.
            \]%
        \item\label{construction-of-equalisers-of-sets-the-cone}\SloganFont{The Cone. }The inclusion map
            \[
                \eq(f,g)%
                \colon%
                \Eq(f,g)%
                \hookrightarrow%
                A.%
            \]%
    \end{enumerate}
\end{construction}
\begin{Proof}{Proof of \cref{construction-of-equalisers-of-sets}}%
    We claim that $\Eq(f,g)$ is the categorical equaliser of $f$ and $g$ in $\Sets$. First we need to check that the relevant equaliser diagram commutes, i.e.\ that we have
    \[
        f\circ\eq(f,g)%
        =%
        g\circ\eq(f,g),%
    \]%
    which indeed holds by the definition of the set $\Eq(f,g)$. Next, we prove that $\Eq(f,g)$ satisfies the universal property of the equaliser. Suppose we have a diagram of the form
    \[
        \begin{tikzcd}[row sep={4.0*\the\DL,between origins}, column sep={4.0*\the\DL,between origins}, background color=backgroundColor, ampersand replacement=\&]
            {\Eq(f,g)}
            \arrow[r,"{\eq(f,g)}",hook]
            \&[3.0*\the\DL]
            A
            \arrow[r,"f", shift left =0.8]
            \arrow[r,"g"',shift right=0.8]
            \&
            B
            \\
            E
            \arrow[ru,"e"']
            \&[3.0*\the\DL]
            \&
            \&
        \end{tikzcd}
    \]%
    in $\Sets$. Then there exists a unique map $\phi\colon E\to\Eq(f,g)$ making the diagram
    \[
        \begin{tikzcd}[row sep={4.0*\the\DL,between origins}, column sep={4.0*\the\DL,between origins}, background color=backgroundColor, ampersand replacement=\&]
            {\Eq(f,g)}
            \arrow[r,"{\eq(f,g)}",hook]
            \&[3.0*\the\DL]
            A
            \arrow[r,"f", shift left =0.8]
            \arrow[r,"g"',shift right=0.8]
            \&
            B
            \\
            E
            \arrow[ru,"e"']
            \arrow[u,"\phi","\exists!"',dashed]
            \&[3.0*\the\DL]
            \&
            \&
        \end{tikzcd}
    \]%
    commute, being uniquely determined by the condition%
    \[
        \eq(f,g)\circ\phi%
        =%
        e%
    \]%
    via
    \[
        \phi(x)%
        =%
        e(x)
    \]%
    for each $x\in E$, where we note that $e(x)\in A$ indeed lies in $\Eq(f,g)$ by the condition
    \[
        f\circ e%
        =%
        g\circ e,%
    \]%
    which gives
    \[
        f(e(x))%
        =%
        g(e(x))%
    \]%
    for each $x\in E$, so that $e(x)\in\Eq(f,g)$.
\end{Proof}
\begin{proposition}{Properties of Equalisers of Sets}{properties-of-equalisers-of-sets}%
    Let $A$, $B$, and $C$ be sets.
    \begin{enumerate}
        \item\label{properties-of-equalisers-of-sets-associativity}\SloganFont{Associativity. }We have isomorphisms of sets%
            %--- Begin Footnote ---%
            \footnote{%
                That is, the following three ways of forming \say{the} equaliser of $(f,g,h)$ agree:
                \begin{enumerate}
                    \item\label{footnote-properties-of-equalisers-of-sets-associativity-1}Take the equaliser of $(f,g,h)$, i.e.\ the limit of the diagram
                        \[
                            \begin{tikzcd}[row sep={5.0*\the\DL,between origins}, column sep={3.5*\the\DL,between origins}, background color=backgroundColor, ampersand replacement=\&]
                                A%
                                \arrow[r,"f",shift left=2.25]%
                                \arrow[r,"g"description]%
                                \arrow[r,"h"',shift right=2.25]%
                                \&
                                B%
                            \end{tikzcd}
                        \]%
                        in $\Sets$.
                    \item\label{footnote-properties-of-equalisers-of-sets-associativity-2}First take the equaliser of $f$ and $g$, forming a diagram
                        \[
                            \Eq(f,g)%
                            \xlonghookrightarrow{\eq(f,g)}%
                            A%
                            \xlongrightrightarrows{f}{g}%
                            B%
                        \]%
                        and then take the equaliser of the composition
                        \[
                            \Eq(f,g)%
                            \xlonghookrightarrow{\eq(f,g)}%
                            A%
                            \xlongrightrightarrows{f}{h}%
                            B,%
                        \]%
                        obtaining a subset%
                        \[%
                            \Eq(f\circ\eq(f,g),h\circ\eq(f,g))%
                            =%
                            \Eq(g\circ\eq(f,g),h\circ\eq(f,g))%
                        \]%
                        of $\Eq(f,g)$.
                    \item\label{footnote-properties-of-equalisers-of-sets-associativity-3}First take the equaliser of $g$ and $h$, forming a diagram
                        \[
                            \Eq(g,h)%
                            \xlonghookrightarrow{\eq(g,h)}%
                            A%
                            \xlongrightrightarrows{g}{h}%
                            B%
                        \]%
                        and then take the equaliser of the composition
                        \[
                            \Eq(g,h)%
                            \xlonghookrightarrow{\eq(g,h)}%
                            A%
                            \xlongrightrightarrows{f}{g}%
                            B,%
                        \]%
                        obtaining a subset%
                        \[%
                            \Eq(f\circ\eq(g,h),g\circ\eq(g,h))%
                            =%
                            \Eq(f\circ\eq(g,h),h\circ\eq(g,h))%
                        \]%
                        of $\Eq(g,h)$.
                \end{enumerate}
                \par\vspace*{\TCBBoxCorrection}
            }%
            %---  End Footnote  ---%
            \begin{envsmallsize}
                \[
                    \underbrace{\Eq(f\circ\eq(g,h),g\circ\eq(g,h))}_{{}=\Eq(f\circ\eq(g,h),h\circ\eq(g,h))}%
                    \cong
                    \Eq(f,g,h)
                    \cong
                    \underbrace{\Eq(f\circ\eq(f,g),h\circ\eq(f,g))}_{{}=\Eq(g\circ\eq(f,g),h\circ\eq(f,g))},%
                \]%
            \end{envsmallsize}
            where $\Eq(f,g,h)$ is the limit of the diagram
            \[
                \begin{tikzcd}[row sep={5.0*\the\DL,between origins}, column sep={4.0*\the\DL,between origins}, background color=backgroundColor, ampersand replacement=\&]
                    A%
                    \arrow[r,"f",shift left=2.2]%
                    \arrow[r,"g"description]%
                    \arrow[r,"h"',shift right=2.2]%
                    \&
                    B%
                \end{tikzcd}
            \]%
            in $\Sets$, being explicitly given by
            \[
                \Eq(f,g,h)%
                \cong%
                \{a\in A\ \middle|\ f(a)=g(a)=h(a)\}.%
            \]%
        \item\label{properties-of-equalisers-of-sets-unitality}\SloganFont{Unitality. }We have an isomorphism of sets
            \[
                \Eq(f,f)%
                \cong%
                A.%
            \]%
        \item\label{properties-of-equalisers-of-sets-commutativity}\SloganFont{Commutativity. }We have an isomorphism of sets
            \[
                \Eq(f,g)
                \cong
                \Eq(g,f).
            \]%
        \item\label{properties-of-equalisers-of-sets-interaction-with-composition}\SloganFont{Interaction With Composition. }Let
            \[
                A
                \xlongrightrightarrows{f}{g}
                B
                \xlongrightrightarrows{h}{k}
                C
            \]%
            be functions. We have an inclusion of sets
            \[
                \Eq(h\circ f\circ\eq(f,g),k\circ g\circ\eq(f,g))
                \subset
                \Eq(h\circ f,k\circ g),
            \]%
            where $\Eq(h\circ f\circ\eq(f,g),k\circ g\circ\eq(f,g))$ is the equaliser of the composition
            \[
                \Eq(f,g)%
                \xlonghookrightarrow{\eq(f,g)}%
                A%
                \xlongrightrightarrows{f}{g}%
                B%
                \xlongrightrightarrows{h}{k}%
                C.%
            \]%
        %\item\label{properties-of-equalisers-of-sets-}\SloganFont{. }
    \end{enumerate}
\end{proposition}
\begin{Proof}{Proof of \cref{properties-of-equalisers-of-sets}}%
    \FirstProofBox{\cref{properties-of-equalisers-of-sets-associativity}: Associativity}%
    We first prove that $\Eq(f,g,h)$ is indeed given by
    \[
        \Eq(f,g,h)%
        \cong%
        \{a\in A\ \middle|\ f(a)=g(a)=h(a)\}.%
    \]%
    Indeed, suppose we have a diagram of the form
    \[
        \begin{tikzcd}[row sep={4.0*\the\DL,between origins}, column sep={4.0*\the\DL,between origins}, background color=backgroundColor, ampersand replacement=\&]
            {\Eq(f,g,h)}
            \arrow[r,"{\eq(f,g,h)}",hook]
            \&[3.5*\the\DL]
            A
            \arrow[r,"f", shift left =2.0]
            \arrow[r,"g"description]
            \arrow[r,"h"',shift right=2.0]
            \&
            B
            \\
            E
            \arrow[ru,"e"']
            \&[3.5*\the\DL]
            \&
            \&
        \end{tikzcd}
    \]%
    in $\Sets$. Then there exists a unique map $\phi\colon E\to\Eq(f,g,h)$, uniquely determined by the condition%
    \[
        \eq(f,g)\circ\phi%
        =%
        e%
    \]%
    being necessarily given by
    \[
        \phi(x)%
        =%
        e(x)
    \]%
    for each $x\in E$, where we note that $e(x)\in A$ indeed lies in $\Eq(f,g,h)$ by the condition
    \[
        f\circ e%
        =%
        g\circ e%
        =%
        h\circ e,%
    \]%
    which gives
    \[
        f(e(x))%
        =%
        g(e(x))%
        =%
        h(e(x))%
    \]%
    for each $x\in E$, so that $e(x)\in\Eq(f,g,h)$.

    We now check the equalities
    \[
        \Eq(f\circ\eq(g,h),g\circ\eq(g,h))%
        \cong
        \Eq(f,g,h)
        \cong
        \Eq(f\circ\eq(f,g),h\circ\eq(f,g)).%
    \]%
    Indeed, we have
    \begin{envfootnotesize}
        \begin{align*}
            \Eq(f\circ\eq(g,h),g\circ\eq(g,h)) &\cong \{x\in\Eq(g,h)\ \middle|\ [f\circ\eq(g,h)](a)=[g\circ\eq(g,h)](a)\}\\%
                                               &\cong \{x\in\Eq(g,h)\ \middle|\ f(a)=g(a)\}\\%
                                               &\cong \{x\in A\ \middle|\ \text{$f(a)=g(a)$ and $g(a)=h(a)$}\}\\%
                                               &\cong \{x\in A\ \middle|\ \text{$f(a)=g(a)=h(a)$}\}\\%
                                               &\cong \Eq(f,g,h).%
        \end{align*}
    \end{envfootnotesize}
    Similarly, we have
    \begin{envfootnotesize}
        \begin{align*}
            \Eq(f\circ\eq(f,g),h\circ\eq(f,g)) &\cong \{x\in\Eq(f,g)\ \middle|\ [f\circ\eq(f,g)](a)=[h\circ\eq(f,g)](a)\}\\%
                                               &\cong \{x\in\Eq(f,g)\ \middle|\ f(a)=h(a)\}\\%
                                               &\cong \{x\in A\ \middle|\ \text{$f(a)=h(a)$ and $f(a)=g(a)$}\}\\%
                                               &\cong \{x\in A\ \middle|\ \text{$f(a)=g(a)=h(a)$}\}\\%
                                               &\cong \Eq(f,g,h).%
        \end{align*}
    \end{envfootnotesize}

    \ProofBox{\cref{properties-of-equalisers-of-sets-unitality}: Unitality}%
    Indeed, we have
    \begin{align*}
        \Eq(f,f) &\eqdef \{a\in A\ \middle|\ f(a)=f(a)\}\\%
                 &=      A.
    \end{align*}

    \ProofBox{\cref{properties-of-equalisers-of-sets-commutativity}: Commutativity}%
    Indeed, we have
    \begin{align*}
        \Eq(f,g) &\eqdef \{a\in A\ \middle|\ f(a)=g(a)\}\\%
                 &=      \{a\in A\ \middle|\ g(a)=f(a)\}\\%
                 &\eqdef \Eq(g,f).
    \end{align*}

    \ProofBox{\cref{properties-of-equalisers-of-sets-interaction-with-composition}: Interaction With Composition}%
    Indeed, we have
    \begin{envfootnotesize}
        \begin{align*}
            \Eq(h\circ f\circ\eq(f,g),k\circ g\circ\eq(f,g)) &\cong \{a\in\Eq(f,g)\ \middle|\ h(f(a))=k(g(a))\}\\%
                                                             &\cong \{a\in A\ \middle|\ \text{$f(a)=g(a)$ and $h(f(a))=k(g(a))$}\}.%
        \end{align*}
    \end{envfootnotesize}
    and
    \[
        \Eq(h\circ f,k\circ g)%
        \cong%
        \{a\in A\ \middle|\ h(f(a))=k(g(a))\},%
    \]%
    and thus there's an inclusion from $\Eq(h\circ f\circ\eq(f,g),k\circ g\circ\eq(f,g))$ to $\Eq(h\circ f,k\circ g)$.
\end{Proof}
\subsection{Inverse Limits}\label{subsection-limits-of-sets-inverse-limits}
Let $(X_{\alpha},f_{\alpha\beta})_{\alpha,\beta\in I}\colon(I,\preceq)\to\Sets$ be an inverse system of sets.
\begin{definition}{Inverse Limits of Sets}{inverse-limits-of-sets}%
    The \index[set-theory]{inverse limit of sets}\textbf{inverse limit of $(X_{\alpha},f_{\alpha\beta})_{\alpha,\beta\in I}$} is the inverse limit of $(X_{\alpha},f_{\alpha\beta})_{\alpha,\beta\in I}$ in $\Sets$ as in \ChapterRef{\ChapterLimitsAndColimits, \cref{limits-and-colimits:inverse-limits}}{\cref{inverse-limits}}.
\end{definition}
\begin{construction}{Construction of Inverse Limits of Sets}{construction-of-inverse-limits-of-sets}%
    Concretely, the inverse limit of $(X_{\alpha},f_{\alpha\beta})_{\alpha,\beta\in I}$ is the pair \index[notation]{limalphainIXalpha@$\invlim_{\alpha\in I}(X_{\alpha})$}$\smash{\Big(\displaystyle\invlim_{\alpha\in I}(X_{\alpha})}$, $\smash{\{\pr_{\alpha}\}_{\alpha\in I}\Big)}$ consisting of:
    \begin{enumerate}
        \item\label{construction-of-inverse-limits-of-sets-the-limit}\SloganFont{The Limit. }The set $\displaystyle\invlim_{\alpha\in I}(X_{\alpha})$ given by
            \[
                \invlim_{\alpha\in I}(X_{\alpha})%
                =%
                \{%
                    (x_{\alpha})_{\alpha\in I}\in\prod_{\alpha\in I}X_{\alpha}%
                    \ \middle|\ %
                    \begin{aligned}
                        &\text{for each $\alpha,\beta\in I$, if $\alpha\preceq\beta$,}\\
                        &\text{then we have $x_{\alpha}=f_{\alpha\beta}(x_{\beta})$}
                    \end{aligned}
                \}.%
            \]%
        \item\label{construction-of-inverse-limits-of-sets-the-cone}\SloganFont{The Cone.}The collection
            \[
                \{%
                    \pr_{\gamma}%
                    \colon%
                    \invlim_{\alpha\in I}(X_{\alpha})%
                    \to%
                    X_{\gamma}%
                \}_{\gamma\in I}%
            \]%
            of maps of sets defined as the restriction of the maps
            \[
                \{%
                    \pr_{\gamma}%
                    \colon%
                    \prod_{\alpha\in I}X_{\alpha}%
                    \to%
                    X_{\gamma}%
                \}_{\gamma\in I}%
            \]%
            of \cref{construction-of-the-product-of-a-family-of-sets-the-cone} of \cref{construction-of-the-product-of-a-family-of-sets} to $\displaystyle\invlim_{\alpha\in I}(X_{\alpha})$ and hence given by
            \[
                \pr_{\gamma}((x_{\alpha})_{\alpha\in I})%
                =%
                x_{\gamma}%
            \]%
            for each $\gamma\in I$ and each $(x_{\alpha})_{\alpha\in I}\in\displaystyle\invlim_{\alpha\in I}(X_{\alpha})$.%
    \end{enumerate}
\end{construction}
\begin{Proof}{Proof of \cref{construction-of-inverse-limits-of-sets}}%
    We claim that $\invlim_{\alpha\in I}(X_{\alpha})$ is the limit of the inverse system of sets $(X_{\alpha},f_{\alpha\beta})_{\alpha,\beta\in I}$. First we need to check that the limit diagram defined by it commutes, i.e.\ that we have
    \begin{webcompile}
        f_{\alpha\beta}\circ\pr_{\alpha}%
        =%
        \pr_{\beta},%
        \quad%
        \begin{tikzcd}[row sep={5.0*\the\DL,between origins}, column sep={4.0*\the\DL,between origins}, background color=backgroundColor, ampersand replacement=\&]
            \&
            \displaystyle\invlim_{\alpha\in I}(X_{\alpha})
            \arrow[ld,"\pr_{\alpha}"']
            \arrow[rd,"\pr_{\beta}"]
            \&
            \\
            X_{\alpha}
            \arrow[rr,"f_{\alpha\beta}"']
            \&
            \&
            X_{\beta}
        \end{tikzcd}
    \end{webcompile}
    for each $\alpha,\beta\in I$ with $\alpha\preceq\beta$. Indeed, given $(x_{\gamma})_{\gamma\in I}\in\invlim_{\gamma\in I}(X_{\gamma})$, we have
    \begin{align*}
        [f_{\alpha\beta}\circ\pr_{\alpha}]((x_{\gamma})_{\gamma\in I}) &\eqdef f_{\alpha\beta}(\pr_{\alpha}((x_{\gamma})_{\gamma\in I}))\\
                                                                       &\eqdef f_{\alpha\beta}(x_{\alpha})\\
                                                                       &=      x_{\beta}\\
                                                                       &\eqdef \pr_{\beta}((x_{\gamma})_{\gamma\in I}),
    \end{align*}
    where the third equality comes from the definition of $\invlim_{\alpha\in I}(X_{\alpha})$. Next, we prove that $\invlim_{\alpha\in I}(X_{\alpha})$ satisfies the universal property of an inverse limit. Suppose that we have, for each $\alpha,\beta\in I$ with $\alpha\preceq\beta$, a diagram of the form
    \[
        \begin{tikzcd}[row sep={5.0*\the\DL,between origins}, column sep={4.0*\the\DL,between origins}, background color=backgroundColor, ampersand replacement=\&]
            \&
            L
            \arrow[ldd,"p_{\alpha}"',bend right=20]
            \arrow[rdd,"p_{\beta}",bend left=20]
            \&
            \\
            \&
            \displaystyle\invlim_{\alpha\in I}(X_{\alpha})
            \arrow[ld,"\pr_{\alpha}"description,pos=0.45]
            \arrow[rd,"\pr_{\beta}"'description,pos=0.45]
            \&
            \\[0.75*\the\DL]
            X_{\alpha}
            \arrow[rr,"f_{\alpha\beta}"']
            \&
            \&
            X_{\beta}
        \end{tikzcd}
    \]%
    in $\Sets$. Then there indeed exists a unique map $\phi\colon L\uearrow\smash{\displaystyle\invlim_{\alpha\in I}(X_{\alpha})}$ making the diagram
    \[
        \begin{tikzcd}[row sep={5.0*\the\DL,between origins}, column sep={4.0*\the\DL,between origins}, background color=backgroundColor, ampersand replacement=\&]
            \&
            L
            \arrow[ldd,"p_{\alpha}"',bend right=20]
            \arrow[rdd,"p_{\beta}",bend left=20]
            \arrow[d,"\phi"',dashed]
            \arrow[d,"\exists!",dashed]
            \&
            \\
            \&
            \displaystyle\invlim_{\alpha\in I}(X_{\alpha})
            \arrow[ld,"\pr_{\alpha}"description,pos=0.45]
            \arrow[rd,"\pr_{\beta}"'description,pos=0.45]
            \&
            \\[0.75*\the\DL]
            X_{\alpha}
            \arrow[rr,"f_{\alpha\beta}"']
            \&
            \&
            X_{\beta}
        \end{tikzcd}
    \]%
    commute, being uniquely determined by the family of conditions%
    \[
        \{%
            p_{\alpha}%
            =%
            \pr_{\alpha}\circ\phi%
        \}_{\alpha\in I}%
    \]%
    via
    \[
        \phi(\ell)%
        =%
        (p_{\alpha}(\ell))_{\alpha\in I}
    \]%
    for each $\ell\in L$, where we note that $(p_{\alpha}(\ell))_{\alpha\in I}\in\prod_{\alpha\in I}X_{\alpha}$ indeed lies in $\invlim_{\alpha\in I}(X_{\alpha})$, as we have
    \begin{align*}
        f_{\alpha\beta}(p_{\alpha}(\ell)) &\eqdef [f_{\alpha\beta}\circ p_{\alpha}](\ell)\\
                                          &\eqdef p_{\beta}(\ell)
    \end{align*}
    for each $\beta\in I$ with $\alpha\preceq\beta$ by the commutativity of the diagram for $(L,\{p_{\alpha}\}_{\alpha\in I})$.
\end{Proof}
\begin{example}{Examples of Inverse Limits of Sets}{examples-of-inverse-limits-of-sets}%
    Here are some examples of inverse limits of sets.
    \begin{enumerate}
        \item\label{examples-of-inverse-limits-of-sets-the-p-adic-integers}\SloganFont{The $p$-Adic Integers. }The ring of $p$-adic integers $\Z_{p}$ of \cref{TODO4} is the inverse limit
            \[
                \Z_{p}%
                \cong%
                \invlim_{n\in\N}(\Zn{p^{n}});%
            \]%
            see \cref{TODO5}.
        \item\label{examples-of-inverse-limits-of-sets-rings-of-formal-power-series}\SloganFont{Rings of Formal Power Series. }The ring $R\llbracket t\rrbracket$ of formal power series in a variable $t$ is the inverse limit
            \[
                R\llbracket t\rrbracket%
                \cong%
                \invlim_{n\in\N}(R[t]/t^{n}R[t]);%
            \]%
            see \cref{TODO6}.
        \item\label{examples-of-inverse-limits-of-sets-profinite-groups}\SloganFont{Profinite Groups. }Profinite groups are inverse limits of finite groups; see \cref{TODO7}.
        %\item\label{examples-of-inverse-limits-of-sets-}\SloganFont{. }
    \end{enumerate}
\end{example}
\section{Colimits of Sets}\label{section-colimits-of-sets}
\subsection{The Initial Set}\label{subsection-the-initial-set}
\begin{definition}{The Initial Set}{the-initial-set}%
    The \index[set-theory]{initial set}\textbf{initial set} is the initial object of $\Sets$ as in \ChapterRef{\ChapterLimitsAndColimits, \cref{limits-and-colimits:initial-objects}}{\cref{initial-objects}}.
\end{definition}
\begin{construction}{Construction of the Initial Set}{construction-of-the-initial-set}%
    Concretely, the initial set is the pair \index[notation]{emptyset@$\emptyset$}$\smash{(\emptyset,\{\iota_{A}\}_{A\in\Obj(\Sets)})}$ consisting of:
    \begin{enumerate}
        \item\label{construction-of-the-initial-set-the-colimit}\SloganFont{The Colimit. }The empty set $\emptyset$ of \cref{the-empty-set}.
        \item\label{construction-of-the-initial-set-the-cocone}\SloganFont{The Cocone. }The collection of maps
            \[
                \{%
                    \iota_{A}%
                    \colon%
                    \emptyset%
                    \to%
                    A%
                \}_{A\in\Obj(\Sets)}%
            \]%
            given by the inclusion maps from $\emptyset$ to $A$.
    \end{enumerate}
\end{construction}
\begin{Proof}{Proof of \cref{construction-of-the-initial-set}}%
    We claim that $\emptyset$ is the initial object of $\Sets$. Indeed, suppose we have a diagram of the form
    \[
        \begin{tikzcd}[row sep={3.0*\the\DL,between origins}, column sep={4.0*\the\DL,between origins}, background color=backgroundColor, ampersand replacement=\&]
            \emptyset
            \&
            A
        \end{tikzcd}
    \]%
    in $\Sets$. Then there exists a unique map $\phi\colon\emptyset\to A$ making the diagram
    \[
        \begin{tikzcd}[row sep={3.0*\the\DL,between origins}, column sep={4.0*\the\DL,between origins}, background color=backgroundColor, ampersand replacement=\&]
            \emptyset
            \arrow[r,"\phi"{pos=0.5},"\exists!"'{pos=0.5}, dashed]
            \&
            A
        \end{tikzcd}
    \]%
    commute, namely the inclusion map $\iota_{A}$.
\end{Proof}
\subsection{Coproducts of Families of Sets}\label{subsection-coproducts-of-families-of-sets}
Let $\{A_{i}\}_{i\in I}$ be a family of sets.%
\begin{definition}{The Coproduct of a Family of Sets}{the-coproduct-of-a-family-of-sets}%
    The \index[set-theory]{coproduct!of a family of sets}\textbf{coproduct of $\{A_{i}\}_{i\in I}$}%
    %--- Begin Footnote ---%
    \footnote{%
        \SloganFont{Further Terminology: }Also called the \index[set-theory]{disjoint union!of a family of sets}\textbf{disjoint union of the family $\{A_{i}\}_{i\in I}$}.
        \par\vspace*{\TCBBoxCorrection}
    } %
    %---  End Footnote  ---%
    is the coproduct of $\{A_{i}\}_{i\in I}$ in $\Sets$ as in \ChapterRef{\ChapterLimitsAndColimits, \cref{limits-and-colimits:the-coproduct-of-a-family-of-objects}}{\cref{the-coproduct-of-a-family-of-objects}}.
\end{definition}
\begin{construction}{Construction of the Coproduct of a Family of Sets}{construction-of-the-coproduct-of-a-family-of-sets}%
    Concretely, the disjoint union of $\{A_{i}\}_{i\in I}$ is the pair \index[notation]{coprodiiniAi@$\coprod_{i\in I}A_{i}$}$(\coprod_{i\in I}A_{i},\{\inj_{i}\}_{i\in I})$ consisting of:
    \begin{enumerate}
        \item\label{construction-of-the-coproduct-of-a-family-of-sets-the-colimit}\SloganFont{The Colimit. }The set $\coprod_{i\in I}A_{i}$ given by%
            \[
                \coprod_{i\in I}A_{i}%
                =%
                \{%
                    (i,x)\in I\times\left(\bigcup_{i\in I}A_{i}\right)%
                    \ \middle|\ %
                    \text{%
                        $x\in A_{i}$%
                    }%
                \}.%
            \]%
        \item\label{construction-of-the-coproduct-of-a-family-of-sets-the-cocone}\SloganFont{The Cocone. }The collection
            \[
                \{%
                    \inj_{i}
                    \colon%
                    A_{i}%
                    \to%
                    \coprod_{i\in I}A_{i}%
                \}_{i\in I}%
            \]%
            of maps defined by
            \[
                \inj_{i}(x)%
                \defeq%
                (i,x)%
            \]%
            for each $x\in A_{i}$ and each $i\in I$.
    \end{enumerate}
\end{construction}
\begin{Proof}{Proof of \cref{construction-of-the-coproduct-of-a-family-of-sets}}%
    We claim that $\coprod_{i\in I}A_{i}$ is the categorical coproduct of $\{A_{i}\}_{i\in I}$ in $\Sets$. Indeed, suppose we have, for each $i\in I$, a diagram of the form
    \[
        \begin{tikzcd}[row sep={5.0*\the\DL,between origins}, column sep={5.0*\the\DL,between origins}, background color=backgroundColor, ampersand replacement=\&]
            \&
            C
            \\
            A_{i}
            \arrow[ru,"\iota_{i}"]
            \arrow[r,"\inj_{i}"']
            \&
            {\displaystyle\coprod_{i\in I}A_{i}}
        \end{tikzcd}
    \]%
    in $\Sets$. Then there exists a unique map $\phi\colon\coprod_{i\in I}A_{i}\to C$ making the diagram
    \[
        \begin{tikzcd}[row sep={5.0*\the\DL,between origins}, column sep={5.0*\the\DL,between origins}, background color=backgroundColor, ampersand replacement=\&]
            \&
            C
            \\
            A_{i}
            \arrow[ru,"\iota_{i}"]
            \arrow[r,"\inj_{i}"']
            \&
            {\displaystyle\coprod_{i\in I}A_{i}}
            \arrow[u,"\phi"{pos=0.45},"\exists!"'{pos=0.45},dashed]
        \end{tikzcd}
    \]%
    commute, being uniquely determined by the condition $\phi\circ\inj_{i}=\iota_{i}$ for each $i\in I$ via
    \[
        \phi((i,x))%
        =%
        \iota_{i}(x)
    \]%
    for each $(i,x)\in\coprod_{i\in I}A_{i}$.
\end{Proof}
\begin{proposition}{Properties of Coproducts of Families of Sets}{properties-of-coproducts-of-families-of-sets}%
    Let $\{A_{i}\}_{i\in I}$ be a family of sets.%
    \begin{enumerate}
        \item\label{properties-of-coproducts-of-families-of-sets-functoriality}\SloganFont{Functoriality. }The assignment $\{A_{i}\}_{i\in I}\mapsto\coprod_{i\in I}A_{i}$ defines a functor
            \[
                \coprod_{i\in I}%
                \colon%
                \Fun(I_{\disc},\Sets)%
                \to%
                \Sets%
            \]%
            where
            \begin{itemize}
                \item\SloganFont{Action on Objects. }For each $(A_{i})_{i\in I}\in\Obj(\Fun(I_{\disc},\Sets))$, we have
                    \[
                        \left[\coprod_{i\in I}\right]((A_{i})_{i\in I})%
                        \defeq%
                        \coprod_{i\in I}A_{i}%
                    \]%
                \item\SloganFont{Action on Morphisms. }For each $(A_{i})_{i\in I},(B_{i})_{i\in I}\in\Obj(\Fun(I_{\disc},\Sets))$, the action on $\Hom$-sets
                    \[
                        \left(\coprod_{i\in I}\right)_{(A_{i})_{i\in I},(B_{i})_{i\in I}}
                        \colon
                        \Nat((A_{i})_{i\in I},(B_{i})_{i\in I})%
                        \to%
                        \Sets\left(\coprod_{i\in I}A_{i},\coprod_{i\in I}B_{i}\right)%
                    \]%
                    of $\coprod_{i\in I}$ at $((A_{i})_{i\in I},(B_{i})_{i\in I})$ is defined by sending a map
                    \[
                        \{%
                            f_{i}%
                            \colon%
                            A_{i}%
                            \to%
                            B_{i}
                        \}_{i\in I}%
                        %
                    \]%
                    in $\Nat((A_{i})_{i\in I},(B_{i})_{i\in I})$ to the map of sets
                    \[
                        \coprod_{i\in I}f_{i}%
                        \colon
                        \coprod_{i\in I}A_{i}%
                        \to%
                        \coprod_{i\in I}B_{i}%
                    \]%
                    defined by
                    \[
                        \left[\coprod_{i\in I}f_{i}\right](i,a)
                        \defeq%
                        f_{i}(a)
                    \]%
                    for each $(i,a)\in\coprod_{i\in I}A_{i}$.
            \end{itemize}
        %\item\label{properties-of-coproducts-of-families-of-sets-}\SloganFont{. }
    \end{enumerate}
\end{proposition}
\begin{Proof}{Proof of \cref{properties-of-coproducts-of-families-of-sets}}%
    \FirstProofBox{\cref{properties-of-coproducts-of-families-of-sets-functoriality}: Functoriality}%
    This follows from \ChapterRef{\ChapterLimitsAndColimits, \cref{limits-and-colimits:properties-of-co-limits-functoriality} of \cref{limits-and-colimits:properties-of-co-limits}}{\cref{properties-of-co-limits-functoriality} of \cref{properties-of-co-limits}}.
\end{Proof}
\subsection{Binary Coproducts}\label{subsection-binary-coproducts-of-sets}
Let $A$ and $B$ be sets.%
\begin{definition}{Coproducts of Sets}{coproducts-of-sets}%
    The \index[set-theory]{coproduct}\textbf{coproduct of $A$ and $B$}%
    %--- Begin Footnote ---%
    \footnote{%
        \SloganFont{Further Terminology: }Also called the \index[set-theory]{disjoint union}\textbf{disjoint union of $A$ and $B$}.
        \par\vspace*{\TCBBoxCorrection}
    } %
    %---  End Footnote  ---%
    is the coproduct of $A$ and $B$ in $\Sets$ as in \ChapterRef{\ChapterLimitsAndColimits, \cref{limits-and-colimits:binary-coproducts}}{\cref{binary-coproducts}}.
\end{definition}
\begin{construction}{Construction of Coproducts of Sets}{construction-of-coproducts-of-sets}%
    Concretely, the coproduct of $A$ and $B$ is the pair \index[notation]{AicoprodB@$A\icoprod B$}$(A\coprod B,\{\inj_{1},\inj_{2}\})$ consisting of:
    \begin{enumerate}
        \item\label{construction-of-coproducts-of-sets-the-colimit}\SloganFont{The Colimit. }The set $A\icoprod B$ given by%
            \begin{align*}
                A\icoprod B &=      \coprod_{z\in\{A,B\}}z\\
                            &\eqdef \{(0,a)\in S\ \middle|\ a\in A\}\cup\{(1,b)\in S\ \middle|\ b\in B\},
            \end{align*}
            where $S=\{0,1\}\times(A\cup B)$.
        \item\label{construction-of-coproducts-of-sets-the-cocone}\SloganFont{The Cocone. }The maps
            \begin{align*}
                \inj_{1} &\colon A \to A\icoprod B,\\
                \inj_{2} &\colon B \to A\icoprod B,
            \end{align*}
            defined by
            \begin{align*}
                \inj_{1}(a) &\defeq (0,a),\\
                \inj_{2}(b) &\defeq (1,b),
            \end{align*}
            for each $a\in A$ and each $b\in B$.
    \end{enumerate}
\end{construction}
\begin{Proof}{Proof of \cref{construction-of-coproducts-of-sets}}%
    We claim that $A\icoprod B$ is the categorical coproduct of $A$ and $B$ in $\Sets$. Indeed, suppose we have a diagram of the form
    \[
        \begin{tikzcd}[row sep={4.5*\the\DL,between origins}, column sep={4.5*\the\DL,between origins}, background color=backgroundColor, ampersand replacement=\&,coproductArrows={4.5*\the\DL}{\iota_{1}}{\iota_{2}}]
            {}%
            \&
            C
            \&
            {}%
            \\
            A
            \&
            A\icoprod B
            \arrow[from=l,"\inj_{1}"',hook]
            \arrow[from=r,"\inj_{2}",hook']
            \&
            B
        \end{tikzcd}
    \]%
    in $\Sets$. Then there exists a unique map $\phi\colon A\icoprod B\to C$ making the diagram
    \[
        \begin{tikzcd}[row sep={4.5*\the\DL,between origins}, column sep={4.5*\the\DL,between origins}, background color=backgroundColor, ampersand replacement=\&,coproductArrows={4.5*\the\DL}{\iota_{1}}{\iota_{2}}]
            {}%
            \&
            C
            \arrow[from=d,"\phi","\exists!"', dashed]
            \&
            {}%
            \\
            A
            \&
            A\icoprod B
            \arrow[from=l,"\inj_{1}"',hook]
            \arrow[from=r,"\inj_{2}",hook']
            \&
            B
        \end{tikzcd}
    \]%
    commute, being uniquely determined by the conditions
    \begin{align*}
        \phi\circ\inj_{A} &= \iota_{A},\\
        \phi\circ\inj_{B} &= \iota_{B}
    \end{align*}
    via
    \[
        \phi(x)%
        =%
        \begin{cases}
            \iota_{A}(a) &\text{if $x=(0,a)$,}\\%
            \iota_{B}(b) &\text{if $x=(1,b)$}
        \end{cases}
    \]%
    for each $x\in A\icoprod B$.
\end{Proof}
\begin{proposition}{Properties of Coproducts of Sets}{properties-of-coproducts-of-sets}%
    Let $A$, $B$, $C$, and $X$ be sets.
    \begin{enumerate}
        \item\label{properties-of-coproducts-of-sets-functoriality}\SloganFont{Functoriality. }The assignment $A,B,(A,B)\mapsto A\icoprod B$ defines functors
            \[
                \Bifunctoriality{A\icoprod-}{-\icoprod B}{-_{1}\icoprod-_{2}}{\Sets}{\Sets}{\Sets\times\Sets}{\Sets}%
            \]%
            where $-_{1}\icoprod-_{2}$ is the functor where
            \begin{itemize}
                \item\SloganFont{Action on Objects. }For each $(A,B)\in\Obj(\Sets\times\Sets)$, we have
                    \[
                        [-_{1}\icoprod-_{2}](A,B)%
                        \defeq
                        A\icoprod B.%
                    \]%
                \item\SloganFont{Action on Morphisms. }For each $(A,B),(X,Y)\in\Obj(\Sets)$, the action on $\Hom$-sets
                    \[
                        \ipushout{(A,B),(X,Y)}
                        \colon
                        \Sets(A,X)\times\Sets(B,Y)%
                        \to%
                        \Sets(A\icoprod B,X\icoprod Y)%
                    \]%
                    of $\icoprod$ at $((A,B),(X,Y))$ is defined by sending $(f,g)$ to the function
                    \[
                        f\icoprod g%
                        \colon%
                        A\icoprod B%
                        \to%
                        X\icoprod Y%
                    \]%
                    defined by
                    \[
                        [f\icoprod g](x)%
                        \defeq%
                        \begin{cases}
                            (0,f(a)) &\text{if $x=(0,a)$,}\\%
                            (1,g(b)) &\text{if $x=(1,b)$,}%
                        \end{cases}
                    \]%
                    for each $x\in A\icoprod B$.
            \end{itemize}
            and where $A\icoprod-$ and $-\icoprod B$ are the partial functors of $-_{1}\icoprod-_{2}$ at $A,B\in\Obj(\Sets)$.
        \item\label{properties-of-coproducts-of-sets-adjointness}\SloganFont{Adjointness. }We have an adjunction
            \begin{webcompile}
                \Adjunction#{-_{1}\icoprod-_{2}}#\Delta_{\Sets}#\Sets\times\Sets#\Sets,#\\
            \end{webcompile}%
            witnessed by a bijection
            \[
                \Sets(A\icoprod B,C),%
                \cong%
                \Hom_{\Sets\times\Sets}((A,B),(C,C))%
            \]%
            natural in $(A,B)\in\Obj(\Sets\times\Sets)$ and in $C\in\Obj(\Sets)$.
        \item\label{properties-of-coproducts-of-sets-associativity}\SloganFont{Associativity. }We have an isomorphism of sets
            \[
                \alpha^{\Sets,\icoprod}_{X,Y,Z}
                \colon
                (X\icoprod Y)\icoprod Z
                \isorightarrow
                X\icoprod(Y\icoprod Z),
            \]%
            natural in $X,Y,Z\in\Obj(\Sets)$.
        \item\label{properties-of-coproducts-of-sets-unitality}\SloganFont{Unitality. }We have isomorphisms of sets
            \begin{align*}
                \LUnitor^{\Sets,\icoprod}_{X} &\colon \emptyset\icoprod X \isorightarrow X,\\
                \RUnitor^{\Sets,\icoprod}_{X} &\colon X\icoprod\emptyset  \isorightarrow X,
            \end{align*}
            natural in $X\in\Obj(\Sets)$.
        \item\label{properties-of-coproducts-of-sets-commutativity}\SloganFont{Commutativity. }We have an isomorphism of sets
            \[
                \sigma^{\Sets,\icoprod}_{X,Y}
                \colon
                X\icoprod Y
                \isorightarrow
                Y\icoprod X,
            \]%
            natural in $X,Y\in\Obj(\Sets)$.
        \item\label{properties-of-coproducts-of-sets-symmetric-monoidality}\SloganFont{Symmetric Monoidality. }The 7-tuple $\left(\phantom{\mrp{\alpha^{\Sets}}}\Sets\right.$, $\icoprod$, $\emptyset$, $\alpha^{\Sets}_{\icoprod}$, $\LUnitor^{\Sets}_{\icoprod}$, $\RUnitor^{\Sets}_{\icoprod}$, $\left.\sigma^{\Sets}\right)$ is a symmetric monoidal category.
        %\item\label{properties-of-coproducts-of-sets-}\SloganFont{. }
    \end{enumerate}
\end{proposition}
\begin{Proof}{Proof of \cref{properties-of-coproducts-of-sets}}%
    \FirstProofBox{\cref{properties-of-coproducts-of-sets-functoriality}: Functoriality}%
    This follows from \ChapterRef{\ChapterLimitsAndColimits, \cref{limits-and-colimits:properties-of-co-limits-functoriality} of \cref{limits-and-colimits:properties-of-co-limits}}{\cref{properties-of-co-limits-functoriality} of \cref{properties-of-co-limits}}.

    \ProofBox{\cref{properties-of-coproducts-of-sets-adjointness}: Adjointness}%
    This follows from the universal property of the coproduct.

    \ProofBox{\cref{properties-of-coproducts-of-sets-associativity}: Associativity}%
    This is proved in the proof of \ChapterRef{\ChapterMonoidalStructuresOnTheCategoryOfSets, \cref{monoidal-structures-on-the-category-of-sets:the-associator-of-the-coproduct-of-sets}}{\cref{the-associator-of-the-coproduct-of-sets}}.

    \ProofBox{\cref{properties-of-coproducts-of-sets-unitality}: Unitality}%
    This is proved in the proof of \ChapterRef{\ChapterMonoidalStructuresOnTheCategoryOfSets, \cref{monoidal-structures-on-the-category-of-sets:the-left-unitor-of-the-coproduct-of-sets,monoidal-structures-on-the-category-of-sets:the-right-unitor-of-the-coproduct-of-sets}}{\cref{the-left-unitor-of-the-coproduct-of-sets,the-right-unitor-of-the-coproduct-of-sets}}.

    \ProofBox{\cref{properties-of-coproducts-of-sets-commutativity}: Commutativity}%
    This is proved in the proof of \ChapterRef{\ChapterMonoidalStructuresOnTheCategoryOfSets, \cref{monoidal-structures-on-the-category-of-sets:the-symmetry-of-the-coproduct-of-sets}}{\cref{the-symmetry-of-the-coproduct-of-sets}}.

    \ProofBox{\cref{properties-of-coproducts-of-sets-symmetric-monoidality}: Symmetric Monoidality}%
    This is a repetition of \ChapterRef{\ChapterMonoidalStructuresOnTheCategoryOfSets, \cref{monoidal-structures-on-the-category-of-sets:the-monoidal-structure-on-sets-associated-to-the-coproduct}}{\cref{the-monoidal-structure-on-sets-associated-to-the-coproduct}}, and is proved there.
\end{Proof}
\subsection{Pushouts}\label{subsection-pushouts-of-sets}
Let $A$, $B$, and $C$ be sets and let $f\colon C\to A$ and $g\colon C\to B$ be functions.
\begin{definition}{Pushouts of Sets}{pushouts-of-sets}%
    The \index[set-theory]{pushout of sets}\textbf{pushout of $A$ and $B$ over $C$ along $f$ and $g$}%
    %--- Begin Footnote ---%
    \footnote{%
        \SloganFont{Further Terminology: }Also called the \index[set-theory]{fibre coproduct of sets}\textbf{fibre coproduct of $A$ and $B$ over $C$ along $f$ and $g$}.
        \par\vspace*{\TCBBoxCorrection}
    } %
    %---  End Footnote  ---%
    is the pushout of $A$ and $B$ over $C$ along $f$ and $g$ in $\Sets$ as in \ChapterRef{\ChapterLimitsAndColimits, \cref{limits-and-colimits:pushouts}}{\cref{pushouts}}.
\end{definition}
\begin{construction}{Construction of Pushouts of Sets}{construction-of-pushouts-of-sets}%
    Concretely, the pushout of $A$ and $B$ over $C$ along $f$ and $g$ is the pair \index[notation]{AicoprodCB@$A\ipushout{C}B$}$(A\ipushout{C}B,\{\inj_{1},\inj_{2}\})$ consisting of:
    \begin{enumerate}
        \item\label{construction-of-pushouts-of-sets-the-colimit}\SloganFont{The Colimit. }The set $A\ipushout{C}B$ given by
            \[
                A\ipushout{C}B%
                =%
                A\icoprod B/\unsim_{C},
            \]%
            where $\unsim_{C}$ is the equivalence relation on $A\icoprod B$ generated by $(0,f(c))\sim_{C}(1,g(c))$.
        \item\label{construction-of-pushouts-of-sets-the-cocone}\SloganFont{The Cocone. }The maps
            \begin{align*}
                \inj_{1} &\colon A\to A\ipushout{C}B,\\
                \inj_{2} &\colon B\to A\ipushout{C}B
            \end{align*}
            defined by
            \begin{align*}
                \inj_{1}(a) &\defeq [(0,a)]\\%
                \inj_{2}(b) &\defeq [(1,b)]%
            \end{align*}
            for each $a\in A$ and each $b\in B$.
    \end{enumerate}
\end{construction}
\begin{Proof}{Proof of \cref{construction-of-pushouts-of-sets}}%
    We claim that $A\ipushout{C}B$ is the categorical pushout of $A$ and $B$ over $C$ with respect to $(f,g)$ in $\Sets$. First we need to check that the relevant pushout diagram commutes, i.e.\ that we have
    \begin{webcompile}
        \inj_{1}\circ f%
        =%
        \inj_{2}\circ g,%
        \qquad
        \begin{tikzcd}[row sep={5.0*\the\DL,between origins}, column sep={5.0*\the\DL,between origins}, background color=backgroundColor, ampersand replacement=\&]
            A\ipushout{C}B
            \arrow[from=r,"\inj_{2}"']
            \arrow[from=d,"\inj_{1}"]
            \&
            B
            \arrow[from=d,"g"']
            \\
            A
            \arrow[from=r,"f"]
            \&
            C\mrp{.}
        \end{tikzcd}
    \end{webcompile}
    Indeed, given $c\in C$, we have
    \begin{align*}
        [\inj_{1}\circ f](c) &= \inj_{1}(f(c))\\%
                             &= [(0,f(c))]\\%
                             &= [(1,g(c))]\\%
                             &= \inj_{2}(g(c))\\%
                             &= [\inj_{2}\circ g](c),%
    \end{align*}
    where $[(0,f(c))]=[(1,g(c))]$ by the definition of the relation $\unsim$ on $A\icoprod B$. Next, we prove that $A\icoprod_{C}B$ satisfies the universal property of the pushout. Suppose we have a diagram of the form
    \[
        \begin{tikzcd}[row sep={6.0*\the\DL,between origins}, column sep={6.0*\the\DL,between origins}, background color=backgroundColor, ampersand replacement=\&]
            P
            \arrow[from=rrd, "\iota_{2}"',bend right=25]
            \arrow[from=rdd, "\iota_{1}", bend left =27.5]
            \&[-2.0*\the\DL]
            \&
            \\[-2.0*\the\DL]
            \&[-2.0*\the\DL]
            A\ipushout{C}B
            \arrow[rd, phantom, "\ulcorner", very near start]
            \arrow[from=r, "\inj_{2}"description]
            \arrow[from=d, "\inj_{1}"'description]
            \&
            B
            \arrow[from=d, "g"']
            \\
            \&[-2.0*\the\DL]
            A
            \arrow[from=r, "f"]
            \&
            C
        \end{tikzcd}
    \]%
    in $\Sets$. Then there exists a unique map $\phi\colon A\ipushout{C}B\to P$ making the diagram
    \[
        \begin{tikzcd}[row sep={6.0*\the\DL,between origins}, column sep={6.0*\the\DL,between origins}, background color=backgroundColor, ampersand replacement=\&]
            P
            \arrow[from=rrd, "\iota_{2}"',bend right=25]
            \arrow[from=rdd, "\iota_{1}", bend left =27.5]
            \arrow[from=rd,  "\phi","\exists!"',dashed]
            \&[-2.0*\the\DL]
            \&
            \\[-2.0*\the\DL]
            \&[-2.0*\the\DL]
            A\ipushout{C}B
            \arrow[rd, phantom, "\ulcorner", very near start]
            \arrow[from=r, "\inj_{2}"description]
            \arrow[from=d, "\inj_{1}"'description]
            \&
            B
            \arrow[from=d, "g"']
            \\
            \&[-2.0*\the\DL]
            A
            \arrow[from=r, "f"]
            \&
            C
        \end{tikzcd}
    \]%
    commute, being uniquely determined by the conditions%
    \begin{align*}
        \phi\circ\inj_{1} &= \iota_{1},\\%
        \phi\circ\inj_{2} &= \iota_{2}
    \end{align*}
    via
    \[
        \phi(x)%
        =%
        \begin{cases}
            \iota_{1}(a) &\text{if $x=[(0,a)]$,}\\
            \iota_{2}(b) &\text{if $x=[(1,b)]$}
        \end{cases}
    \]%
    for each $x\in A\ipushout{C}B$, where the well-definedness of $\phi$ is guaranteed by the equality $\iota_{1}\circ f=\iota_{2}\circ g$ and the definition of the relation $\unsim$ on $A\icoprod B$ as follows:
    \begin{enumerate}
        \item\label{proof-of-construction-of-pushouts-of-sets-case-1}\SloganFont{Case 1: }Suppose we have $x=[(0,a)]=[(0,a')]$ for some $a,a'\in A$. Then, by \cref{unwinding-pushouts-of-sets}, we have a sequence
            \[
                (0,a)\sim'x_{1}\sim'\cdots\sim'x_{n}\sim'(0,a').%
            \]%
        \item\label{proof-of-construction-of-pushouts-of-sets-case-2}\SloganFont{Case 2: }Suppose we have $x=[(1,b)]=[(1,b')]$ for some $b,b'\in B$. Then, by \cref{unwinding-pushouts-of-sets}, we have a sequence
            \[
                (1,b)\sim'x_{1}\sim'\cdots\sim'x_{n}\sim'(1,b').%
            \]%
        \item\label{proof-of-construction-of-pushouts-of-sets-case-3}\SloganFont{Case 3: }Suppose we have $x=[(0,a)]=[(1,b)]$ for some $a\in A$ and $b\in B$. Then, by \cref{unwinding-pushouts-of-sets}, we have a sequence
            \[
                (0,a)\sim'x_{1}\sim'\cdots\sim'x_{n}\sim'(1,b).%
            \]%
    \end{enumerate}
    In all these cases, we declare $x\sim'y$ iff there exists some $c\in C$ such that $x=(0,f(c))$ and $y=(1,g(c))$ or $x=(1,g(c))$ and $y=(0,f(c))$. Then, the equality $\iota_{1}\circ f=\iota_{2}\circ g$ gives
    \begin{align*}
        \phi([x]) &=      \phi([(0,f(c))])\\
                  &\eqdef \iota_{1}(f(c))\\
                  &=      \iota_{2}(g(c))\\
                  &\eqdef \phi([(1,g(c))])\\
                  &=      \phi([y]),
    \end{align*}
    with the case where $x=(1,g(c))$ and $y=(0,f(c))$ similarly giving $\phi([x])=\phi([y])$. Thus, if $x\sim'y$, then $\phi([x])=\phi([y])$. Applying this equality pairwise to the sequences
    \begin{align*}
        (0,a)&\sim'x_{1}\sim'\cdots\sim'x_{n}\sim'(0,a'),\\%
        (1,b)&\sim'x_{1}\sim'\cdots\sim'x_{n}\sim'(1,b'),\\%
        (0,a)&\sim'x_{1}\sim'\cdots\sim'x_{n}\sim'(1,b)%
    \end{align*}
    gives
    \begin{align*}
        \phi([(0,a)]) &= \phi([(0,a')]),\\
        \phi([(1,b)]) &= \phi([(1,b')]),\\
        \phi([(0,a)]) &= \phi([(1,b)]),
    \end{align*}
    showing $\phi$ to be well-defined.
\end{Proof}
\begin{remark}{Unwinding \cref{pushouts-of-sets}}{unwinding-pushouts-of-sets}%
    In detail, by \ChapterRef{\ChapterConditionsOnRelations, \cref{conditions-on-relations:construction-of-the-equivalence-closure-of-a-relation}}{\cref{construction-of-the-equivalence-closure-of-a-relation}}, the relation $\unsim$ of \cref{pushouts-of-sets} is given by declaring $a\sim b$ \textiff one of the following conditions is satisfied:%
    \begin{enumerate}
        \item\label{unwinding-pushouts-of-sets-1}We have $a,b\in A$ and $a=b$.
        \item\label{unwinding-pushouts-of-sets-2}We have $a,b\in B$ and $a=b$.
        \item\label{unwinding-pushouts-of-sets-3}There exist $x_{1},\ldots,x_{n}\in A\icoprod B$ such that $a\sim'x_{1}\sim'\cdots\sim'x_{n}\sim'b$, where we declare $x\sim'y$ if one of the following conditions is satisfied:
            \begin{enumerate}
                \item\label{unwinding-pushouts-of-sets-3-a}There exists $c\in C$ such that $x=(0,f(c))$ and $y=(1,g(c))$.
                \item\label{unwinding-pushouts-of-sets-3-b}There exists $c\in C$ such that $x=(1,g(c))$ and $y=(0,f(c))$.
            \end{enumerate}
            In other words, there exist $x_{1},\ldots,x_{n}\in A\icoprod B$ satisfying the following conditions:
            \begin{enumerate}
                \setcounter{enumii}{2}
                \item\label{unwinding-pushouts-of-sets-3-c}There exists $c_{0}\in C$ satisfying one of the following conditions:
                    \begin{enumerate}
                        \item\label{unwinding-pushouts-of-sets-3-c-i}We have $a=f(c_{0})$ and $x_{1}=g(c_{0})$.
                        \item\label{unwinding-pushouts-of-sets-3-c-ii}We have $a=g(c_{0})$ and $x_{1}=f(c_{0})$.
                    \end{enumerate}
                \item\label{unwinding-pushouts-of-sets-3-d}For each $1\leq i\leq n-1$, there exists $c_{i}\in C$ satisfying one of the following conditions:
                    \begin{enumerate}
                        \item\label{unwinding-pushouts-of-sets-3-d-i}We have $x_{i}=f(c_{i})$ and $x_{i+1}=g(c_{i})$.
                        \item\label{unwinding-pushouts-of-sets-3-d-ii}We have $x_{i}=g(c_{i})$ and $x_{i+1}=f(c_{i})$.
                    \end{enumerate}
                \item\label{unwinding-pushouts-of-sets-3-e}There exists $c_{n}\in C$ satisfying one of the following conditions:
                    \begin{enumerate}
                        \item\label{unwinding-pushouts-of-sets-3-e-i}We have $x_{n}=f(c_{n})$ and $b=g(c_{n})$.
                        \item\label{unwinding-pushouts-of-sets-3-e-ii}We have $x_{n}=g(c_{n})$ and $b=f(c_{n})$.
                    \end{enumerate}
            \end{enumerate}
    \end{enumerate}
\end{remark}
\begin{remark}{Pushouts of Sets Depend on the Maps}{pushouts-of-sets-depend-on-the-maps}%
    It is common practice to write $A\icoprod_{C}B$ for the pushout of $A$ and $B$ over $C$ along $f$ and $g$, omitting the maps $f$ and $g$ from the notation and instead leaving them implicit, to be understood from the context.

    \indent However, the set $A\icoprod_{C}B$ depends very much on the maps $f$ and $g$, and sometimes it is necessary or useful to note this dependence explicitly. In such situations, we will write \index[notation]{AicoprodfCgB@$A\icoprod_{f,C,g}B$}$A\icoprod_{f,C,g}B$ or \index[notation]{AicoprodfgCB@$A\icoprod^{f,g}_{C}B$}$A\icoprod^{f,g}_{C}B$ for $A\icoprod_{C}B$.
\end{remark}
\begin{example}{Examples of Pushouts of Sets}{examples-of-pushouts-of-sets}%
    Here are some examples of pushouts of sets.
    \begin{enumerate}
        \item\label{examples-of-pushouts-of-sets-wedge-sums-of-pointed-sets}\SloganFont{Wedge Sums of Pointed Sets. }The wedge sum of two pointed sets of \ChapterRef{\ChapterPointedSets, \cref{pointed-sets:coproducts-of-pointed-sets}}{\cref{coproducts-of-pointed-sets}} is an example of a pushout of sets.
        \item\label{examples-of-pushouts-of-sets-intersections-via-unions}\SloganFont{Intersections via Unions. }Let $X$ be a set. We have
            \begin{webcompile}
                A\cup B%
                \cong%
                A\ipushout{A\cap B}B,%
                \quad
                \begin{tikzcd}[row sep={5.0*\the\DL,between origins}, column sep={5.0*\the\DL,between origins}, background color=backgroundColor, ampersand replacement=\&]
                    A\cup B
                    \arrow[from=r]
                    \arrow[from=d]
                    \arrow[rd,very near start,phantom,"\ulcorner"]
                    \&
                    B
                    \arrow[from=d,hook]
                    \\
                    A
                    \arrow[from=r,hook']
                    \&
                    A\cap B
                \end{tikzcd}
            \end{webcompile}
            for each $A,B\in\mathcal{P}(X)$.
    \end{enumerate}
\end{example}
\begin{Proof}{Proof of \cref{examples-of-pushouts-of-sets}}%
    \FirstProofBox{\cref{examples-of-pushouts-of-sets-wedge-sums-of-pointed-sets}: Wedge Sums of Pointed Sets}%
    This follows by definition, as the wedge sum of two pointed sets is defined as a pushout.

    \ProofBox{\cref{examples-of-pushouts-of-sets-intersections-via-unions}: Intersections via Unions}%
    Indeed, $A\ipushout{A\cap B}B$ is the quotient of $A\icoprod B$ by the equivalence relation obtained by declaring $(0,a)\sim(1,b)$ \textiff $a=b\in A\cap B$, which is in bijection with $A\cup B$ via the map with $[(0,a)]\mapsto a$ and $[(1,b)]\mapsto b$.
\end{Proof}
\begin{proposition}{Properties of Pushouts of Sets}{properties-of-pushouts-of-sets}%
    Let $A$, $B$, $C$, and $X$ be sets.
    \begin{enumerate}
        \item\label{properties-of-pushouts-of-sets-functoriality}\SloganFont{Functoriality. }The assignment $(A,B,C,f,g)\mapsto A\ipushout{f,C,g}B$ defines a functor
            \[
                -_{1}\ipushout{-_{3}}-_{1}%
                \colon%
                \Fun(\CatFont{P},\Sets)%
                \to%
                \Sets,%
            \]%
            where $\CatFont{P}$ is the category that looks like this:
            \[
                \begin{tikzcd}[row sep={3.0*\the\DL,between origins}, column sep={3.0*\the\DL,between origins}, background color=backgroundColor, ampersand replacement=\&]
                    \&
                    \bullet
                    \arrow[from=d]
                    \\
                    \bullet
                    \arrow[from=r]
                    \&
                    \bullet\mrp{.}
                \end{tikzcd}
            \]%
            In particular, the action on morphisms of $-_{1}\ipushout{-_{3}}-_{1}$ is given by sending a morphism
            \[
                \begin{tikzcd}[row sep={4.0*\the\DL,between origins}, column sep={4.0*\the\DL,between origins}, background color=backgroundColor, ampersand replacement=\&]
                    A\ipushout{C}B
                    \arrow[from=rr]
                    \arrow[from=dd]
                    %\arrow[rd, dashed]
                    \arrow[rd,very near start,phantom,"\ulcorner"{xshift=0.675em}]
                    \&
                    \&
                    B
                    \arrow[from=dd, "g"{description,pos=0.25}]
                    \arrow[rd, "\psi"]
                    \&
                    \\
                    \&
                    A'\ipushout{C'}B'
                    \arrow[from=rr,crossing over]
                    \arrow[rd,very near start,phantom,"\ulcorner"]
                    \&
                    \&
                    B'
                    \arrow[from=dd, "g'"']
                    \\
                    A
                    \arrow[from=rr, "f"'{pos=0.25}]
                    \arrow[rd, "\phi"']
                    \&
                    \&
                    C
                    \arrow[rd,"\chi"description]
                    \&
                    \\
                    \&
                    A'
                    \arrow[from=rr, "f'"]
                    \arrow[uu, "", crossing over]
                    \&\&
                    C'
                \end{tikzcd}
            \]%
            in $\Fun(\CatFont{P},\Sets)$ to the map $\xi\colon A\ipushout{C}B\uearrow A'\ipushout{C'}B'$ defined by
            \[
                \xi(x)%
                \defeq%
                \begin{cases}
                    \phi(a) &\text{if $x=[(0,a)]$},\\
                    \psi(b) &\text{if $x=[(1,b)]$}
                \end{cases}
            \]%
            for each $x\in A\ipushout{C}B$, which is the unique map making the diagram
            \[
                \begin{tikzcd}[row sep={4.0*\the\DL,between origins}, column sep={4.0*\the\DL,between origins}, background color=backgroundColor, ampersand replacement=\&]
                    A\ipushout{C}B
                    \arrow[from=rr]
                    \arrow[from=dd]
                    \arrow[rd, dashed]
                    \arrow[rd,very near start,phantom,"\ulcorner"{xshift=0.675em}]
                    \&
                    \&
                    B
                    \arrow[from=dd, "g"{description,pos=0.25}]
                    \arrow[rd, "\psi"]
                    \&
                    \\
                    \&
                    A'\ipushout{C'}B'
                    \arrow[from=rr,crossing over]
                    \arrow[rd,very near start,phantom,"\ulcorner"]
                    \&
                    \&
                    B'
                    \arrow[from=dd, "g'"']
                    \\
                    A
                    \arrow[from=rr, "f"'{pos=0.25}]
                    \arrow[rd, "\phi"']
                    \&
                    \&
                    C
                    \arrow[rd,"\chi"description]
                    \&
                    \\
                    \&
                    A'
                    \arrow[from=rr, "f'"]
                    \arrow[uu, "", crossing over]
                    \&\&
                    C'
                \end{tikzcd}
            \]%
            commute.
        \item\label{properties-of-pushouts-of-sets-adjointness}\SloganFont{Adjointness. }We have an adjunction
            \begin{webcompile}
                \Adjunction#{-_{1}\icoprod_{X}-_{2}}#\Delta_{\Sets_{X/}}#\Sets_{X/}\times\Sets_{X/}#\Sets_{X/},#\\
            \end{webcompile}%
            witnessed by a bijection
            \[
                \Sets_{X/}(A\icoprod_{X}B,C),%
                \cong%
                \Hom_{\Sets_{X/}\times\Sets_{X/}}((A,B),(C,C))%
            \]%
            natural in $(A,B)\in\Obj(\Sets_{X/}\times\Sets_{X/})$ and in $C\in\Obj(\Sets_{X/})$.
        \item\label{properties-of-pushouts-of-sets-associativity}\SloganFont{Associativity. }Given a diagram
            \[
                \begin{tikzcd}[row sep={3.5*\the\DL,between origins}, column sep={3.5*\the\DL,between origins}, background color=backgroundColor, ampersand replacement=\&]
                    A
                    \&
                    \&
                    B
                    \&
                    \&
                    C
                    \\
                    \&
                    X
                    \&
                    \&
                    Y
                    \&
                    % 1-Arrows
                    \arrow[from=2-2,to=1-1,"f"]%
                    \arrow[from=2-2,to=1-3,"g"']%
                    %
                    \arrow[from=2-4,to=1-3,"h"]%
                    \arrow[from=2-4,to=1-5,"k"']%
                \end{tikzcd}
            \]%
            in $\Sets$, we have isomorphisms of sets
            \[
                (A\ipushout{X}B)\ipushout{Y}C%
                \cong
                (A\ipushout{X}B)\ipushout{B}(B\ipushout{Y}C)
                \cong
                A\ipushout{X}(B\ipushout{Y}C)%
            \]%
            where these pullbacks are built as in the diagrams
            \begin{scalemath}
                \begin{tikzcd}[row sep={3.5*\the\DL,between origins}, column sep={3.5*\the\DL,between origins}, background color=backgroundColor, ampersand replacement=\&]
                    \&
                    \&
                    (A\ipushout{X}B)\ipushout{Y}C
                    \&
                    \&
                    \\
                    \&
                    A\ipushout{X}B
                    \&
                    \&
                    \&
                    \\
                    A
                    \&
                    \&
                    B
                    \&
                    \&
                    C\mrp{,}
                    \\
                    \&
                    X
                    \&
                    \&
                    Y
                    \&
                    % 1-Arrows
                    \arrow[from=2-2,to=1-3]%
                    \arrow[from=3-5,to=1-3]%
                    %
                    \arrow[from=3-1,to=2-2]%
                    \arrow[from=3-3,to=2-2]%
                    %
                    \arrow[from=4-2,to=3-1,"f"]%
                    \arrow[from=4-2,to=3-3,"g"']%
                    \arrow[from=4-4,to=3-3,"h"]%
                    \arrow[from=4-4,to=3-5,"k"']%
                    %
                    \arrow[from=1-3,to=3-3,very near start,phantom,"\urcorner"{rotate=45}]
                    \arrow[from=2-2,to=4-2,very near start,phantom,"\urcorner"{rotate=45}]
                \end{tikzcd}
                \begin{tikzcd}[row sep={3.5*\the\DL,between origins}, column sep={3.5*\the\DL,between origins}, background color=backgroundColor, ampersand replacement=\&]
                    \&
                    \&
                    (A\ipushout{X}B)\ipushout{B}(B\ipushout{Y}C)
                    \&
                    \&
                    \\
                    \&
                    A\ipushout{X}B
                    \&
                    \&
                    B\ipushout{Y}C
                    \&
                    \\
                    A
                    \&
                    \&
                    B
                    \&
                    \&
                    C\mrp{,}
                    \\
                    \&
                    X
                    \&
                    \&
                    Y
                    \&
                    % 1-Arrows
                    \arrow[from=2-2,to=1-3]%
                    \arrow[from=2-4,to=1-3]%
                    %
                    \arrow[from=3-1,to=2-2]%
                    \arrow[from=3-3,to=2-2]%
                    \arrow[from=3-3,to=2-4]%
                    \arrow[from=3-5,to=2-4]%
                    %
                    \arrow[from=4-2,to=3-1,"f"]%
                    \arrow[from=4-2,to=3-3,"g"']%
                    \arrow[from=4-4,to=3-3,"h"]%
                    \arrow[from=4-4,to=3-5,"k"']%
                    %
                    \arrow[from=1-3,to=3-3,very near start,phantom,"\ulcorner"{rotate=-45}]
                    \arrow[from=2-2,to=4-2,very near start,phantom,"\ulcorner"{rotate=-45}]
                    \arrow[from=2-4,to=4-4,very near start,phantom,"\ulcorner"{rotate=-45}]
                \end{tikzcd}
                \begin{tikzcd}[row sep={3.5*\the\DL,between origins}, column sep={3.5*\the\DL,between origins}, background color=backgroundColor, ampersand replacement=\&]
                    \&
                    \&
                    A\ipushout{X}(B\ipushout{Y}C)
                    \&
                    \&
                    \\
                    \&
                    \&
                    \&
                    B\ipushout{Y}C
                    \&
                    \\
                    A
                    \&
                    \&
                    B
                    \&
                    \&
                    C\mrp{.}
                    \\
                    \&
                    X
                    \&
                    \&
                    Y
                    \&
                    % 1-Arrows
                    \arrow[from=3-1,to=1-3]%
                    \arrow[from=2-4,to=1-3]%
                    %
                    \arrow[from=3-3,to=2-4]%
                    \arrow[from=3-5,to=2-4]%
                    %
                    \arrow[from=4-2,to=3-1,"f"]%
                    \arrow[from=4-2,to=3-3,"g"']%
                    \arrow[from=4-4,to=3-3,"h"]%
                    \arrow[from=4-4,to=3-5,"k"']%
                    %
                    \arrow[from=1-3,to=3-3,very near start,phantom,"\ulcorner"{rotate=-45}]
                    \arrow[from=2-4,to=4-4,very near start,phantom,"\ulcorner"{rotate=-45}]
                \end{tikzcd}
            \end{scalemath}
        \item\label{properties-of-pushouts-of-sets-interaction-with-composition}\SloganFont{Interaction With Composition. }Given a diagram
            \[
                \begin{tikzcd}[row sep={3.5*\the\DL,between origins}, column sep={3.5*\the\DL,between origins}, background color=backgroundColor, ampersand replacement=\&]
                    X
                    \&
                    \&
                    \&
                    \&
                    Y
                    \\
                    \&
                    A
                    \&
                    \&
                    B
                    \&
                    \\
                    \&
                    \&
                    K
                    \&
                    \&
                    % 1-Arrows
                    \arrow[from=2-2,to=1-1,"\phi"]%
                    \arrow[from=3-3,to=2-2,"f"]%
                    %
                    \arrow[from=2-4,to=1-5,"\psi"']%
                    \arrow[from=3-3,to=2-4,"g"']%
                \end{tikzcd}
            \]%
            in $\Sets$, we have isomorphisms of sets
            \begin{align*}
                X\icoprod^{\phi\circ f,\psi\circ g}_{K}Y &\cong (X\icoprod^{\phi,j_{1}}_{A}(A\icoprod^{f,g}_{K}B))\icoprod^{i_{2},i_{1}}_{A\icoprod^{f,g}_{K}B}((A\icoprod^{f,g}_{K}B)\icoprod^{j_{2},\psi}_{B}Y)\\
                                                         &\cong X\icoprod^{\phi,i}_{A}((A\icoprod^{f,g}_{K}B)\icoprod^{j_{2},\psi}_{B}Y)\\
                                                         &\cong (X\icoprod^{\phi,i_{1}}_{A}(A\icoprod^{f,g}_{K}B))\icoprod^{j,\psi}_{B}Y
            \end{align*}
            where
            \[
                \begin{aligned}
                    j_{1} &= \inj^{A\times^{f,g}_{K}B}_{1},\\
                    i_{1} &= \inj^{(A\times^{f,g}_{K}B)\times^{q_{2},\psi}_{Y}}_{1},\\
                    i     &= j_{1}\circ\inj^{(A\times^{f,g}_{K}B)\times^{q_{2},\psi}_{B}Y}_{1},
                \end{aligned}
                \qquad
                \begin{aligned}
                    j_{2} &= \inj^{A\times^{f,g}_{K}B}_{2},\\
                    i_{2} &= \inj^{X\times^{\phi,q_{1}}_{A\times^{f,g}_{K}B}(A\times^{f,g}_{K}B)}_{2},\\
                    j     &= j_{2}\circ\inj^{X\times^{\phi,q_{1}}_{A}(A\times^{f,g}_{K}B)}_{2},
                \end{aligned}
            \]%
            and where these pullbacks are built as in the diagrams
            \begin{scalemath}
                \begin{tikzcd}[row sep={5.0*\the\DL,between origins}, column sep={5.0*\the\DL,between origins}, background color=backgroundColor, ampersand replacement=\&]
                    \&
                    \&
                    (X\icoprod_{A}(A\icoprod_{K}B))\icoprod_{A\icoprod_{K}B}((A\icoprod_{K}B)\icoprod_{B}Y)
                    \&
                    \&
                    \\
                    \&
                    X\icoprod_{A}(A\icoprod_{K}B)
                    \&
                    \&
                    (A\icoprod_{K}B)\icoprod_{B}Y
                    \&
                    \\
                    X
                    \&
                    \&
                    A\icoprod_{K}B
                    \&
                    \&
                    Y\mrp{,}
                    \\
                    \&
                    A
                    \&
                    \&
                    B
                    \&
                    \\
                    \&
                    \&
                    K
                    \&
                    \&
                    % 1-Arrows
                    %
                    \arrow[from=2-2,to=1-3,hook']%
                    \arrow[from=2-4,to=1-3,hook]%
                    %
                    \arrow[from=3-1,to=2-2,hook']%
                    \arrow[from=3-3,to=2-2,hook']%
                    \arrow[from=3-3,to=2-4,hook]%
                    \arrow[from=3-5,to=2-4,hook]%
                    %
                    \arrow[from=4-2,to=3-1,"\phi"]%
                    \arrow[from=5-3,to=4-2,"f"]%
                    %
                    \arrow[from=4-2,to=3-3,hook']%
                    \arrow[from=4-4,to=3-3,hook]%
                    %
                    \arrow[from=4-4,to=3-5,"\psi"']%
                    \arrow[from=5-3,to=4-4,"g"']%
                    %
                    \arrow[from=1-3,to=3-3,very near start,phantom,"\lrcorner"{rotate=-225}]
                    \arrow[from=2-2,to=4-2,very near start,phantom,"\lrcorner"{rotate=-225}]
                    \arrow[from=2-4,to=4-4,very near start,phantom,"\lrcorner"{rotate=-225}]
                    \arrow[from=3-3,to=5-3,very near start,phantom,"\lrcorner"{rotate=-225}]
                \end{tikzcd}
                \quad
                \begin{tikzcd}[row sep={5.0*\the\DL,between origins}, column sep={5.0*\the\DL,between origins}, background color=backgroundColor, ampersand replacement=\&]
                    \&
                    \&
                    X\icoprod_{A}((A\icoprod_{K}B)\icoprod_{B}Y)
                    \&
                    \&
                    \\
                    \&
                    \&
                    \&
                    (A\icoprod_{K}B)\icoprod_{B}Y
                    \&
                    \\
                    X
                    \&
                    \&
                    A\icoprod_{K}B
                    \&
                    \&
                    Y\mrp{,}
                    \\
                    \&
                    A
                    \&
                    \&
                    B
                    \&
                    \\
                    \&
                    \&
                    K
                    \&
                    \&
                    % 1-Arrows
                    %
                    \arrow[from=2-4,to=1-3,hook]%
                    \arrow[from=3-1,to=1-3,hook']%
                    %
                    \arrow[from=3-3,to=2-4,hook]%
                    \arrow[from=3-5,to=2-4,hook]%
                    %
                    \arrow[from=4-2,to=3-1,"\phi"]%
                    \arrow[from=5-3,to=4-2,"f"]%
                    %
                    \arrow[from=4-2,to=3-3,hook']%
                    \arrow[from=4-4,to=3-3,hook]%
                    %
                    \arrow[from=4-4,to=3-5,"\psi"']%
                    \arrow[from=5-3,to=4-4,"g"']%
                    %
                    \arrow[from=1-3,to=3-3,very near start,phantom,"\lrcorner"{rotate=-225}]
                    \arrow[from=2-4,to=4-4,very near start,phantom,"\lrcorner"{rotate=-225}]
                    \arrow[from=3-3,to=5-3,very near start,phantom,"\lrcorner"{rotate=-225}]
                \end{tikzcd}
            \end{scalemath}
            \begin{scalemath}
                \begin{tikzcd}[row sep={5.0*\the\DL,between origins}, column sep={5.0*\the\DL,between origins}, background color=backgroundColor, ampersand replacement=\&]
                    \&
                    \&
                    (X\icoprod_{A}(A\icoprod_{K}B))\icoprod_{B}Y
                    \&
                    \&
                    \\
                    \&
                    X\icoprod_{A}(A\icoprod_{K}B)
                    \&
                    \&
                    \&
                    \\
                    X
                    \&
                    \&
                    A\icoprod_{K}B
                    \&
                    \&
                    Y\mrp{,}
                    \\
                    \&
                    A
                    \&
                    \&
                    B
                    \&
                    \\
                    \&
                    \&
                    K
                    \&
                    \&
                    % 1-Arrows
                    %
                    \arrow[from=2-2,to=1-3,hook']%
                    \arrow[from=3-5,to=1-3,hook]%
                    %
                    \arrow[from=3-1,to=2-2,hook']%
                    \arrow[from=3-3,to=2-2,hook]%
                    %
                    \arrow[from=4-2,to=3-1,"\phi"]%
                    \arrow[from=5-3,to=4-2,"f"]%
                    %
                    \arrow[from=4-2,to=3-3,hook']%
                    \arrow[from=4-4,to=3-3,hook]%
                    %
                    \arrow[from=4-4,to=3-5,"\psi"']%
                    \arrow[from=5-3,to=4-4,"g"']%
                    %
                    \arrow[from=1-3,to=3-3,very near start,phantom,"\lrcorner"{rotate=-225}]
                    \arrow[from=2-2,to=4-2,very near start,phantom,"\lrcorner"{rotate=-225}]
                    \arrow[from=3-3,to=5-3,very near start,phantom,"\lrcorner"{rotate=-225}]
                \end{tikzcd}
                \quad
                \begin{tikzcd}[row sep={5.0*\the\DL,between origins}, column sep={5.0*\the\DL,between origins}, background color=backgroundColor, ampersand replacement=\&]
                    \&
                    \&
                    X\icoprod_{K}Y
                    \&
                    \&
                    \\
                    \&
                    \&
                    \&
                    \&
                    \\
                    X
                    \&
                    \&
                    {}
                    \&
                    \&
                    Y\mrp{.}
                    \\
                    \&
                    A
                    \&
                    \&
                    B
                    \&
                    \\
                    \&
                    \&
                    K
                    \&
                    \&
                    % 1-Arrows
                    %
                    \arrow[from=3-1,to=1-3,hook']%
                    \arrow[from=3-5,to=1-3,hook]%
                    %
                    \arrow[from=4-2,to=3-1,"\phi"]%
                    \arrow[from=5-3,to=4-2,"f"]%
                    %
                    \arrow[from=4-4,to=3-5,"\psi"']%
                    \arrow[from=5-3,to=4-4,"g"']%
                    %
                    \arrow[from=1-3,to=3-3,very near start,phantom,"\lrcorner"{rotate=-225}]
                \end{tikzcd}
            \end{scalemath}
        \item\label{properties-of-pushouts-of-sets-unitality}\SloganFont{Unitality. }We have isomorphisms of sets
            \begin{webcompile}
                \begin{tikzcd}[row sep={5.0*\the\DL,between origins}, column sep={5.0*\the\DL,between origins}, background color=backgroundColor, ampersand replacement=\&]
                    A
                    \arrow[r,Equals]
                    \arrow[from=d,"f"]
                    \arrow[rd,very near start,phantom,"\ulcorner"]
                    \&
                    A
                    \arrow[from=d,"f"']
                    \\
                    X
                    \arrow[r,Equals]
                    \&
                    X
                \end{tikzcd}
                \qquad
                \begin{aligned}
                    \LUnitor^{\Sets_{X/}}_{A} &\colon X\ipushout{X}A \isorightarrow A,\\
                    \RUnitor^{\Sets_{X/}}_{A} &\colon A\ipushout{X}X \isorightarrow A,
                \end{aligned}
                \qquad
                \begin{tikzcd}[row sep={5.0*\the\DL,between origins}, column sep={5.0*\the\DL,between origins}, background color=backgroundColor, ampersand replacement=\&]
                    A
                    \arrow[from=r,"f"']
                    \arrow[d,Equals]
                    \arrow[rd,very near start,phantom,"\ulcorner"]
                    \&
                    X
                    \arrow[d,Equals]
                    \\
                    X
                    \arrow[from=r,"f"]
                    \&
                    X\mrp{,}
                \end{tikzcd}
            \end{webcompile}
            natural in $(A,f)\in\Obj(\Sets_{X/})$.
        \item\label{properties-of-pushouts-of-sets-commutativity}\SloganFont{Commutativity. }We have an isomorphism of sets
            \begin{scalemath}
                \begin{tikzcd}[row sep={5.0*\the\DL,between origins}, column sep={5.0*\the\DL,between origins}, background color=backgroundColor, ampersand replacement=\&]
                    A\ipushout{X}B
                    \arrow[from=r]
                    \arrow[from=d]
                    \arrow[rd,very near start,phantom,"\ulcorner"]
                    \&
                    B
                    \arrow[from=d,"g"']
                    \\
                    A
                    \arrow[from=r,"f"]
                    \&
                    X\mrp{,}
                \end{tikzcd}
                \quad
                \sigma^{\Sets_{X/}}_{A}%
                \colon%
                A\ipushout{X}B%
                \isorightarrow%
                B\ipushout{X}A%
                \quad
                \begin{tikzcd}[row sep={5.0*\the\DL,between origins}, column sep={5.0*\the\DL,between origins}, background color=backgroundColor, ampersand replacement=\&]
                    B\ipushout{X}A
                    \arrow[from=r]
                    \arrow[from=d]
                    \arrow[rd,very near start,phantom,"\ulcorner"]
                    \&
                    A
                    \arrow[from=d,"f"']
                    \\
                    B
                    \arrow[from=r,"g"]
                    \&
                    X\mrp{.}
                \end{tikzcd}
            \end{scalemath}
            natural in $(A,f),(B,g)\in\Obj(\Sets_{X/})$.
        \item\label{properties-of-pushouts-of-sets-interaction-with-coproducts}\SloganFont{Interaction With Coproducts. }We have
            \begin{webcompile}
                A\ipushout{\emptyset}B%
                \cong%
                A\icoprod B,%
                \quad
                \begin{tikzcd}[row sep={5.0*\the\DL,between origins}, column sep={5.0*\the\DL,between origins}, background color=backgroundColor, ampersand replacement=\&]
                    A\icoprod B
                    \arrow[from=r]
                    \arrow[from=d]
                    \arrow[rd,very near start,phantom,"\ulcorner"]
                    \&
                    B
                    \arrow[from=d,"\iota_{B}"',hook]
                    \\
                    A
                    \arrow[from=r,"\iota_{A}",hook']
                    \&
                    \emptyset\mrp{.}
                \end{tikzcd}
            \end{webcompile}
        \item\label{properties-of-pushouts-of-sets-symmetric-monoidality}\SloganFont{Symmetric Monoidality. }The triple $(\Sets_{X/},\ipushout{X},X)$ is a symmetric monoidal category.
        %\item\label{properties-of-pushouts-of-sets-}\SloganFont{. }
    \end{enumerate}
\end{proposition}
\begin{Proof}{Proof of \cref{properties-of-pushouts-of-sets}}%
    \FirstProofBox{\cref{properties-of-pushouts-of-sets-functoriality}: Functoriality}%
    This is a special case of functoriality of co/limits, \ChapterRef{\ChapterLimitsAndColimits, \cref{limits-and-colimits:properties-of-co-limits-functoriality} of \cref{limits-and-colimits:properties-of-co-limits}}{\cref{properties-of-co-limits-functoriality} of \cref{properties-of-co-limits}}, with the explicit expression for $\xi$ following from the commutativity of the cube pushout diagram.

    \ProofBox{\cref{properties-of-pushouts-of-sets-adjointness}: : Adjointness}%
    This follows from the universal property of the coproduct (pushouts are coproducts in $\Sets_{X/}$).

    \ProofBox{\cref{properties-of-pushouts-of-sets-associativity}: Associativity}%
    Omitted.

    \ProofBox{\cref{properties-of-pushouts-of-sets-interaction-with-composition}: Interaction With Composition}%
    Omitted.

    \ProofBox{\cref{properties-of-pushouts-of-sets-unitality}: Unitality}%
    Omitted.

    \ProofBox{\cref{properties-of-pushouts-of-sets-commutativity}: Commutativity}%
    Omitted.

    \ProofBox{\cref{properties-of-pushouts-of-sets-interaction-with-coproducts}: Interaction With Coproducts}%
    Omitted.

    \ProofBox{\cref{properties-of-pushouts-of-sets-symmetric-monoidality}: Symmetric Monoidality}%
    Omitted.
\end{Proof}
\subsection{Coequalisers}\label{subsection-coequalisers-of-sets}
Let $A$ and $B$ be sets and let $f,g\colon A\rightrightarrows B$ be functions.
\begin{definition}{Coequalisers of Sets}{coequalisers-of-sets}%
    The \index[set-theory]{coequaliser of sets}\textbf{coequaliser of $f$ and $g$} is the coequaliser of $f$ and $g$ in $\Sets$ as in \ChapterRef{\ChapterLimitsAndColimits, \cref{limits-and-colimits:coequalisers}}{\cref{coequalisers}}.
\end{definition}
\begin{construction}{Construction of Coequalisers of Sets}{construction-of-coequalisers-of-sets}%
    Concretely, the coequaliser of $f$ and $g$ is the pair \index[notation]{CoEqfg@$\CoEq(f,g)$}$(\CoEq(f,g),\coeq(f,g))$ consisting of:
    \begin{enumerate}
        \item\label{construction-of-coequalisers-of-sets-the-colimit}\SloganFont{The Colimit. }The set $\CoEq(f,g)$ given by%
            \[
                \CoEq(f,g)%
                =%
                B/\unsim,
            \]%
            where $\unsim$ is the equivalence relation on $B$ generated by $f(a)\sim g(a)$.
        \item\label{construction-of-coequalisers-of-sets-the-cocone}\SloganFont{The Cocone. }The map
            \[
                \coeq(f,g)%
                \colon%
                B%
                \twoheadsrightarrow%
                \CoEq(f,g)%
            \]%
            given by the quotient map $\pi\colon B\twoheadsrightarrow B/\unsim$ with respect to the equivalence relation generated by $f(a)\sim g(a)$.
    \end{enumerate}
\end{construction}
\begin{Proof}{Proof of \cref{construction-of-coequalisers-of-sets}}%
    We claim that $\CoEq(f,g)$ is the categorical coequaliser of $f$ and $g$ in $\Sets$. First we need to check that the relevant coequaliser diagram commutes, i.e.\ that we have
    \[
        \coeq(f,g)\circ f%
        =%
        \coeq(f,g)\circ g.%
    \]%
    Indeed, we have
    \begin{align*}
        [\coeq(f,g)\circ f](a) &\eqdef [\coeq(f,g)](f(a))\\
                               &\eqdef [f(a)]\\
                               &=      [g(a)]\\
                               &\eqdef [\coeq(f,g)](g(a))\\
                               &\eqdef [\coeq(f,g)\circ g](a)%
    \end{align*}
    for each $a\in A$. Next, we prove that $\CoEq(f,g)$ satisfies the universal property of the coequaliser. Suppose we have a diagram of the form
    \[
        \begin{tikzcd}[row sep={4.0*\the\DL,between origins}, column sep={4.0*\the\DL,between origins}, background color=backgroundColor, ampersand replacement=\&]
            A
            \arrow[r,"f", shift left =0.8]
            \arrow[r,"g"',shift right=0.8]
            \&
            B
            \arrow[r,"{\coeq(f,g)}",hook]
            \arrow[rd,"c"']
            \&[4.0*\the\DL]
            {\CoEq(f,g)}
            \\
            \&
            \&[4.0*\the\DL]
            C
        \end{tikzcd}
    \]%
    in $\Sets$. Then, since $c(f(a))=c(g(a))$ for each $a\in A$, it follows from \ChapterRef{\ChapterConditionsOnRelations, \cref{conditions-on-relations:properties-of-quotient-sets-descending-functions-to-quotient-sets-1,conditions-on-relations:properties-of-quotient-sets-descending-functions-to-quotient-sets-2} of \cref{conditions-on-relations:properties-of-quotient-sets}}{\cref{properties-of-quotient-sets-descending-functions-to-quotient-sets-1,properties-of-quotient-sets-descending-functions-to-quotient-sets-2} of \cref{properties-of-quotient-sets}} that there exists a unique map $\CoEq(f,g)\uearrow C$ making the diagram
    \[
        \begin{tikzcd}[row sep={4.0*\the\DL,between origins}, column sep={4.0*\the\DL,between origins}, background color=backgroundColor, ampersand replacement=\&]
            A
            \arrow[r,"f", shift left =0.8]
            \arrow[r,"g"',shift right=0.8]
            \&
            B
            \arrow[r,"{\coeq(f,g)}",hook]
            \arrow[rd,"c"']
            \&[4.0*\the\DL]
            {\CoEq(f,g)}
            \arrow[d,"\exists!"{pos=0.475},dashed]
            \\
            \&
            \&[4.0*\the\DL]
            C
        \end{tikzcd}
    \]%
    commute.
\end{Proof}
\begin{remark}{Unwinding \cref{coequalisers-of-sets}}{unwinding-coequalisers-of-sets}%
    In detail, by \ChapterRef{\ChapterConditionsOnRelations, \cref{conditions-on-relations:construction-of-the-equivalence-closure-of-a-relation}}{\cref{construction-of-the-equivalence-closure-of-a-relation}}, the relation $\unsim$ of \cref{coequalisers-of-sets} is given by declaring $a\sim b$ \textiff one of the following conditions is satisfied:%
    \begin{enumerate}
        \item\label{unwinding-coequalisers-of-sets-1}We have $a=b$;
        \item\label{unwinding-coequalisers-of-sets-2}There exist $x_{1},\ldots,x_{n}\in B$ such that $a\sim'x_{1}\sim'\cdots\sim'x_{n}\sim'b$, where we declare $x\sim'y$ if one of the following conditions is satisfied:
            \begin{enumerate}
                \item\label{unwinding-coequalisers-of-sets-2-a}There exists $z\in A$ such that $x=f(z)$ and $y=g(z)$.
                \item\label{unwinding-coequalisers-of-sets-2-b}There exists $z\in A$ such that $x=g(z)$ and $y=f(z)$.
            \end{enumerate}
            In other words, there exist $x_{1},\ldots,x_{n}\in B$ satisfying the following conditions:
            \begin{enumerate}
                \item\label{unwinding-coequalisers-of-sets-2-c}There exists $z_{0}\in A$ satisfying one of the following conditions:
                    \begin{enumerate}
                        \item\label{unwinding-coequalisers-of-sets-2-c-i} We have $a=f(z_{0})$ and $x_{1}=g(z_{0})$.
                        \item\label{unwinding-coequalisers-of-sets-2-c-ii} We have $a=g(z_{0})$ and $x_{1}=f(z_{0})$.
                    \end{enumerate}
                \item\label{unwinding-coequalisers-of-sets-2-d} For each $1\leq i\leq n-1$, there exists $z_{i}\in A$ satisfying one of the following conditions:
                    \begin{enumerate}
                        \item\label{unwinding-coequalisers-of-sets-2-d-i} We have $x_{i}=f(z_{i})$ and $x_{i+1}=g(z_{i})$.
                        \item\label{unwinding-coequalisers-of-sets-2-d-ii} We have $x_{i}=g(z_{i})$ and $x_{i+1}=f(z_{i})$.
                    \end{enumerate}
                \item\label{unwinding-coequalisers-of-sets-2-e} There exists $z_{n}\in A$ satisfying one of the following conditions:
                    \begin{enumerate}
                        \item\label{unwinding-coequalisers-of-sets-2-e-i} We have $x_{n}=f(z_{n})$ and $b=g(z_{n})$.
                        \item\label{unwinding-coequalisers-of-sets-2-e-ii} We have $x_{n}=g(z_{n})$ and $b=f(z_{n})$.
                    \end{enumerate}
            \end{enumerate}
    \end{enumerate}
\end{remark}
\begin{example}{Examples of Coequalisers of Sets}{examples-of-coequalisers-of-sets}%
    Here are some examples of coequalisers of sets.
    \begin{enumerate}
        \item\label{examples-of-coequalisers-of-sets-quotients-by-equivalence-relations}\SloganFont{Quotients by Equivalence Relations. }Let $R$ be an equivalence relation on a set $X$. We have a bijection of sets
            \[
                X/\unsim_{R}%
                \cong%
                \CoEq(R\hookrightarrow X\times X\xlongrightrightarrows{\pr_{1}}{\pr_{2}}X).
            \]%
    \end{enumerate}
\end{example}
\begin{Proof}{Proof of \cref{examples-of-coequalisers-of-sets}}%
    \FirstProofBox{\cref{examples-of-coequalisers-of-sets-quotients-by-equivalence-relations}: Quotients by Equivalence Relations}%
    See \cite{proof-wiki:quotient-map-is-coequaliser}.
\end{Proof}
\begin{proposition}{Properties of Coequalisers of Sets}{properties-of-coequalisers-of-sets}%
    Let $A$, $B$, and $C$ be sets.
    \begin{enumerate}
        \item\label{properties-of-coequalisers-of-sets-associativity}\SloganFont{Associativity. }We have isomorphisms of sets%
            %--- Begin Footnote ---%
            \footnote{%
                That is, the following three ways of forming \say{the} coequaliser of $(f,g,h)$ agree:
                \begin{enumerate}
                    \item\label{footnote-properties-of-coequalisers-of-sets-associativity-1}Take the coequaliser of $(f,g,h)$, i.e.\ the colimit of the diagram
                        \[
                            \begin{tikzcd}[row sep={5.0*\the\DL,between origins}, column sep={3.5*\the\DL,between origins}, background color=backgroundColor, ampersand replacement=\&]
                                A%
                                \arrow[r,"f",shift left=2.25]%
                                \arrow[r,"g"description]%
                                \arrow[r,"h"',shift right=2.25]%
                                \&
                                B%
                            \end{tikzcd}
                        \]%
                        in $\Sets$.
                    \item\label{footnote-properties-of-coequalisers-of-sets-associativity-2}First take the coequaliser of $f$ and $g$, forming a diagram
                        \[
                            A%
                            \xlongrightrightarrows{f}{g}%
                            B%
                            \xlongtwoheadsrightarrow{\coeq(f,g)}%
                            \CoEq(f,g)%
                        \]%
                        and then take the coequaliser of the composition
                        \[
                            A%
                            \xlongrightrightarrows{f}{h}%
                            B%
                            \xlongtwoheadsrightarrow{\coeq(f,g)}%
                            \CoEq(f,g),%
                        \]%
                        obtaining a quotient
                        \begin{envfootnotesize}
                            \[%
                                \CoEq(\coeq(f,g)\circ f,\coeq(f,g)\circ h)%
                                =%
                                \CoEq(\coeq(f,g)\circ g,\coeq(f,g)\circ h)%
                            \]%
                        \end{envfootnotesize}
                        of $\CoEq(f,g)$%.
                    \item\label{footnote-properties-of-coequalisers-of-sets-associativity-3}First take the coequaliser of $g$ and $h$, forming a diagram
                        \[
                            A%
                            \xlongrightrightarrows{g}{h}%
                            B%
                            \xlongtwoheadsrightarrow{\coeq(g,h)}%
                            \CoEq(g,h)%
                        \]%
                        and then take the coequaliser of the composition
                        \[
                            A%
                            \xlongrightrightarrows{f}{g}%
                            B%
                            \xlongtwoheadsrightarrow{\coeq(g,h)}%
                            \CoEq(g,h),%
                        \]%
                        obtaining a quotient
                        \begin{envfootnotesize}
                            \[%
                                \CoEq(\coeq(g,h)\circ f,\coeq(g,h)\circ g)%
                                =%
                                \CoEq(\coeq(g,h)\circ f,\coeq(g,h)\circ h)%
                            \]%
                        \end{envfootnotesize}
                        of $\CoEq(g,h)$.
                \end{enumerate}
                \par\vspace*{\TCBBoxCorrection}
            }%
            %---  End Footnote  ---%
            \begin{envscriptsize}
                \[
                    \underbrace{\CoEq(\coeq(f,g)\circ f,\coeq(f,g)\circ h)}_{{}=\CoEq(\coeq(f,g)\circ g,\coeq(f,g)\circ h)}%
                    \cong
                    \CoEq(f,g,h)
                    \cong
                    \underbrace{\CoEq(\coeq(g,h)\circ f,\coeq(g,h)\circ g)}_{{}=\CoEq(\coeq(g,h)\circ f,\coeq(g,h)\circ h)},%
                \]%
            \end{envscriptsize}
            where $\CoEq(f,g,h)$ is the colimit of the diagram
            \[
                \begin{tikzcd}[row sep={5.0*\the\DL,between origins}, column sep={4.0*\the\DL,between origins}, background color=backgroundColor, ampersand replacement=\&]
                    A%
                    \arrow[r,"f",shift left=2.0]%
                    \arrow[r,"g"description]%
                    \arrow[r,"h"',shift right=2.0]%
                    \&
                    B%
                \end{tikzcd}
            \]%
            in $\Sets$.
        \item\label{properties-of-coequalisers-of-sets-unitality}\SloganFont{Unitality. }We have an isomorphism of sets
            \[
                \CoEq(f,f)%
                \cong%
                B.%
            \]%
        \item\label{properties-of-coequalisers-of-sets-commutativity}\SloganFont{Commutativity. }We have an isomorphism of sets
            \[
                \CoEq(f,g)
                \cong
                \CoEq(g,f).
            \]%
        \item\label{properties-of-coequalisers-of-sets-interaction-with-composition}\SloganFont{Interaction With Composition. }Let
            \[
                A
                \xlongrightrightarrows{f}{g}
                B
                \xlongrightrightarrows{h}{k}
                C
            \]%
            be functions. We have a surjection
            \[
                \CoEq(h\circ f,k\circ g)%
                \twoheadsrightarrow%
                \CoEq(\coeq(h,k)\circ h\circ f,\coeq(h,k)\circ k\circ g)%
            \]%
            exhibiting $\CoEq(\coeq(h,k)\circ h\circ f,\coeq(h,k)\circ k\circ g)$ as a quotient of $\CoEq(h\circ f,k\circ g)$ by the relation generated by declaring $h(y)\sim k(y)$ for each $y\in B$.
        %\item\label{properties-of-coequalisers-of-sets-}\SloganFont{. }
    \end{enumerate}
\end{proposition}
\begin{Proof}{Proof of \cref{properties-of-coequalisers-of-sets}}%
    \FirstProofBox{\cref{properties-of-coequalisers-of-sets-associativity}: Associativity}%
    Omitted.

    \ProofBox{\cref{properties-of-coequalisers-of-sets-unitality}: Unitality}%
    Omitted.

    \ProofBox{\cref{properties-of-coequalisers-of-sets-commutativity}: Commutativity}%
    Omitted.

    \ProofBox{\cref{properties-of-coequalisers-of-sets-interaction-with-composition}: Interaction With Composition}%
    Omitted.
\end{Proof}
\subsection{Direct Colimits}\label{subsection-limits-of-sets-direct-colimits}
Let $(X_{\alpha},f_{\alpha\beta})_{\alpha,\beta\in I}\colon(I,\preceq)\to\Top$ be a direct system of sets.
\begin{definition}{Direct Colimits of Sets}{direct-colimits-of-sets}%
    The \index[set-theory]{direct colimit of sets}\textbf{direct colimit of $(X_{\alpha},f_{\alpha\beta})_{\alpha,\beta\in I}$} is the direct colimit of $(X_{\alpha},f_{\alpha\beta})_{\alpha,\beta\in I}$ in $\Sets$ as in \ChapterRef{\ChapterLimitsAndColimits, \cref{limits-and-colimits:direct-colimits}}{\cref{direct-colimits}}.
\end{definition}
\begin{construction}{Construction of Direct Colimits of Sets}{construction-of-direct-colimits-of-sets}%
    Concretely, the direct colimit of $(X_{\alpha},f_{\alpha\beta})_{\alpha,\beta\in I}$ is the pair \index[notation]{colimalphainIXalpha@$\dircolim_{\alpha\in I}(X_{\alpha})$}$\smash{\Big(\displaystyle\dircolim_{\alpha\in I}(X_{\alpha})}$, $\smash{\{\inj_{\alpha}\}_{\alpha\in I}\Big)}$ consisting of:
    \begin{enumerate}
        \item\label{construction-of-direct-colimits-of-sets-the-colimit}\SloganFont{The Colimit. }The set $\displaystyle\dircolim_{\alpha\in I}(X_{\alpha})$ given by
            \[
                \dircolim_{\alpha\in I}(X_{\alpha})%
                =%
                \left.\left(\coprod_{\alpha\in I}X_{\alpha}\right)\middle/\unsim\right.,%
            \]%
            where $\unsim$ is the equivalence relation on $\coprod_{\alpha\in I}X_{\alpha}$ generated by declaring $(\alpha,x)\sim(\beta,y)$ \textiff there exists some $\gamma\in I$ satisfying the following conditions:
            \begin{enumerate}
                \item\label{construction-of-direct-colimits-of-sets-the-colimit-1}We have $\alpha\preceq\gamma$.
                \item\label{construction-of-direct-colimits-of-sets-the-colimit-2}We have $\beta\preceq\gamma$.
                \item\label{construction-of-direct-colimits-of-sets-the-colimit-3}We have $f_{\alpha\gamma}(x)=f_{\beta\gamma}(y)$.
            \end{enumerate}
        \item\label{construction-of-direct-colimits-of-sets-the-cocone}\SloganFont{The Cocone.}The collection
            \[
                \{%
                    \inj_{\gamma}%
                    \colon%
                    X_{\gamma}%
                    \to%
                    \dircolim_{\alpha\in I}(X_{\alpha})%
                \}_{\gamma\in I}%
            \]%
            of maps of sets defined by
            \[
                \inj_{\gamma}(x)%
                \defeq%
                [(\gamma,x)]%
            \]%
            for each $\gamma\in I$ and each $x\in X_{\gamma}$.
    \end{enumerate}
\end{construction}
\begin{Proof}{Proof of \cref{construction-of-direct-colimits-of-sets}}%
    We will prove \cref{construction-of-direct-colimits-of-sets} below in a bit, but first we need a lemma (which is interesting in its own right).
\end{Proof}
\begin{lemma}{Identification of $x$ with $f_{\alpha\beta}(x)$ in Direct Colimits}{identification-of-x-with-f-alpha-beta-x-in-direct-colimits}%
    For each $\alpha,\beta\in I$ and each $x\in X_{\alpha}$, if $\alpha\preceq\beta$, then we have
    \[
        (\alpha,x)%
        \sim%
        (\beta,f_{\alpha\beta}(x))%
    \]%
    in $\displaystyle\dircolim_{\alpha\in I}(X_{\alpha})$.
\end{lemma}
\begin{Proof}{Proof of \cref{identification-of-x-with-f-alpha-beta-x-in-direct-colimits}}%
    Taking $\gamma=\beta$, we have $f_{\alpha\gamma}=f_{\alpha\beta}$, we have $f_{\beta\gamma}=f_{\beta\beta}\eqdef\id_{X_{\beta}}$, and we have
    \begin{align*}
        f_{\alpha\beta}(x) &=      f_{\beta\beta}(f_{\alpha\beta}(x))\\%
                           &\eqdef \id_{X_{\beta}}(f_{\alpha\beta}(x)),\\%
                           &=      f_{\alpha\beta}(x).%
    \end{align*}
    As a result, since $\alpha\preceq\beta$ and $\beta\preceq\beta$ as well, \cref{construction-of-direct-colimits-of-sets-the-colimit-1,construction-of-direct-colimits-of-sets-the-colimit-2,construction-of-direct-colimits-of-sets-the-colimit-3} of \cref{construction-of-direct-colimits-of-sets} are met. Thus we have $(\alpha,x)\sim(\beta,f_{\alpha\beta}(x))$.
\end{Proof}
We can now prove \cref{construction-of-direct-colimits-of-sets}:
\begin{Proof}{Proof of \cref{construction-of-direct-colimits-of-sets}}%
    We claim that $\dircolim_{\alpha\in I}(X_{\alpha})$ is the colimit of the direct system of sets $(X_{\alpha},f_{\alpha\beta})_{\alpha,\beta\in I}$.

    \ProofBox{Commutativity of the Colimit Diagram}%
    First, we need to check that the colimit diagram defined by $\dircolim_{\alpha\in I}(X_{\alpha})$ commutes, i.e.\ that we have
    \begin{webcompile}
        \inj_{\alpha}%
        =%
        \inj_{\beta}\circ f_{\alpha\beta},%
        \quad%
        \begin{tikzcd}[row sep={5.0*\the\DL,between origins}, column sep={3.0*\the\DL,between origins}, background color=backgroundColor, ampersand replacement=\&]
            \&
            \displaystyle\dircolim_{\alpha\in I}(X_{\alpha})
            \arrow[from=ld,"\inj_{\alpha}",pos=0.3]
            \arrow[from=rd,"\inj_{\beta}"',pos=0.3]
            \&
            \\
            X_{\alpha}
            \arrow[rr,"f_{\alpha\beta}"']
            \&
            \&
            X_{\beta}
        \end{tikzcd}
    \end{webcompile}
    for each $\alpha,\beta\in I$ with $\alpha\preceq\beta$. Indeed, given $x\in X_{\alpha}$, we have
    \begin{align*}
        [\inj_{\beta}\circ f_{\alpha\beta}](x) &\eqdef \inj_{\beta}(f_{\alpha\beta}(x))\\
                                               &\eqdef [(\beta,f_{\alpha\beta}(x))]\\
                                               &=      [(\alpha,x)]\\
                                               &\eqdef \inj_{\alpha}(x),
    \end{align*}
    where we have used \cref{identification-of-x-with-f-alpha-beta-x-in-direct-colimits} for the third equality.

    \ProofBox{Proof of the Universal Property of the Colimit}%
    Next, we prove that $\dircolim_{\alpha\in I}(X_{\alpha})$ as constructed in \cref{construction-of-direct-colimits-of-sets} satisfies the universal property of a direct colimit. Suppose that we have, for each $\alpha,\beta\in I$ with $\alpha\preceq\beta$, a diagram of the form
    \[
        \begin{tikzcd}[row sep={5.0*\the\DL,between origins}, column sep={4.0*\the\DL,between origins}, background color=backgroundColor, ampersand replacement=\&]
            \&
            C
            \arrow[from=ldd,"i_{\alpha}",bend left=20]
            \arrow[from=rdd,"i_{\beta}"',bend right=20]
            \&
            \\
            \&
            \displaystyle\dircolim_{\alpha\in I}(X_{\alpha})
            \arrow[from=ld,"\inj_{\alpha}"description,pos=0.45]
            \arrow[from=rd,"\inj_{\beta}"'description,pos=0.45]
            \&
            \\[0.75*\the\DL]
            X_{\alpha}
            \arrow[rr,"f_{\alpha\beta}"']
            \&
            \&
            X_{\beta}
        \end{tikzcd}
    \]%
    in $\Sets$. We claim that there exists a unique map $\phi\colon\smash{\displaystyle\dircolim_{\alpha\in I}(X_{\alpha})}\uearrow C$ making the diagram
    \[
        \begin{tikzcd}[row sep={5.0*\the\DL,between origins}, column sep={4.0*\the\DL,between origins}, background color=backgroundColor, ampersand replacement=\&]
            \&
            C
            \arrow[from=ldd,"i_{\alpha}",bend left =20]
            \arrow[from=rdd,"i_{\beta}"',bend right=20]
            \arrow[from=d,"\phi",dashed]
            \arrow[from=d,"\exists!"',dashed]
            \&
            \\
            \&
            \displaystyle\dircolim_{\alpha\in I}(X_{\alpha})
            \arrow[from=ld,"\inj_{\alpha}"description,pos=0.45]
            \arrow[from=rd,"\inj_{\beta}"'description,pos=0.45]
            \&
            \\[0.75*\the\DL]
            X_{\alpha}
            \arrow[rr,"f_{\alpha\beta}"']
            \&
            \&
            X_{\beta}
        \end{tikzcd}
    \]%
    commute. To this end, first consider the diagram
    \[
        \begin{tikzcd}[row sep={6.5*\the\DL,between origins}, column sep={6.5*\the\DL,between origins}, background color=backgroundColor, ampersand replacement=\&]
            \displaystyle\coprod_{\alpha\in I}X_{\alpha}
            \arrow[r,"\pr",two heads]
            \arrow[rd,"\scalebox{0.8}{$\displaystyle\coprod_{\alpha\in I}i_{\alpha}$}"']
            \&
            \displaystyle\dircolim_{\alpha\in I}(X_{\alpha})
            \\
            \&
            C\mrp{.}
        \end{tikzcd}
    \]%
    \textbf{Lemma. }If $(\alpha,x)\sim(\beta,y)$, then we have
    \[
        \left[\coprod_{\alpha\in I}i_{\alpha}\right](x)%
        =%
        \left[\coprod_{\alpha\in I}i_{\alpha}\right](y).%
    \]%
    \textit{Proof. }Indeed, if $(\alpha,x)\sim(\beta,y)$, then there exists some $\gamma\in I$ satisfying the following conditions:
    \begin{enumerate}
        \item\label{proof-of-construction-of-direct-colimits-of-sets-1}We have $\alpha\preceq\gamma$.
        \item\label{proof-of-construction-of-direct-colimits-of-sets-2}We have $\beta\preceq\gamma$.
        \item\label{proof-of-construction-of-direct-colimits-of-sets-3}We have $f_{\alpha\gamma}(x)=f_{\beta\gamma}(y)$.
    \end{enumerate}
    We then have
    \begin{align*}
        \left[\coprod_{\alpha\in I}i_{\alpha}\right](x) &\eqdef i_{\alpha}(x)\\
                                                        &\eqdef [i_{\gamma}\circ f_{\alpha\gamma}](x)\\
                                                        &\eqdef i_{\gamma}(f_{\alpha\gamma}(x))\\
                                                        &=      i_{\gamma}(f_{\beta\gamma}(x))\\
                                                        &\eqdef [i_{\gamma}\circ f_{\beta\gamma}](x)\\
                                                        &=      i_{\beta}(y)\\
                                                        &\eqdef \left[\coprod_{\alpha\in I}i_{\alpha}\right](y).
    \end{align*}
    This finishes the proof of the lemma. Continuing, by \ChapterRef{\ChapterConditionsOnRelations, \cref{conditions-on-relations:properties-of-quotient-sets-descending-functions-to-quotient-sets-for-arbitrary-relations} of \cref{conditions-on-relations:properties-of-quotient-sets}}{\cref{properties-of-quotient-sets-descending-functions-to-quotient-sets-for-arbitrary-relations} of \cref{properties-of-quotient-sets}}, there then exists a map $\phi\colon\smash{\displaystyle\dircolim_{\alpha\in I}(X_{\alpha})}\uearrow C$ making the diagram
    \[
        \begin{tikzcd}[row sep={6.0*\the\DL,between origins}, column sep={7.0*\the\DL,between origins}, background color=backgroundColor, ampersand replacement=\&]
            \displaystyle\coprod_{\alpha\in I}X_{\alpha}
            \arrow[r,"\pr",two heads]
            \arrow[rd,"\scalebox{0.8}{$\displaystyle\coprod_{\alpha\in I}i_{\alpha}$}"']
            \&
            \displaystyle\dircolim_{\alpha\in I}(X_{\alpha})
            \arrow[d,"\phi"]
            \\
            \&
            C
        \end{tikzcd}
    \]%
    commute. In particular, this implies that the diagram
    \[
        \begin{tikzcd}[row sep={6.0*\the\DL,between origins}, column sep={6.0*\the\DL,between origins}, background color=backgroundColor, ampersand replacement=\&]
            X_{\alpha}
            \arrow[r,"\inj_{\alpha}",two heads]
            \arrow[rd,"i_{\alpha}"']
            \&
            \displaystyle\dircolim_{\alpha\in I}(X_{\alpha})
            \arrow[d,"\phi"]
            \\
            \&
            C
        \end{tikzcd}
    \]%
    also commutes, and thus so does the diagram
    \[
        \begin{tikzcd}[row sep={5.0*\the\DL,between origins}, column sep={4.0*\the\DL,between origins}, background color=backgroundColor, ampersand replacement=\&]
            \&
            C
            \arrow[from=ldd,"i_{\alpha}",bend left =20]
            \arrow[from=rdd,"i_{\beta}"',bend right=20]
            \arrow[from=d,"\phi",dashed]
            \arrow[from=d,"\exists!"',dashed]
            \&
            \\
            \&
            \displaystyle\dircolim_{\alpha\in I}(X_{\alpha})
            \arrow[from=ld,"\inj_{\alpha}"description,pos=0.45]
            \arrow[from=rd,"\inj_{\beta}"'description,pos=0.45]
            \&
            \\[0.75*\the\DL]
            X_{\alpha}
            \arrow[rr,"f_{\alpha\beta}"']
            \&
            \&
            X_{\beta}\mrp{.}
        \end{tikzcd}
    \]%
    This finishes the proof.%
    %--- Begin Footnote ---%
    \footnote{%
        Incidentally, the conditions
        \[
            \{%
                i_{\alpha}%
                =%
                \phi\circ\inj_{\alpha}%
            \}_{\alpha\in I}%
        \]%
        show that $\phi$ must be given by
        \[
            \phi([(\alpha,x)])%
            =%
            (i_{\alpha}(x))_{\alpha\in I}
        \]%
        for each $[(\alpha,x)]\in\dircolim_{\alpha\in I}(X_{\alpha})$, although we would need to show that this assignment is well-defined were we to prove \cref{construction-of-direct-colimits-of-sets} in this way. Instead, invoking \ChapterRef{\ChapterConditionsOnRelations, \cref{conditions-on-relations:properties-of-quotient-sets-descending-functions-to-quotient-sets-for-arbitrary-relations} of \cref{conditions-on-relations:properties-of-quotient-sets}}{\cref{properties-of-quotient-sets-descending-functions-to-quotient-sets-for-arbitrary-relations} of \cref{properties-of-quotient-sets}} gave us a way to avoid having to prove this, leading to a cleaner alternative proof.
        \par\vspace*{\TCBBoxCorrection}
    }%
    %---  End Footnote  ---%
\end{Proof}
\begin{example}{Examples of Direct Colimits of Sets}{examples-of-direct-colimits-of-sets}%
    Here are some examples of direct colimits of sets.
    \begin{enumerate}
        \item\label{examples-of-direct-colimits-of-sets-the-prüfer-group}\SloganFont{The Prüfer Group. }The Prüfer group $\Z(p^{\infty})$ is defined as the direct colimit
            \[
                \Z(p^{\infty})%
                \defeq%
                \dircolim_{n\in\N}(\Zn{p^{n}});%
            \]%
            see \cref{TODO8}.
        %\item\label{examples-of-direct-colimits-of-sets-}\SloganFont{. }
    \end{enumerate}
\end{example}
\section{More Constructions With Sets}\label{section-more-constructions-with-sets}
\subsection{Sets of Maps}\label{subsection-sets-of-maps}
Let $A$ and $B$ be sets.
\begin{definition}{Sets of Maps}{sets-of-maps}%
    The \index[set-theory]{set!of maps}\textbf{set of maps from $A$ to $B$}%
    %--- Begin Footnote ---%
    \footnote{%
        \SloganFont{Further Terminology: }Also called the \index[set-theory]{set!Hom}\textbf{Hom set from $A$ to $B$}.
    } %
    %---  End Footnote  ---%
    is the set \index[notation]{SetsAB@$\Sets(A,B)$}$\Sets(A,B)$%
    %--- Begin Footnote ---%
    \footnote{%
        \SloganFont{Further Notation: }Also written \index[notation]{HomSetsAB@$\Hom_{\Sets}(A,B)$}$\Hom_{\Sets}(A,B)$.
        \par\vspace*{\TCBBoxCorrection}
    } %
    %---  End Footnote  ---%
    whose elements are the functions from $A$ to $B$.
\end{definition}
\begin{proposition}{Properties of Sets of Maps}{properties-of-sets-of-maps}%
    Let $A$ and $B$ be sets.
    \begin{enumerate}
        \item\label{properties-of-sets-of-maps-functoriality}\SloganFont{Functoriality. }The assignments $X,Y,(X,Y)\mapsto\Hom_{\Sets}(X,Y)$ define functors
            \[
                \BifunctorialityPeriod{\Sets(X,-)}{\Sets(-,Y)}{\Sets(-_{1},-_{2})}{\Sets}{\Sets^{\mrp{\op}}}{\Sets^{\op}\times\Sets}{\Sets}%
            \]%
        \item\label{properties-of-sets-of-maps-adjointness}\SloganFont{Adjointness. }We have adjunctions
            \begin{webcompile}
                \begin{gathered}
                    \AdjunctionShort#A\times -#{\Sets(A,-)}#\Sets#\Sets,#\\
                    \AdjunctionShort#-\times B#{\Sets(B,-)}#\Sets#\Sets,#
                \end{gathered}
            \end{webcompile}%
            witnessed by bijections
            \begin{align*}
                \Sets(A\times B,C) &\cong \Sets(A,\Sets(B,C)),\\
                \Sets(A\times B,C) &\cong \Sets(B,\Sets(A,C)),
            \end{align*}
            natural in $A,B,C\in\Obj(\Sets)$.
        \item\label{properties-of-sets-of-maps-maps-from-the-punctual-set}\SloganFont{Maps From the Punctual Set. }We have a bijection
            \[
                \Sets(\pt,A)%
                \cong%
                A,%
            \]%
            natural in $A\in\Obj(\Sets)$.
        \item\label{properties-of-sets-of-maps-maps-to-the-punctual-set}\SloganFont{Maps to the Punctual Set. }We have a bijection
            \[
                \Sets(A,\pt)%
                \cong%
                \pt,%
            \]%
            natural in $A\in\Obj(\Sets)$.
        %\item\label{properties-of-sets-of-maps-}\SloganFont{. }
    \end{enumerate}
\end{proposition}
\begin{Proof}{Proof of \cref{properties-of-sets-of-maps}}%
    \FirstProofBox{\cref{properties-of-sets-of-maps-functoriality}: Functoriality}%
    This follows from \ChapterRef{\ChapterCategories, \cref{categories:properties-of-pre-postcomposition-interaction-with-composition-1,categories:properties-of-pre-postcomposition-interaction-with-identities} of \cref{categories:properties-of-pre-postcomposition}}{\cref{properties-of-pre-postcomposition-interaction-with-composition-1,properties-of-pre-postcomposition-interaction-with-identities} of \cref{properties-of-pre-postcomposition}}.

    \ProofBox{\cref{properties-of-sets-of-maps-adjointness}: Adjointness}%
    This is a repetition of \cref{properties-of-products-of-sets-adjointness-1} of \cref{properties-of-products-of-sets} and is proved there.

    \ProofBox{\cref{properties-of-sets-of-maps-maps-from-the-punctual-set}: Maps From the Punctual Set}%
    The bijection
    \[
        \Phi_{A}%
        \colon%
        \Sets(\pt,A)%
        \isorightarrow%
        A%
    \]%
    is defined by
    \[
        \Phi_{A}(f)%
        \defeq%
        f(\point)%
    \]%
    for each $f\in\Sets(\pt,A)$, admitting an inverse
    \[
        \Phi^{-1}_{A}%
        \colon%
        A%
        \isorightarrow%
        \Sets(\pt,A)%
    \]%
    defined by
    \[
        \Phi^{-1}_{A}(a)%
        \defeq%
        \llbracket\point\mapsto a\rrbracket%
    \]%
    for each $a\in A$. Indeed, we have
    \begin{align*}
        [\Phi^{-1}_{A}\circ\Phi_{A}](f) &\eqdef \Phi^{-1}_{A}(\Phi_{A}(f))\\
                                        &\eqdef \Phi^{-1}_{A}(f(\point))\\
                                        &\eqdef \llbracket\point\mapsto f(\point)\rrbracket\\
                                        &\eqdef f\\
                                        &\eqdef [\id_{\Sets(\pt,A)}](f)
    \end{align*}
    for each $f\in\Sets(\pt,A)$ and
    \begin{align*}
        [\Phi_{A}\circ\Phi^{-1}_{A}](a) &\eqdef \Phi_{A}(\Phi^{-1}_{A}(a))\\
                                        &\eqdef \Phi_{A}(\llbracket\point\mapsto a\rrbracket)\\
                                        &\eqdef \ev_{\point}(\llbracket\point\mapsto a\rrbracket)\\
                                        &\eqdef a\\
                                        &\eqdef [\id_{A}](a)
    \end{align*}
    for each $a\in A$, and thus we have
    \begin{align*}
        \Phi^{-1}_{A}\circ\Phi_{A} &= \id_{\Sets(\pt,A)}\\
        \Phi_{A}\circ\Phi^{-1}_{A} &= \id_{A}.
    \end{align*}
    To prove naturality, we need to show that the diagram
    \[
        \begin{tikzcd}[row sep={5.0*\the\DL,between origins}, column sep={8.0*\the\DL,between origins}, background color=backgroundColor, ampersand replacement=\&]
            \Sets(\pt,A)
            \arrow[r,"f_{!}"]
            \arrow[d,"\Phi_{A}"',dash pattern=on 1.65pt off 1.65pt]
            \arrow[d,isoarrow]
            \&
            \Sets(\pt,B)
            \arrow[d,"\Phi_{B}",dash pattern=on 1.65pt off 1.65pt]
            \arrow[d,isoarrowprime]
            \\
            A
            \arrow[r,"f"']
            \&
            B
        \end{tikzcd}
    \]%
    commutes. Indeed, we have
    \begin{align*}
        [f\circ\Phi_{A}](\phi) &\eqdef f(\Phi_{A}(\phi))\\
                               &\eqdef f(\phi(\point))\\
                               &\eqdef [f\circ\phi](\point)\\
                               &\eqdef \Phi_{B}(f\circ\phi)\\
                               &\eqdef \Phi_{B}(f_{!}(\phi))\\
                               &\eqdef [\Phi_{B}\circ f_{!}](\phi)
    \end{align*}
    for each $\phi\in\Sets(\pt,A)$. This finishes the proof.

    \ProofBox{\cref{properties-of-sets-of-maps-maps-to-the-punctual-set}: Maps to the Punctual Set}%
    This follows from the universal property of $\pt$ as the terminal set, \cref{the-terminal-set}.
\end{Proof}
\begin{appendices}
\input{ABSOLUTEPATH/chapters2.tex}
\end{appendices}
\end{document}
