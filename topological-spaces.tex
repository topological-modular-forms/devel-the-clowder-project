\input{preamble}

% OK, start here.
%
\usepackage{fontspec}
\let\hyperwhite\relax
\let\hyperred\relax
\newcommand{\hyperwhite}{\hypersetup{citecolor=white,filecolor=white,linkcolor=white,urlcolor=white}}
\newcommand{\hyperred}{%
\hypersetup{%
    citecolor=TitlingRed,%
    filecolor=TitlingRed,%
    linkcolor=TitlingRed,%
     urlcolor=TitlingRed%
}}
\let\ChapterRef\relax
\newcommand{\ChapterRef}[2]{#1}
\setcounter{tocdepth}{2}
%▓▓▓▓▓▓▓▓▓▓▓▓▓▓▓▓▓▓▓▓▓▓▓▓▓▓▓▓▓▓▓▓▓
%▓▓ ╔╦╗╦╔╦╗╦  ╔═╗  ╔═╗╔═╗╔╗╔╔╦╗ ▓▓
%▓▓  ║ ║ ║ ║  ║╣   ╠╣ ║ ║║║║ ║  ▓▓
%▓▓  ╩ ╩ ╩ ╩═╝╚═╝  ╚  ╚═╝╝╚╝ ╩  ▓▓
%▓▓▓▓▓▓▓▓▓▓▓▓▓▓▓▓▓▓▓▓▓▓▓▓▓▓▓▓▓▓▓▓▓
%\usepackage{titlesec}
%▓▓▓▓▓▓▓▓▓▓▓▓▓▓▓▓▓▓▓▓▓▓▓▓▓▓▓▓▓▓▓▓▓▓▓▓▓▓▓▓▓▓▓▓▓▓▓▓▓▓▓▓▓▓▓
%▓▓ ╔╦╗╔═╗╔╗ ╦  ╔═╗  ╔═╗╔═╗  ╔═╗╔═╗╔╗╔╔╦╗╔═╗╔╗╔╔╦╗╔═╗ ▓▓
%▓▓  ║ ╠═╣╠╩╗║  ║╣   ║ ║╠╣   ║  ║ ║║║║ ║ ║╣ ║║║ ║ ╚═╗ ▓▓
%▓▓  ╩ ╩ ╩╚═╝╩═╝╚═╝  ╚═╝╚    ╚═╝╚═╝╝╚╝ ╩ ╚═╝╝╚╝ ╩ ╚═╝ ▓▓
%▓▓▓▓▓▓▓▓▓▓▓▓▓▓▓▓▓▓▓▓▓▓▓▓▓▓▓▓▓▓▓▓▓▓▓▓▓▓▓▓▓▓▓▓▓▓▓▓▓▓▓▓▓▓▓
\newcommand{\ChapterTableOfContents}{%
    \begingroup
    \addfontfeature{Numbers={Lining,Monospaced}}
    \hypersetup{hidelinks}\tableofcontents%
    \endgroup
}%

\let\DotFill\relax
\makeatletter
\newcommand \DotFill {\leavevmode \cleaders \hb@xt@ .33em{\hss .\hss }\hfill \kern \z@}
\makeatother

\definecolor{ToCGrey}{rgb}{0.4,0.4,0.4}
\definecolor{mainColor}{rgb}{0.82745098,0.18431373,0.18431373}
\usepackage{titletoc}
\titlecontents{part}
[0.0em]
{\addvspace{1pc}\color{TitlingRed}\large\bfseries\text{Part }}
{\bfseries\textcolor{TitlingRed}{\contentslabel{0.0em}}\hspace*{1.35em}}
{}
{\textcolor{TitlingRed}{{\hfill\bfseries\contentspage\nobreak}}}
[]
\titlecontents{section}
[0.0em]
{\addvspace{1pc}}
{\color{black}\bfseries\textcolor{TitlingRed}{\contentslabel{0.0em}}\hspace*{1.65em}}
{}
{\textcolor{black}{\textbf{\DotFill}{\bfseries\contentspage\nobreak}}}
[]
\titlecontents{subsection}
[0.0em]
{}
{\hspace*{1.65em}\color{ToCGrey}{\contentslabel{0.0em}}\hspace*{2.5em}}
{}
{{\textcolor{ToCGrey}\DotFill}\textcolor{ToCGrey}{\contentspage}\nobreak}
[]
\usepackage{marginnote}
\renewcommand*{\marginfont}{\normalfont}
\usepackage{inconsolata}
\setmonofont{inconsolata}%
\let\ChapterRef\relax
\newcommand{\ChapterRef}[2]{#1}
\AtBeginEnvironment{subappendices}{%%
    \section*{\huge Appendices}%
}%

\begin{document}

\title{Topological Spaces}

\maketitle

\phantomsection
\label{section-phantom}

This chapter contains some material about topological spaces.

\ChapterTableOfContents

\section{Topologies}\label{section-topologies}
\subsection{Foundations}\label{subsection-topologies-foundations}
Let $X$ be a set.
\begin{definition}{Topologies}{topologies}%
    A \index[topology]{topology on a set}\textbf{topology on $X$} is equivalently:%
    \begin{enumerate}
        \item\label{topologies-1}A coreflective subcategory $\rmOpen(X)$ of $\mathcal{P}(X)$ for which the embedding
            \[
                \iota%
                \colon%
                \rmOpen(X)%
                \hookrightarrow%
                \mathcal{P}(X)%
            \]%
            is right-exact.

            A subset $\rmOpen(X)$ of $\mathcal{P}(X)$, called the \textbf{set of open sets of $X$}, satisfying the following conditions:
            \begin{enumerate}
                \item\label{topologies-1-a}\SloganFont{Containment of the Empty and Total Sets. }We have $\emptyset,X\in\rmOpen(X)$.%
                \item\label{topologies-1-b}\SloganFont{Closure Under Unions. }Unions of open sets are open.
                \item\label{topologies-1-c}\SloganFont{Closure Under Finite Intersections. }Finite intersections of open sets are open.
            \end{enumerate}
        \item\label{topologies-2}A subset $\rmCld(X)$ of $\mathcal{P}(X)$, called the \textbf{set of closed sets of $X$}, satisfying the following conditions:
            \begin{enumerate}
                \item\label{topologies-2-a}\SloganFont{Containment of the Empty and Total Sets. }We have $\emptyset,X\in\mathcal{T}$.%
                \item\label{topologies-2-b}\SloganFont{Closure Under Intersections. }Intersections of closed sets are closed.
                \item\label{topologies-2-c}\SloganFont{Closure Under Finite Unions. }Finite unions of closed sets are closed.
            \end{enumerate}
        \item\label{topologies-3}A collection of fundamental neighbourhoods of $X$ as in  \cref{systems-of-neighbourhoods}.
        \item\label{topologies-4}A Kuratowski closure operator  on $X$ as in \cref{kuratowski-closure-operators}.
        \item\label{topologies-5}A Kuratowski interior operator on $X$ as in \cref{kuratowski-interior-operators}.
    \end{enumerate}
\end{definition}
\begin{Proof}{Proof of the Equivalences in \cref{topologies}}%
    We claim that the notions defined by \cref{topologies-1,topologies-2,topologies-3,topologies-4,topologies-5} are indeed equivalent.

    \ProofBox{\cref{topologies-1}$\iff$\cref{topologies-2}}%
    Omitted.

    \ProofBox{\cref{topologies-1}$\iff$\cref{topologies-3}}%
    Omitted.

    \ProofBox{\cref{topologies-1}$\iff$\cref{topologies-4}}%
    Omitted.

    \ProofBox{\cref{topologies-1}$\iff$\cref{topologies-5}}%
    Omitted.
\end{Proof}
\begin{definition}{Comparing Topologies}{comparing-topologies}%
    Let $\mathcal{T},\mathcal{T}'$ be two topologies on a set $X$.\index[topology]{topology on a set!finer/coarser than another topology}%
    \begin{itemize}
        \item\label{comparing-topologies-1}The topology $\mathcal{T}'$ is \textbf{finer}%
            %--- Begin Footnote ---%
            \footnote{%
                \SloganFont{Further Terminology: }We also say that $\mathcal{T}$ is \textbf{coarser} than $\mathcal{T}'$.
            } %
            %---  End Footnote  ---%
            than $\mathcal{T}$ if $\mathcal{T}\subset\mathcal{T}'$.
        \item\label{comparing-topologies-2}The topology $\mathcal{T}'$ is \textbf{strictly finer}%
            %--- Begin Footnote ---%
            \footnote{%
                \SloganFont{Further Terminology: }We also say that $\mathcal{T}$ is \textbf{strictly coarser} than $\mathcal{T}'$.
                \par\vspace*{-1.75\baselineskip}
            } %
            %---  End Footnote  ---%
            than $\mathcal{T}$ if $\mathcal{T}\subsetneq\mathcal{T}'$.
        \item\label{comparing-topologies-3}The topology $\mathcal{T}$ and $\mathcal{T}'$ are \textbf{comparable} if either $\mathcal{T}'\subset\mathcal{T}$ or $\mathcal{T}\subset\mathcal{T}'$.
    \end{itemize}
\end{definition}
\begin{proposition}{Properties of Topologies}{properties-of-topologies}%
    Let $X$ be a set.
    \begin{enumerate}
        \item\label{properties-of-topologies-lattice-structure}\SloganFont{Lattice Structure. }The set of all topologies on $X$ together with set inclusion forms a complete lattice:
            \begin{enumerate}
                \item\label{properties-of-topologies-lattice-structure-infima}\SloganFont{Infima. }Given a family of topologies $\{\mathcal{T}_{\alpha}\}_{\alpha\in I}$ on $X$, we have
                    \[
                        \inf_{\alpha\in I}(\mathcal{T}_{\alpha})%
                        =%
                        \bigcap_{\alpha\in I}\mathcal{T}_{\alpha}.%
                    \]%
                \item\label{properties-of-topologies-lattice-structure-suprema}\SloganFont{Suprema. }Given a family of topologies $\{\mathcal{T}_{\alpha}\}_{\alpha\in I}$ on $X$ generated by subbases $\{\mathcal{B}_{\alpha}\}_{\alpha\in I}$, we have
                    \[
                        \sup_{\alpha\in I}(\mathcal{T}_{\alpha})%
                        =%
                        \langle\bigcup_{\alpha\in I}\mathcal{B}_{\alpha}\rangle,%
                    \]%
                    i.e.\ $\sup_{\alpha\in I}(\mathcal{T}_{\alpha})$ is generated by the subbasis $\bigcup_{\alpha\in I}\mathcal{B}_{\alpha}$.
            \end{enumerate}
        %\item\label{properties-of-topologies-}\SloganFont{. }
    \end{enumerate}
\end{proposition}
\begin{Proof}{Proof of \cref{properties-of-topologies}}%
    \FirstProofBox{\cref{properties-of-topologies-lattice-structure}: Lattice Structure}%
    Omitted.
\end{Proof}
\subsection{Systems of Neighbourhoods}\label{subsection-fundamental-neighbourhoods}
Let $X$ be a set.
\begin{definition}{Systems of Neighbourhoods}{systems-of-neighbourhoods}%
    A \index[topology]{system of neighbourhoods}\textbf{system of neighbourhoods of $X$}%
    %--- Begin Footnote ---%
    \footnote{%
        \SloganFont{Further Terminology: }Also called a \textbf{complete system of neighbourhoods of $X$} or a \textbf{neighbourhood filter of $X$}.
            \par\vspace*{-1.75\baselineskip}
    } %
    %---  End Footnote  ---%
    is a set $\mathcal{N}=(\mathcal{N}_{p})_{p\in X}$ satisfying the following conditions for each $p\in X$:
    \begin{enumerate}
        \item\label{systems-of-neighbourhoods-1}For each $S\in\mathcal{P}(X)$, if there exists some $N\in\mathcal{N}_{p}$ such that $N\subset S$, then $S\in\mathcal{N}_{p}$.
        \item\label{systems-of-neighbourhoods-2}For each $x\in X$, finite intersections of sets in $\mathcal{N}_{p}$ are in $\mathcal{N}_{p}$.
        \item\label{systems-of-neighbourhoods-3}For each $N\in\mathcal{N}_{p}$, we have $p\in N$.
        \item\label{systems-of-neighbourhoods-4}For each $N\in\mathcal{N}_{p}$, there exists a set $N'\in\mathcal{N}_{p}$ such that, for each $q\in N'$, we have $N\in\mathcal{N}_{q}$.%
            %--- Begin Footnote ---%
            \footnote{%
                \SloganFont{Slogan }(from \cite[p.~19]{bourbaki-general-topology-1-4})\SloganFont{: }A neighbourhood of $p$ is also a neighbourhood of all points sufficiently near $p$.
                \par\vspace*{-1.75\baselineskip}
            }%
            %---  End Footnote  ---%
    \end{enumerate}
\end{definition}
\begin{proposition}{Properties of Systems of Neighbourhoods}{properties-of-systems-of-neighbourhoods}%
    Let $X$ be a set.
    \begin{enumerate}
        \item\label{properties-of-systems-of-neighbourhoods-equivalence-with-topologies}\SloganFont{Equivalence With Topologies. }We have a bijection
            \[
                \{%
                    \begin{gathered}
                        \text{systems of neigh-}\\%
                        \text{bourhoods of $X$}%
                    \end{gathered}
                \}%
                \cong
                \{\text{topologies on $X$}\}%
            \]%
            given by:
            \begin{enumerate}
                \item\label{properties-of-systems-of-neighbourhoods-equivalence-with-topological-spaces-1}Sending a system of neighbourhoods $\mathcal{N}$ of $X$ to the topology $\mathcal{T}^{\mathcal{N}}_{X}$ on $X$ defined by
                    \[
                        \mathcal{T}^{\mathcal{N}}_{X}%
                        \defeq%
                        \{%
                            U\in\mathcal{P}(X)%
                            \ \middle|\ %
                            \text{for each $p\in U$, we have $U\in\mathcal{N}(p)$}%
                        \},%
                    \]%
                    following \cref{properties-of-neighbourhoods-characterisations-of-open-neighbourhoods} of \cref{properties-of-neighbourhoods}.
                \item\label{properties-of-systems-of-neighbourhoods-equivalence-with-topological-spaces-2}Sending a topology $\mathcal{T}$ on $X$ to the system of neighbourhoods $\mathcal{N}_{\mathcal{T}}=(\mathcal{N}_{\mathcal{T}}(p))_{p\in X}$ defined by
                    \[
                        \mathcal{N}_{\mathcal{T}}(p)%
                        \defeq%
                        \{%
                            N\in\mathcal{P}(X)%
                            \ \middle|\ %
                            \text{$N$ is a neighbourhood of $p$ in $X$}%
                        \},%
                    \]%
                    where a neighbourhood of $p$ in $X$ is defined as in \cref{neighbourhoods}.
            \end{enumerate}
        %\item\label{properties-of-systems-of-neighbourhoods-}\SloganFont{. }
    \end{enumerate}
\end{proposition}
\begin{Proof}{Proof of \cref{properties-of-systems-of-neighbourhoods}}%
    \FirstProofBox{\cref{properties-of-systems-of-neighbourhoods-equivalence-with-topologies}: Equivalence With Topologies}%
    See \cite[Proposition 2 in p.~19]{bourbaki-general-topology-1-4}.
\end{Proof}
\subsection{Kuratowski Closure Operators}\label{subsection-kuratowski-closure-operators}
Let $X$ be a set.
\begin{definition}{Kuratowski Closure Operators}{kuratowski-closure-operators}%
    A \index[topology]{Kuratowski closure operator}\textbf{Kuratowski closure operator on $X$} is a right exact idempotent monad on $\mathcal{P}(X)$.
\end{definition}
\begin{remark}{Unwinding \cref{kuratowski-closure-operators}}{unwinding-kuratowski-closure-operators}%
    In detail, a \textbf{Kuratowski closure operator on $X$} is a functor
    \[
        \rmCl_{X}%
        \colon%
        \mathcal{P}(X)%
        \to%
        \mathcal{P}(X),%
    \]%
    satisfying the following conditions:%
    %--- Begin Footnote ---%
    \footnote{%
        We can also express some separation conditions on $X$ in terms of closure operators:
        \begin{itemize}
            \item The topological space $X$ is $\rmT_{0}$ \textiff $x\neq y$ implies $\rmCl_{X}(\{x\})\neq\rmCl_{X}(\{y\})$.
            \item The topological space $X$ is $\rmT_{1}$ \textiff for each $p\in X$, we have $\rmCl_{X}(\{p\})=\{p\}$.
            \item The topological space $X$ is $\rmT_{2}$ \textiff $x\neq y$ implies there exists $U\in\mathcal{P}(X)$ such that $x\nin\rmCl_{X}(U)$ and $y\nin\rmCl_{Y}(X\setminus U)$.
        \end{itemize}
        \par\vspace*{-2.0\baselineskip}
    }%
    %---  End Footnote  ---%
    \begin{enumerate}
        \item\label{unwinding-kuratowski-closure-operators-functoriality}\SloganFont{Functoriality. }For each $U,V\in\mathcal{P}(X)$, if $U\subset V$, then $\rmCl_{X}(U)\subset\rmCl_{X}(V)$.
        \item\label{unwinding-kuratowski-closure-operators-idempotency}\SloganFont{Idempotency. }For each $U\in\mathcal{P}(A)$, we have $\rmCl_{X}(\rmCl_{X}(A))=\rmCl_{X}(A)$.
        \item\label{unwinding-kuratowski-closure-operators-unitality}\SloganFont{Unitality. }We have a natural transformation
            \[
                \eta%
                \colon%
                \id_{\mathcal{P}(X)}%
                \Rightarrow%
                \rmCl_{X}%
            \]%
            in $\Fun(\mathcal{P}(X),\mathcal{P}(X))$, having components%
            \[
                \eta_{U}%
                \colon%
                U%
                \subset%
                \rmCl_{X}(U)%
            \]%
            with $U\in\mathcal{P}(X)$.
        \item\label{unwinding-kuratowski-closure-operators-right-exactness}\SloganFont{Right Exactness. }The functor $\rmCl_{X}$ is right exact:
            \begin{enumerate}
                \item\label{unwinding-kuratowski-closure-operators-right-exactness-preservation-of-the-empty-set}\SloganFont{Preservation of the Empty Set. }We have $\rmCl_{X}(\emptyset)=\emptyset$.
                \item\label{unwinding-kuratowski-closure-operators-right-exactness-preservation-of-finite-unions}\SloganFont{Preservation of Finite Unions. }We have
                    \[
                        \rmCl_{X}(U\cup V)%
                        =%
                        \rmCl_{X}(U)\cup\rmCl_{X}(V)%
                    \]%
                    for each $U,V\in\mathcal{P}(X)$.
            \end{enumerate}
    \end{enumerate}
\end{remark}
\subsection{Kuratowski Interior Operators}\label{subsection-kuratowski-interior-operators}
Let $X$ be a set.
\begin{definition}{Kuratowski Interior Operators}{kuratowski-interior-operators}%
    A \index[topology]{Kuratowski interior operator}\textbf{Kuratowski interior operator on $X$} is a left exact idempotent comonad on $\mathcal{P}(X)$.
\end{definition}
\begin{remark}{Unwinding \cref{kuratowski-interior-operators}}{unwinding-kuratowski-interior-operators}%
    In detail, a \textbf{Kuratowski interior operator on $X$} is a functor
    \[
        \rmint_{X}%
        \colon%
        \mathcal{P}(X)%
        \to%
        \mathcal{P}(X),%
    \]%
    satisfying the following conditions:
    \begin{enumerate}
        \item\label{unwinding-kuratowski-interior-operators-functoriality}\SloganFont{Functoriality. }For each $U,V\in\mathcal{P}(X)$, if $U\subset V$, then $\rmint_{X}(U)\subset\rmint_{X}(V)$.
        \item\label{unwinding-kuratowski-interior-operators-idempotency}\SloganFont{Idempotency. }For each $U\in\mathcal{P}(A)$, we have $\rmint_{X}(A)=\rmint_{X}(\rmint_{X}(A))$.
        \item\label{unwinding-kuratowski-interior-operators-counitality}\SloganFont{Counitality. }We have a natural transformation
            \[
                \epsilon%
                \colon%
                \rmint_{X}%
                \Rightarrow%
                \id_{\mathcal{P}(X)}%
            \]%
            in $\Fun(\mathcal{P}(X),\mathcal{P}(X))$, having components
            \[
                \epsilon_{U}%
                \colon%
                \rmint_{X}(U)%
                \subset%
                U%
            \]%
            with $U\in\mathcal{P}(X)$.
        \item\label{unwinding-kuratowski-interior-operators-left-exactness}\SloganFont{Left Exactness. }The functor $\rmint_{X}$ is left exact:
            \begin{enumerate}
                \item\label{unwinding-kuratowski-interior-operators-left-exactness-preservation-of-the-total-set}\SloganFont{Preservation of the Total Set. }We have $\rmint_{X}(X)=X$.
                \item\label{unwinding-kuratowski-interior-operators-left-exactness-preservation-of-finite-intersections}\SloganFont{Preservation of Finite Intersections. }We have
                    \[
                        \rmint_{X}(U\cap V)%
                        =%
                        \rmint_{X}(U)\cap\rmint_{X}(V)%
                    \]%
                    for each $U,V\in\mathcal{P}(X)$.
            \end{enumerate}
    \end{enumerate}
\end{remark}
\begin{appendices}
\input{ABSOLUTEPATH/chapters2.tex}
\end{appendices}
\end{document}
