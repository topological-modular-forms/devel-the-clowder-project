\input{preamble}

% OK, start here.
%
\usepackage{fontspec}
\let\hyperwhite\relax
\let\hyperred\relax
\newcommand{\hyperwhite}{\hypersetup{citecolor=white,filecolor=white,linkcolor=white,urlcolor=white}}
\newcommand{\hyperred}{%
\hypersetup{%
    citecolor=TitlingRed,%
    filecolor=TitlingRed,%
    linkcolor=TitlingRed,%
     urlcolor=TitlingRed%
}}
\let\ChapterRef\relax
\newcommand{\ChapterRef}[2]{#1}
\setcounter{tocdepth}{2}
%▓▓▓▓▓▓▓▓▓▓▓▓▓▓▓▓▓▓▓▓▓▓▓▓▓▓▓▓▓▓▓▓▓
%▓▓ ╔╦╗╦╔╦╗╦  ╔═╗  ╔═╗╔═╗╔╗╔╔╦╗ ▓▓
%▓▓  ║ ║ ║ ║  ║╣   ╠╣ ║ ║║║║ ║  ▓▓
%▓▓  ╩ ╩ ╩ ╩═╝╚═╝  ╚  ╚═╝╝╚╝ ╩  ▓▓
%▓▓▓▓▓▓▓▓▓▓▓▓▓▓▓▓▓▓▓▓▓▓▓▓▓▓▓▓▓▓▓▓▓
%\usepackage{titlesec}
%▓▓▓▓▓▓▓▓▓▓▓▓▓▓▓▓▓▓▓▓▓▓▓▓▓▓▓▓▓▓▓▓▓▓▓▓▓▓▓▓▓▓▓▓▓▓▓▓▓▓▓▓▓▓▓
%▓▓ ╔╦╗╔═╗╔╗ ╦  ╔═╗  ╔═╗╔═╗  ╔═╗╔═╗╔╗╔╔╦╗╔═╗╔╗╔╔╦╗╔═╗ ▓▓
%▓▓  ║ ╠═╣╠╩╗║  ║╣   ║ ║╠╣   ║  ║ ║║║║ ║ ║╣ ║║║ ║ ╚═╗ ▓▓
%▓▓  ╩ ╩ ╩╚═╝╩═╝╚═╝  ╚═╝╚    ╚═╝╚═╝╝╚╝ ╩ ╚═╝╝╚╝ ╩ ╚═╝ ▓▓
%▓▓▓▓▓▓▓▓▓▓▓▓▓▓▓▓▓▓▓▓▓▓▓▓▓▓▓▓▓▓▓▓▓▓▓▓▓▓▓▓▓▓▓▓▓▓▓▓▓▓▓▓▓▓▓
\newcommand{\ChapterTableOfContents}{%
    \begingroup
    \addfontfeature{Numbers={Lining,Monospaced}}
    \hypersetup{hidelinks}\tableofcontents%
    \endgroup
}%

\let\DotFill\relax
\makeatletter
\newcommand \DotFill {\leavevmode \cleaders \hb@xt@ .33em{\hss .\hss }\hfill \kern \z@}
\makeatother

\definecolor{ToCGrey}{rgb}{0.4,0.4,0.4}
\definecolor{mainColor}{rgb}{0.82745098,0.18431373,0.18431373}
\usepackage{titletoc}
\titlecontents{part}
[0.0em]
{\addvspace{1pc}\color{TitlingRed}\large\bfseries\text{Part }}
{\bfseries\textcolor{TitlingRed}{\contentslabel{0.0em}}\hspace*{1.35em}}
{}
{\textcolor{TitlingRed}{{\hfill\bfseries\contentspage\nobreak}}}
[]
\titlecontents{section}
[0.0em]
{\addvspace{1pc}}
{\color{black}\bfseries\textcolor{TitlingRed}{\contentslabel{0.0em}}\hspace*{1.65em}}
{}
{\textcolor{black}{\textbf{\DotFill}{\bfseries\contentspage\nobreak}}}
[]
\titlecontents{subsection}
[0.0em]
{}
{\hspace*{1.65em}\color{ToCGrey}{\contentslabel{0.0em}}\hspace*{2.5em}}
{}
{{\textcolor{ToCGrey}\DotFill}\textcolor{ToCGrey}{\contentspage}\nobreak}
[]
\usepackage{marginnote}
\renewcommand*{\marginfont}{\normalfont}
\usepackage{inconsolata}
\setmonofont{inconsolata}%
\let\ChapterRef\relax
\newcommand{\ChapterRef}[2]{#1}
\AtBeginEnvironment{subappendices}{%%
    \section*{\huge Appendices}%
}%

\begin{document}

\title{Homotopy Type Theory}

\maketitle

\phantomsection
\label{section-phantom}

This chapter contains material on homotopy type theory. Note: the current material is heavily based on Egbert Rijke's book, as I'm writing this material as I go through it.

\ChapterTableOfContents

TODO:
\begin{enumerate}
    \item AAAAA
\end{enumerate}

\section{Equivalences}\label{section-hott-equivalences}
\subsection{Homotopies}\label{subsection-hott-homotopies}
Let $f,g\oftype\prod_{x\oftype A}B(x)$ be dependent functions.
\begin{definition}{Homotopies}{hott-homotopies}%
    The \index[type-theory]{type of homotopies}\textbf{type of homotopies} from $f$ to $g$ is the type \index[notation]{fAsimB@$f\htpy_{[A,B]}g$}$f\htpy_{[A,B]}g$ defined by%
    %--- Begin Footnote ---%
    \footnote{%
        \SloganFont{Further Terminology: }A term of type $f\htpy_{[A,B]}g$ is called a \index[type-theory]{homotopy}\textbf{homotopy} from $f$ to $g$.
    }%
    %---  End Footnote  ---%
    %--- Begin Footnote ---%
    \footnote{%
        \SloganFont{Further Notation: }We will sometimes simply write \index[notation]{fsimg@$f\sim g$}$f\sim g$ for $f\htpy_{[A,B]}g$.
        \par\vspace*{\TCBBoxCorrection}
    }%
    %---  End Footnote  ---%
    \[
        f\subhtpy{A}{B}g%
        \defeq%
        \prod_{\chigh{x\oftype A}}\chigh{f(x)\subequals{B(x)}g(x)}.%
    \]%
\end{definition}
\begin{remark}{Commutativity of Diagrams}{hott-commutativity-of-diagrams}%
    Homotopies are used to express commutativity for diagrams. For example:
    \begin{itemize}
        \item Saying that the triangle
            \[
                \begin{tikzcd}[row sep={3.75*\the\DL,between origins}, column sep={2.5*\the\DL,between origins}, background color=backgroundColor, ampersand replacement=\&]
                    A
                    \arrow[rr,"h"]
                    \arrow[rd,"f"']
                    \&
                    \&
                    B
                    \arrow[ld,"g"]
                    \\
                    \&
                    X%
                    \&
                \end{tikzcd}
            \]%
            \textbf{commutes} means that there exists a homotopy $H\oftype f\htpy_{[A,X]}g\circ h$.%
        \item Saying that the square
            \[
                \begin{tikzcd}[row sep={5.0*\the\DL,between origins}, column sep={5.0*\the\DL,between origins}, background color=backgroundColor, ampersand replacement=\&]
                    A
                    \arrow[r,"f"]
                    \arrow[d,"\phi"']
                    \&
                    B
                    \arrow[d,"\psi"]
                    \\
                    X
                    \arrow[r,"g"']
                    \&
                    Y
                \end{tikzcd}
            \]%
            commutes means that there exists a homotopy $H\oftype g\circ\phi\htpy_{[A,Y]}\psi\circ f$.
    \end{itemize}
\end{remark}
\begin{remark}{Higher Homotopies}{hott-higher-homotopies}%
    Given homotopies $H,K\oftype f\htpy_{[A,B]}g$, we may also consider the type
    \[
        H\htpy_{[f,g]}K%
        \defeq%
        \prod_{x\oftype A}H(x)\subequals{f(x)\subequals{B(x)}g(x)}K(x).%
    \]%
    of homotopies between $H$ and $K$.
\end{remark}
\begin{proposition}{The Groupoidal Structure of Homotopies}{the-groupoidal-structure-of-homotopies}%
    Let $B$ be a type family over $A$.
    \begin{enumerate}
        \item\label{the-groupoidal-structure-of-homotopies-reflexivity}\SloganFont{Reflexivity. }There exists a homotopy
            \[
                \reflhtpy_{A,B}%
                \oftype%
                \prod_{\scriptdisplaystyle{f\oftype\prod_{x\oftype A}B(x)}}f\subhtpy{A}{B}f%
            \]%
            defined by
            \[
                \reflhtpy_{A,B}(f)%
                \defeq%
                \llbracket x\mapsto\refl_{f(x)}\rrbracket.%
            \]%
        \item\label{the-groupoidal-structure-of-homotopies-inversion}\SloganFont{Inversion. }There exists a homotopy
            \[
                \invhtpy_{A,B}%
                \oftype%
                \prod_{\chigh{\scriptdisplaystyle{f,g\oftype\prod_{x\oftype A}B(x)}}}\chigh{(f\subhtpy{A}{B}g)\to(g\subhtpy{A}{B}f)}%
            \]%
            defined by
            \[
                \invhtpy_{A,B}(H)%
                \defeq%
                \llbracket x\mapsto H(x)^{-1}\rrbracket.%
            \]%
        %\item\label{the-groupoidal-structure-of-homotopies-}\SloganFont{. }
    \end{enumerate}
\end{proposition}
\begin{Proof}{Proof of \cref{the-groupoidal-structure-of-homotopies}}%
    \FirstProofBox{\cref{the-groupoidal-structure-of-homotopies-reflexivity}: Reflexivity}%
    There is nothing to prove.

    \ProofBox{\cref{the-groupoidal-structure-of-homotopies-inversion}: Inversion}%
    There is nothing to prove.
\end{Proof}
\begin{appendices}
\input{ABSOLUTEPATH/chapters2.tex}
\end{appendices}
\end{document}
