\input{preamble}

% OK, start here.
%
\usepackage{fontspec}
\let\hyperwhite\relax
\let\hyperred\relax
\newcommand{\hyperwhite}{\hypersetup{citecolor=white,filecolor=white,linkcolor=white,urlcolor=white}}
\newcommand{\hyperred}{%
\hypersetup{%
    citecolor=TitlingRed,%
    filecolor=TitlingRed,%
    linkcolor=TitlingRed,%
     urlcolor=TitlingRed%
}}
\let\ChapterRef\relax
\newcommand{\ChapterRef}[2]{#1}
\setcounter{tocdepth}{2}
%▓▓▓▓▓▓▓▓▓▓▓▓▓▓▓▓▓▓▓▓▓▓▓▓▓▓▓▓▓▓▓▓▓
%▓▓ ╔╦╗╦╔╦╗╦  ╔═╗  ╔═╗╔═╗╔╗╔╔╦╗ ▓▓
%▓▓  ║ ║ ║ ║  ║╣   ╠╣ ║ ║║║║ ║  ▓▓
%▓▓  ╩ ╩ ╩ ╩═╝╚═╝  ╚  ╚═╝╝╚╝ ╩  ▓▓
%▓▓▓▓▓▓▓▓▓▓▓▓▓▓▓▓▓▓▓▓▓▓▓▓▓▓▓▓▓▓▓▓▓
%\usepackage{titlesec}
%▓▓▓▓▓▓▓▓▓▓▓▓▓▓▓▓▓▓▓▓▓▓▓▓▓▓▓▓▓▓▓▓▓▓▓▓▓▓▓▓▓▓▓▓▓▓▓▓▓▓▓▓▓▓▓
%▓▓ ╔╦╗╔═╗╔╗ ╦  ╔═╗  ╔═╗╔═╗  ╔═╗╔═╗╔╗╔╔╦╗╔═╗╔╗╔╔╦╗╔═╗ ▓▓
%▓▓  ║ ╠═╣╠╩╗║  ║╣   ║ ║╠╣   ║  ║ ║║║║ ║ ║╣ ║║║ ║ ╚═╗ ▓▓
%▓▓  ╩ ╩ ╩╚═╝╩═╝╚═╝  ╚═╝╚    ╚═╝╚═╝╝╚╝ ╩ ╚═╝╝╚╝ ╩ ╚═╝ ▓▓
%▓▓▓▓▓▓▓▓▓▓▓▓▓▓▓▓▓▓▓▓▓▓▓▓▓▓▓▓▓▓▓▓▓▓▓▓▓▓▓▓▓▓▓▓▓▓▓▓▓▓▓▓▓▓▓
\newcommand{\ChapterTableOfContents}{%
    \begingroup
    \addfontfeature{Numbers={Lining,Monospaced}}
    \hypersetup{hidelinks}\tableofcontents%
    \endgroup
}%

\let\DotFill\relax
\makeatletter
\newcommand \DotFill {\leavevmode \cleaders \hb@xt@ .33em{\hss .\hss }\hfill \kern \z@}
\makeatother

\definecolor{ToCGrey}{rgb}{0.4,0.4,0.4}
\definecolor{mainColor}{rgb}{0.82745098,0.18431373,0.18431373}
\usepackage{titletoc}
\titlecontents{part}
[0.0em]
{\addvspace{1pc}\color{TitlingRed}\large\bfseries\text{Part }}
{\bfseries\textcolor{TitlingRed}{\contentslabel{0.0em}}\hspace*{1.35em}}
{}
{\textcolor{TitlingRed}{{\hfill\bfseries\contentspage\nobreak}}}
[]
\titlecontents{section}
[0.0em]
{\addvspace{1pc}}
{\color{black}\bfseries\textcolor{TitlingRed}{\contentslabel{0.0em}}\hspace*{1.65em}}
{}
{\textcolor{black}{\textbf{\DotFill}{\bfseries\contentspage\nobreak}}}
[]
\titlecontents{subsection}
[0.0em]
{}
{\hspace*{1.65em}\color{ToCGrey}{\contentslabel{0.0em}}\hspace*{2.5em}}
{}
{{\textcolor{ToCGrey}\DotFill}\textcolor{ToCGrey}{\contentspage}\nobreak}
[]
\usepackage{marginnote}
\renewcommand*{\marginfont}{\normalfont}
\usepackage{inconsolata}
\setmonofont{inconsolata}%
\let\ChapterRef\relax
\newcommand{\ChapterRef}[2]{#1}
\AtBeginEnvironment{subappendices}{%%
    \section*{\huge Appendices}%
}%

\begin{document}

\title{Centres and Traces of Categories}

\maketitle

\phantomsection
\label{section-phantom}

This chapter contains some material about centres and traces of categories. Notably, we discuss and explore:

TODO:
\begin{enumerate}
    \item The trace via the cyclic nerve (and relation to cyclic coends). 1-skeleton has the pure relations $p\circ q\sim q\circ p$ come out of the $d_{0}$ and $d_{1}$ maps. The graph associated to this 1-skeleton is to $\Tr(\CatFont{C})$ as $\Fun(\B\N,\CatFont{C})$ is to $\pi_{0}\Fun(\B\N,\CatFont{C})$.
    \item \url{https://arxiv.org/abs/2109.02144} --- this one is very important!!
    \item Centre and trace of the core of the category of finite $\F_{q}$-vector spaces and isomorphisms
    \item Centre and trace of divisibility posets $D_{n}=\{k\in\N\ \middle|\ k\mid n\}$.
    \item Centre and trace of divisibility posets $D_{n}=\{k\in\N\ \middle|\ k\leq n\}$.
    \item $\pi_{0}(\Fun(\B\N,\CatFont{C}))$ and $\K_{0}(\Fun(\B\N,\CatFont{C}))$ for all categories $\CatFont{C}$ below.
    \item centre and trace of full subcategory of $\Grp$ spanned by the cyclic   groups
    \item centre and trace of full subcategory of $\Grp$ spanned by the dihedral groups
    \item centre and trace of $\FinSets_{G}$, $\Rep(G)$, orbit category of $G$, etc.
        \begin{enumerate}
            \item centre of $\Sets_{A}$ is $A$
        \end{enumerate}
    \item centre of $\GrMod{G}{R}$: morphisms $G\to R$
    \item centre of $\LMod{R}$: elements of $\rmZ(R)$
    \item centre of $\GrLMod{G}{R}$: morphisms $G\to\rmZ(R)$
    \item \url{https://math.stackexchange.com/questions/3085187/a-coend-in-the-category-of-vector-spaces}
    \item Idea: decomposing the trace in a spectrum of different classes, like we do with operators, decomposing them into a point spectrum, a continuous spectrum, and a residual spectrum.
    \item \say{Idempotent spectrum of the trace}: If $f\sim g$ with $f$ and $g$ idempotent endomorphisms in $\CatFont{C}$, then $\Im(f)\cong\Im(g)$.
    \item Split idempotents get identified with the identity
    \item Sending a morphism to its fixed points can often be used to show that $\tr(\id_{A})\neq\tr(\id_{B})$. (Can we generalise this using points like $\pt\to A$?). E.g.\ If $f\sim g$, then $\#\mathrm{Fix}(f)=\#\mathrm{Fix}(g)$.
    \item What is the trace of the category of divisibility posets?
    \item What is the trace of the full subcategory of $\Pos$ spanned by the posets of the form $[n]\times[m]$?
    \item Be careful with adding the above to Clowder if the relevant problem gets accepted by Epoch AI and uses the stuff above.
    \item Every endomorphism is equivalent to its eventual image in $\Tr(\CatFont{C})$, since the diagram
        \[
            \begin{tikzcd}[row sep={5.0*\the\DL,between origins}, column sep={5.0*\the\DL,between origins}, background color=backgroundColor, ampersand replacement=\&]
                \Im_{\infty}(f)
                \arrow[r,hook']
                \arrow[d,"\restriction{f}{\Im_{\infty}(f)}"']
                \&
                X
                \arrow[d,"f"]
                \arrow[ld,"f"description]
                \\
                \Im_{\infty}(f)
                \arrow[r,hook]
                \&
                X
            \end{tikzcd}
        \]%
        commutes.
    \item Eventual image:
        \begin{enumerate}
            \item \url{https://golem.ph.utexas.edu/category/2011/12/the_eventual_image_part_2.html}
            \item \url{https://golem.ph.utexas.edu/category/2011/12/the_eventual_image.html}
            \item \url{https://golem.ph.utexas.edu/category/2011/10/can_you_describe_this_idempote.html}
            \item \url{https://golem.ph.utexas.edu/category/2022/10/the_eventual_image_eventually.html}
            \item \url{https://ncatlab.org/nlab/show/eventual+image}
                \url{https://arxiv.org/abs/2210.00302}
            \item Left and right adjoints of the functor $\Fun(\B\bbB,\CatFont{C})\hookrightarrow\Fun(\B\N,\CatFont{C})$?
            \item Is there a condition for when $\Tr(\CatFont{C})\cong\pi_{0}(\B\N,\Core(\CatFont{C}))$? This happens for finite sets via the eventual image, and maybe it happens for finite abelian groups as well.
        \end{enumerate}
    \item $\Tr(\IrrRep_{\C}(G))\cong\C^{n}$ for $n$ the number of irreducible representations of $G$.
    \item Does Karoubi completion preserve traces?
    \item simplicial Hochschild homology of a category $\CatFont{C}$
\end{enumerate}

\ChapterTableOfContents

\section{Centres, Isomorphism Classes and Traces}\label{section-centres-isomorphism-classes-and-traces}
\subsection{The Centre of a Category}\label{subsection-the-centre-of-a-category}
Let $\CatFont{C}$ be a category.
\begin{definition}{The Centre of a Category}{the-centre-of-a-category}%
    The \index[categories]{centre of a category}\index[categories]{category!centre of}\textbf{centre} of $\CatFont{C}$ is the monoid \index[notation]{ZC@$\rmZ(\CatFont{C})$}$\rmZ(\CatFont{C})$ consisting of
    \begin{itemize}
        \item\SloganFont{The Underlying Set. }The set $\rmZ(\CatFont{C})$ defined by
            \begin{align*}
                \rmZ(\CatFont{C}) &\defeq \int_{A\in\CatFont{C}}\Hom_{\CatFont{C}}(A,A)\\
                                  &\cong  \Nat(\id_{\CatFont{C}},\id_{\CatFont{C}}),
            \end{align*}
            where the isomorphism comes from \ChapterRef{\ChapterEndsAndCoends, \cref{ends-and-coends:properties-of-co-ends-natural-transformations-via-ends} of \cref{ends-and-coends:properties-of-co-ends}}{\cref{properties-of-co-ends-natural-transformations-via-ends} of \cref{properties-of-co-ends}}.
        \item\SloganFont{The Multiplication Map. }The map
            \[
                \mu_{\rmZ(\CatFont{C})}%
                \colon%
                \rmZ(\CatFont{C})\times\rmZ(\CatFont{C})%
                \to%
                \rmZ(\CatFont{C})%
            \]%
            defined by
            \[
                \beta\cdot_{\rmZ(\CatFont{C})}\alpha%
                \defeq%
                \beta\circ\alpha%
            \]%
            for each $\alpha,\beta\in\rmZ(\CatFont{C})$, where $\beta\circ\alpha$ is the vertical composition of $\alpha$ and $\beta$ of \ChapterRef{\ChapterCategories, \cref{categories:vertical-composition-of-natural-transformations}}{\cref{vertical-composition-of-natural-transformations}}.
        \item\SloganFont{The Unit Map. }The map
            \[
                \eta_{\rmZ(\CatFont{C})}%
                \colon%
                \pt%
                \to%
                \rmZ(\CatFont{C})%
            \]%
            picking the identity natural transformation
            \[
                \id_{\id_{\CatFont{C}}}%
                \colon%
                \id_{\CatFont{C}}%
                \Rightarrow%
                \id_{\CatFont{C}}%
            \]%
            of \ChapterRef{\ChapterCategories, \cref{categories:identity-natural-transformations}}{\cref{identity-natural-transformations}}.
    \end{itemize}
\end{definition}
\begin{remark}{Unwinding \cref{the-centre-of-a-category}}{unwinding-the-centre-of-a-category}%
    The elements of $\rmZ(\CatFont{C})$ are natural transformation from $\id_{\CatFont{C}}$ to itself. These consist of collections
    \[
        \{%
            \alpha_{A}%
            \colon%
            A%
            \to%
            A%
        \}_{A\in\Obj(\CatFont{C})}%
    \]%
    of morphisms of $\CatFont{C}$ such that, for each morphism $f\colon A\to B$ of $\CatFont{C}$, the diagram
    \[
        \begin{tikzcd}[row sep={5.0*\the\DL,between origins}, column sep={5.0*\the\DL,between origins}, background color=backgroundColor, ampersand replacement=\&]
            A
            \arrow[r,"f"]
            \arrow[d,"\alpha_{A}"']
            \&
            B
            \arrow[d,"\alpha_{B}"]
            \\
            A
            \arrow[r,"f"']
            \&
            B
        \end{tikzcd}
    \]%
    commutes.
\end{remark}
\begin{example}{Examples of Centres of Categories}{examples-of-centres-of-categories}%
    Here are some examples of centres of categories.
    \begin{enumerate}
        \item\label{examples-of-centres-of-categories-the-centre-of-a-monoid}\SloganFont{The Centre of a Monoid. }Let $A$ be a monoid. We have
            \begin{align*}
                \rmZ(\B{A}) &\cong  \rmZ(A)\\
                            &\defeq \{a\in A\ \middle|\ \text{for each $b\in A$, we have $ab=ba$}\}.
            \end{align*}
        \item\label{examples-of-centres-of-categories-the-centre-of-the-simplex-category}\SloganFont{The Centre of the Simplex Category. }We have
            \[
                \rmZ(\SimplexCategory)%
                \cong%
                \pt.%
            \]%
        \item\label{examples-of-centres-of-categories-the-centre-of-the-cube-category}\SloganFont{The Centre of the Cube Category. }We have
        \item\label{examples-of-centres-of-categories-the-centre-of-the-cube-category-with-connections}\SloganFont{The Centre of the Cube Category With Connections. }We have
        \item\label{examples-of-centres-of-categories-the-centre-of-the-category-of-finite-sets}\SloganFont{The Centre of $\FinSets$. }We have
        \item\label{examples-of-centres-of-categories-the-centre-of-the-category-of-sets}\SloganFont{The Centre of $\Sets$. }We have
        \item\label{examples-of-centres-of-categories-the-centre-of-the-category-of-groups}\SloganFont{The Centre of $\Grp$. }
        \item\label{examples-of-centres-of-categories-the-centre-of-the-category-of-abelian-groups}\SloganFont{The Centre of $\Ab$. }
    \end{enumerate}
\end{example}
\begin{Proof}{Proof of \cref{examples-of-centres-of-categories}}%
    \FirstProofBox{\cref{examples-of-centres-of-categories-the-centre-of-a-monoid}: The Centre of a Monoid}%

    \ProofBox{\cref{examples-of-centres-of-categories-the-centre-of-the-simplex-category}: The Centre of the Simplex Category}%
\end{Proof}
\begin{proposition}{Properties of the Centre of a Category}{properties-of-the-centre-of-a-category}%
    Let $\CatFont{C}$ be a category.
    \begin{enumerate}
        \item\label{properties-of-the-centre-of-a-category-interaction-with-monoidal-categories}\SloganFont{Interaction With Monoidal Categories. }
        %\item\label{properties-of-the-centre-of-a-category-}\SloganFont{. }
    \end{enumerate}
\end{proposition}
\begin{Proof}{Proof of \cref{properties-of-the-centre-of-a-category}}%
    \FirstProofBox{\cref{properties-of-the-centre-of-a-category-interaction-with-monoidal-categories}: Interaction With Monoidal Categories}%
    Omitted.
\end{Proof}
\subsection{The Set of Isomorphism Classes of a Category}\label{subsection-the-set-of-isomorphism-classes-of-a-category}
\section{The Trace of a Category}\label{section-the-trace-of-a-category}
\subsection{Foundations}\label{subsection-the-trace-of-a-category-foundations}
\begin{definition}{The Trace of a Category}{the-trace-of-a-category}%
    Let $\CatFont{C}$ be a small category.
    \begin{enumerate}
        \item\label{the-trace-of-a-category-the-trace-of-a-category}The \index[categories]{trace!of a category}\textbf{trace of $\CatFont{C}$} is the set \index[notation]{TrC@$\Tr(\CatFont{C})$}$\Tr(\CatFont{C})$ defined by%
            \begin{align*}%
                \Tr(\CatFont{C}) &\defeq \int^{A\in\CatFont{C}}\Hom_{\CatFont{C}}(A,A)\\%
                                 &\cong  \rmEnd(\CatFont{C})/\unsim,%
            \end{align*}%
            where $\unsim$ is the equivalence relation on $\rmEnd(\CatFont{C})$ generated by $f\circ g\sim g\circ f$.%
        \item\label{the-trace-of-a-category-the-trace-map}The \index[categories]{trace map!of a category}\textbf{trace map of $\CatFont{C}$} is the map\index[notation]{trC@$\tr_{\CatFont{C}}$}%
            %--- Begin Footnote ---%
            \footnote{%
                \SloganFont{Further Notation: }Also written simply \index[notation]{tr@$\tr$}$\tr$.
                \par\vspace*{-1.75\baselineskip}
            }%
            %---  End Footnote  ---%
            \[
                \tr_{\CatFont{C}}%
                \colon%
                \rmEnd(\CatFont{C})%
                \to%
                \Tr(\CatFont{C})%
            \]%
            given by the quotient map $\rmEnd(\CatFont{C})\twoheadsrightarrow\rmEnd(\CatFont{C})/\unsim\cong\Tr(\CatFont{C})$.
    \end{enumerate}
\end{definition}
\begin{remark}{On the Name \say{Trace}}{on-the-name-trace}%
    The name \say{trace} comes from the fact that the \emph{trace} of a matrix is the unique (up to scaling) linear map
    \[
        \Tr
        \colon
        \Mat_{n\times n}(V)
        \longrightarrow
        k
    \]%
    satisfying
    \[
        \Tr(\mathbf{A}\mathbf{B})%
        =%
        \Tr(\mathbf{B}\mathbf{A})%
    \]%
    for each $\mathbf{A},\mathbf{B}\in\Mat_{n\times n}(V)$; see \cite[Exercise 2]{trace-as-an-alternative-decategorification-functor}. In other words, it is the unique (up to scaling) linear map
    \[
        \Tr
        \colon
        \rmEnd_{k}(V)
        \longrightarrow
        k
    \]%
    satisfying
    \[
        \Tr(f\circ g)
        =
        \Tr(g\circ f)
    \]%
    for each $f,g\in\rmEnd_{k}(V)$.
\end{remark}
\begin{remark}{Unwinding \cref{the-trace-of-a-category}}{unwinding-the-trace-of-a-category}%
    In detail, the \textbf{trace of $\CatFont{C}$} is the set $\Tr(\CatFont{C})$ given by
    \begin{webcompile}
        \begin{aligned}
            \Tr(\CatFont{C}) &\defeq \int^{A\in\CatFont{C}}\Hom_{\CatFont{C}}(A,A)\\
                             &\cong  \CoEq\mspace{3.0mu}(%
                                                            \begin{tikzcd}[row sep={10.0*\the\DL,between origins}, column sep={10.0*\the\DL,between origins}, background color=backgroundColor, ampersand replacement=\&,cramped]
                                                                \displaystyle\coprod_{\substack{A,B\in\Obj(\CatFont{C})\\f\in\Hom_{\CatFont{C}}(A,B)}}h^{B}_{A}
                                                                \arrow[r,"{g\,\,\mapsto\,\,g\circ f}", shift left =1.0]
                                                                \arrow[r,"{g\,\,\mapsto\,\,f\circ g}"',shift right=1.0]
                                                                \&
                                                                \displaystyle\coprod_{A\in\Obj(\CatFont{C})}h^{A}_{A}
                                                            \end{tikzcd}
                                                        )\\
                             &\cong  \left.\coprod_{A\in\Obj(\CatFont{C})}\Hom_{\CatFont{C}}(A,A)\middle/\unsim\right.\\
                             &\cong  \rmEnd(\CatFont{C})/\unsim,
        \end{aligned}
    \end{webcompile}
    where $\unsim$ is the equivalence relation on $\rmEnd(\CatFont{C})$ generated by $g\circ f\mathbin{\unsim'}f\circ g$.%
    % BEGIN RAW HTML %
    <br>
    % BEGIN LATEX HTML %
    \vspace{0.5\baselineskip}
    % END RAW HTML %
    Invoking \ChapterRef{\ChapterEquivalenceRelationsAndApartnessRelations, \cref{equivalence-relations-and-apartness-relations:construction-of-the-equivalence-closure-of-a-relation}}{\cref{construction-of-the-equivalence-closure-of-a-relation}}, we see that $f\sim^{\rmeq}g$ for two endomorphisms $f$ and $g$ of $\CatFont{C}$ \textiff there exists a diagram in $\CatFont{C}$ of the form%
    \[%
        \begin{tikzcd}[row sep={6.0*\the\DL,between origins}, column sep={6.0*\the\DL,between origins}, background color=backgroundColor, ampersand replacement=\&]
            A_{1}
            \arrow[r,"p_{1}"]
            \arrow[d,"f=f_{1}"']
            \&
            A_{2}
            \arrow[d,"f_{2}"description]
            \arrow[r,"p_{2}"]
            \&[0.5em]
            \cdots
            \arrow[d, shift left=2.125]
            \arrow[d,"\scriptstyle\cdots",phantom]
            \arrow[d,shift right=2.125]
            \arrow[r,"p_{n-2}"]
            \&[0.625em]
            A_{n-1}
            \arrow[r,"p_{n-1}"]
            \arrow[d,"f_{n-1}"description]
            \&
            A_{n}
            \arrow[d,"f_{n}=g"]
            \\
            A_{1}
            \arrow[r,"p_{1}"']
            \arrow[from=ru,"q_{1}"description]
            \&
            A_{2}
            \arrow[r,"p_{2}"']
            \arrow[from=ru,"q_{2}"description,end anchor={[xshift=-0.5em]}]
            \&[0.5em]
            \cdots
            \arrow[r,"p_{n-2}"']
            \arrow[from=ru,"q_{n-2}"description,start anchor={[xshift=+0.625em]}]
            \&[0.625em]
            A_{n-1}
            \arrow[r,"p_{n-1}"']
            \arrow[from=ru,"q_{n-1}"description]
            \&
            A_{n}\mrp{.}
        \end{tikzcd}%
    \]%
\end{remark}
\begin{proposition}{Properties of Traces of Categories}{properties-of-traces-of-categories}%
    Let $\CatFont{C}$ be a category.
    \begin{enumerate}
        \item\label{properties-of-traces-of-categories-functoriality}\SloganFont{Functoriality. }
        %\item\label{properties-of-traces-of-categories-}\SloganFont{. }
    \end{enumerate}
\end{proposition}
\begin{Proof}{Proof of \cref{properties-of-traces-of-categories}}%
\end{Proof}
\subsection{Examples of Traces}\label{subsection-examples-of-traces}
\begin{example}{Examples of Traces of Categories}{examples-of-traces-of-categories}%
    Here we explicitly work out some examples of traces of categories. First, here's a quick summary, where $A$ is a monoid and $G$ is a group in the following table:
    % BEGIN RAW HTML %
    <br>
    <div class="trace-table">
      <table>
        <tr>
          <th style="text-align: center; background-color: #803300; color: white;"><b>Category</b></th>
          <th style="text-align: center; background-color: #803300; color: white;"><b>Trace</b></th>
          <th style="text-align: center; background-color: #803300; color: white;"><b>Trace map</b></th>
        </tr>
        <tr>
          <td>$\B{A}$</td>
          <td>$A/\{ab\sim ba\}$</td>
          <td>$a\mapsto[a]$</td>
        </tr>
        <tr>
          <td>$\B{G}$</td>
          <td>$\Cl(G)$</td>
          <td>$a\mapsto\cl(a)$</td>
        </tr>
        <tr>
          <td>$\SimplexCategory$</td>
          <td>$\N_{\geq1}$</td>
          <td>$\sigma\mapsto\Card{\Fix(\sigma)}$</td>
        </tr>
        <tr>
          <td>$\AugmentedSimplexCategory$</td>
          <td>$\N$</td>
          <td>$\sigma\mapsto\Card{\Fix(\sigma)}$</td>
        </tr>
        <tr>
          <td>$\GlobeCategory$</td>
          <td>$\N$</td>
          <td>$([n]\xrightarrow{\sigma}[n])\mapsto n$</td>
        </tr>
        <tr>
          <td>$\CubeCategory$</td>
          <td>$\N_{\geq1}$</td>
          <td>$\sigma\mapsto\Card{\Fix(\sigma)}$</td>
        </tr>
        <tr>
          <td>$\FinSets$</td>
          <td>$\displaystyle\coprod^{\infty}_{n=0}\Cl(\Sigma_{n})$</td>
          <td>$f\mapsto\cl(\restriction{f}{\Im^{\infty}(f)})\in\Cl(\Sigma_{\Card{\Im^{\infty}(f)}})$</td>
        </tr>
        <tr>
          <td>$\GammaCategory$</td>
          <td>$\displaystyle\coprod^{\infty}_{n=0}\Cl(\Sigma_{n})$</td>
          <td>$f\mapsto\cl(\restriction{f}{\Im^{\infty}(f)})\in\Cl(\Sigma_{\Card{\Im^{\infty}(f)}})$</td>
        </tr>
        <tr>
          <td>$\CycleCategory$</td>
          <td>?</td>
          <td>?</td>
        </tr>
        <tr>
          <td>$\ParacycleCategory$</td>
          <td>?</td>
          <td>?</td>
        </tr>
        <tr>
          <td>$\TreeCategory$</td>
          <td>?</td>
          <td>?</td>
        </tr>
        <tr>
          <td>$\ThetaCategory_{n}$</td>
          <td>?</td>
          <td>?</td>
        </tr>
        <tr>
          <td>$\Orb_{G}$</td>
          <td>?</td>
          <td>?</td>
        </tr>
      </table>
    </div>
    <br>
    % BEGIN LATEX HTML %
    % END RAW HTML %
    Next, we describe each example in detail:
    \begin{enumerate}
        \item\label{examples-of-traces-of-categories-the-trace-of-a-monoid}\SloganFont{The Trace of a Monoid. }Let $A$ be a monoid. We have
            \[
                \Tr(\B{A})%
                \cong%
                A/\unsim,%
            \]%
            where $\unsim$ is the equivalence relation generated by $ab\sim ba$.
            \begin{enumerate}
                \item When $A$ is a group, this recovers the set of conjugacy classes of $A$ (\cref{TODO}).
                \item When $A$ is a general monoid, this forms one among a few different important notions of conjugacy class for monoids; see \cite{four-notions-of-conjugacy-for-abstract-semigroups}.
            \end{enumerate}
        \item\label{examples-of-traces-of-categories-the-trace-of-the-fundamental-groupoid-of-a-topological-space}\SloganFont{The Trace of the Fundamental Groupoid of a Topological Space. }If $X$ is path-connected, then we have
            \begin{align*}
                \Tr(\Pi_{\leq1}(X)) &\cong \Cl(\pi_{0}(X;x_{0}))\\%
                                    &\cong [S^{1},X],%
            \end{align*}
            where $x_{0}$ denotes any point of $X$.
        \item\label{examples-of-traces-of-categories-the-trace-of-the-simplex-category}\SloganFont{The Trace of the Simplex Category. }Let $\SimplexCategory$ be the simplex category of \cref{TODO}. We have
            \[
                \Tr(\SimplexCategory)%
                \cong%
                \N_{\geq1}%
            \]%
            and the trace map
            \[
                \tr%
                \colon%
                \rmEnd(\SimplexCategory)%
                \to%
                \Tr(\SimplexCategory)%
            \]%
            is given (under the identification $\Tr(\SimplexCategory)\cong\N_{\geq1}$) by
            \[
                \tr(\sigma)%
                \defeq%
                \Card{\Fix(\sigma)}%
            \]%
            for each $\sigma\in\rmEnd(\SimplexCategory)$, i.e.\ by sending an endomorphism $\sigma\colon[n]\to[n]$ of $\SimplexCategory$ to the cardinality of its set of fixed points.
        \item\label{examples-of-traces-of-categories-the-trace-of-the-augmented-simplex-category}\SloganFont{The Trace of the Augmented Simplex Category. }Let $\AugmentedSimplexCategory$ be the augmented simplex category of \cref{TODO}. We have
            \[
                \Tr(\AugmentedSimplexCategory)%
                \cong%
                \N%
            \]%
            and the trace map
            \[
                \tr%
                \colon%
                \rmEnd(\AugmentedSimplexCategory)%
                \to%
                \Tr(\AugmentedSimplexCategory)%
            \]%
            is given (under the identification $\Tr(\AugmentedSimplexCategory)\cong\N$) by
            \[
                \tr(\sigma)%
                \defeq%
                \Card{\Fix(\sigma)}%
            \]%
            for each $\sigma\in\rmEnd(\AugmentedSimplexCategory)$.
        \item\label{examples-of-traces-of-categories-the-trace-of-the-cube-category}\SloganFont{The Trace of the Globe Category. }We have
            \[
                \Tr(\GlobeCategory)%
                \cong%
                \N%
            \]%
            and the trace map
            \[
                \tr%
                \colon%
                \rmEnd(\GlobeCategory)%
                \to%
                \Tr(\GlobeCategory)%
            \]%
            is given (under the identification $\Tr(\GlobeCategory)\cong\N$) by
            \[
                \tr([n]\xrightarrow{\sigma}[n])%
                \defeq%
                n%
            \]%
            for each $\sigma\in\rmEnd(\GlobeCategory)$.
        \item\label{examples-of-traces-of-categories-the-trace-of-the-cube-category}\SloganFont{The Trace of the Cube Category. }We have
            \[
                \Tr(\CubeCategory)%
                \cong%
                \N_{\geq1}%
            \]%
            and the trace map
            \[
                \tr%
                \colon%
                \rmEnd(\CubeCategory)%
                \to%
                \Tr(\CubeCategory)%
            \]%
            is given (under the identification $\Tr(\CubeCategory)\cong\N_{\geq1}$) by
            \[
                \tr(\sigma)%
                \defeq%
                \Card{\Fix(\sigma)}%
            \]%
            for each $\sigma\in\rmEnd(\CubeCategory)$.
        \item\label{examples-of-traces-of-categories-the-trace-of-the-cube-category-with-connections}\SloganFont{The Trace of the Cube Category With Connections. }We have
        \item\label{examples-of-traces-of-categories-the-trace-of-the-category-of-sets}\SloganFont{The Trace of $\FinSets$. }We have
            \begin{align*}
                \Tr(\FinSets) &\cong \Tr(\FinSets^{\simeq})\\%
                              &\cong \Tr(\coprod^{\infty}_{n=0}\B{\Sigma_{n}})\\%
                              &\cong \coprod^{\infty}_{n=0}\Cl(\Sigma_{n})\\%
                              &\cong \{%
                                         (a_{n})_{n\in\N}\in\N^{\N}%
                                         \ \middle|\ %
                                         \begin{aligned}
                                             &\text{we have $a_{n}=0$ for all but}\\%
                                             &\text{finitely many $n\in\N$}%
                                         \end{aligned}
                                     \},%
            \end{align*}
            where:
            \begin{itemize}
                \item $\FinSets^{\simeq}$ is the wide subcategory of $\FinSets$ spanned by the bijections, i.e.\ the category of finite sets and bijections between them.
                \item $\Sigma_{n}$ is the $n$th symmetric group of \ChapterRef{\ChapterGroups, \cref{TODO}}{\cref{TODO}}.
                \item $\Cl(\Sigma_{n})$ is the set of conjugacy classes of $\Sigma_{n}$, given by Young diagrams of weight $n$.
            \end{itemize}
            The trace map
            \[
                \tr%
                \colon%
                \rmEnd(\FinSets)%
                \to%
                \Tr(\FinSets)%
            \]%
            is defined as follows:
            \begin{itemize}
                \item Start with an endomorphism $f\colon X\to X$ of a finite set $X$;
                \item Form the \textit{eventual image} $\Im^{\infty}(f)$ of $f$ by
                    \[
                        \Im^{\infty}(f)%
                        \defeq%
                        \bigcap^{\infty}_{n=1}f^{\circ n}(X),%
                    \]%
                    which will be a finite set of size $n$.
                \item Consider the map of finite sets
                    \[
                        \restriction{f}{\Im^{\infty}(f)}%
                        \colon%
                        \Im^{\infty}(f)%
                        \to%
                        \Im^{\infty}(f),%
                    \]%
                    which is a bijection.
                \item Take the conjugacy class $\sigma$ of $\restriction{f}{\Im^{\infty}(f)}$ in $\Sigma_{n}$.
                \item The pair $(n,\sigma)$ then gives an element of $\Tr(\FinSets)$ (under the identification $\Tr(\FinSets)\cong\coprod^{\infty}_{n=0}\Cl(\Sigma_{n})$).
            \end{itemize}
        \item\label{examples-of-traces-of-categories-the-trace-of-the-category-of-sets}\SloganFont{The Trace of $\FinSets^{\rmL}_{A}$. }
            \begin{enumerate}
                \item Does the isomorphism $\FinSets_{G}\cong\Fun(\B{G},\FinSets)$ make this easier to compute?
                \item Does $G\mapsto\Tr(\FinSets_{G})$ form a monad?
            \end{enumerate}
        \item\label{examples-of-traces-of-categories-the-trace-of-the-category-of-sets}\SloganFont{The Trace of $\Sets$. }We have
        \item\label{examples-of-traces-of-categories-the-trace-of-the-category-of-groups}\SloganFont{The Trace of $\Grp$. }
        \item\label{examples-of-traces-of-categories-the-trace-of-the-category-of-abelian-groups}\SloganFont{The Trace of $\Ab$. }
    \end{enumerate}
\end{example}
\begin{example}{The Trace of the Category of Finite $G$-Sets \rmI}{the-trace-of-the-category-of-finite-g-sets-1}%
    Let $\mathcal{G}_{n}$ be the full subgroupoid of $\FinSets^{\simeq}_{G}$ spanned by the $G$-sets whose underlying set is given by $[n]=\{1,\ldots,n\}$. We have
    \[
        \Tr(\mathcal{G}_{n})%
        \cong%
        \coprod_{\alpha\in\Hom_{\Grp}(G,\Sigma_{n})_{\Sigma_{n}}}\Stab_{\Sigma_{n}}(\alpha),%
    \]%
    where:
    \begin{itemize}
        \item $\Hom_{\Grp}(G,\Sigma_{n})_{\Sigma_{n}}$ denotes the space of orbits of $\Hom_{\Grp}(G,\Sigma_{n})$ under the conjugaction action $\alpha\mapsto\sigma^{-1}\alpha\sigma$ of $\Sigma_{n}$.
        \item $\Stab_{\Sigma_{n}}(\alpha)$ is the stabilizer subgroup of $\Sigma_{n}$ under the conjugation action $\alpha\mapsto\sigma^{-1}\alpha\sigma$ of $\Sigma_{n}$ on $\Hom_{\Grp}(G,\Sigma_{n})$.
    \end{itemize}
\end{example}
\begin{Proof}{Proof of \cref{the-trace-of-the-category-of-finite-g-sets-1}}%
    First, note that:
    \begin{enumerate}
        \item\label{the-trace-of-the-category-of-finite-g-sets-1-1}An action of $G$ on $[n]$ is equivalently specified by a morphism of groups $\alpha\colon G\to\Sigma_{n}$.
        \item\label{the-trace-of-the-category-of-finite-g-sets-1-2}Given two such actions $\alpha,\beta\colon G\to\Sigma_{n}$, an isomorphism of $G$-sets from $([n],\alpha)$ to $([n],\beta)$ is a bijection $\sigma\colon[n]\to[n]$ such that the diagram
            \[
                \begin{tikzcd}[row sep={5.0*\the\DL,between origins}, column sep={5.0*\the\DL,between origins}, background color=backgroundColor, ampersand replacement=\&]
                    G
                    \arrow[r,"\beta"]
                    \arrow[d,"\alpha"']
                    \&
                    \Sigma_{n}
                    \arrow[d,"\sigma_{*}"]
                    \\
                    \Sigma_{n}
                    \arrow[r,"\sigma^{*}"']
                    \&
                    \Sigma_{n}
                \end{tikzcd}
            \]%
            commutes, i.e.\ such that $\alpha=\sigma^{-1}\beta\sigma$.
    \end{enumerate}
    In other words, the orbits for the conjugation action of $\Sigma_{n}$ on $\Hom_{\Grp}(G,\Sigma_{n})$ are precisely the isomorphism classes of objects of $\mathcal{G}_{n}$. In symbols:
    \[
        \K_{0}(\mathcal{G}_{n})%
        =%
        \Hom_{\Grp}(G,\Sigma_{n})/\Sigma_{n}.%
    \]%
    Similarly, given a $G$-set $([n],\alpha)$ in $\mathcal{G}_{n}$, its automorphism group consists of those bijections $\sigma$ of $[n]$ (i.e.\ elements of $\Sigma_{n}$) making the diagram
    \[
        \begin{tikzcd}[row sep={5.0*\the\DL,between origins}, column sep={5.0*\the\DL,between origins}, background color=backgroundColor, ampersand replacement=\&]
            {[n]}
            \arrow[r,"\sigma"]
            \arrow[d,"\alpha_{g}"']
            \&
            {[n]}
            \arrow[d,"\alpha_{g}"]
            \\
            {[n]}
            \arrow[r,"\sigma"']
            \&
            {[n]}
        \end{tikzcd}
    \]%
    commute for each $g\in G$. In other words, we must have $\sigma^{-1}\alpha\sigma=\alpha$, and thus
    \[
        \Aut_{\mathcal{G}_{n}}([n],\alpha)%
        =%
        \Stab_{\Sigma_{n}}(\alpha).%
    \]%
    Since $\mathcal{G}_{n}$ is a groupoid, we may apply \cref{a} of \cref{a}, giving
    \begin{align*}
        \Tr(\mathcal{G}_{n}) &\cong \coprod_{\alpha\in\K_{0}(\mathcal{G}_{n})}\Aut_{\mathcal{G}_{n}}(\alpha)\\%
                             &\cong \coprod_{\alpha\in\Hom_{\Grp}(G,\Sigma_{n})/\Sigma_{n}}\Stab_{\Sigma_{n}}(\alpha).%
    \end{align*}
    This finishes the proof.
\end{Proof}
\begin{example}{The Trace of the Category of Finite $G$-Sets \rmII}{the-trace-of-the-category-of-finite-g-sets-2}%
    Let $\mathcal{G}_{n}$ be as in \cref{the-trace-of-the-category-of-finite-g-sets-1}. We have
    \begin{align*}
        \Tr(\FinSets_{G}) &\cong \Tr(\FinSets^{\simeq}_{G})\\%
                          &\cong \coprod^{\infty}_{n=0}\Tr(\mathcal{G}_{n})\\
                          &\cong \coprod_{([n],\alpha)\in\K_{0}(\FinSets^{\simeq}_{G})}\Stab_{\Sigma_{n}}(\alpha).%
    \end{align*}
\end{example}
\begin{Proof}{Proof of \cref{the-trace-of-the-category-of-finite-g-sets-2}}%
\end{Proof}
% The example below applies only to transitive $G$-sets. I should use that FinSets_G is the free symmetric monoidal groupoid on the groupoid of finite transitive $G$-sets (https://qchu.wordpress.com/2015/11/04/the-categorical-exponential-formula/) to get a general result
% Tr(Orb_G) is \coprod_{\substack{[H]\leq G\\ [G:H]\less\infty}}\rmN_{G}(H)/H,
%\begin{example}{The Trace of the Category of Finite $G$-Sets \rmIII}{the-trace-of-the-category-of-finite-g-sets-3}%
%    Let $\mathcal{G}_{n}$ be as in \cref{the-trace-of-the-category-of-finite-g-sets-1}. We have
%    \[
%        \Tr(\FinSets^{\simeq}_{G})%
%        \cong%
%        \coprod_{\substack{[H]\leq G\\ [G:H]\less\infty}}\rmN_{G}(H)/H,%
%    \]%
%    where $\rmN_{G}(H)$ is the normalizer of $G$ in $H$.
%\end{example}
%\begin{Proof}{Proof of \cref{the-trace-of-the-category-of-finite-g-sets-2}}%
%    By \cref{the-trace-of-the-category-of-finite-g-sets-1}, we have
%    \[
%        \Tr(\mathcal{G}_{n})%
%        \cong%
%        \coprod_{\alpha\in\Hom_{\Grp}(G,\Sigma_{n})/\Sigma_{n}}\Stab_{\Sigma_{n}}(\alpha).%
%    \]%
%    Since every transitive $G$-set structure on $[n]$ is isomorphic to a coset space of the form $G/H$ for $H$ a subgroup of index $n$ in $G$ and in that case $\Aut(G/H)\cong\rmN_{G}(H)/H$, we may rewrite this isomorphism as
%    \[
%        \Tr(\mathcal{G}_{n})%
%        \cong%
%        \coprod_{\substack{[H]\leq G\\ [G:H]=n}}\rmN_{G}(H)/H.%
%    \]%
%    Summing over $n\in\N$ and applying \cref{TODO} of \cref{TODO} then gives the result.
%\end{Proof}
\begin{appendices}
\input{ABSOLUTEPATH/chapters2.tex}
\end{appendices}
\end{document}
