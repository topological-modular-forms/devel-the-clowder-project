\input{preamble}

% OK, start here.
\usepackage{fontspec}
\let\hyperwhite\relax
\let\hyperred\relax
\newcommand{\hyperwhite}{\hypersetup{citecolor=white,filecolor=white,linkcolor=white,urlcolor=white}}
\newcommand{\hyperred}{%
\hypersetup{%
    citecolor=TitlingRed,%
    filecolor=TitlingRed,%
    linkcolor=TitlingRed,%
     urlcolor=TitlingRed%
}}
\let\ChapterRef\relax
\newcommand{\ChapterRef}[2]{#1}
\setcounter{tocdepth}{2}
%▓▓▓▓▓▓▓▓▓▓▓▓▓▓▓▓▓▓▓▓▓▓▓▓▓▓▓▓▓▓▓▓▓
%▓▓ ╔╦╗╦╔╦╗╦  ╔═╗  ╔═╗╔═╗╔╗╔╔╦╗ ▓▓
%▓▓  ║ ║ ║ ║  ║╣   ╠╣ ║ ║║║║ ║  ▓▓
%▓▓  ╩ ╩ ╩ ╩═╝╚═╝  ╚  ╚═╝╝╚╝ ╩  ▓▓
%▓▓▓▓▓▓▓▓▓▓▓▓▓▓▓▓▓▓▓▓▓▓▓▓▓▓▓▓▓▓▓▓▓
%\usepackage{titlesec}
%▓▓▓▓▓▓▓▓▓▓▓▓▓▓▓▓▓▓▓▓▓▓▓▓▓▓▓▓▓▓▓▓▓▓▓▓▓▓▓▓▓▓▓▓▓▓▓▓▓▓▓▓▓▓▓
%▓▓ ╔╦╗╔═╗╔╗ ╦  ╔═╗  ╔═╗╔═╗  ╔═╗╔═╗╔╗╔╔╦╗╔═╗╔╗╔╔╦╗╔═╗ ▓▓
%▓▓  ║ ╠═╣╠╩╗║  ║╣   ║ ║╠╣   ║  ║ ║║║║ ║ ║╣ ║║║ ║ ╚═╗ ▓▓
%▓▓  ╩ ╩ ╩╚═╝╩═╝╚═╝  ╚═╝╚    ╚═╝╚═╝╝╚╝ ╩ ╚═╝╝╚╝ ╩ ╚═╝ ▓▓
%▓▓▓▓▓▓▓▓▓▓▓▓▓▓▓▓▓▓▓▓▓▓▓▓▓▓▓▓▓▓▓▓▓▓▓▓▓▓▓▓▓▓▓▓▓▓▓▓▓▓▓▓▓▓▓
\newcommand{\ChapterTableOfContents}{%
    \begingroup
    \addfontfeature{Numbers={Lining,Monospaced}}
    \hypersetup{hidelinks}\tableofcontents%
    \endgroup
}%

\let\DotFill\relax
\makeatletter
\newcommand \DotFill {\leavevmode \cleaders \hb@xt@ .33em{\hss .\hss }\hfill \kern \z@}
\makeatother

\definecolor{ToCGrey}{rgb}{0.4,0.4,0.4}
\definecolor{mainColor}{rgb}{0.82745098,0.18431373,0.18431373}
\usepackage{titletoc}
\titlecontents{part}
[0.0em]
{\addvspace{1pc}\color{TitlingRed}\large\bfseries\text{Part }}
{\bfseries\textcolor{TitlingRed}{\contentslabel{0.0em}}\hspace*{1.35em}}
{}
{\textcolor{TitlingRed}{{\hfill\bfseries\contentspage\nobreak}}}
[]
\titlecontents{section}
[0.0em]
{\addvspace{1pc}}
{\color{black}\bfseries\textcolor{TitlingRed}{\contentslabel{0.0em}}\hspace*{1.65em}}
{}
{\textcolor{black}{\textbf{\DotFill}{\bfseries\contentspage\nobreak}}}
[]
\titlecontents{subsection}
[0.0em]
{}
{\hspace*{1.65em}\color{ToCGrey}{\contentslabel{0.0em}}\hspace*{2.5em}}
{}
{{\textcolor{ToCGrey}\DotFill}\textcolor{ToCGrey}{\contentspage}\nobreak}
[]
\usepackage{marginnote}
\renewcommand*{\marginfont}{\normalfont}
\usepackage{inconsolata}
\setmonofont{inconsolata}%
\let\ChapterRef\relax
\newcommand{\ChapterRef}[2]{#1}
\AtBeginEnvironment{subappendices}{%%
    \section*{\huge Appendices}%
}%

\begin{document}

\title{Notes}

\maketitle

\phantomsection
\label{section-phantom}

This chapter contains some notes.

\ChapterTableOfContents

\section{TikZ Code for Commutative Diagrams}\label{section-tikz-code-for-commutative-diagrams}
In this section we gather some useful examples of \texttt{tikzcd} code for commutative diagrams.
\subsection{Product Diagram With Circular Arrows}\label{subsection-tikz-code-for-commutative-diagrams-product-diagram-with-circular-arrows}
Define
\begin{verbatim}
\newlength{\DL}
\setlength{\DL}{0.9em}
\end{verbatim}
in the preamble, as well as
\begin{verbatim}
\tikzcdset{
    productArrows/.style args={#1#2#3}{
    execute at end picture={
        % FIRST ARROW
        % Step 1: Draw arrow body
        \begin{scope}
            \clip (\tikzcdmatrixname-1-2.east) -- (\tikzcdmatrixname-2-2.center) -- (\tikzcdmatrixname-2-3.north) -- (\tikzcdmatrixname-1-3.center) -- cycle;
            \path[draw,line width=rule_thickness] (\tikzcdmatrixname-1-2) arc[start angle=90,end angle=0,radius=#1];
        \end{scope}
        % Step 2: Draw arrow head
        % Step 2.1: Find the point at which to place the arrowhead
        \path[name path=curve-1-a] (\tikzcdmatrixname-1-2.east) -- (\tikzcdmatrixname-2-2.center) -- (\tikzcdmatrixname-2-3.north) -- (\tikzcdmatrixname-1-3.center) -- cycle;
        \path[name path=curve-1-b] (\tikzcdmatrixname-1-2) arc[start angle=90,end angle=0,radius=#1];
        \fill [name intersections={of=curve-1-a and curve-1-b}] (intersection-2);
        % Step 2.2: Find the angle at which to place the arrowhead
        \coordinate (arc-start) at (\tikzcdmatrixname-1-2.east);
        \coordinate (arc-center) at (\tikzcdmatrixname-2-2.center);
        \draw let
            \p1 = ($(intersection-2) - (arc-center)$), % \p1 is the vector from the arc's centre to the intersection point (we use i-2 for the 2nd intersection)
            \n1 = {atan2(\y1, \x1)}, % \n1 is the angle of that vector in degrees
            \n2 = {\n1 - 90} % \n2 is the angle of the tangent (90 degrees from the radius vector for a circle)
          in [->] (intersection-2) -- ++(\n2:0.1pt);
        % SECOND ARROW
        % Step 1: Draw arrow body
        \begin{scope}
            \clip (\tikzcdmatrixname-1-2.west) -- (\tikzcdmatrixname-2-2.center) -- (\tikzcdmatrixname-2-1.north) -- (\tikzcdmatrixname-1-1.center) -- cycle;
            \path[draw,line width=rule_thickness] (\tikzcdmatrixname-1-2) arc[start angle=90,end angle=180,radius=#1];
        \end{scope}
        % Step 2: Draw arrow head
        % Step 2.1: Find the point at which to place the arrowhead
        \path[name path=curve-2-a] (\tikzcdmatrixname-1-2.west) -- (\tikzcdmatrixname-2-2.center) -- (\tikzcdmatrixname-2-1.north) -- (\tikzcdmatrixname-1-1.center) -- cycle;
        \path[name path=curve-2-b] (\tikzcdmatrixname-1-2) arc[start angle=90,end angle=180,radius=#1];
        \fill [name intersections={of=curve-2-a and curve-2-b}] (intersection-2);
        % Step 2.2: Find the angle at which to place the arrowhead
        \coordinate (arc-start) at (\tikzcdmatrixname-1-2.west);
        \coordinate (arc-center) at (\tikzcdmatrixname-2-2.center);
        \draw let
            \p1 = ($(intersection-2) - (arc-center)$), % \p1 is the vector from the arc's centre to the intersection point (we use i-2 for the 2nd intersection)
            \n1 = {atan2(\y1, \x1)}, % \n1 is the angle of that vector in degrees
            \n2 = {\n1 - 90} % \n2 is the angle of the tangent (90 degrees from the radius vector for a circle)
          in [<-] (intersection-2) -- ++(\n2:0.1pt);
          % Labels
          \path (\tikzcdmatrixname-1-2) arc[start angle=90,end angle=180,radius=#1] node[above left,pos=0.5] {$\scriptstyle #2$};
          \path (\tikzcdmatrixname-1-2) arc[start angle=90,end angle=0,radius=#1] node[above right,pos=0.5] {$\scriptstyle #3$};
    }
  }
}
\end{verbatim}
The code
\begin{verbatim}
\begin{tikzcd}[row sep={4.5*\the\DL,between origins}, column sep={4.5*\the\DL,between origins}, background color=backgroundColor, ampersand replacement=\&,productArrows={4.5*\the\DL}{p_{1}}{p_{2}}]
    {}% Don't remove this line, it's important!
    \&
    P
    \arrow[d,"\phi"'{pos=0.475},"\exists!"{pos=0.475}, dashed]
    \&
    {}% Don't remove this line, it's important!
    \\
    A
    \&
    A\times B
    \arrow[l,"\pr_{1}"{pos=0.425},two heads]
    \arrow[r,"\pr_{2}"'{pos=0.425},two heads]
    \&
    B
\end{tikzcd}
\end{verbatim}
will then produce the following diagram:
\[
    \begin{tikzcd}[row sep={4.5*\the\DL,between origins}, column sep={4.5*\the\DL,between origins}, background color=backgroundColor, ampersand replacement=\&,productArrows={4.5*\the\DL}{p_{1}}{p_{2}}]
        {}%
        \&
        P
        \arrow[d,"\phi"'{pos=0.475},"\exists!"{pos=0.475}, dashed]
        \&
        {}%
        \\
        A
        \&
        A\times B
        \arrow[l,"\pr_{1}"{pos=0.425},two heads]
        \arrow[r,"\pr_{2}"'{pos=0.425},two heads]
        \&
        B
    \end{tikzcd}
\]%
\subsection{Coproduct Diagram With Circular Arrows}\label{subsection-tikz-code-for-commutative-diagrams-coproduct-diagram-with-circular-arrows}
Define
\begin{verbatim}
\newlength{\DL}
\setlength{\DL}{0.9em}
\end{verbatim}
in the preamble, as well as
\begin{verbatim}
\tikzcdset{
    coproductArrows/.style args={#1#2#3}{
    execute at end picture={
        % FIRST ARROW
        % Step 1: Draw arrow body
        \begin{scope}
            \clip (\tikzcdmatrixname-1-2.east) -- (\tikzcdmatrixname-2-2.center) -- (\tikzcdmatrixname-2-3.north) -- (\tikzcdmatrixname-1-3.center) -- cycle;
            \path[draw,line width=rule_thickness] (\tikzcdmatrixname-1-2) arc[start angle=90,end angle=0,radius=#1];
        \end{scope}
        % Step 2: Draw arrow head
        % Step 2.1: Find the point at which to place the arrowhead
        \path[name path=curve-1-a] (\tikzcdmatrixname-1-2.east) -- (\tikzcdmatrixname-2-2.center) -- (\tikzcdmatrixname-2-3.north) -- (\tikzcdmatrixname-1-3.center) -- cycle;
        \path[name path=curve-1-b] (\tikzcdmatrixname-1-2) arc[start angle=90,end angle=0,radius=#1];
        \fill [name intersections={of=curve-1-a and curve-1-b}] (intersection-1);
        % Step 2.2: Find the angle at which to place the arrowhead
        \coordinate (arc-start) at (\tikzcdmatrixname-1-2.east);
        \coordinate (arc-center) at (\tikzcdmatrixname-2-2.center);
        \draw let
            \p1 = ($(intersection-1) - (arc-center)$), % \p1 is the vector from the arc's centre to the intersection point (we use i-2 for the 2nd intersection)
            \n1 = {atan2(\y1, \x1)}, % \n1 is the angle of that vector in degrees
            \n2 = {\n1 - 90} % \n2 is the angle of the tangent (90 degrees from the radius vector for a circle)
          in [<-] (intersection-1) -- ++(\n2:0.1pt);
        % SECOND ARROW
        % Step 1: Draw arrow body
        \begin{scope}
            \clip (\tikzcdmatrixname-1-2.west) -- (\tikzcdmatrixname-2-2.center) -- (\tikzcdmatrixname-2-1.north) -- (\tikzcdmatrixname-1-1.center) -- cycle;
            \path[draw,line width=rule_thickness] (\tikzcdmatrixname-1-2) arc[start angle=90,end angle=180,radius=#1];
        \end{scope}
        % Step 2: Draw arrow head
        % Step 2.1: Find the point at which to place the arrowhead
        \path[name path=curve-2-a] (\tikzcdmatrixname-1-2.west) -- (\tikzcdmatrixname-2-2.center) -- (\tikzcdmatrixname-2-1.north) -- (\tikzcdmatrixname-1-1.center) -- cycle;
        \path[name path=curve-2-b] (\tikzcdmatrixname-1-2) arc[start angle=90,end angle=180,radius=#1];
        \fill [name intersections={of=curve-2-a and curve-2-b}] (intersection-1);
        % Step 2.2: Find the angle at which to place the arrowhead
        \coordinate (arc-start) at (\tikzcdmatrixname-1-2.west);
        \coordinate (arc-center) at (\tikzcdmatrixname-2-2.center);
        \draw let
            \p1 = ($(intersection-1) - (arc-center)$), % \p1 is the vector from the arc's centre to the intersection point (we use i-2 for the 2nd intersection)
            \n1 = {atan2(\y1, \x1)}, % \n1 is the angle of that vector in degrees
            \n2 = {\n1 - 90} % \n2 is the angle of the tangent (90 degrees from the radius vector for a circle)
          in [->] (intersection-1) -- ++(\n2:0.1pt);
          % Labels
          \path (\tikzcdmatrixname-1-2) arc[start angle=90,end angle=180,radius=#1] node[above left,pos=0.5] {$\scriptstyle #2$};
          \path (\tikzcdmatrixname-1-2) arc[start angle=90,end angle=0,radius=#1] node[above right,pos=0.5] {$\scriptstyle #3$};
    }
  }
}
\end{verbatim}
The code
\begin{verbatim}
\begin{tikzcd}[row sep={4.5*\the\DL,between origins}, column sep={4.5*\the\DL,between origins}, background color=backgroundColor, ampersand replacement=\&,coproductArrows={4.5*\the\DL}{\iota_{1}}{\iota_{2}}]
    {}% Don't remove this line, it's important!
    \&
    C
    \arrow[from=d,"\phi","\exists!"', dashed]
    \&
    {}% Don't remove this line, it's important!
    \\
    A
    \&
    A\icoprod B
    \arrow[from=l,"\inj_{1}"',hook]
    \arrow[from=r,"\inj_{2}",hook']
    \&
    B
\end{tikzcd}
\end{verbatim}
will then produce the following diagram:
\[
    \begin{tikzcd}[row sep={4.5*\the\DL,between origins}, column sep={4.5*\the\DL,between origins}, background color=backgroundColor, ampersand replacement=\&,coproductArrows={4.5*\the\DL}{\iota_{1}}{\iota_{2}}]
        {}%
        \&
        C
        \arrow[from=d,"\phi","\exists!"', dashed]
        \&
        {}%
        \\
        A
        \&
        A\icoprod B
        \arrow[from=l,"\inj_{1}"',hook]
        \arrow[from=r,"\inj_{2}",hook']
        \&
        B
    \end{tikzcd}
\]%
\subsection{Cube Diagram}\label{subsection-tikz-code-for-commutative-diagrams-cube-diagram}
Define
\begin{verbatim}
\newlength{\DL}
\setlength{\DL}{0.9em}
\end{verbatim}
The code
\begin{verbatim}
\begin{tikzcd}[row sep={4.0*\the\DL,between origins}, column sep={4.0*\the\DL,between origins}, background color=backgroundColor, ampersand replacement=\&]
    1
    \&
    \&
    2
    \&
    \\
    \&
    1'
    \&
    \&
    2'
    \\
    3
    \&
    \&
    4
    \&
    \\
    \&
    3'
    \&
    \&
    4'
    % 1-Arrows
    % First Square
    \arrow[from=1-1,to=3-1,"f"']%
    \arrow[from=3-1,to=3-3,"h"{description,pos=0.25}]%
    \arrow[from=1-1,to=1-3,"g"]%
    \arrow[from=1-3,to=3-3,"k"{description,pos=0.25}]%
    % Second Square
    \arrow[from=2-2,to=4-2,"f'"{description,pos=0.3},crossing over]%
    \arrow[from=4-2,to=4-4,"h'"']%
    \arrow[from=2-2,to=2-4,"g'"{description,pos=0.3},crossing over]%
    \arrow[from=2-4,to=4-4,"k'"]%
    % Connecting Arrows
    \arrow[from=1-1,to=2-2,"c_{1}"description]%
    \arrow[from=1-3,to=2-4,"c_{2}"]%
    \arrow[from=3-1,to=4-2,"c_{3}"']%
    \arrow[from=3-3,to=4-4,"c_{4}"description]%
\end{tikzcd}
\end{verbatim}
will produce the following diagram:
\[
    \begin{tikzcd}[row sep={4.0*\the\DL,between origins}, column sep={4.0*\the\DL,between origins}, background color=backgroundColor, ampersand replacement=\&]
        1
        \&
        \&
        2
        \&
        \\
        \&
        1'
        \&
        \&
        2'
        \\
        3
        \&
        \&
        4
        \&
        \\
        \&
        3'
        \&
        \&
        4'
        % 1-Arrows
        % First Square
        \arrow[from=1-1,to=3-1,"f"']%
        \arrow[from=3-1,to=3-3,"h"{description,pos=0.25}]%
        \arrow[from=1-1,to=1-3,"g"]%
        \arrow[from=1-3,to=3-3,"k"{description,pos=0.25}]%
        % Second Square
        \arrow[from=2-2,to=4-2,"f'"{description,pos=0.3},crossing over]%
        \arrow[from=4-2,to=4-4,"h'"']%
        \arrow[from=2-2,to=2-4,"g'"{description,pos=0.3},crossing over]%
        \arrow[from=2-4,to=4-4,"k'"]%
        % Connecting Arrows
        \arrow[from=1-1,to=2-2,"c_{1}"description]%
        \arrow[from=1-3,to=2-4,"c_{2}"]%
        \arrow[from=3-1,to=4-2,"c_{3}"']%
        \arrow[from=3-3,to=4-4,"c_{4}"description]%
    \end{tikzcd}
\]%
\subsection{Cube Diagram With Labelled Faces}\label{subsection-tikz-code-for-commutative-diagrams-cube-diagram-with-labelled-faces}
Define
\begin{verbatim}
\newlength{\DL}
\setlength{\DL}{0.9em}
\end{verbatim}
The code
\begin{verbatim}
\begin{tikzcd}[row sep={4.0*\the\DL,between origins}, column sep={4.0*\the\DL,between origins}, background color=backgroundColor, ampersand replacement=\&]
    1
    \&
    \&
    2
    \&
    \\
    \&
    1'
    \&
    \&
    2'
    \\
    3
    \&
    \&
    \&
    \\
    \&
    3'
    \&
    \&
    4'
    % 1-Arrows
    % First Square
    \arrow[from=1-1,to=3-1,"f"']%
    \arrow[from=1-1,to=1-3,"g"]%
    % Second Square
    \arrow[from=2-2,to=4-2,"f'"{description},crossing over]%
    \arrow[from=4-2,to=4-4,"h'"']%
    \arrow[from=2-2,to=2-4,"g'"{description},crossing over]%
    \arrow[from=2-4,to=4-4,"k'"]%
    % Connecting Arrows
    \arrow[from=1-1,to=2-2,"c_{1}"description]%
    \arrow[from=1-3,to=2-4,"c_{2}"]%
    \arrow[from=3-1,to=4-2,"c_{3}"']%
    % Subdiagrams
    \arrow[from=2-2,to=1-3,"\scriptstyle(1)"{rotate=-0.3,xslant=-0.903569337,yslant=0,xscale=7.0341,yscale=4.4454,xscale=0.225,yscale=0.225},phantom]%
    \arrow[from=3-1,to=2-2,"\scriptstyle(2)"{rotate=-44.6,xslant=-0.965688775,yslant=0,xscale=8.6931,yscale=8.2852,xscale=0.15,yscale=0.15},phantom]%
    \arrow[from=4-2,to=2-4,"\scriptstyle(3)"{rotate=0,xslant=0,yslant=0,xscale=1.5,yscale=1.5},phantom]%
\end{tikzcd}
\qquad
\begin{tikzcd}[row sep={4.0*\the\DL,between origins}, column sep={4.0*\the\DL,between origins}, background color=backgroundColor, ampersand replacement=\&]
    1
    \&
    \&
    2
    \&
    \\
    \&
    \&
    \&
    2'
    \\
    3
    \&
    \&
    4
    \&
    \\
    \&
    3'
    \&
    \&
    4'
    % 1-Arrows
    % First Square
    \arrow[from=1-1,to=3-1,"f"']%
    \arrow[from=3-1,to=3-3,"h"{description}]%
    \arrow[from=1-1,to=1-3,"g"]%
    \arrow[from=1-3,to=3-3,"k"{description}]%
    % Second Square
    \arrow[from=4-2,to=4-4,"h'"']%
    \arrow[from=2-4,to=4-4,"k'"]%
    % Connecting Arrows
    \arrow[from=1-3,to=2-4,"c_{2}"]%
    \arrow[from=3-1,to=4-2,"c_{3}"']%
    \arrow[from=3-3,to=4-4,"c_{4}"description]%
    % Subdiagrams
    \arrow[from=1-1,to=3-3,"\scriptstyle(4)"{rotate=0,xslant=0,yslant=0,xscale=1.5,yscale=1.5},phantom]%
    \arrow[from=3-3,to=2-4,"\scriptstyle(5)"{rotate=-44.6,xslant=-0.965688775,yslant=0,xscale=8.6931,yscale=8.2852,xscale=0.15,yscale=0.15},phantom]%
    \arrow[from=4-2,to=3-3,"\scriptstyle(6)"{rotate=-0.3,xslant=-0.903569337,yslant=0,xscale=7.0341,yscale=4.4454,xscale=0.225,yscale=0.225},phantom]%
\end{tikzcd}
\end{verbatim}
will produce the following diagram:
\begin{webcompile}
    \begin{tikzcd}[row sep={4.0*\the\DL,between origins}, column sep={4.0*\the\DL,between origins}, background color=backgroundColor, ampersand replacement=\&]
        1
        \&
        \&
        2
        \&
        \\
        \&
        1'
        \&
        \&
        2'
        \\
        3
        \&
        \&
        \&
        \\
        \&
        3'
        \&
        \&
        4'
        % 1-Arrows
        % First Square
        \arrow[from=1-1,to=3-1,"f"']%
        \arrow[from=1-1,to=1-3,"g"]%
        % Second Square
        \arrow[from=2-2,to=4-2,"f'"{description},crossing over]%
        \arrow[from=4-2,to=4-4,"h'"']%
        \arrow[from=2-2,to=2-4,"g'"{description},crossing over]%
        \arrow[from=2-4,to=4-4,"k'"]%
        % Connecting Arrows
        \arrow[from=1-1,to=2-2,"c_{1}"description]%
        \arrow[from=1-3,to=2-4,"c_{2}"]%
        \arrow[from=3-1,to=4-2,"c_{3}"']%
        % Subdiagrams
        \arrow[from=2-2,to=1-3,"\scriptstyle(1)"{rotate=-0.3,xslant=-0.903569337,yslant=0,xscale=7.0341,yscale=4.4454,xscale=0.225,yscale=0.225},phantom]%
        \arrow[from=3-1,to=2-2,"\scriptstyle(2)"{rotate=-44.6,xslant=-0.965688775,yslant=0,xscale=8.6931,yscale=8.2852,xscale=0.15,yscale=0.15},phantom]%
        \arrow[from=4-2,to=2-4,"\scriptstyle(3)"{rotate=0,xslant=0,yslant=0,xscale=1.5,yscale=1.5},phantom]%
    \end{tikzcd}
    \qquad
    \begin{tikzcd}[row sep={4.0*\the\DL,between origins}, column sep={4.0*\the\DL,between origins}, background color=backgroundColor, ampersand replacement=\&]
        1
        \&
        \&
        2
        \&
        \\
        \&
        \&
        \&
        2'
        \\
        3
        \&
        \&
        4
        \&
        \\
        \&
        3'
        \&
        \&
        4'
        % 1-Arrows
        % First Square
        \arrow[from=1-1,to=3-1,"f"']%
        \arrow[from=3-1,to=3-3,"h"{description}]%
        \arrow[from=1-1,to=1-3,"g"]%
        \arrow[from=1-3,to=3-3,"k"{description}]%
        % Second Square
        \arrow[from=4-2,to=4-4,"h'"']%
        \arrow[from=2-4,to=4-4,"k'"]%
        % Connecting Arrows
        \arrow[from=1-3,to=2-4,"c_{2}"]%
        \arrow[from=3-1,to=4-2,"c_{3}"']%
        \arrow[from=3-3,to=4-4,"c_{4}"description]%
        % Subdiagrams
        \arrow[from=1-1,to=3-3,"\scriptstyle(4)"{rotate=0,xslant=0,yslant=0,xscale=1.5,yscale=1.5},phantom]%
        \arrow[from=3-3,to=2-4,"\scriptstyle(5)"{rotate=-44.6,xslant=-0.965688775,yslant=0,xscale=8.6931,yscale=8.2852,xscale=0.15,yscale=0.15},phantom]%
        \arrow[from=4-2,to=3-3,"\scriptstyle(6)"{rotate=-0.3,xslant=-0.903569337,yslant=0,xscale=7.0341,yscale=4.4454,xscale=0.225,yscale=0.225},phantom]%
    \end{tikzcd}
\end{webcompile}
\subsection{Pentagon Diagram}\label{subsection-tikz-code-for-commutative-diagrams-pentagon-diagram}
Define
\begin{verbatim}
\newlength{\ThreeCm}
\setlength{\ThreeCm}{3.0cm}
\end{verbatim}
The code
\begin{verbatim}
\begin{tikzcd}[row sep={0*\the\DL,between origins}, column sep={0*\the\DL,between origins}, background color=backgroundColor, ampersand replacement=\&]
    \&[0.30901699437\ThreeCm]
    \&[0.5\ThreeCm]
    A\otimes_{R}(A\otimes_{R}A)
    \&[0.5\ThreeCm]
    \&[0.30901699437\ThreeCm]
    \\[0.58778525229\ThreeCm]
    (A\otimes_{R}A)\otimes_{R}A
    \&[0.30901699437\ThreeCm]
    \&[0.5\ThreeCm]
    \&[0.5\ThreeCm]
    \&[0.30901699437\ThreeCm]
    A\otimes_{R}A
    \\[0.95105651629\ThreeCm]
    \&[0.30901699437\ThreeCm]
    A\otimes_{R}A
    \&[0.5\ThreeCm]
    \&[0.5\ThreeCm]
    A
    \&[0.30901699437\ThreeCm]
    % 1-Arrows
    % Left Boundary
    \arrow[from=2-1,to=1-3,"\alpha^{\Mod_{R}}_{A,A,A}"{pos=0.4125}]%
    \arrow[from=1-3,to=2-5,"\id_{A}\otimes_{R}\mu^{\times}_{A}"{pos=0.6}]%
    \arrow[from=2-5,to=3-4,"\mu^{\times}_{A}"{pos=0.425}]%
    % Right Boundary
    \arrow[from=2-1,to=3-2,"\mu^{\times}_{A}\otimes_{R}\id_{A}"'{pos=0.425}]%
    \arrow[from=3-2,to=3-4,"\mu^{\times}_{A}"']%
\end{tikzcd}
\end{verbatim}
will produce the following pentagon diagram:
\[
    \begin{tikzcd}[row sep={0*\the\DL,between origins}, column sep={0*\the\DL,between origins}, background color=backgroundColor, ampersand replacement=\&]
        \&[0.30901699437\ThreeCm]
        \&[0.5\ThreeCm]
        A\otimes_{R}(A\otimes_{R}A)
        \&[0.5\ThreeCm]
        \&[0.30901699437\ThreeCm]
        \\[0.58778525229\ThreeCm]
        (A\otimes_{R}A)\otimes_{R}A
        \&[0.30901699437\ThreeCm]
        \&[0.5\ThreeCm]
        \&[0.5\ThreeCm]
        \&[0.30901699437\ThreeCm]
        A\otimes_{R}A
        \\[0.95105651629\ThreeCm]
        \&[0.30901699437\ThreeCm]
        A\otimes_{R}A
        \&[0.5\ThreeCm]
        \&[0.5\ThreeCm]
        A
        \&[0.30901699437\ThreeCm]
        % 1-Arrows
        % Left Boundary
        \arrow[from=2-1,to=1-3,"\alpha^{\Mod_{R}}_{A,A,A}"{pos=0.4125}]%
        \arrow[from=1-3,to=2-5,"\id_{A}\otimes_{R}\mu^{\times}_{A}"{pos=0.6}]%
        \arrow[from=2-5,to=3-4,"\mu^{\times}_{A}"{pos=0.425}]%
        % Right Boundary
        \arrow[from=2-1,to=3-2,"\mu^{\times}_{A}\otimes_{R}\id_{A}"'{pos=0.425}]%
        \arrow[from=3-2,to=3-4,"\mu^{\times}_{A}"']%
    \end{tikzcd}
\]%
To make the diagram larger, one could use e.g.
\begin{verbatim}
\newlength{\FourCm}
\setlength{\FourCm}{2.0cm}
\end{verbatim}
and replace all instances of \texttt{\textbackslash ThreeCm} with \texttt{\textbackslash FourCm} in the code above.
\subsection{Hexagon Diagram}\label{subsection-tikz-code-for-commutative-diagrams-hexagon-diagram}
Define
\begin{verbatim}
\newlength{\OneCmPlusHalf}
\setlength{\OneCmPlusHalf}{1.5cm}
\end{verbatim}
The code
\begin{verbatim}
\begin{tikzcd}[row sep={0.0*\the\DL,between origins}, column sep={0.0*\the\DL,between origins}, background color=backgroundColor, ampersand replacement=\&]
    \&[0.86602540378\OneCmPlusHalf]
    1
    \&[0.86602540378\OneCmPlusHalf]
    \\[0.5\OneCmPlusHalf]
    2
    \&[0.86602540378\OneCmPlusHalf]
    \&[0.86602540378\OneCmPlusHalf]
    3
    \\[\OneCmPlusHalf]
    4
    \&[0.86602540378\OneCmPlusHalf]
    \&[0.86602540378\OneCmPlusHalf]
    5
    \\[0.5\OneCmPlusHalf]
    \&[0.86602540378\OneCmPlusHalf]
    6
    \&[0.86602540378\OneCmPlusHalf]
    % 1-Arrows
    % Left Boundary
    \arrow[from=1-2,to=2-1,"L_{1}"']%
    \arrow[from=2-1,to=3-1,"L_{2}"']%
    \arrow[from=3-1,to=4-2,"L_{3}"']%
    % Right Boundary
    \arrow[from=1-2,to=2-3,"R_{1}"]%
    \arrow[from=2-3,to=3-3,"R_{2}"]%
    \arrow[from=3-3,to=4-2,"R_{3}"]%
\end{tikzcd}
\end{verbatim}
will produce the following hexagon diagram:
\[
    \begin{tikzcd}[row sep={0.0*\the\DL,between origins}, column sep={0.0*\the\DL,between origins}, background color=backgroundColor, ampersand replacement=\&]
        \&[0.86602540378\OneCmPlusHalf]
        1
        \&[0.86602540378\OneCmPlusHalf]
        \\[0.5\OneCmPlusHalf]
        2
        \&[0.86602540378\OneCmPlusHalf]
        \&[0.86602540378\OneCmPlusHalf]
        3
        \\[\OneCmPlusHalf]
        4
        \&[0.86602540378\OneCmPlusHalf]
        \&[0.86602540378\OneCmPlusHalf]
        5
        \\[0.5\OneCmPlusHalf]
        \&[0.86602540378\OneCmPlusHalf]
        6
        \&[0.86602540378\OneCmPlusHalf]
        % 1-Arrows
        % Left Boundary
        \arrow[from=1-2,to=2-1,"L_{1}"']%
        \arrow[from=2-1,to=3-1,"L_{2}"']%
        \arrow[from=3-1,to=4-2,"L_{3}"']%
        % Right Boundary
        \arrow[from=1-2,to=2-3,"R_{1}"]%
        \arrow[from=2-3,to=3-3,"R_{2}"]%
        \arrow[from=3-3,to=4-2,"R_{3}"]%
    \end{tikzcd}
\]%
To make the diagram larger, one could use e.g.
\begin{verbatim}
\newlength{\TwoCm}
\setlength{\TwoCm}{2.0cm}
\end{verbatim}
and replace all instances of \texttt{\textbackslash OneCmPlusHalf} with \texttt{\textbackslash TwoCm} in the code above.
\subsection{Double Square Diagram}\label{subsection-tikz-code-for-commutative-diagrams-double-square-diagram}
Define
\begin{verbatim}
\newlength{\DL}
\setlength{\DL}{0.9cm}
\end{verbatim}
The code
\begin{verbatim}
\begin{tikzcd}[row sep={10.0*\the\DL,between origins}, column sep={10.0*\the\DL,between origins}, background color=backgroundColor, ampersand replacement=\&]
    \bullet
    \&
    \&
    \&
    \bullet
    \\
    \&
    \bullet
    \&
    \bullet
    \&
    \\
    \&
    \bullet
    \&
    \bullet
    \&
    \\
    \bullet
    \&
    \&
    \&
    \bullet
    % Arrows
    % Outer Square
    \arrow[from=1-1,to=1-4]%
    \arrow[from=1-4,to=4-4]%
    %
    \arrow[from=1-1,to=4-1]%
    \arrow[from=4-1,to=4-4]%
    % Inner Square
    \arrow[from=2-2,to=2-3]%
    \arrow[from=2-3,to=3-3]%
    %
    \arrow[from=2-2,to=3-2]%
    \arrow[from=3-2,to=3-3]%
    % Connecting Arrows
    \arrow[from=1-1,to=2-2]%
    \arrow[from=1-4,to=2-3]%
    \arrow[from=3-2,to=4-1]%
    \arrow[from=3-3,to=4-4]%
    % Subdiagrams
    \arrow[from=2-2,to=3-3,"\scriptstyle(1)",phantom,yshift=10.0*\the\DL]%
    \arrow[from=2-2,to=3-2,"\scriptstyle(2)",phantom,xshift=-5.0*\the\DL]%
    \arrow[from=2-2,to=3-3,"\scriptstyle(3)",phantom]%
    \arrow[from=2-3,to=3-3,"\scriptstyle(4)",phantom,xshift=5.0*\the\DL]%
    \arrow[from=2-2,to=3-3,"\scriptstyle(5)",phantom,yshift=-10.0*\the\DL]%
\end{tikzcd}
\end{verbatim}
will produce the following double square diagram:
\[
    \begin{tikzcd}[row sep={5.0*\the\DL,between origins}, column sep={5.0*\the\DL,between origins}, background color=backgroundColor, ampersand replacement=\&]
        \bullet
        \&
        \&
        \&
        \bullet
        \\
        \&
        \bullet
        \&
        \bullet
        \&
        \\
        \&
        \bullet
        \&
        \bullet
        \&
        \\
        \bullet
        \&
        \&
        \&
        \bullet
        % Arrows
        % Outer Square
        \arrow[from=1-1,to=1-4]%
        \arrow[from=1-4,to=4-4]%
        %
        \arrow[from=1-1,to=4-1]%
        \arrow[from=4-1,to=4-4]%
        % Inner Square
        \arrow[from=2-2,to=2-3]%
        \arrow[from=2-3,to=3-3]%
        %
        \arrow[from=2-2,to=3-2]%
        \arrow[from=3-2,to=3-3]%
        % Connecting Arrows
        \arrow[from=1-1,to=2-2]%
        \arrow[from=1-4,to=2-3]%
        \arrow[from=3-2,to=4-1]%
        \arrow[from=3-3,to=4-4]%
        % Subdiagrams
        \arrow[from=2-2,to=3-3,"\scriptstyle(1)",phantom,yshift=5.0*\the\DL]%
        \arrow[from=2-2,to=3-2,"\scriptstyle(2)",phantom,xshift=-2.5*\the\DL]%
        \arrow[from=2-2,to=3-3,"\scriptstyle(3)",phantom]%
        \arrow[from=2-3,to=3-3,"\scriptstyle(4)",phantom,xshift=2.5*\the\DL]%
        \arrow[from=2-2,to=3-3,"\scriptstyle(5)",phantom,yshift=-5.0*\the\DL]%
    \end{tikzcd}
\]%
\subsection{Double Hexagon Diagram}\label{subsection-tikz-code-for-commutative-diagrams-double-hexagon-diagram}
Define
\begin{verbatim}
\newlength{\OneCm}
\setlength{\OneCm}{1.0cm}
\end{verbatim}
The code
\begin{verbatim}
\begin{tikzcd}[row sep={0.0*\the\DL,between origins}, column sep={0.0*\the\DL,between origins}, background color=backgroundColor, ampersand replacement=\&]
    \&[1.73205081*\OneCm]
    \&[1.73205081*\OneCm]
    \text{1-3}
    \&[1.73205081*\OneCm]
    \&[1.73205081*\OneCm]
    \\[2.0*\OneCm]
    \text{2-1}
    \&[1.73205081*\OneCm]
    \&[1.73205081*\OneCm]
    \text{2-3}
    \&[1.73205081*\OneCm]
    \&[1.73205081*\OneCm]
    \text{2-5}
    \\[1.0*\OneCm]
    \&[1.73205081*\OneCm]
    \text{3-2}
    \&[1.73205081*\OneCm]
    \&[1.73205081*\OneCm]
    \text{3-4}
    \&[1.73205081*\OneCm]
    \\[2.0*\OneCm]
    \&[1.73205081*\OneCm]
    \text{4-2}
    \&[1.73205081*\OneCm]
    \&[1.73205081*\OneCm]
    \text{4-4}
    \&[1.73205081*\OneCm]
    \\[1.0*\OneCm]
    \text{5-1}
    \&[1.73205081*\OneCm]
    \&[1.73205081*\OneCm]
    \text{5-3}
    \&[1.73205081*\OneCm]
    \&[1.73205081*\OneCm]
    \text{5-5}
    \\[2.0*\OneCm]
    \&[1.73205081*\OneCm]
    \&[1.73205081*\OneCm]
    \text{6-3}
    \&[1.73205081*\OneCm]
    \&[1.73205081*\OneCm]
    % Arrows
    \arrow[from=1-3,to=2-1,"1"']%
    \arrow[from=2-1,to=5-1,"2"']%
    \arrow[from=5-1,to=6-3,"3"']%
    %
    \arrow[from=1-3,to=2-5,"4"]%
    \arrow[from=2-5,to=5-5,"5"]%
    \arrow[from=5-5,to=6-3,"6"]%
    %
    \arrow[from=2-3,to=3-2,"1'"description]%
    \arrow[from=3-2,to=4-2,"2'"description]%
    \arrow[from=4-2,to=5-3,"3'"description]%
    %
    \arrow[from=2-3,to=3-4,"4'"description]%
    \arrow[from=3-4,to=4-4,"5'"description]%
    \arrow[from=4-4,to=5-3,"6'"description]%
    %
    \arrow[from=1-3,to=2-3,"a"description]%
    \arrow[from=2-1,to=3-2,"b"description]%
    \arrow[from=2-5,to=3-4,"c"description]%
    \arrow[from=5-1,to=4-2,"d"description]%
    \arrow[from=5-5,to=4-4,"e"description]%
    \arrow[from=5-3,to=6-3,"f"description]%
\end{tikzcd}
\end{verbatim}
will produce the following double hexagon diagram:
\[
    \begin{tikzcd}[row sep={0.0*\the\DL,between origins}, column sep={0.0*\the\DL,between origins}, background color=backgroundColor, ampersand replacement=\&]
        \&[1.73205081*\OneCm]
        \&[1.73205081*\OneCm]
        \text{1-3}
        \&[1.73205081*\OneCm]
        \&[1.73205081*\OneCm]
        \\[2.0*\OneCm]
        \text{2-1}
        \&[1.73205081*\OneCm]
        \&[1.73205081*\OneCm]
        \text{2-3}
        \&[1.73205081*\OneCm]
        \&[1.73205081*\OneCm]
        \text{2-5}
        \\[1.0*\OneCm]
        \&[1.73205081*\OneCm]
        \text{3-2}
        \&[1.73205081*\OneCm]
        \&[1.73205081*\OneCm]
        \text{3-4}
        \&[1.73205081*\OneCm]
        \\[2.0*\OneCm]
        \&[1.73205081*\OneCm]
        \text{4-2}
        \&[1.73205081*\OneCm]
        \&[1.73205081*\OneCm]
        \text{4-4}
        \&[1.73205081*\OneCm]
        \\[1.0*\OneCm]
        \text{5-1}
        \&[1.73205081*\OneCm]
        \&[1.73205081*\OneCm]
        \text{5-3}
        \&[1.73205081*\OneCm]
        \&[1.73205081*\OneCm]
        \text{5-5}
        \\[2.0*\OneCm]
        \&[1.73205081*\OneCm]
        \&[1.73205081*\OneCm]
        \text{6-3}
        \&[1.73205081*\OneCm]
        \&[1.73205081*\OneCm]
        % Arrows
        \arrow[from=1-3,to=2-1,"1"']%
        \arrow[from=2-1,to=5-1,"2"']%
        \arrow[from=5-1,to=6-3,"3"']%
        %
        \arrow[from=1-3,to=2-5,"4"]%
        \arrow[from=2-5,to=5-5,"5"]%
        \arrow[from=5-5,to=6-3,"6"]%
        %
        \arrow[from=2-3,to=3-2,"1'"description]%
        \arrow[from=3-2,to=4-2,"2'"description]%
        \arrow[from=4-2,to=5-3,"3'"description]%
        %
        \arrow[from=2-3,to=3-4,"4'"description]%
        \arrow[from=3-4,to=4-4,"5'"description]%
        \arrow[from=4-4,to=5-3,"6'"description]%
        %
        \arrow[from=1-3,to=2-3,"a"description]%
        \arrow[from=2-1,to=3-2,"b"description]%
        \arrow[from=2-5,to=3-4,"c"description]%
        \arrow[from=5-1,to=4-2,"d"description]%
        \arrow[from=5-5,to=4-4,"e"description]%
        \arrow[from=5-3,to=6-3,"f"description]%
    \end{tikzcd}
\]%
To make the diagram larger, one could use e.g.
\begin{verbatim}
\newlength{\TwoCm}
\setlength{\TwoCm}{2.0cm}
\end{verbatim}
and replace all instances of \texttt{\textbackslash OneCm} with \texttt{\textbackslash TwoCm} in the code above.
\section{Retired Tags}\label{section-retired-tags}
\subsection{Relations}\label{subsection-retired-tags-relations}
\begin{oldtag}{Equivalent Definitions of Relations}{equivalent-definitions-of-relations}%
    The content of this tag has been moved to \ChapterRef{\ChapterRelations, \cref{relations:relations}}{\cref{relations}}.
\end{oldtag}
\begin{oldtag}{Interaction Between Composition and Characteristic Relations}{properties-of-composition-of-relations-interaction-with-composition}%
    The original statement of this tag was false.
\end{oldtag}
\begin{oldtag}{Interaction Between Composition and Characteristic Relations}{properties-of-converses-of-relations-interaction-with-composition-2}%
    The original statement of this tag was false.
\end{oldtag}
\begin{oldtag}{Explicit Description of Internal Left Kan Extensions Along Functions}{explicit-description-of-internal-left-kan-extensions-along-functions}%
    This was a question. Now an explicit description is available as \ChapterRef{\ChapterRelations, \cref{relations:left-kan-extensions-along-functions}}{\cref{left-kan-extensions-along-functions}}. 
\end{oldtag}
\begin{oldtag}{Explicit Description of Internal Left Kan Lifts Along Functions}{explicit-description-of-internal-left-kan-lifts-along-functions}%
    This was a question. Now an explicit description is available as \ChapterRef{\ChapterRelations, \cref{relations:left-kan-lifts-along-functions}}{\cref{left-kan-lifts-along-functions}}. 
\end{oldtag}
\begin{oldtag}{Internal Kan Extensions and Lifts}{subsection-internal-kan-extensions-and-lifts-in-rel}%
    This tag is obsolete; see \ChapterRef{\ChapterRelations, \cref{relations:subsection-internal-left-kan-extensions-in-rel,relations:subsection-internal-left-kan-lifts-in-rel,relations:subsection-internal-right-kan-extensions-in-rel,relations:subsection-internal-right-kan-lifts-in-rel}}{\cref{subsection-internal-left-kan-extensions-in-rel,subsection-internal-left-kan-lifts-in-rel,subsection-internal-right-kan-extensions-in-rel,subsection-internal-right-kan-lifts-in-rel}} instead.
\end{oldtag}
\begin{oldtag}{Internal Kan Extensions and Lifts}{kan-extensions-and-kan-lifts-in-rel}%
    This tag is obsolete; see \ChapterRef{\ChapterRelations, \cref{relations:subsection-internal-left-kan-extensions-in-rel,relations:subsection-internal-left-kan-lifts-in-rel,relations:subsection-internal-right-kan-extensions-in-rel,relations:subsection-internal-right-kan-lifts-in-rel}}{\cref{subsection-internal-left-kan-extensions-in-rel,subsection-internal-left-kan-lifts-in-rel,subsection-internal-right-kan-extensions-in-rel,subsection-internal-right-kan-lifts-in-rel}} instead.
\end{oldtag}
\begin{oldtag}{Internal Kan Extensions and Lifts}{section-kan-extensions-and-kan-lifts-in-the-2-category-of-relations}%
    This tag is obsolete; see \ChapterRef{\ChapterRelations, \cref{relations:subsection-internal-left-kan-extensions-in-rel,relations:subsection-internal-left-kan-lifts-in-rel,relations:subsection-internal-right-kan-extensions-in-rel,relations:subsection-internal-right-kan-lifts-in-rel}}{\cref{subsection-internal-left-kan-extensions-in-rel,subsection-internal-left-kan-lifts-in-rel,subsection-internal-right-kan-extensions-in-rel,subsection-internal-right-kan-lifts-in-rel}} instead.
\end{oldtag}
\begin{oldtag}{Better Characterisations of Representably Full Morphisms in $\sfbfRel$}{better-characterisations-of-representably-full-morphisms-in-rel}%
    This was originally a question. It has been answered in \ChapterRef{\ChapterRelations, \cref{relations:subsection-monomorphisms-and-2-categorical-monomorphisms-in-rel}}{\cref{subsection-monomorphisms-and-2-categorical-monomorphisms-in-rel}}.
\end{oldtag}
\begin{oldtag}{Better Characterisations of Corepresentably Full Morphisms in $\sfbfRel$}{better-characterisations-of-corepresentably-full-morphisms-in-rel}%
    This was originally a question. It has been answered in \ChapterRef{\ChapterRelations, \cref{relations:subsection-epimorphisms-and-2-categorical-epimorphisms-in-rel}}{\cref{subsection-epimorphisms-and-2-categorical-epimorphisms-in-rel}}.
\end{oldtag}
\begin{oldtag}{Characterisation of Monomorphisms in $\sfRel$}{characterisations-of-monomorphisms-in-rel}%
    Superseded by \ChapterRef{\ChapterRelations, \cref{relations:subsection-monomorphisms-and-2-categorical-monomorphisms-in-rel}}{\cref{subsection-monomorphisms-and-2-categorical-monomorphisms-in-rel}}.
\end{oldtag}
\begin{oldtag}{Characterisation of 2-Categorical Monomorphisms in $\sfRel$}{subsection-2-categorical-monomorphisms-in-rel}%
    Superseded by \ChapterRef{\ChapterRelations, \cref{relations:subsection-monomorphisms-and-2-categorical-monomorphisms-in-rel}}{\cref{subsection-monomorphisms-and-2-categorical-monomorphisms-in-rel}}.
\end{oldtag}
\begin{oldtag}{2-Categorical Monomorphisms in $\sfRel$}{2-categorical-monomorphisms-in-rel}%
    Superseded by \ChapterRef{\ChapterRelations, \cref{relations:subsection-monomorphisms-and-2-categorical-monomorphisms-in-rel}}{\cref{subsection-monomorphisms-and-2-categorical-monomorphisms-in-rel}}.
\end{oldtag}
\begin{oldtag}{2-Categorical Monomorphisms in $\sfRel$}{2-categorical-monomorphisms-in-rel-representably-faithful-morphisms-in-rel}
    Superseded by \ChapterRef{\ChapterRelations, \cref{relations:subsection-monomorphisms-and-2-categorical-monomorphisms-in-rel}}{\cref{subsection-monomorphisms-and-2-categorical-monomorphisms-in-rel}}.
\end{oldtag}
\begin{oldtag}{2-Categorical Monomorphisms in $\sfRel$}{2-categorical-monomorphisms-in-rel-representably-full-morphisms-in-rel}
    Superseded by \ChapterRef{\ChapterRelations, \cref{relations:subsection-monomorphisms-and-2-categorical-monomorphisms-in-rel}}{\cref{subsection-monomorphisms-and-2-categorical-monomorphisms-in-rel}}.
\end{oldtag}
\begin{oldtag}{2-Categorical Monomorphisms in $\sfRel$}{2-categorical-monomorphisms-in-rel-representably-full-morphisms-in-rel-1}
    Superseded by \ChapterRef{\ChapterRelations, \cref{relations:subsection-monomorphisms-and-2-categorical-monomorphisms-in-rel}}{\cref{subsection-monomorphisms-and-2-categorical-monomorphisms-in-rel}}.
\end{oldtag}
\begin{oldtag}{2-Categorical Monomorphisms in $\sfRel$}{2-categorical-monomorphisms-in-rel-representably-full-morphisms-in-rel-2}
    Superseded by \ChapterRef{\ChapterRelations, \cref{relations:subsection-monomorphisms-and-2-categorical-monomorphisms-in-rel}}{\cref{subsection-monomorphisms-and-2-categorical-monomorphisms-in-rel}}.
\end{oldtag}
\begin{oldtag}{2-Categorical Monomorphisms in $\sfRel$}{2-categorical-monomorphisms-in-rel-representably-full-morphisms-in-rel-3}
    Superseded by \ChapterRef{\ChapterRelations, \cref{relations:subsection-monomorphisms-and-2-categorical-monomorphisms-in-rel}}{\cref{subsection-monomorphisms-and-2-categorical-monomorphisms-in-rel}}.
\end{oldtag}
\begin{oldtag}{2-Categorical Monomorphisms in $\sfRel$}{2-categorical-monomorphisms-in-rel-representably-full-morphisms-in-rel-4}
    Superseded by \ChapterRef{\ChapterRelations, \cref{relations:subsection-monomorphisms-and-2-categorical-monomorphisms-in-rel}}{\cref{subsection-monomorphisms-and-2-categorical-monomorphisms-in-rel}}.
\end{oldtag}
\begin{oldtag}{2-Categorical Monomorphisms in $\sfRel$}{2-categorical-monomorphisms-in-rel-representably-full-morphisms-in-rel-5}
    Superseded by \ChapterRef{\ChapterRelations, \cref{relations:subsection-monomorphisms-and-2-categorical-monomorphisms-in-rel}}{\cref{subsection-monomorphisms-and-2-categorical-monomorphisms-in-rel}}.
\end{oldtag}
\begin{oldtag}{2-Categorical Monomorphisms in $\sfRel$}{2-categorical-monomorphisms-in-rel-representably-full-morphisms-in-rel-6}
    Superseded by \ChapterRef{\ChapterRelations, \cref{relations:subsection-monomorphisms-and-2-categorical-monomorphisms-in-rel}}{\cref{subsection-monomorphisms-and-2-categorical-monomorphisms-in-rel}}.
\end{oldtag}
\begin{oldtag}{2-Categorical Monomorphisms in $\sfRel$}{2-categorical-monomorphisms-in-rel-representably-fully-faithful-morphisms-in-rel}
    Superseded by \ChapterRef{\ChapterRelations, \cref{relations:subsection-monomorphisms-and-2-categorical-monomorphisms-in-rel}}{\cref{subsection-monomorphisms-and-2-categorical-monomorphisms-in-rel}}.
\end{oldtag}
\begin{oldtag}{Characterisation of Epimorphisms in $\sfRel$}{characterisations-of-epimorphisms-in-rel}%
    Superseded by \ChapterRef{\ChapterRelations, \cref{relations:subsection-epimorphisms-and-2-categorical-epimorphisms-in-rel}}{\cref{subsection-epimorphisms-and-2-categorical-epimorphisms-in-rel}}.
\end{oldtag}
\begin{oldtag}{Characterisation of Epimorphisms in $\sfRel$}{characterisations-of-epimorphisms-in-rel-4}
    Superseded by \ChapterRef{\ChapterRelations, \cref{relations:subsection-epimorphisms-and-2-categorical-epimorphisms-in-rel}}{\cref{subsection-epimorphisms-and-2-categorical-epimorphisms-in-rel}}.
\end{oldtag}
\begin{oldtag}{2-Categorical Epimorphisms in $\sfRel$}{subsection-2-categorical-epimorphisms-in-rel}
    Superseded by \ChapterRef{\ChapterRelations, \cref{relations:subsection-epimorphisms-and-2-categorical-epimorphisms-in-rel}}{\cref{subsection-epimorphisms-and-2-categorical-epimorphisms-in-rel}}.
\end{oldtag}
\begin{oldtag}{2-Categorical Epimorphisms in $\sfRel$}{2-categorical-epimorphisms-in-rel}
    Superseded by \ChapterRef{\ChapterRelations, \cref{relations:subsection-epimorphisms-and-2-categorical-epimorphisms-in-rel}}{\cref{subsection-epimorphisms-and-2-categorical-epimorphisms-in-rel}}.
\end{oldtag}
\begin{oldtag}{2-Categorical Epimorphisms in $\sfRel$}{2-categorical-epimorphisms-in-rel-corepresentably-faithful-morphisms-in-rel}
    Superseded by \ChapterRef{\ChapterRelations, \cref{relations:subsection-epimorphisms-and-2-categorical-epimorphisms-in-rel}}{\cref{subsection-epimorphisms-and-2-categorical-epimorphisms-in-rel}}.
\end{oldtag}
\begin{oldtag}{2-Categorical Epimorphisms in $\sfRel$}{2-categorical-epimorphisms-in-rel-corepresentably-full-morphisms-in-rel}
    Superseded by \ChapterRef{\ChapterRelations, \cref{relations:subsection-epimorphisms-and-2-categorical-epimorphisms-in-rel}}{\cref{subsection-epimorphisms-and-2-categorical-epimorphisms-in-rel}}.
\end{oldtag}
\begin{oldtag}{2-Categorical Epimorphisms in $\sfRel$}{2-categorical-epimorphisms-in-rel-corepresentably-full-morphisms-in-rel-1}
    Superseded by \ChapterRef{\ChapterRelations, \cref{relations:subsection-epimorphisms-and-2-categorical-epimorphisms-in-rel}}{\cref{subsection-epimorphisms-and-2-categorical-epimorphisms-in-rel}}.
\end{oldtag}
\begin{oldtag}{2-Categorical Epimorphisms in $\sfRel$}{2-categorical-epimorphisms-in-rel-corepresentably-full-morphisms-in-rel-2}
    Superseded by \ChapterRef{\ChapterRelations, \cref{relations:subsection-epimorphisms-and-2-categorical-epimorphisms-in-rel}}{\cref{subsection-epimorphisms-and-2-categorical-epimorphisms-in-rel}}.
\end{oldtag}
\begin{oldtag}{2-Categorical Epimorphisms in $\sfRel$}{2-categorical-epimorphisms-in-rel-corepresentably-full-morphisms-in-rel-3}
    Superseded by \ChapterRef{\ChapterRelations, \cref{relations:subsection-epimorphisms-and-2-categorical-epimorphisms-in-rel}}{\cref{subsection-epimorphisms-and-2-categorical-epimorphisms-in-rel}}.
\end{oldtag}
\begin{oldtag}{2-Categorical Epimorphisms in $\sfRel$}{2-categorical-epimorphisms-in-rel-corepresentably-full-morphisms-in-rel-4}
    Superseded by \ChapterRef{\ChapterRelations, \cref{relations:subsection-epimorphisms-and-2-categorical-epimorphisms-in-rel}}{\cref{subsection-epimorphisms-and-2-categorical-epimorphisms-in-rel}}.
\end{oldtag}
\begin{oldtag}{2-Categorical Epimorphisms in $\sfRel$}{2-categorical-epimorphisms-in-rel-corepresentably-full-morphisms-in-rel-5}
    Superseded by \ChapterRef{\ChapterRelations, \cref{relations:subsection-epimorphisms-and-2-categorical-epimorphisms-in-rel}}{\cref{subsection-epimorphisms-and-2-categorical-epimorphisms-in-rel}}.
\end{oldtag}
\begin{oldtag}{2-Categorical Epimorphisms in $\sfRel$}{2-categorical-epimorphisms-in-rel-corepresentably-full-morphisms-in-rel-6}
    Superseded by \ChapterRef{\ChapterRelations, \cref{relations:subsection-epimorphisms-and-2-categorical-epimorphisms-in-rel}}{\cref{subsection-epimorphisms-and-2-categorical-epimorphisms-in-rel}}.
\end{oldtag}
\begin{oldtag}{2-Categorical Epimorphisms in $\sfRel$}{2-categorical-epimorphisms-in-rel-corepresentably-fully-faithful-morphisms-in-rel}
    Superseded by \ChapterRef{\ChapterRelations, \cref{relations:subsection-epimorphisms-and-2-categorical-epimorphisms-in-rel}}{\cref{subsection-epimorphisms-and-2-categorical-epimorphisms-in-rel}}.
\end{oldtag}
\begin{oldtag}{Epimorphisms in $\sfRel$}{characterisations-of-epimorphisms-in-rel-5}
    Superseded by \ChapterRef{\ChapterRelations, \cref{relations:subsection-epimorphisms-and-2-categorical-epimorphisms-in-rel}}{\cref{subsection-epimorphisms-and-2-categorical-epimorphisms-in-rel}}.
\end{oldtag}
\begin{oldtag}{Epimorphisms in $\sfRel$}{characterisations-of-epimorphisms-in-rel-6}
    Superseded by \ChapterRef{\ChapterRelations, \cref{relations:subsection-epimorphisms-and-2-categorical-epimorphisms-in-rel}}{\cref{subsection-epimorphisms-and-2-categorical-epimorphisms-in-rel}}.
\end{oldtag}
\begin{oldtag}{Interaction With Inverse Images \rmII}{properties-of-coinverse-image-functions-associated-to-relations-1-relation-to-inverse-images-2}%
    Superseded by \ChapterRef{\ChapterRelations, \cref{relations:properties-of-coinverse-image-functions-associated-to-relations-1-interaction-with-functional-relations,relations:properties-of-coinverse-image-functions-associated-to-relations-1-interaction-with-total-relations,relations:properties-of-coinverse-image-functions-associated-to-relations-1-interaction-with-functions-1,relations:properties-of-coinverse-image-functions-associated-to-relations-1-interaction-with-functions-2} of \cref{relations:properties-of-coinverse-image-functions-associated-to-relations-1}}{\cref{properties-of-coinverse-image-functions-associated-to-relations-1-interaction-with-functional-relations,properties-of-coinverse-image-functions-associated-to-relations-1-interaction-with-total-relations,properties-of-coinverse-image-functions-associated-to-relations-1-interaction-with-functions-1,properties-of-coinverse-image-functions-associated-to-relations-1-interaction-with-functions-2} of \cref{properties-of-coinverse-image-functions-associated-to-relations-1}}.%
\end{oldtag}
\begin{oldtag}{Interaction With Coinverse Images \rmII}{properties-of-coinverse-image-functions-associated-to-relations-1-relation-to-coinverse-images-2}%
    Superseded by \ChapterRef{\ChapterRelations, \cref{relations:properties-of-inverse-image-functions-associated-to-relations-1-interaction-with-functional-relations,relations:properties-of-inverse-image-functions-associated-to-relations-1-interaction-with-total-relations,relations:properties-of-inverse-image-functions-associated-to-relations-1-interaction-with-functions-1,relations:properties-of-inverse-image-functions-associated-to-relations-1-interaction-with-functions-2} of \cref{relations:properties-of-inverse-image-functions-associated-to-relations-1}}{\cref{properties-of-inverse-image-functions-associated-to-relations-1-interaction-with-functional-relations,properties-of-inverse-image-functions-associated-to-relations-1-interaction-with-total-relations,properties-of-inverse-image-functions-associated-to-relations-1-interaction-with-functions-1,properties-of-inverse-image-functions-associated-to-relations-1-interaction-with-functions-2} of \cref{properties-of-inverse-image-functions-associated-to-relations-1}}.%
\end{oldtag}
\subsection{Pointed Sets}\label{subsection-retired-tags-pointed-sets}
\begin{oldtag}{The Underlying Pointed Set of a Semimodule}{the-underlying-pointed-set-of-a-semimodule}%
    The \index[set-theory]{pointed set!underlying a semimodule}\index[set-theory]{semimodule!underlying pointed set of}\textbf{underlying pointed set} of a semimodule $(M,\alpha_{M})$ is the pointed set $(M,0_{M})$.
\end{oldtag}
\begin{oldtag}{The Underlying Pointed Set of a Module}{the-underlying-pointed-set-of-a-module}%
    The \index[set-theory]{pointed set!underlying a module}\index[set-theory]{module!underlying pointed set of}\textbf{underlying pointed set} of a module $(M,\alpha_{M})$ is the pointed set $(M,0_{M})$.
\end{oldtag}
\subsection{Tensor Products of Pointed Sets}\label{section-tensor-products-of-pointed-sets}
\begin{oldtag}{Section on Universal Properties of the Smash Product of Pointed Sets \rmI}{subsection-universal-properties-of-the-smash-product-of-pointed-sets-1}%
    Absorbed into \ChapterRef{\ChapterTensorProductsOfPointedSets, \cref{tensor-products-of-pointed-sets:subsection-the-universal-property-of-the-smash-product-of-pointed-sets}}{\cref{subsection-the-universal-property-of-the-smash-product-of-pointed-sets}}.
\end{oldtag}
\begin{oldtag}{Section on Universal Properties of the Smash Product of Pointed Sets \rmII}{subsection-universal-properties-of-the-smash-product-of-pointed-sets-2}%
    Absorbed into \ChapterRef{\ChapterTensorProductsOfPointedSets, \cref{tensor-products-of-pointed-sets:subsection-the-universal-property-of-the-smash-product-of-pointed-sets}}{\cref{subsection-the-universal-property-of-the-smash-product-of-pointed-sets}}.
\end{oldtag}
\begin{oldtag}{Universal Properties of the Smash Product of Pointed Sets \rmI}{universal-properties-of-the-smash-product-of-pointed-sets-1}%
    Absorbed into \ChapterRef{\ChapterTensorProductsOfPointedSets, \cref{tensor-products-of-pointed-sets:subsection-the-universal-property-of-the-smash-product-of-pointed-sets}}{\cref{subsection-the-universal-property-of-the-smash-product-of-pointed-sets}}.
\end{oldtag}
\begin{oldtag}{Universal Properties of the Smash Product of Pointed Sets \rmII}{universal-properties-of-the-smash-product-of-pointed-sets-2}%
    Absorbed into \ChapterRef{\ChapterTensorProductsOfPointedSets, \cref{tensor-products-of-pointed-sets:subsection-the-universal-property-of-the-smash-product-of-pointed-sets}}{\cref{subsection-the-universal-property-of-the-smash-product-of-pointed-sets}}.
\end{oldtag}
\subsection{Categories}\label{subsection-retired-tags-categories}
\begin{oldtag}{Picturing Natural Transformations in Diagrams}{picturing-natural-transformations-in-diagrams}%
    We denote natural transformations in diagrams as
    \[
        \begin{tikzcd}[row sep={5.0*\the\DL,between origins}, column sep={5.0*\the\DL,between origins}, background color=backgroundColor, ampersand replacement=\&]
            \CatFont{C}
            \arrow[r, "F"{name=F}, bend left=30]
            \arrow[r, "G"'{name=G}, bend right=30]
            \&
            \CatFont{D}\mrp{.}
            %--- Adjunction Symbol
            \arrow[from=F, to=G, Rightarrow, shorten=0.25em, "\alpha"'{pos=0.45}]
        \end{tikzcd}
    \]%
    (This tag has been removed and is now part of \ChapterRef{\ChapterCategories, \cref{categories:further-terminology-and-notation-for-natural-transformations}}{\cref{further-terminology-and-notation-for-natural-transformations}}.)
\end{oldtag}
\begin{oldtag}{Interaction Between Fullness and Postcomposition Functors}{interaction-between-fullness-and-postcomposition-functors}%
    (This Tag was an item of \ChapterRef{\ChapterCategories, \cref{categories:properties-of-full-functors}}{\cref{properties-of-full-functors}}, but has since been removed because its statement is incorrect. Naïm Camille Favier provided a counterexample, and the corrected statements now appear as \ChapterRef{\ChapterCategories, \cref{categories:properties-of-full-functors-interaction-with-postcomposition-1,categories:properties-of-full-functors-interaction-with-postcomposition-2} of \cref{categories:properties-of-full-functors}}{\cref{properties-of-full-functors-interaction-with-postcomposition-1,properties-of-full-functors-interaction-with-postcomposition-2} of \cref{properties-of-full-functors}}.)
    \begin{enumerate}
        \item\label{properties-of-full-functors-interaction-with-postcomposition}\SloganFont{Interaction With Postcomposition. }The following conditions are equivalent:
            \begin{enumerate}
                \item\label{properties-of-full-functors-interaction-with-postcomposition-a}The functor $F\colon\CatFont{C}\to\CatFont{D}$ is full.
                \item\label{properties-of-full-functors-interaction-with-postcomposition-b}For each $\CatFont{X}\in\Obj(\Cats)$, the postcomposition functor
                    \[
                        F_{*}
                        \colon
                        \Fun(\CatFont{X},\CatFont{C})
                        \to
                        \Fun(\CatFont{X},\CatFont{D})
                    \]%
                    is full.
                \item\label{properties-of-full-functors-interaction-with-postcomposition-c}The functor $F\colon\CatFont{C}\to\CatFont{D}$ is a representably full morphism in $\TwoCategoryOfCategories$ in the sense of \ChapterRef{\ChapterTypesOfMorphismsInBicategories, \cref{types-of-morphisms-in-bicategories:representably-full-morphisms}}{\cref{representably-full-morphisms}}.
            \end{enumerate}
    \end{enumerate}
\end{oldtag}
\section{Miscellany}\label{section-notes-miscellany}
\subsection{List of Things To Explore/Add}\label{subsection-things-to-explore-add}
Here we list things to be explored in or added to this work in the future. This is a very quick and dirty list; some items may not be fully intelligible.
\begin{remark}{Things To Explore/Add}{things-to-explore-add}%
    Set Theory:
    \begin{enumerate}
        \item \url{https://math.stackexchange.com/questions/200389/show-that-the-set-of-all-finite-subsets-of-mathbbn-is-countable}
        \item \url{https://mathoverflow.net/a/479528}
        \item \url{https://www.maths.ed.ac.uk/~tl/ast/ast.pdf}
        \item Leinster ETCS \url{https://webhomes.maths.ed.ac.uk/~tl/ast/ast.pdf}
    \end{enumerate}
    Type Theory:
    \begin{enumerate}
        \item \url{https://arxiv.org/abs/2508.14320}
        \item \url{https://mathoverflow.net/questions/497570/universes-dont-need-to-be-indexed-by-natural-numbers}
    \end{enumerate}
    Pointed sets:
    \begin{enumerate}
        \item Universal properties (plural!) of the left tensor product of pointed sets
        \item Universal properties (plural!) of the right tensor product of pointed sets
    \end{enumerate}
    Relations:
    \begin{enumerate}
        \item Internal fibrations in $\sfbfRel$, like discrete fibrations and Street fibrations
        \item Return to Eilenberg--Moore and Kleisli objects in $\sfbfRel$ once the general theory has been set up for internal monads
    \end{enumerate}
    Spans:
    \begin{enumerate}
        \item \url{https://arxiv.org/abs/2505.22832}
        \item Spans: study certain compositions of spans like composing $B\xleftarrow{f}A=A$ and $A=A\xleftarrow{g}B$ into a span $B\xleftarrow{f}A\xleftarrow{g}B$
        \item Comparison \emph{double functor} from Span to Rel and vice versa
        \item Apartness composition for spans and alternate compositions for spans in general
        \item non-Cartesian analogue of spans
            \begin{enumerate}
                \item View spans as morphisms $S\to A\times B$ and consider instead morphisms $S\to A\otimes_{\CatFont{C}}B$
            \end{enumerate}
        \item Record the universal property of the bicategory of spans of \url{https://ncatlab.org/nlab/show/span}
        \item \url{https://ncatlab.org/nlab/show/span+trace}
        \item Cospans.
        \item Multispans.
    \end{enumerate}
    Un/Straightening for Indexed and Fibred Sets:
    \begin{enumerate}
        \item Analogue of adjoints for Grothendieck construction for indexed and fibred sets
        \item Write proper sections on straightening for lax functors from Sets to Rel or Span (displayed sets)
        \item co/units for un/straightening adjunction
    \end{enumerate}
    Categories:
    \begin{enumerate}
        \item \url{https://www.numdam.org/actas/SE/}, \url{https://www.numdam.org/journals/CTGDC/}
        \item \url{https://www.numdam.org/item/CTGDC_1966__8__A5_0.pdf}
        \item \url{https://mathoverflow.net/questions/493931/is-the-category-of-posets-locally-cartesian-closed}
        \item From Keith: Presheaves on a topological space $X$ valued in $\TTV$
            \begin{enumerate}
                \item They are the same as collections of open subsets of $X$
                \item They are sheaves \textiff that collection is closed under union
                \item Their sheafification is the closure of that collection under unions
            \end{enumerate}
        \item \url{https://arxiv.org/abs/2504.20949}
        \item Notion of equality that is weaker than equivalence but stronger than adjunction
        \item Tangent categories, Beck modules, categorical derivations
        \item Flat functors
        \item Is the classifying space of a category isomorphic to $\Ex^{\infty}$ of the nerve of the category? If so, an intuition for having an initial/terminal object implying being homotopically contractible is that taking the free $\infty$-groupoid generated by that identifies every object with the terminal one.
        \item \url{https://en.wikipedia.org/wiki/Category_algebra}
        \item simple objects
        \item \url{https://mathoverflow.net/questions/442212/properties-of-categorical-zeta-function}
        \item Polynomial functors, \url{https://ncatlab.org/nlab/show/polynomial+functor}, \url{https://arxiv.org/abs/2312.00990}
        \item \url{https://ncatlab.org/nlab/show/simple+object}
        \item \url{https://mathoverflow.net/questions/442212/properties-of-categorical-zeta-function}
        \item \url{https://arxiv.org/abs/2409.17489}
        \item \url{https://mathoverflow.net/a/478644}
        \item Posetal category associated to a poset as a right adjoint
        \item \say{Presetal category} associated to a preordered set
        \item Vopenka's principle simplifies stuff in the theory of locally presentable categories. If we build categories using type theory or HoTT, what stuff from vopenka holds?
        \item Are pseudoepic functors those functors whose restricted Yoneda embedding is pseudomonic and Yoneda preserves absolute colimits?
        \item Absolutely dense functors enriched over $\R^{+}$ apparently reduce to topological density
        \item Is there a reasonable notion of category homology? It is very common for the geometric realisation of a category to be contractible (e.g.\ having an initial or terminal object), but maybe some notion of directed homology could work here
        \item Nerves of categories:
            \begin{enumerate}
                \item Dihedral and symmetric nerves of categories via groupoids (define them first for groupoids and then Kan extend along $\Grpd\hookrightarrow\Cats$)
                    \begin{enumerate}
                        \item Same applies to twisted nerves
                    \end{enumerate}
                \item Cyclic nerve of a category
                \item Crossed Simplicial Group Categorical Nerves, \url{https://arxiv.org/abs/1603.08768}
            \end{enumerate}
        \item Define contractible categories and add a discussion of universal properties as stating that certain categories are contractible. (Example of non-unique isomorphisms as e.g.\ being a group of order $5$ corresponds to all objects being isomorphic but the category not being contractible)
        \item Expand \cref{construction-of-the-groupoid-completion-of-a-category} and add a proof to it.
        \item Sections and retractions; retracts, \url{https://ncatlab.org/nlab/show/retract}.
        \item Groupoid cardinality
            \begin{enumerate}
                \item \url{https://mathoverflow.net/questions/376175/category-theory-and-arithmetical-identities/376223#376223}
                \item \url{https://mathoverflow.net/questions/420088/groupoid-cardinality-of-the-class-of-abelian-p-groups?rq=1}
                \item \url{https://mathoverflow.net/questions/363292/what-is-the-groupoid-cardinality-of-the-category-of-vector-spaces-over-a-finite}
                \item The groupoid cardinality of the core of the category of finite sets is $\e$. What is the groupoid cardinality of the core of $\FinSets_{G}$?
                \item groupoid cardinality of the core of the category of finite G-sets, \url{https://www.arxiv.org/pdf/2502.03585}
                \item \url{https://ncatlab.org/nlab/show/groupoid+cardinality}
                \item \url{https://arxiv.org/abs/2104.11399}
                \item \url{https://terrytao.wordpress.com/2017/04/13/counting-objects-up-to-isomorphism-groupoid-cardinality/}
                \item \url{https://arxiv.org/abs/0809.2130}
                \item \url{https://qchu.wordpress.com/2012/11/08/groupoid-cardinality/}
                \item \url{https://mathoverflow.net/questions/363292/what-is-the-groupoid-cardinality-of-the-category-of-vector-spaces-over-a-finite}
            \end{enumerate}
        \item combinatorial species
            \begin{enumerate}
                \item \url{https://ncatlab.org/nlab/show/Schur+functor}
                    \begin{enumerate}
                        \item Equivalence between twisted commutative algebras and algebras on categories of polynomial functors, \url{https://mathweb.ucsd.edu/~ssam/talks/2014/ihp-tca.pdf}
                    \end{enumerate}
                \item \url{https://mathoverflow.net/questions/22462/what-are-some-examples-of-interesting-uses-of-the-theory-of-combinatorial-specie}
                \item \url{https://en.wikipedia.org/wiki/Combinatorial_species}
            \end{enumerate}
        \item Leinster's the eventual image, \url{https://arxiv.org/abs/2210.00302}
            \begin{enumerate}
                \item Telescope notation $\mathrm{tel}_{\phi}(X)\defeq\colim(X\xrightarrow{\phi}X\xrightarrow{\phi}\xrightarrow{\phi}\cdots)$ introduced in \url{https://arxiv.org/abs/2505.06979}
            \end{enumerate}
        \item \url{https://ncatlab.org/nlab/show/separable+functor}
        \item Dagger categories:
            \begin{enumerate}
                \item \url{https://en.wikipedia.org/wiki/Dagger_category}
                \item \url{https://ncatlab.org/nlab/show/dagger+category}
                \item Dagger compact categories, \url{https://en.wikipedia.org/wiki/Dagger_compact_category}
                \item \url{https://mathoverflow.net/questions/220032/are-dagger-categories-truly-evil}
                \item generalisation of dagger categories to categories with duality, i.e.\ categories $\CatFont{C}$ together with a functor $\dagger\colon\CatFont{C}^{\op}\to\CatFont{C}$
                    \begin{enumerate}
                        \item Perhaps with the additional condition that $\dagger\circ\dagger=\id$
                        \item categories with involutions in general
                    \end{enumerate}
            \end{enumerate}
    \end{enumerate}
    Regular Categories:
    \begin{enumerate}
        \item \url{https://arxiv.org/pdf/2004.08964.pdf}.
        \item Internal relations
    \end{enumerate}
    Types of Morphisms in Categories:
    \begin{enumerate}
        \item \url{https://mathoverflow.net/questions/490476/duality-of-injectivity-surjectivity-of-precomposition-map} for motivation of monomorphisms/epimorphisms
        \item Characterisation of epimorphisms in the category of fields, \url{https://math.stackexchange.com/q/4941660}
        \item Strong epimorphisms
        \item Behaviour in $\Fun(\CatFont{C},\CatFont{D})$, e.g.\ pointwise sections vs.\ sections in $\Fun(\CatFont{C},\CatFont{D})$.
        \item Faithful functors from balanced categories are conservative
        \item Natural cotransformations:
            \begin{enumerate}
                \item If there is a natural transformation between functors between categories, taking nerves gives a homotopy equivalence (or something like that). What happens for natural cotransformations?
                \item Natural transformations come with a vertical composition map
                    \[
                        \circ%
                        \colon%
                        \coprod_{G\in\Fun(\CatFont{C},\CatFont{D})}\Nat(G,H)\times\Nat(F,G)%
                        \to%
                        \Nat(F,H).%
                    \]%
                    As Morgan Rogers shows \href{https://categorytheory.zulipchat.com/#narrow/stream/229136-theory.3A-category-theory/topic/.E2.80.9CNatural.20cotransformations.E2.80.9D/near/436863628}{here}, there's no vertical cocomposition map of the form
                    \[
                        \CoNat(F,H)%
                        \to%
                        \prod_{G\in\Fun(\CatFont{C},\CatFont{D})}\CoNat(G,H)\times\CoNat(F,G)%
                    \]%
                    or of the form
                    \[
                        \CoNat(F,H)%
                        \to%
                        \prod_{G\in\Fun(\CatFont{C},\CatFont{D})}\CoNat(G,H)\icoprod\CoNat(F,G)%
                    \]%
                    for natural cotransformations.
                \item Cap product for CoNat and Nat
                    \begin{enumerate}
                        \item recovers map $\rmZ(G)\times\Cl(G)\to\Cl(G)$.
                    \end{enumerate}
                \item What is the geometric realisation of $\mathrm{CoTrans}(F,G)$?
                    \begin{enumerate}
                        \item Related: \url{https://mathoverflow.net/questions/89753/geometric-realization-of-hochschild-complex}
                    \end{enumerate}
                \item What is the totalisation of $\mathrm{Trans}(F,G)$?
                    \begin{enumerate}
                        \item If we view sets as discrete topological spaces, what are the homotopy/homology groups of it? The nLab says this (\url{https://ncatlab.org/nlab/show/totalization}):
                            \begin{quote}
                                The homotopy groups of the totalization of a cosimplicial space are computed by a Bousfield-Kan spectral sequence.

                                The homology groups by an Eilenberg-Moore spectral sequence.
                            \end{quote}
                    \end{enumerate}
                \item Abstract
            \end{enumerate}
    \end{enumerate}
    Adjunctions:
    \begin{enumerate}
        \item Relative adjunctions: message Alyssa asking for her notes
        \item Adjunctions, units, counits, and fully faithfulness as in \url{https://mathoverflow.net/questions/100808/properties-of-functors-and-their-adjoints}.
        \item Morphisms between adjunctions and bicategory $\Adj(\CatFont{C})$.
        \item \url{https://ncatlab.org/nlab/show/transformation+of+adjoints}
    \end{enumerate}
    Presheaves and the Yoneda Lemma:
    \begin{enumerate}
        \item \url{https://mathoverflow.net/questions/498069/products-and-coproducts-in-the-category-of-elements-of-a-presheaf}
        \item Yoneda extension along $\yo_{\CatFont{D}}\circ F\colon\CatFont{C}\to\PSh(\CatFont{D})$, giving a functor left adjoint to the precomposition functor $F^{*}\colon\PSh(\CatFont{D})\to\PSh(\CatFont{C})$.
        \item Consider the diagram
            \[
                \begin{tikzcd}[row sep={5.0*\the\DL,between origins}, column sep={5.0*\the\DL,between origins}, background color=backgroundColor, ampersand replacement=\&]
                    \&
                    \PSh(\CatFont{C})
                    \arrow[d,dashed]
                    \arrow[rd,dashed]
                    \&
                    \\
                    \CatFont{C}
                    \arrow[r]
                    \arrow[ru,hook]
                    \&
                    \CatFont{D}
                    \arrow[r,hook]
                    \&
                    \PSh(\CatFont{D})
                \end{tikzcd}
            \]%
        \item Does the functor tensor product admit a right adjoint (\say{Hom}) in some sense?
        \item Yoneda embedding preserves limits
        \item universal objects and universal elements
        \item adjoints to the Yoneda embedding and total categories
        \item The co-Yoneda lemma: co/presheaves are colimits of co/representables
        \item Properties of categories of copresheaves
        \item Contravariant restricted Yoneda embedding
        \item Contravariant Yoneda extensions
        \item Make table of $\Lift_{\yo}(\yo)$, $\Ran_{\yo}(\yo)$, $\Ran_{\yo}(\coyo)$, etc.
        \item Properties of restricted Yoneda embedding, e.g.\ if the restricted Yoneda embedding is full, then what can we conclude? Related: \url{https://qchu.wordpress.com/2015/05/17/generators/}
        \item Tensor product of functors and relation to profunctors
        \item rifts and rans and lifts and lans involving yoneda in $\Cats$ and $\Prof$
        \item Tensor product of functors and relation to rifts and rans of profunctors
    \end{enumerate}
    Isbell Duality:
    \begin{enumerate}
        \item enriched Isbell over walking chain complex
        \item Isbell self-dual presheaves for Lawvere metric spaces; when
            \[
                f(x)%
                =%
                \sup_{x\in X}(\abs{f(x)-\sup_{y\in X}(\abs{f(y)-\d_{X}(y,x)})})%
            \]%
            holds.
        \item \url{https://ncatlab.org/nlab/show/Fr\%C3\%B6licher+spaces+and+Isbell+envelopes}
        \item \url{https://ncatlab.org/nlab/show/envelope+of+an+adjunction}
        \item \url{https://ncatlab.org/nlab/show/nucleus+of+a+profunctor}
        \item \url{https://ncatlab.org/nlab/show/nuclear+adjunction}
        \item \url{https://ncatlab.org/nlab/show/fixed+point+of+an+adjunction}
        \item \textbf{Important: }I should reconsider going with the notation $\IsbellO$ and $\IsbellSpec$. Although a bit common in the (somewhat scarce) literature on Isbell duality, I have doubts regarding how useful/nice of a choice $\IsbellO$ and $\IsbellSpec$ are, and whether there are better choices of notation for them.
        \item Interaction with $\times$, $\Hom$, $F_{!}$, $F^{*}$, and $F_{*}$
        \item Interactions between presheaves and copresheaves:
            \begin{enumerate}
                \item Natural transformations from a presheaf to a copresheaf and vice versa
                \item Mixed Day convolution?
            \end{enumerate}
        \item Isbell duality for monoids:
            \begin{enumerate}
                \item Set up a dictionary between properties of $\Sets^{\rmL}_{A}$ or $\Sets^{\rmR}_{A}$ and properties of $A$
                \item Do the same for $\IsbellO$ given by $A\mapsto\Sets^{\rmL}_{A}(X,A)$
                \item Do the same for $\IsbellSpec$ given by $A\mapsto\Sets^{\rmR}_{A}(X,A)$
                \item Do the same for $\IsbellO\circ\IsbellSpec$
                \item Do the same for $\IsbellSpec\circ\IsbellO$
                \item Algebras for $\IsbellSpec\circ\IsbellO$
                \item Coalgebras for $\IsbellO\circ\IsbellSpec$
            \end{enumerate}
        \item Properties of $\IsbellSpec$ (e.g.\ fully faithfulness) vs.\ properties of $\CatFont{C}$
        \item Properties of $\IsbellO$ (e.g.\ fully faithfulness) vs.\ properties of $\CatFont{C}$
        \item co/unit being monomorphism/epimorphism
        \item reflexive completion
        \item Isbell duality for simplicial sets; what's the reflexive completion?
        \item Isbell envelope
        \item What does Isbell duality look like, when Cat(Aop,Set) is identified with the category of discrete opfibrations over A, using A.5.14?
        \item Generalizations of Isbell duality:
            \begin{enumerate}
                \item Monoidal Isbell duality: monoidality for Isbell adjunction with day convolution (6.3 of coend cofriend)
                \item Isbell duality with sheaves
                \item Isbell duality with Lawvere theories, product preserving functors or whatever
                \item Isbell duality for profunctors
                    \begin{enumerate}
                        \item In view of \cref{properties-of-the-isbell-o-functor-as-a-right-kan-lift-in-prof} of \cref{properties-of-the-isbell-o-functor}, can we just use right Kan lifts/extensions?
                        \item Right Kan lift/extension of Hom functors (there's probably a version of the Yoneda lemma here)
                            \begin{enumerate}
                                \item What is $\Rift_{F}(\Hom_{\CatFont{C}})$
                                \item What is $\Ran_{F}(\Hom_{\CatFont{C}})$
                                \item What is $\Rift_{\Hom_{\CatFont{C}}}(F)$
                                \item What is $\Ran_{\Hom_{\CatFont{C}}}(F)$
                                \item What is $\Lift_{F}(\Hom_{\CatFont{C}})$
                                \item What is $\Lan_{F}(\Hom_{\CatFont{C}})$
                                \item What is $\Lift_{\Hom_{\CatFont{C}}}(F)$
                                \item What is $\Lan_{\Hom_{\CatFont{C}}}(F)$
                            \end{enumerate}
                    \end{enumerate}
            \end{enumerate}
        \item Tensor product of functors and Isbell duality
            \begin{enumerate}
                \item What is $\SheafFont{F}\boxtimes_{\CatFont{C}}\IsbellO(\SheafFont{F})$?
                \item What is $\IsbellSpec(F)\boxtimes_{\CatFont{C}}F$?
                \item I think there is a canonical morphism
                    \[
                        \SheafFont{F}\boxtimes_{\CatFont{C}}\IsbellO(\SheafFont{F})%
                        \to%
                        \Tr(\CatFont{C}).%
                    \]%
                    By the way, what is $\Tr(\SimplexCategory)$? What is $\Tr(\B{A})$? What about $\Nat(\id_{\CatFont{C}},\id_{\CatFont{C}})$ for $\CatFont{C}=\B{A}$ or $\CatFont{C}=\SimplexCategory$
            \end{enumerate}
        \item Isbell with coends:
            \begin{enumerate}
                \item $\Hom(F(A),h_A)$ but it's a coend
                \item Conatural transformations and all that
            \end{enumerate}
        \item Co/limit preservation for O/Spec
        \item Isbell duality for N vs. N + N
        \item What do we get if we replace $\IsbellO\defeq\Nat(-,h_{X})$ by $\Nat^{[W]}(-,h_{X})$, and in particular by $\DiNat(-,h_{X})$?
    \end{enumerate}
    Species:
    \begin{enumerate}
        \item Joyal--Street's $q$-species; via promonoidal structures \url{https://arxiv.org/pdf/1201.2991#page=22}
        \item associators, braidings, unitors; $\F^{n}_{q}\to\F^{n}_{q}$ centre of $\GL_{n}(\F_{q})$ trick
        \item group completion of $\mathcal{GL}(\F_{q})$ as algebraic k-theory
    \end{enumerate}
    Constructions With Categories:
    \begin{enumerate}
        \item \url{https://arxiv.org/abs/2504.21764}
        \item Comparison between pseudopullbacks and isocomma categories: the \say{evident} functor $\CatFont{C}\times^{\sfps}_{\CatFont{E}}\CatFont{D}\to\CatFont{C}\isocomma_{\CatFont{E}}\CatFont{D}$ is essentially surjective and full, but not faithful in general.
        \item Quotients of categories by actions of monoidal categories
            \begin{enumerate}
                \item Quotients of categories by actions of monoids $\B{A}$
                \item Quotients of categories by actions of monoids $A_{\disc}$
                \item Lax, oplax, pseudo, strict, etc.\ quotients of categories
                \item lax Kan extensions along $\B{\CatFont{C}}\to\B{\CatFont{D}}$ for $\CatFont{C}\to\CatFont{D}$ a monoidal functor
            \end{enumerate}
        \item Quotient of $\Fun(\B{A},\CatFont{C})$ by the $A$-action.
            \begin{enumerate}
                \item This is used to build the cycle and $p$-cycle categories from the paracycle category.
                \item The quotient of $\Fun(\B\N,\CatFont{C})$ by the $\N$-action should act as a kind of cyclic directed loop space of $\CatFont{C}$
            \end{enumerate}
        \item $\Fun(\B\N,\CatFont{C})$ as a homotopy pullback in $\TwoCategoryOfCategories$
            \begin{enumerate}
                \item $\Fun(\B\Z,\CatFont{C})$ as a homotopy pullback in $\TwoCategoryOfGroupoids$
                \item Free loop space objects
            \end{enumerate}
    \end{enumerate}
    Limits and colimits:
    \begin{enumerate}
        \item adjunction between co/product and diagonal; abstract version of \cref{properties-of-products-of-sets-adjointness-2} and \cref{properties-of-coproducts-of-sets-adjointness}
        \item Examples of kan extensions along functors of the form $\FinSets\hookrightarrow\Sets$
        \item Initial/terminal objects as left/right adjoints to $!_{\CatFont{C}}\colon\CatFont{C}\to\PunctualCategory$.
        \item A small cocomplete category is a poset, \url{https://mathoverflow.net/questions/108737/small-categories-and-completeness}
        \item Co/limits in $\B{A}$, including e.g.\ co/equalisers in $\B{A}$
        \item Add the characterisations of absolutely dense functors given in \cref{https://ncatlab.org/nlab/show/absolutely+dense+functor} to \cref{properties-of-fully-faithful-functors-interaction-with-precomposition-4}.
        \item Absolutely dense functors, \url{https://ncatlab.org/nlab/show/absolutely+dense+functor}. Also theorem 1.1 here: \url{http://www.tac.mta.ca/tac/volumes/8/n20/n20.pdf}.
        \item Dense functors, codense functors, and absolutely codense functors.
        \item van kampen colimits
    \end{enumerate}
    Completions and cocompletions:
    \begin{enumerate}
        \item \url{https://mathoverflow.net/questions/429003/manifolds-as-cauchy-completed-objects}
        \item what is the conservative cocompletion of smooth manifolds? Is it related to diffeological spaces?
        \item what is the conservative completion of smooth manifolds? Is it related to diffeological spaces?
        \item what is the conservative bicompletion of smooth manifolds? Is it related to diffeological spaces?
        \item completion of a category under exponentials
        \item \url{https://mathoverflow.net/questions/468897/cocompletion-without-cocontinuous-functors}
        \item The free cocompletion of a category;
        \item The free completion of a category;
        \item The free completion under finite products;
        \item The free cocompletion under finite coproducts;
        \item The free bicompletion of a category;
        \item The free bicompletion of a category under nonempty products and nonempty coproducts (\url{https://ncatlab.org/nlab/show/free+bicompletion});
        \item Cauchy completions
        \item Dedekind--MacNeille completions
        \item Isbell completion (\url{https://ncatlab.org/nlab/show/reflexive+completion})
        \item Isbell envelope
    \end{enumerate}
    Ends and Coends:
    \begin{enumerate}
        \item motivate co/ends as co/limits of profunctors
        \item Ask Fosco about whether composition of dinatural transformations into higher dinaturals could be useful for https://arxiv.org/abs/2409.10237
        \item Cyclic co/ends
            \begin{enumerate}
                \item Try to mimic the construction given in Haugseng for the cycle, paracycle, cube, etc.\ categories
                \item cyclotomic stuff for cyclic co/ends
                    \begin{enumerate}
                        \item Check out Ayala--Mazel-Gee--Rozenblyum's \textit{\href{https://arxiv.org/abs/2405.03897}{Symmetries of the cyclic nerve}}
                        \item isogenetic $\N^{\times}$-action (what the fuck does this mean?)
                    \end{enumerate}
            \end{enumerate}
        \item After stating the co/ends
            \[
                \begin{aligned}
                    \int^{A\in\CatFont{C}}h_{A}\odot\SheafFont{F}^{A},\\
                    \int^{A\in\CatFont{C}}h^{A}\odot F_{A},
                \end{aligned}
                \qquad
                \begin{aligned}
                    \int_{A\in\CatFont{C}}\Sets(h_{A},\SheafFont{F}^{A}),\\
                    \int_{A\in\CatFont{C}}\Sets(h^{A},F_{A})
                \end{aligned}
            \]%
            in the co/end version of the Yoneda lemma, add a remark explaining what the co/ends
            \[
                \begin{aligned}
                    \int_{A\in\CatFont{C}}h_{A}\odot\SheafFont{F}^{A},\\
                    \int_{A\in\CatFont{C}}h^{A}\odot F_{A},
                \end{aligned}
                \qquad
                \begin{aligned}
                    \int^{A\in\CatFont{C}}\Sets(h_{A},\SheafFont{F}^{A}),\\
                    \int^{A\in\CatFont{C}}\Sets(h^{A},F_{A})
                \end{aligned}
            \]%
            and the co/ends
            \[
                \begin{aligned}
                    \int^{A\in\CatFont{C}}\SheafFont{F}^{A}\odot h_{A},\\
                    \int^{A\in\CatFont{C}}F_{A}\odot h^{A},\\
                    \int_{A\in\CatFont{C}}\SheafFont{F}^{A}\odot h_{A},\\
                    \int_{A\in\CatFont{C}}F_{A}\odot h^{A},\\
                \end{aligned}
                \qquad
                \begin{aligned}
                    \int_{A\in\CatFont{C}}\Sets(\SheafFont{F}^{A},h_{A}),\\
                    \int_{A\in\CatFont{C}}\Sets(F_{A},h^{A}),\\
                    \int^{A\in\CatFont{C}}\Sets(\SheafFont{F}^{A},h_{A}),\\
                    \int^{A\in\CatFont{C}}\Sets(F_{A},h^{A})
                \end{aligned}
            \]%
            are.
        \item ends $\CatFont{C}\to\CatFont{D}$ with $\odot$ is a special case of ends for a certain enrichment over $\CatFont{D}$
        \item try to figure out what the end/coend
            \[
                \int^{X\in\CatFont{C}}h^{A}_{X}\times h^{X}_{B},%
                \qquad%
                \int_{X\in\CatFont{C}}h^{A}_{X}\times h^{X}_{B}%
            \]%
            are for $\CatFont{C}=\B{A}$. (I think the coend is like tensor product of $A$ as a left $A$-set with it as a right $A$-set)
        \item Cyclic ends
        \item Dihedral ends
        \item Does Haugseng's constructions give a way to define cyclic co/homology with coefficients in a bimodule?
        \item Category of elements of dinatural transformation classifier
        \item Examples of co/ends: \url{https://mathoverflow.net/a/461814}
        \item Cofinality for co/ends, \url{https://mathoverflow.net/questions/353876}
        \item \say{Fourier transforms} as in \url{https://arxiv.org/pdf/1501.02503#page=168} or \url{https://tetrapharmakon.github.io/stuff/itaca.pdf}
    \end{enumerate}
    Weighted/diagonal category theory:
    \begin{enumerate}
        \item co/ends as centre/trace-infused co/limits: compare the co/end of $\Hom_{\CatFont{C}}$ with the co/limit of $\Hom_{\CatFont{C}}$
        \item Codensity $W$-weighted monads, $\Ran^{[W]}_{F}(F)$;
        \item Codensity diagonal monads, $\mathrm{DiRan}_{F}(F)$;
    \end{enumerate}
    Profunctors:
    \begin{enumerate}
        \item Apartness defines a composition for relations, but its analogue
            \[
                \mathfrak{q}\mathbin{\square}\mathfrak{p}%
                \defeq%
                \int_{A\in\CatFont{C}}\mathfrak{p}^{-_{1}}_{A}\icoprod\mathfrak{q}^{A}_{-_{2}}%
            \]%
            fails to be unital for profunctors with the unit $h^{A}_{-}$. The issue is that while $\mathcal{P}(X)$ is *-autonomous, $\PSh(\CatFont{C})$ need not be so.

            However, if $\CatFont{V}$ is $*$-autonomous, then $\mathsf{Prof}_{\CatFont{V}}$ is a linear bicategory (Proposition 6.6 of \url{https://arxiv.org/abs/2209.05693}). In that case, there's probably a calculus of left/right Kan extensions/lifts one can develop. What is it?
        \item What monoidal category structures on $\Sets$ induce associative and unital compositions on $\Prof$ via co/ends?
        \item \url{https://mathoverflow.net/questions/470213/a-distributor-between-categories-induces-a-distributor-between-their-categories}
        \item Different compositions for profunctors from monoidal structures on the category of sets (e.g.\ \url{https://mathoverflow.net/questions/155939/what-other-monoidal-structures-exist-on-the-category-of-sets})
        \item Nucleus of a profunctor;
        \item Isbell duality for profunctors:
            \begin{enumerate}
                \item \url{https://mathoverflow.net/questions/259525/isbell-duality-for-profunctors}
                \item \url{https://mathoverflow.net/questions/260322/the-mathfrak-l-functor-on-textsfprof}
                \item \url{https://mathoverflow.net/questions/262462/again-on-the-mathfrak-l-functor-on-mathsfprof}
            \end{enumerate}
    \end{enumerate}
    Centres and Traces of Categories:
    \begin{enumerate}
        \item $\K_{0}(\Fun(\B\N,\CatFont{C}))$ vs.\ $\pi_{0}(\Fun(\B\N,\CatFont{C}))$ vs.\ $\Tr(\CatFont{C})$, and how these are generalisations of conjugacy classes for monoids
        \item Explicitly work out the trace and $\pi_{0}\Fun(\B\N,-)$ for monoids with few elements.
        \item $[1_{A}]$ can contain more than one element. An example is $\Sets(\N,\N)$ and the maps given by
            \begin{align*}
                \{0,1,2,3,\ldots\} &\mapsto \{0,0,1,2,\ldots\},\\
                \{0,1,2,3,\ldots\} &\mapsto \{2,3,4,5,\ldots\}.
            \end{align*}
            Show also that if $c\in[1_{A}]$, then $c$ is idempotent.
        \item Drinfeld centre
        \item trace of the symmetric simplex category; it's probably different from that of $\FinSets$
        \item Trace of $\FontForCategories{Rep}_{G}$ and interaction with induction, restriction, etc.
        \item $\pi_{0}(\B\N,\B{A})$, $K(\B\N,\B{A})$, and $\Tr(\B\N,\B{A})$ as concepts of conjugacy for monoids, their equivalents for categories, and comparison with traces
        \item Comparison between $\pi_{0}(\Fun(\B\N,\CatFont{C}))$ and $K(\Fun(\B\N,\CatFont{C}))$
        \item Lax, oplax, pseudo, and strict trace of simplex 2-category
        \item duality over $\GammaCategory$ might give a map from product of a monoid with a set to $\Tr(\GammaCategory)$
        \item Studying the set $\Nat(\id_{\CatFont{C}},F)$ as a notion of categorical trace:
            \begin{enumerate}
                \item Ganter–Kapranov define the trace of a $1$-endomorphism $f\colon A\to A$ in a $2$-category $\CatFont{C}$ to be the set $\Hom_{\CatFont{C}}(\id_{A},f)$;
                    \begin{enumerate}
                        \item \url{https://arxiv.org/abs/math/0602510}
                        \item \url{https://golem.ph.utexas.edu/string/archives/000757.html}
                        \item \url{https://ncatlab.org/nlab/show/categorical+trace}
                    \end{enumerate}
                    We should study this notion in detail, and also study $\Nat(F,\id_{\CatFont{C}})$ as well as $\CoNat(\id_{\CatFont{C}},F)$ and $\CoNat(F,\id_{\CatFont{C}})$.
            \end{enumerate}
        \item Centre of bicategories
        \item Lax centres and lax traces
        \item Examples of traces:
            \begin{enumerate}
                \item Discrete categories
                \item Posets
                    \begin{enumerate}
                        \item $\FontForCategories{Open}(X)$
                    \end{enumerate}
                \item Trace of small but non-finite categories:
                    \begin{enumerate}
                        \item $\Sets$
                        \item $\FontForCategories{Rep}(G)$
                        \item category of finite groups
                        \item category of finite abelian groups
                        \item category of finite $p$-groups for fixed $p$
                        \item category of finite $p$-groups for all $p$
                        \item category of finite fields
                        \item category of finite topological spaces
                        \item category of finite [insert a mathematical object here]
                    \end{enumerate}
            \end{enumerate}
        \item When is the trace of a groupoid just the disjoint sum of sets of conjugacy classes?
        \item Set-theoretical issues when defining traces
            \begin{enumerate}
                \item Sets is a large category, and yet we can speak of its centre
                    \begin{align*}
                        \rmZ(\Sets) &\defeq \int_{A\in\Sets}\Sets(X,X)\\%
                                    &\cong  \Nat(\id_{\Sets},\id_{\Sets})\\
                                    &\cong  \pt.
                    \end{align*}
                    Is there a way to do the same for the trace of sets, or otherwise work with traces of large categories?
            \end{enumerate}
        \item Understand how traces are defined via universal properties in Xinwen Zhu's \href{https://arxiv.org/abs/1810.07375}{Geometric Satake, categorical traces, and arithmetic of Shimura varieties}.
        \item trace as an $\Obj(\CatFont{C})$-indexed set
            \begin{enumerate}
                \item properties, functoriality, etc.
            \end{enumerate}
        \item Maybe actually call $\Fun(\B\N,\CatFont{C})$ the categorical directed loop space of $\CatFont{C}$?
        \item Cyclic version of $\Fun(\B\N,\CatFont{C})$
        \item Traces of categories, nerves of categories, and the cycle category
    \end{enumerate}
    Categorical Hochschild Homology:
    \begin{enumerate}
        \item To any functor we have an associated natural transformation (\cref{the-natural-transformation-associated-to-a-functor}). Do we have sharp transformations associated to natural transformation?
        \item build Hochschild co/simplicial set and study its homotopy groups
        \item $\Fun(\B\N,X_{\bullet})$ vs.\ $\Fun(\Delta^{1}/\partial\Delta^{1},X_{\bullet})$
            \begin{enumerate}
                \item Their $\pi_{0}$'s vs.\ the $\pi_{0}$'s of $\Hom_{X_{\bullet}}(x,x)$, of $\Hom^{\rmL}_{X_{\bullet}}(x,x)$, and $\Hom^{\rmR}_{X_{\bullet}}(x,x)$.
            \end{enumerate}
    \end{enumerate}
    Monoidal Categories:
    \begin{enumerate}
        \item \url{https://mathoverflow.net/questions/380302}
        \item Analogue of Picard rings for dualisable objects
        \item Moduli of associators, braidings, etc.\ for species, $q$-species
        \item When is the left Kan extension along a fully faithful functor of monoidal categories a strong monoidal functor?
        \item Interaction between Day convolution and Isbell duality
        \item general theory for lifting pseudomonads from Cat to Prof along the equipment embedding
        \item definition of prostrength on a functor between promonoidal categories, differential 2-rigs fosco
        \item Promonoidal structure in \url{https://arxiv.org/pdf/1201.2991#page=22}
        \item Day convolution as a colimit over category of factorizations $F(A)\otimes_{\CatFont{C}}G(B)\to V$
        \item Day convolution with respect to Cartesian monoidal structure is Cartesian monoidal. There's an easy proof of this with coend Yoneda
        \item \url{https://mathoverflow.net/questions/491234}
        \item \url{https://mathoverflow.net/questions/488426/adjunction-of-monoidal-closed-categories}
        \item \url{https://arxiv.org/abs/2502.02532}
        \item Does the forgetful functor $\Wasureru\colon\IdemMon(\CatFont{C})\to\Mon(\CatFont{C})$ admit a left adjoint? What about $\Wasureru\colon\IdemMon(\CatFont{C})\to\CatFont{C}$?
        \item Clifford algebras in monoidal categories
        \item Exterior algebras in monoidal categories
            \begin{enumerate}
                \item \url{https://mathoverflow.net/questions/70607/exterior-powers-in-tensor-categories}
                \item \url{https://mathoverflow.net/questions/127476/analogy-between-the-exterior-power-and-the-power-set}
                \item \url{https://mathoverflow.net/questions/182476/delignes-exterior-power}
                \item martin brandenburg's phd thesis
            \end{enumerate}
        \item Different monoidal products in $\Fun(\CatFont{C},\CatFont{C})$ and their distributivity
            \begin{enumerate}
                \item Composition
                \item Pointwise product
                \item Day convolution
                \item Relative monad version of Day convolution
            \end{enumerate}
        \item Classification of monoidal structures on $\SimplexCategory$
        \item Classification of monoidal structures on $\CycleCategory$
        \item Tensor Categories, 8.5.4
        \item \url{https://ncatlab.org/nlab/show/monoidal+action+of+a+monoidal+category}
        \item \url{https://arxiv.org/abs/2203.16351}
        \item Para construction
        \item Drinfeld center; Symmetric center; JY's books on bimonoidal categories
        \item Picard and Brauer 2-groups
            \begin{enumerate}
                \item Differential Picard and Brauer Groups via $\Fun(\B\N,\Mod_{R})$.
                \item Brauer and Picard groups of $(\Fun(\CatFont{C},\CatFont{C}),\circ,\id_{\CatFont{C}})$
                \item Brauer and Picard groups of $\Rep(G)$
                \item Brauer and Picard groups of $\Sets$
                \item Brauer and Picard groups of $\Ch_{\Z}(R)$
                \item Brauer and Picard groups of $\Shv(X)$
                \item Brauer and Picard groups of $\dgMod_{R}$
            \end{enumerate}
        \item Explore examples in which Day convolution gives weird things, like $\Fun(\B\Zn{n},\Sets)$.
        \item Day convolution is a left Kan extension; explore the right Kan extension
        \item Further develop the theory of moduli categories of monoidal structures
        \item Picard group
            \begin{enumerate}
                \item Picard group for Day convolution. A special case is one of Kaplansky's conjectures, \url{https://en.wikipedia.org/wiki/Kaplansky\%27s_conjectures}, about units of group rings
            \end{enumerate}
        \item Day convolution between representable and an arbitrary presheaf $\mathcal{F}$ --- can we prove something nice using the colimit formula for $\mathcal{F}$ in terms of representables?
        \item Notion of braided monoidal categories in which the braiding is not an isomorphism. Relation to \url{https://arxiv.org/abs/1307.5969}
        \item Proving a certain diagram between free monoidal categories commutes involves Fermat's little theorem. Can we reverse this and prove Fermat's little theorem from the commutativty of that diagram?
        \item \url{https://nilesjohnson.net/notes/grPic-P2S.pdf}
        \item Proof that monoidal equivalences $F$ of monoidal categories automatically admit monoidal natural isomorphisms $\id_{\CatFont{C}}\cong F^{-1}\circ F$ and $\id_{\CatFont{D}}\cong F\circ F^{-1}$.
        \item Proof that category with products is monoidal under the Cartesian monoidal structure, \cite{MO382264}.
        \item Explore 2-categorical algebra:
            \begin{enumerate}
                \item Find a construction of the free 2-group on a monoidal category. Apply it to the multiplicative structure on the category of finite sets and permutations, as well as to the multiplicative structure on the 1-truncation of the sphere spectrum, and try to figure out whether this looks like a categorification of $\Q$.
                \item What is the free 2-group on $(\SimplexCategory,\oplus,[0])$?
            \end{enumerate}
        \item Categorify the preorder $\leq$ on $\N$ to a promonad $\mathfrak{p}$ on the groupoid of finite sets and permutations $\F$:
            \begin{enumerate}
                \item A preorder is a monad in $\Rel$
                \item A promonad is a monad in $\Prof$.
                \item There's a promonad $\mathfrak{p}$ in $\F$ defined by
                    \[
                        \mathfrak{p}(m,n)%
                        \defeq%
                        \{%
                            \text{surjections from $\{1,\ldots,m\}$ to $\{1,\ldots,n\}$}%
                        \}%
                    \]%
                    This promonad categorifies $\leq$ in that its values are the witnesses to the fact that $m$ is bigger than $n$ (i.e.\ surjections).
                \item Figure out whether this promonad extends to the 1-truncation of the sphere spectrum, and perhaps to other categorified analogues of monoids/groups/rings.
            \end{enumerate}
        \item \url{https://arxiv.org/abs/1307.5969}
        \item \url{https://arxiv.org/abs/1306.3215}
        \item \url{https://mathoverflow.net/questions/477219/reference-for-the-monoidal-category-structure-x-otimes-y-x-y-x-times-y}
        \item Include an explicit proof of \cref{properties-of-products-of-sets-symmetric-monoidality}
        \item Include an explicit proof of \cref{properties-of-coproducts-of-sets-symmetric-monoidality}
        \item \cref{the-cartesian-product-of-sets-as-an-e-k-e-ell-tensor-product}
        \item obstruction theory for braided enhancements of monoidal categories, using the \say{moduli category of braided enhancements}
        \item Define symmetric and exterior algebras internal to braided monoidal categories
            \begin{enumerate}
                \item \url{https://mathoverflow.net/questions/471372/is-there-an-alternating-power-functor-on-braided-monoidal-categories}
                \item \url{https://arxiv.org/abs/math/0504155}
            \end{enumerate}
        \item \url{https://mathoverflow.net/q/382364}
        \item \url{https://mathoverflow.net/q/471490}
        \item Concepts of bicategories applied to monoidal categories (e.g.\ internal adjunctions lead to dualisable objects)
        \item Involutive Category Theory
        \item \url{https://mathoverflow.net/questions/474662/the-analogy-between-dualizable-categories-and-compact-hausdorff-spaces}
    \end{enumerate}
    Bimonoidal Categories:
    \begin{enumerate}
        \item Bimonoidal structures on the category of species
        \item Include an explicit proof of \cref{properties-of-products-of-sets-symmetric-bimonoidality}
    \end{enumerate}
    Six Functor Formalisms:
    \begin{enumerate}
        \item Michael Shulman: 
            \begin{quote}
                A lot of the "six functor formalism" makes sense in the context of an arbitrary indexed monoidal category (= monoidal fibration), particularly with cartesian base. In particular, I studied the external tensor product in this generality in my paper on Framed bicategories and monoidal fibrations.

                The internal-hom of powersets in particular, with $\emptyset$ as a dualizing object, is well-known in constructive mathematics and topos theory, where powersets are in general a Heyting algebra rather than a Boolean algebra.
            \end{quote}
            Morgan Rogers:
            \begin{quote}
                I second this: you're discovering (and making pleasingly explicit, I might add) a special case of "thin category theory": a lot of what you've discovered will work for posets, with the powerset replaced with the frame of downsets :D
            \end{quote}
        \item A six functor formalism for monoids
        \item \url{https://mathoverflow.net/questions/258159/yoga-of-six-functors-for-group-representations}
        \item Is the 1-categorical analogue of six functor formalisms given by Mann interesting?
            \begin{enumerate}
                \item Mann defines:
                    \begin{quote}
                        A six functor formalism is an $\infty$-functor $f\colon\FontForCategories{Corr}(C,E)\to\Cats_{\infty}$ such that $-\otimes A$, $f^{*}$, and $f_{!}$ admit right adjoints
                    \end{quote}
                \item Is the notion
                    \begin{quote}
                        A 1-categorical six functor formalism is a (lax?) $2$-functor $f\colon\FontForCategories{Corr}(C,E)\to\Cats_{2}$ (or should $\Cats$ be the target?) such that $-\otimes A$, $f^{*}$, and $f_{!}$ admit right adjoints
                    \end{quote}
                    interesting?
            \end{enumerate}
        \item Interaction of the six functors with Kan extensions (e.g.\ how the left Kan extension of $-\otimes A$ may interact with the other functors)
        \item Contexts like Wirthmuller Grothendieck etc
        \item formalisation by cisinski and deglise
        \item How do the following examples fit?
            \begin{enumerate}
                \item base change between $\CatFont{C}_{/X}$ and $\CatFont{C}_{/Y}$
                \item $f_{!}\dashv f_{*}\dashv f^{*}$ adjunction between powersets
                \item $f_{!}\dashv f_{*}\dashv f^{*}$ adjunction between $\Span(\pt,A)$ and $\Span(\pt,B)$
                \item quadruple adjunction between powersets induced by a relation
                \item adjunctions between categories of presheaves induced by a functor or a profunctor
                \item Adjunction between left $A$-sets and left $B$-sets
            \end{enumerate}
            Do they have exceptional $f^{!}$? Is there a notion of Fourier--Mukai transform for them? What kind of compatibility conditions (proper base change, etc.) do we have?
    \end{enumerate}
    Skew Monoidal Categories:
    \begin{enumerate}
        \item \url{https://arxiv.org/abs/2506.06847}
        \item Try to come up with examples of skew monoidal categories by twisting a tensor product $A\otimes B$ into $T(A)\otimes B$. Related idea: product of $G$-sets but twisted on the left by an automorphism of $G$, so that $(ag,b)\sim(a,gb)$ becomes $(a\phi(g),b)\sim(a,gb)$.
        \item Skew monoidal category induced from $G$-sets in analogy to Rel
        \item Free monoidal category on a skew monoidal category
        \item Skew monoidal structures associated to a locally Cartesian closed category
        \item Does the $\E_{1}$ tensor product of monoids admit a skew monoidal category structure?
        \item Is there a (right?) skew monoidal category structure on $\Fun(\CatFont{C},\CatFont{D})$ using right Kan extensions instead of left Kan extensions?
        \item Similarly, are there skew monoidal category structures on the subcategory of $\eRel(A,B)$ spanned by the functions using left Kan extensions and left Kan lifts?
        \item Add example: $\CatFont{C}$ with coproducts, take $\CatFont{C}_{X/}$ and define
            \[
                (X\xrightarrow{f}A)\oplus(X\xrightarrow{g}B)%
                \defeq%
                [%
                    X%
                    \to%
                    X\icoprod X%
                    \xrightarrow{f\icoprod g}%
                    A\icoprod B%
                ]%
            \]%
        \item Duals:
            \begin{enumerate}
                \item Dualisable objects in monoidal categories and traces of endomorphisms of them, including also examples for monoidal categories which are not autonomous/rigid, such as $(\Fun(\CatFont{C},\CatFont{C}),\circ,\id_{\CatFont{C}})$.
                \item compact closed categories
                \item star autonomous categories
                \item Chu construction
                \item Balanced monoidal categories, \url{https://ncatlab.org/nlab/show/balanced+monoidal+category}
                \item Traced monoidal categories, \url{https://ncatlab.org/nlab/show/traced+monoidal+category}
            \end{enumerate}
        \item Invertible objects and Picard groupoids
        \item \url{https://mathoverflow.net/questions/155939/what-other-monoidal-structures-exist-on-the-category-of-sets}
        \item Free braided monoidal category with a braided monoid: \url{https://ncatlab.org/nlab/show/vine}
        \item \url{https://golem.ph.utexas.edu/category/2024/08/skew_monoidal_categories_throu.html}
    \end{enumerate}
    Fibred Category Theory:
    \begin{enumerate}
        \item \url{https://arxiv.org/abs/2402.11644}
        \item \url{https://categorytheory.zulipchat.com/#narrow/channel/229136-theory.3A-category-theory/topic/A.20.22change.20of.20variables.22.20for.20the.20Grothendieck.20construction/near/495776958}
        \item Internal $\eHom$ in categories of co/Cartesian fibrations.
        \item \textit{Tensor structures on fibered categories} by Luca Terenzi: \url{https://arxiv.org/abs/2401.13491}. Check also the other papers by Luca Terenzi.
        \item \url{https://ncatlab.org/nlab/show/cartesian+natural+transformation} (this is a cartesian morphism in $\Fun(\CatFont{C},\CatFont{D})$ apparently)
        \item CoCartesian fibration classifying $\Fun(F,G)$, \url{https://mathoverflow.net/questions/457533/cocartesian-fibration-classifying-mathrmfunf-g}
    \end{enumerate}
    Operads and Multicategories:
    \begin{enumerate}
        \item \href{https://arxiv.org/abs/2405.10072}{Simplicial lists in operad theory I}
    \end{enumerate}
    Monads:
    \begin{enumerate}
        \item Relative monads: message Alyssa asking for her notes
        \item \url{https://ncatlab.org/nlab/show/adjoint+monad}
        \item Kantorovich monad (\url{https://ncatlab.org/nlab/show/Kantorovich+monad}) and probability monads in general, \url{https://ncatlab.org/nlab/show/monads+of+probability\%2C+measures\%2C+and+valuations}.
    \end{enumerate}
    Enriched Categories:
    \begin{enumerate}
        \item $\CatFont{V}$-matrices
    \end{enumerate}
    Bicategories:
    \begin{enumerate}
        \item Bicategories of Lax Fractions, \url{https://arxiv.org/abs/2507.12044}
        \item Linear bicategories, \url{https://ncatlab.org/nlab/show/linear+bicategory}
            \begin{enumerate}
                \item Frobenius algebras, *-autonomous categories
                \item Linearly distributive category, \url{https://ncatlab.org/nlab/show/linearly+distributive+category}
                \item \href{https://arxiv.org/abs/2401.07055}{Diagrammatic Algebra of First Order Logic}
                \item \href{https://arxiv.org/abs/2209.05693}{Constructing linear bicategories}
                \item \href{https://www.math.mcgill.ca/rags/bicats/bicat.pdf}{Introduction to linear bicategories}
                \item quoting from elsewhere here:
                    \begin{quote}
                        Apartness defines a composition for relations, but its analogue
                        \[
                            \mathfrak{q}\mathbin{\square}\mathfrak{p}%
                            \defeq%
                            \int_{A\in\CatFont{C}}\mathfrak{p}^{-_{1}}_{A}\icoprod\mathfrak{q}^{A}_{-_{2}}%
                        \]%
                        fails to be unital for profunctors with the unit $h^{A}_{-}$. The issue is that while $\mathcal{P}(X)$ is *-autonomous, $\PSh(\CatFont{C})$ need not be so.

                        However, if $\CatFont{V}$ is $*$-autonomous, then $\mathsf{Prof}_{\CatFont{V}}$ is a linear bicategory (Proposition 6.6 of \url{https://arxiv.org/abs/2209.05693}). In that case, there's probably a calculus of left/right Kan extensions/lifts one can develop. What is it?
                    \end{quote}
            \end{enumerate}
        \item Allegories, \url{https://ncatlab.org/nlab/show/allegory}
        \item Skew bicategories
        \item Bigroupoid cardinality
        \item Bicategory where objects are groups and a morphism $G\rightproarrow H$ is a representation of $G^{\op}\times H$. (I.e.\ functors $\B{G}^{\op}\times\B{H}\to\Vect_{k}$).
        \item Relative monads internal to a bicategory
        \item Bicategory of monoid actions
        \item \url{https://arxiv.org/abs/0809.1760}
        \item $\Rel_{G}\defeq\Fun(\B{G},\Rel)$
        \item $\Rel$ but for $\Ab$, where morphisms are pairings of the form $A\otimes_{\Z}B\to\Z$.
        \item 2-dimensional co/limits in 2-category of categories and adjoint functors
        \item Category of equivalence classes
            \begin{enumerate}
                \item Given a category $\CatFont{C}$, we have a set $\K_{0}(\CatFont{C})$ of isomorphism classes of objects
                \item Given a bicategory $\CatFont{C}$, there should be a category $\FontForCategories{K}_{0}(\CatFont{C})$ with $\Hom_{\FontForCategories{K}_{0}(\CatFont{C})}(A,B)\defeq\K_{0}(\cHom_{\CatFont{C}}(A,B))$
                \item The set $\K^{\rmeq}_{0}(\CatFont{C})$ of equivalence classes of objects of $\CatFont{C}$ should then satisfy
                    \[
                        \K^{\rmeq}_{0}(\CatFont{C})%
                        \cong%
                        \K_{0}(\FontForCategories{K}_{0}(\CatFont{C})).
                    \]%
            \end{enumerate}
        \item bicategory of chain complexes, section \say{Second Example: Differential Complexes of an Abelian Category} on Gabriel--Zisman's calculus of fractions
        \item 2-vector spaces
        \item Morita equivalence is equivalence internal to bimod
        \item \url{https://mathoverflow.net/questions/478867/2-category-structure-on-modr}
        \item Bicategories of matrices, as in Street's Variation through enrichment, also \url{https://arxiv.org/abs/2410.18877}
        \item \url{https://mathoverflow.net/a/86933}
        \item What are the internal 2-adjunctions in the fundamental $2$-groupoid of a space?
        \item 2-category structure on $\Mod_{R}$, where a $2$-morphism is a commutative square. Characterisation of adjuntions therein
        \item Cook up a very large list of examples of bicategories, like the ones I made for the AI problems. In particular, find an interesting bicategory of representations qualitatively different from the one I described in the Epoch AI problem
        \item 2-category structure on category of $R$-algebras as enriched $\Mod_{R}$-categories
        \item Let $\CatFont{C}$ be a bicategory, let $A,B\in\Obj(\CatFont{C})$, and let $F,G\in\Obj(\cHom_{\CatFont{C}}(A,B))$.
            \begin{enumerate}
                \item Does precomposition with $\LUnitor^{\CatFont{C}}_{A|F}\colon\id_{A}\circ F\Rightarrow F$ induce an isomorphism of sets
                    \[
                        \Hom_{\cHom_{\CatFont{C}}(A,B)}(F,G)%
                        \cong%
                        \Hom_{\cHom_{\CatFont{C}}(A,B)}(F\circ\id_{A},G)%
                    \]%
                    for each $F,G\in\Obj(\cHom_{\CatFont{C}}(A,B))$?
                \item Similarly, do we have an induced isomorphism of the form
                    \[
                        \Hom_{\cHom_{\CatFont{C}}(A,B)}(F,G)%
                        \cong%
                        \Hom_{\cHom_{\CatFont{C}}(A,B)}(F,\id_{B}\circ G)%
                    \]%
                    and so on?
            \end{enumerate}
        \item Are there two Duskin nerve functors? (lax/oplax/etc.?)
        \item Interaction with cotransformations:
            \begin{enumerate}
                \item Can we abstract the structure provided to $\TwoCategoryOfCategories$ by natural cotransformations?
                \item Are there analogues of cotransformations for $\sfbfRel$, $\Span$, $\BiMod$, $\MonAct$, etc.?
                \item Perhaps this might also make sense as a 1-categorical definition, e.g.\ comorphisms of groups from $A$ to $B$ as $\Sets(A,B)$ quotiented by $f(ab)\sim f(a)f(b)$.
            \end{enumerate}
        \item Consider developing the analogue of traces for endomorphisms of dualisable objects in monoidal categories to the setting of bicategories, including e.g.\ the trace of a category as a trace internal to $\Prof$.
        \item Centres of bicategories (lax, strict, etc.)
        \item Concepts of monoidal categories applied to bicategories (e.g.\ traces)
        \item Internal adjunctions in $\Mod$ as in \cite[Section 6.3]{2-categories-book}; see \cite[Example 6.2.6]{2-categories-book}.
        \item Comonads in the bicategory of profunctors.
        \item 2-limit of $\id,\id\colon\Sets\rightrightarrows\Sets$ is $\B\Z$, \url{https://mathoverflow.net/questions/209904/van-kampen-colimits?rq=1#comment520288_209904}
        \item \url{https://mathoverflow.net/questions/473527/universal-property-of-2-presheaves-and-pseudo-lax-colax-natural-transformations}
        \item \url{https://mathoverflow.net/questions/473526/free-cocompletion-of-a-2-category-under-pseudo-colimits-lax-colimits-and-colax}
    \end{enumerate}
    Types of Morphisms in Bicategories:
    \begin{enumerate}
        \item Behaviour in 2-categories of pseudofunctors (or lax functors, etc.), e.g.\ pointwise pseudoepic morphisms in vs.\ pseudoepic morphisms in 2-categories of pseudofunctors.
        \item Statements like \say{coequifiers are lax epimorphisms}, Item 2 of Examples 2.4 of \url{https://arxiv.org/abs/2109.09836}, along with most of the other statements/examples there.
        \item Dense, absolutely dense, etc.\ morphisms in bicategories
    \end{enumerate}
    Internal adjunctions:
    \begin{enumerate}
        \item \url{https://www.google.com/search?q=mate+of+an+adjunction}
        \item Moreover, by uniqueness of adjoints (\ChapterRef{\ChapterInternalAdjunctions, \cref{internal-adjunctions:properties-of-internal-adjunctions-uniqueness-of-adjoints} of \cref{internal-adjunctions:properties-of-internal-adjunctions}}{\cref{properties-of-internal-adjunctions-uniqueness-of-adjoints} of \cref{properties-of-internal-adjunctions}}), this implies also that $S=f^{-1}$.
        \item define bicategory $\Adj(\CatFont{C})$
        \item walking monad
        \item proposition: 2-functors preserve unitors and associators
        \item https://ncatlab.org/nlab/show/2-category+of+adjunctions. Is there a 3-category too?
        \item https://ncatlab.org/nlab/show/free+monad
        \item https://ncatlab.org/nlab/show/CatAdj
        \item https://ncatlab.org/nlab/show/Adj
        \item $\Adj(\Adj(\CatFont{C}))$
        \item Examples of internal adjunctions
            \begin{enumerate}
                \item Internal adjunctions in $\Mod$.
                \item Internal adjunctions in $\PseudoFun(\CatFont{C},\CatFont{D})$.
                \item Internal adjunctions in $\LaxFun(\CatFont{C},\CatFont{D})$.
                \item Internal adjunctions in 2-categories related to fibrations.
            \end{enumerate}
    \end{enumerate}
    2-Categorical Limits:
    \begin{enumerate}
        \item \url{https://sorilee.github.io/posts/strict-bilimit-and-its-proper-examples}
    \end{enumerate}
    Double Categories:
    \begin{enumerate}
        \item Ehresmann
        \item \url{https://arxiv.org/abs/2505.08766}
        \item \url{https://arxiv.org/abs/2504.18065}
        \item \url{https://arxiv.org/abs/2504.11099}
        \item Pinwheel/Yojouhan diagrams and compositionality, section on nLab at \url{https://ncatlab.org/nlab/show/double+category}
    \end{enumerate}
    Homological Algebra:
    \begin{enumerate}
        \item \url{https://arxiv.org/abs/2505.08321}
        \item \url{https://mathoverflow.net/questions/418676/derived-functor-of-functor-tensor-product}
        \item \url{https://math.stackexchange.com/questions/3665036/higher-chain-homotopies}
    \end{enumerate}
    Topos theory:
    \begin{enumerate}
        \item \url{https://arxiv.org/abs/2505.08766}
        \item \url{https://arxiv.org/abs/2304.05338}
        \item \url{https://arxiv.org/abs/2503.20664}
        \item \url{https://arxiv.org/abs/2204.08351}
        \item \url{https://arxiv.org/abs/2404.12313}
        \item \url{https://www.teses.usp.br/teses/disponiveis/45/45131/tde-31082023-163143/en.php}
        \item \url{https://teses.usp.br/teses/disponiveis/45/45131/tde-24042019-195658/pt-br.php}
        \item \url{https://mathoverflow.net/q/479496}
        \item Grothendieck topologies on $\B{A}$
        \item Enriched Grothendieck topologies
            \begin{enumerate}
                \item Borceux--Quintero, \url{https://www.numdam.org/item/CTGDC_1996__37_2_145_0/}
                \item \url{https://arxiv.org/abs/2405.19529}
            \end{enumerate}
        \item Cotopos theory:
            \begin{enumerate}
                \item Copresheaves and copresheaf cotopoi
                \item Elementary cotopoi
                    \begin{enumerate}
                        \item \url{https://mathoverflow.net/questions/474287/intuition-for-the-internal-logic-of-a-cotopos}
                        \item \url{https://mathoverflow.net/questions/394098/what-is-a-cotopos}
                            \begin{quote}
                                In case you haven’t seen it yet, Grothendieck studies (pseudo) cotopos in \href{n case you haven’t seen it yet, Grothendieck studies (pseudo) cotopos in pursuing stacks}{pursuing stacks}
                            \end{quote}
                    \end{enumerate}
            \end{enumerate}
    \end{enumerate}
    Formal category theory:
    \begin{enumerate}
        \item Yosegi boxes \url{https://arxiv.org/abs/1901.01594}
    \end{enumerate}
    Homotopical Algebra:
    \begin{enumerate}
        \item \url{https://arxiv.org/abs/2109.07803}
    \end{enumerate}
    Simplicial stuff:
    \begin{enumerate}
        \item \url{https://arxiv.org/abs/2507.15341}
        \item \url{https://arxiv.org/abs/2503.13663}
        \item \url{https://www.math.univ-paris13.fr/~harpaz/quasi_unital.pdf}
            \begin{enumerate}
                \item slogan: geometric definition of $\infty$-categories should be geometric for identities too
                \item In an $\infty$-category, define a \textbf{quasi-unit} to be a 1-morphism $f$ such that
                    \begin{align*}
                        [f]_{*} &\colon \Hom_{\Ho(\Spaces)}(\Hom_{\mathcal{C}}(X,A)\Hom_{\mathcal{C}}(X,B)),\\
                        [f]^{*} &\colon \Hom_{\Ho(\Spaces)}(\Hom_{\mathcal{C}}(B,X)\Hom_{\mathcal{C}}(A,X))
                    \end{align*}
                    are the identity in $\Ho(\Spaces)$. Explore equivalent conditions,
                \item \url{https://arxiv.org/abs/1606.05669}
                \item \url{https://arxiv.org/abs/1702.08696}
            \end{enumerate}
        \item \url{https://arxiv.org/abs/math/0507116}, \url{https://arxiv.org/abs/2503.11338}
        \item \url{https://arxiv.org/abs/2302.02484} and \url{https://arxiv.org/abs/2411.19751}
        \item Internal adjunctions in $\SimplexCategory$ are the same as Galois connections between $[n]$ and $[m]$.
        \item \url{https://mathoverflow.net/q/478461}
        \item draw coherence for lax functors using the diagram for $\Delta^{2}$
        \item characterisation of simplicial sets such that left, right, and two-sided homotopies agree
        \item every continuous simplicial set arises as the nerve of a poset.
        \item Functor $\sd$ is convolution of $\yo_{\SimplexCategory}$ with itself; see \url{https://arxiv.org/pdf/1501.02503.pdf#page=109}
        \item Extra degeneracies
            \begin{enumerate}
                \item \url{https://www.google.com/search?client=firefox-b-d&q=augmented+simplicial+objects+with+extra+degeneracies}
                \item \url{https://leanprover-community.github.io/mathlib_docs/algebraic_topology/extra_degeneracy.html}
            \end{enumerate}
        \item Comparison between $\Delta^{1}/\partial\Delta^{1}$ and $\B\N$
    \end{enumerate}
    $\infty$-Categories:
    \begin{enumerate}
        \item \url{https://arxiv.org/abs/2508.03145}
        \item \url{https://runegha.folk.ntnu.no/naivecat_web.pdf}
        \item \url{https://arxiv.org/abs/2505.22640}
        \item \url{https://arxiv.org/abs/2410.17102}
        \item \url{https://arxiv.org/abs/2410.02578}, \url{https://scholar.colorado.edu/concern/graduate_thesis_or_dissertations/st74cr650}, \url{https://arxiv.org/abs/2206.00849}
        \item \url{https://mathoverflow.net/questions/479716/non-strictly-unital-functors-of-infinity-categories}
        \item \url{https://mathoverflow.net/questions/472253/whats-the-localization-of-the-infty-category-of-categories-under-inverting-f}
    \end{enumerate}
    Condensed Mathematics:
    \begin{enumerate}
        \item \url{https://golem.ph.utexas.edu/category/2020/03/pyknoticity_versus_cohesivenes.html#c057724}
        \item \url{https://golem.ph.utexas.edu/category/2020/03/pyknoticity_versus_cohesivenes.html#c057810}
        \item \url{https://maths.anu.edu.au/news-events/events/universal-property-category-condensed-sets}
        \item \url{https://grossack.site/2024/07/03/life-in-johnstones-topological-topos}
        \item \url{https://grossack.site/2024/07/03/topological-topos-2-algebras}
        \item \url{https://grossack.site/2024/07/03/topological-topos-3-bonus-axioms}
        \item \url{https://terrytao.wordpress.com/2025/04/23/stonean-spaces-projective-objects-the-riesz-representation-theorem-and-possibly-condensed-mathematics/}
    \end{enumerate}
    Monoids:
    \begin{enumerate}
        \item \url{https://mathoverflow.net/questions/278429/}
        \item Homological algebra of $A$-sets, \url{https://arxiv.org/abs/1503.02309}
        \item Catalan monoids, \url{https://arxiv.org/abs/1309.6120}
        \item \url{https://mathoverflow.net/questions/438305/grothendieck-group-of-the-fibonacci-monoid}
        \item \url{https://math.stackexchange.com/questions/2662005/how-much-of-a-group-g-is-determined-by-the-category-of-g-sets}
        \item \url{https://math.stackexchange.com/a/4996051/603207}, \url{https://arxiv.org/abs/1006.5687}
        \item Six functor formalism for monoids, following \ChapterRef{\ChapterConstructionsWithSets, \cref{constructions-with-sets:subsection-a-six-functor-formalism-for-sets}}{\cref{subsection-a-six-functor-formalism-for-sets}}, but in which $\cap$ and $[-,-]$ are replaced with Day convolution.
        \item Monoid $(\{1,\ldots,n\}\cup\infty,\gcd)$. The element $\infty$ can be replaced by $p^{\min(e^{1}_{1},\ldots,e^{m}_{1})}_{1}\cdots p^{\min(e^{1}_{k},\ldots,e^{m}_{k})}_{k}$.
        \item Universal property of localisation of monoids as a left adjoint to the forgetful functor $\CatFont{C}\to\CatFont{D}$, where:
            \begin{itemize}
                \item $\CatFont{C}$ is the category whose objects are pairs $(A,S)$ with $A$ a monoid and $S$ a submonoid of $A$.
                \item $\CatFont{D}$ is the category whose objects are pairs $(A,S)$ with $A$ a monoid and $S$ a submonoid of $A$ which is also a group.
            \end{itemize}
            Explore this also for localisations of rings

            Explore if we can define field spectra with an approach like this
        \item Adjunction between monoids and monoids with zero corresponding to $(-)^{-}\dashv(-)^{+}$
        \item Rock paper scissors as an example of a non-associative operation
        \item \url{https://mathoverflow.net/questions/438305/grothendieck-group-of-the-fibonacci-monoid}
        \item Witt monoid, \url{https://www.google.com/search?q=Witt+monoid}
        \item semi-direct product of monoids, \url{https://ncatlab.org/nlab/show/semidirect+product+group}
        \item morphisms of monoids as natural transformation between left $A$-sets over $A$ and $B_{A}$.
        \item Figure out if 2-morphisms of monoids coming from $\Fun^{\otimes}(A_{\disc},B_{\disc})$, $\PseudoFun(\B{A},\B{B})$, etc. are interesting
        \item Write sections on the quotient and set of fixed points of a set by a monoid action
        \item Isbell's zigzag theorem for semigroups: the following conditions are equivalent:
            \begin{enumerate}
                \item A morphism $f\colon A\to B$ of semigroups is an epimorphism.
                \item For each $b\in B$, one of the following conditions is satisfied:
                    \begin{itemize}
                        \item We have $f(a)=b$.
                        \item There exist some $m\in\N_{\geq1}$ and two factorisations
                            \begin{align*}
                                b &= a_{0}y_{1},\\
                                b &= x_{m}a_{2m}
                            \end{align*}
                            connected by relations
                            \begin{align*}
                                a_{0}         = x_{1}a_{1},\\
                                a_{1}y_{1}    = a_{2}y_{2},\\
                                x_{1}a_{2}    = x_{2}a_{3},\\
                                a_{2m-1}y_{m} = a_{2m}
                            \end{align*}
                            such that, for each $1\leq i\leq m$, we have $a_{i}\in\Im(f)$.
                    \end{itemize}
            \end{enumerate}
            Wikipedia says in \url{https://en.wikipedia.org/wiki/Isbell\%27s\_zigzag\_theorem}:
            \begin{quote}
                For monoids, this theorem can be written more concisely:
            \end{quote}
        \item Representation theory of monoids
            \begin{enumerate}
                \item \url{https://mathoverflow.net/questions/37115/why-arent-representations-of-monoids-studied-so-much}
                \item Representation theory of groups associated to monoids (groups of units, group completions, etc.)
            \end{enumerate}
    \end{enumerate}
    Monoid Actions:
    \begin{enumerate}
        \item \url{https://link.springer.com/book/10.1007/978-3-642-11297-3}
        \item \url{https://ncatlab.org/schreiber/files/EquivariantInfinityBundles_220809.pdf} has some interesting things, like a fully faithful embedding of $\Mon(\Sets^{\rmL}_{A})$ into $\Mon_{/A}$ whose essential image is given by those monoids of the form $X\rtimes_{\alpha}A$.
        \item $f_{!}\dashv f^{*}\dashv f_{*}$ adjunction
            \begin{enumerate}
                \item Is it related to the Kan extensions adjunction for $f\colon\B{A}\to\B{B}$ and the categories $\Sets^{\rmL}_{A}\cong\PSh(\B{A}^{\op},\Sets)$ and $\Sets^{\rmL}_{B}\cong\PSh(\B{B}^{\op},\Sets)$?
                \item Is it related to the cobase change adjunction of \url{https://ncatlab.org/nlab/show/base+change}? Maybe we can take a morphism of monoids $f\colon A\to B$ and consider $B^{\rmL}_{A}$ as a left $A$-set, and then $(\Sets^{\rmL}_{A})_{A/}$ and $(\Sets^{\rmL}_{A})_{B^{\rmL}_{A}/}$
            \end{enumerate}
        \item \url{https://arxiv.org/abs/2112.10198}
        \item double category of monoid actions
        \item Analogue of Brauer groups for $A$-sets
        \item Hochschild homology for $A$-sets
    \end{enumerate}
    Group Theory:
    \begin{enumerate}
        \item \url{https://mathoverflow.net/questions/45651/is-there-a-q-analog-to-the-braid-group}
        \item \url{https://johncarlosbaez.wordpress.com/2025/03/27/the-mcgee-group/}
        \item \url{https://bookstore.ams.org/memo-1-2/}
        \item \url{https://link.springer.com/book/10.1007/978-3-662-59144-4}
        \item \url{https://en.wikipedia.org/wiki/Tits_group}
        \item \url{https://en.wikipedia.org/wiki/Group_of_Lie_type}
        \item \url{https://mathoverflow.net/questions/251769/what-meanings-does-chevalley-group-have}
        \item \url{https://encyclopediaofmath.org/wiki/Chevalley\_group}
        \item \url{https://en.wikipedia.org/wiki/Group\_of\_Lie\_type}
        \item MO: cardinality of $\Cl(\Aut(\GL_{n}(\F_{q})))$
        \item \url{https://math.stackexchange.com/questions/4419869/do-the-groups-operatornamesl-operatornamepgl-and-operatornamepsl}
        \item \url{https://groupprops.subwiki.org/wiki/Order\_formulas\_for\_linear\_groups}
        \item \url{https://groupprops.subwiki.org/wiki/Order\_of\_semidirect\_product\_is\_product\_of\_orders}
        \item \url{https://groupprops.subwiki.org/wiki/Central\_automorphism\_group\_of\_general\_linear\_group}
        \item \url{https://groupprops.subwiki.org/wiki/Automorphism\_group\_of\_general\_linear\_group\_over\_a\_field}
        \item \url{https://groupprops.subwiki.org/wiki/Inner-centralizing\_automorphism}
        \item \url{https://math.stackexchange.com/questions/2519372/number-of-conjugacy-classes-for-the-modular-group}
        \item $\GL_{n}(K)$ for $K$ a skew field
        \item \url{https://arxiv.org/abs/1212.6157}, \url{https://arxiv.org/abs/0708.1608}, \url{https://en.wikipedia.org/wiki/Wild\_problem}, \url{https://www.google.com/search?q=matrix+pair+problem}, \url{https://arxiv.org/abs/2007.09242}, \url{https://mathoverflow.net/questions/291815/rational-canonical-form-over-mathbbz-pk-mathbbz}, \url{https://mathoverflow.net/questions/291815/rational-canonical-form-over-mathbbz-pk-mathbbz}
        \item \url{https://link.springer.com/book/10.1007/978-981-13-2895-4}
        \item \url{https://ysharifi.wordpress.com/2022/09/14/automorphisms-of-dihedral-groups/}
        \item \url{https://en.wikipedia.org/wiki/PSL(2,7)}
        \item \url{https://arxiv.org/abs/2304.08617}
        \item \url{https://johncarlosbaez.wordpress.com/2016/03/22/the-involute-of-a-cubical-parabola/\#comment-78884}
        \item \url{https://arxiv.org/abs/0904.1876}
        \item finite subgroups of $\mathrm{SU}(2)$, and viewing them as groups of rotations and such
        \item \url{https://arxiv.org/abs/1201.2363}
        \item \url{https://ncatlab.org/nlab/show/group+extension#SchreierTheory}, \url{https://ncatlab.org/nlab/show/nonabelian+cohomology}, \url{https://ncatlab.org/nlab/show/nonabelian+group+cohomology}
        \item \url{https://en.wikipedia.org/wiki/Fibonacci_group}
        \item Study the functoriality properties of $G\mapsto\Aut(G)$ via functoriality of ends
        \item Is $\sum_{[g]\in\Cl(G)}\frac{1}{\abs{g}}$ an interesting invariant of $G$?
        \item Idempotent endomorphism $f\colon A\to A$ is the same as a decomposition $A\cong B\oplus C$ via $B\cong\Im(f)$ and $C\cong\Ker(f)$.
            \begin{enumerate}
                \item \url{https://mathstrek.blog/2015/03/02/idempotents-and-decomposition/}
            \end{enumerate}
        \item \url{https://math.stackexchange.com/questions/34271/order-of-general-and-special-linear-groups-over-finite-fields}
    \end{enumerate}
    Linear Algebra:
    \begin{enumerate}
        \item Size of conjugacy class $[A]$ of $A\in\GL_{n}(\F_{q})$ is given by $\#\GL_{n}(\F_{q})$ divided by the centralizer $\mathrm{Z}_{\GL_{n}(\F_{q})}(A)$ of $A$ in $\GL_{n}(\F_{q})$, whose order is given by
            \begin{align*}
                \#\mathrm{Z}_{\GL_{n}(\F_{q})}(A) &= \prod^{k}_{i=1}\#\GL_{r_{i}}(\F_{q})\\%
                                                  &= q^{\sum^{k}_{i=1}\binom{r_{i}}{2}}\prod^{k}_{i=1}\prod^{r_{i}-1}_{j=0}(q^{r_{i}-j}-1)%
            \end{align*}
            if $A$ is diagonalisable with eigenvalues $\lambda_{1},\ldots,\lambda_{k}$ having multiplicities $r_{1},\ldots,r_{k}$. More generally, see \url{https://groupprops.subwiki.org/wiki/Conjugacy_class_size_formula_in_general_linear_group_over_a_finite_field}
        \item \url{https://en.wikipedia.org/wiki/Semilinear\_map}
        \item conjugacy for $\GL_{n}(\F_{q})$, \url{https://mathoverflow.net/a/104457}
        \item \url{https://en.wikipedia.org/wiki/Dieudonn\%C3\%A9_determinant}, \url{https://ncatlab.org/nlab/show/Dieudonn\%C3\%A9+determinant\#Dieudonne}
        \item \url{https://ncatlab.org/nlab/show/Pfaffian}
        \item \url{https://math.stackexchange.com/questions/1715249/the-number-of-subspaces-over-a-finite-field}
        \item \url{https://math.stackexchange.com/questions/70801/how-many-k-dimensional-subspaces-there-are-in-n-dimensional-vector-space-over}
        \item \url{https://en.wikipedia.org/wiki/Gaussian_binomial_coefficient}
        \item \url{https://en.wikipedia.org/wiki/List_of_q-analogs}
    \end{enumerate}
    Noncommutative Algebra:
    \begin{enumerate}
        \item \url{https://arxiv.org/abs/1608.08140}
        \item \url{https://arxiv.org/abs/2401.12884}
        \item \url{https://ncatlab.org/nlab/show/dihedral+homology}
        \item \url{https://www.sciencedirect.com/science/article/pii/0022404995000836}
        \item \url{https://arxiv.org/abs/2008.11569}, \url{https://www.lakeheadu.ca/sites/default/files/uploads/77/docs/Cox\%20Daniel.pdf}
    \end{enumerate}
    Commutative Algebra:
    \begin{enumerate}
        \item If $M\in\Pic(R)$, then $\Aut(M)\cong R^{\times}$.
        \item \url{https://math.stackexchange.com/questions/637918/}
        \item \url{https://categorytheory.zulipchat.com/#narrow/stream/411257-theory.3A-mathematics/topic/Big.20Witt.20ring}
        \item \url{https://math.stackexchange.com/questions/535623/how-many-irreducible-factors-does-xn-1-have-over-finite-field}
        \item Derivations between morphisms of $R$-algebras, after \url{https://mathoverflow.net/questions/434488}
            \begin{enumerate}
                \item Namely, a derivation from a morphism $f\colon A\to B$ of $R$-algebras to a morphism $g\colon A\to B$ of $R$-algebras is a map $D\colon B\to B$ such that we have
                    \[
                        D(ab)%
                        =%
                        g(a)D(b)%
                        +%
                        D(a)f(b)%
                    \]%
                    for each $a,b\in B$.
            \end{enumerate}
    \end{enumerate}
    Hyper Algebra:
    \begin{enumerate}
        \item \url{https://arxiv.org/abs/2205.02362}
        \item \url{http://www.numdam.org/item/SD_1959-1960__13_1_A9_0/}
        \item \url{https://www.worldscientific.com/worldscibooks/10.1142/13652#t=aboutBook}
    \end{enumerate}
    Coalgebra:
    \begin{enumerate}
        \item \url{https://mathoverflow.net/questions/483668/textrepd-4-and-its-three-fiber-functors}
    \end{enumerate}
    Topological Algebra:
    \begin{enumerate}
        \item \url{https://golem.ph.utexas.edu/category/2014/08/holy\_crap\_do\_you\_know\_what\_a\_c.html}
        \item \url{https://categorytheory.zulipchat.com/#narrow/channel/411257-theory.3A-mathematics/topic/topological.20rings.20and.20fields}
        \item \url{https://mathoverflow.net/q/477757}
        \item \url{https://math.stackexchange.com/questions/2593556/galois-theory-for-topological-fields}
    \end{enumerate}
    Differential Graded Algebras:
    \begin{enumerate}
        \item \url{https://mathoverflow.net/questions/476150/constructing-an-adjunction-between-algebras-and-differential-graded-algebras}
    \end{enumerate}
    Topology:
    \begin{enumerate}
        \item \url{https://arxiv.org/abs/2507.18418}
        \item Topologies on $\mathcal{P}(\mathcal{P}(X))$, \url{https://mathoverflow.net/questions/496630/topological-analogues-of-gromov-hausdorff-convergence}
        \item \url{https://mathoverflow.net/questions/255912/what-is-the-structure-associated-to-almost-everywhere-convergence}
        \item \url{https://arxiv.org/abs/2504.12965}
        \item \url{https://mathoverflow.net/questions/485669/exponential-law-for-topological-spaces-for-the-topology-of-pointwise-convergence} and comments therein
        \item This paper has some cool references on convergence spaces: \url{https://arxiv.org/abs/2410.18245}
        \item \url{https://arxiv.org/abs/2402.12316}
        \item Write about the 6-functor formalism for sheaves on topological spaces and for topological stacks, with lots of examples.
            \begin{enumerate}
                \item MO question titled \emph{6-functor formalism for topological stacks}: \url{https://mathoverflow.net/q/471758}
            \end{enumerate}
    \end{enumerate}
    Measure Theory:
    \begin{enumerate}
        \item \url{https://mathoverflow.net/questions/126994/beck-chevalley-for-measures}
        \item \url{https://mathoverflow.net/questions/483726}
        \item \url{https://en.wikipedia.org/wiki/Valuation_\%28measure_theory\%29}
        \item There's a theorem saying that there does not exist an infinite-dimensional \say{Lebesgue} measure, i.e. (from \url{https://en.wikipedia.org/wiki/Infinite-dimensional_Lebesgue_measure}):
            \begin{quote}
                Let $X$ be an infinite-dimensional, separable Banach space. Then, the only locally finite and translation invariant Borel measure $\mu$ on $X$ is a trivial measure. Equivalently, there is no locally finite, strictly positive, and translation invariant measure on $X$.
            \end{quote}
            What kind of measures exist/not exist that satisfy all conditions above except being locally finite?
        \item \url{https://ncatlab.org/nlab/show/categories+of+measure+theory}
        \item Functions $f_{!}$, $f^{*}$, and $f_{*}$ between spaces of (probability) measures on probability/measurable spaces, mimicking how a map of sets $f\colon X\to Y$ induces morphisms of sets $f_{!}$, $f^{*}$, and $f_{*}$ between $\mathcal{P}(X)$ and $\mathcal{P}(Y)$.
        \item Analogies between representable presheaves and the Yoneda lemma on the one hand and Dirac probability measures on the other hand
            \begin{enumerate}
                \item Universal property of the embedding of a space $X$ into the space of probability measures on $X$
                \item Same question but for distributions
                \item non-symmetric metric on space of probability measures where we define $\mathrm{d}(\mu,\nu)$ to be the measure given by
                    \[
                        U%
                        \mapsto%
                        \int_{U}\rho_{\mu}\,\mathrm{d}\nu,%
                    \]%
                    where $\rho_{\mu}$ is the probability density of $\mu$. Can we make this idea work?
            \end{enumerate}
        \item \url{https://arxiv.org/abs/0801.2250}
        \item \url{https://mathoverflow.net/questions/325861}
            \begin{quote}
                \textit{In particular, I came across a PhD thesis by Martial Agueh. I thought it was interesting because it explicitly investigated the geodesics of Wasserstein space to produce solutions to a type of parabolic PDE.}
            \end{quote}
    \end{enumerate}
    Probability Theory:
    \begin{enumerate}
        \item \url{https://en.wikipedia.org/wiki/Wiener_sausage}
        \item \href{https://link.springer.com/book/10.1007/978-3-319-20828-2}{https://link.springer.com/book/10.1007/978-3-319-20828-2}
        \item \url{https://arxiv.org/abs/2406.10676}
        \item Lévy's forgery theorem
        \item \url{https://www.epatters.org/wiki/stats-ml/categorical-probability-theory}
        \item \url{https://ncatlab.org/nlab/show/category-theoretic+approaches+to+probability+theory}
        \item Categorical probability theory
        \item \url{https://golem.ph.utexas.edu/category/2024/08/introduction_to_categorical_pr.html}
        \item \url{https://arxiv.org/abs/1109.1880}
        \item Connection between fractional differential operators and stochastic processes with jumps
    \end{enumerate}
    Statistics:
    \begin{enumerate}
        \item \url{https://towardsdatascience.com/t-test-from-application-to-theory-5e5051b0f9dc}
    \end{enumerate}
    Metric Spaces:
    \begin{enumerate}
        \item Lawvere metric spaces: object of $\CatFont{V}$-natural transformations corresponds to $\inf(\d(f(x),g(x)))$.
        \item Does the assignment $d(x,y)\mapsto d(x,y)/(1+d(x,y))$ constructing a bounded metric from a metric be given a universal property?
        \item Explore Lawvere metric spaces in a comprehensive manner
        \item metric $\lcm(x,y)/\gcd(x,y)$ on $\N$, \url{https://mathoverflow.net/questions/461588/}. What shape do balls on $\N\times\N$ have with respect to this metric?
        \item \url{https://golem.ph.utexas.edu/category/2023/05/metric_spaces_as_enriched_categories_ii.html}
        \item Simon Willerton's work on the Legendre--Fenchel transform:
            \begin{enumerate}
                \item \url{https://golem.ph.utexas.edu/category/2014/04/enrichment_and_the_legendrefen.html}
                \item \url{https://golem.ph.utexas.edu/category/2014/05/enrichment_and_the_legendrefen_1.html}
                \item \url{https://arxiv.org/abs/1501.03791}
            \end{enumerate}
    \end{enumerate}
    Special Functions:
    \begin{enumerate}
        \item \url{https://en.wikipedia.org/wiki/Dickson_polynomial}
    \end{enumerate}
    $p$-Adic Analysis:
    \begin{enumerate}
        \item \url{https://mathoverflow.net/questions/15673/an-unfamiliar-to-me-form-of-hensels-lemma}
        \item \url{https://arxiv.org/abs/2503.08909}
        \item Analysis of functions $\Z_{p}\to\Q_{q}$, $\Q_{p}\to\Q_{q}$, $\Z_{p}\to\C_{q}$, etc.
            \begin{enumerate}
                \item \url{https://siegelmaxwellc.wordpress.com/publications-pre-prints/}
            \end{enumerate}
    \end{enumerate}
    Partial Differential Equations:
    \begin{enumerate}
        \item Moduli of PDEs
            \begin{enumerate}
                \item \url{https://arxiv.org/abs/2312.05226}, \url{https://arxiv.org/abs/2406.16825}
                \item \url{https://arxiv.org/abs/2304.08671}, \url{https://arxiv.org/abs/2404.07931}
                \item \url{https://arxiv.org/abs/2507.07937}
            \end{enumerate}
        \item \url{https://en.wikipedia.org/wiki/Homotopy_principle}
        \item \url{https://mathoverflow.net/questions/125166/wild-solutions-of-the-heat-equation-how-to-graph-them}
        \item \url{https://math.stackexchange.com/questions/2112841/difference-between-linear-semilinear-and-quasilinear-pdes/5036699\#5036699}
        \item Proof of the smoothing property of the heat equation via:
            \begin{enumerate}
                \item Feynman--Kac formula
                \item Radon--Nikodym + Wiener process has Gaussian as PDF
                \item Convolution of locally integrable with smooth is smooth
            \end{enumerate}
        \item Geometry of PDEs:
            \begin{enumerate}
                \item \url{https://mathoverflow.net/questions/457268/pdes-and-algebraic-varieties}
                \item Can we build a kind of algebraic geometry of PDEs starting with the notion of the zero locus of a differential operator?
                    \begin{enumerate}
                        \item \url{https://ncatlab.org/nlab/show/diffiety}
                    \end{enumerate}
            \end{enumerate}
    \end{enumerate}
    Functional Analysis:
    \begin{enumerate}
        \item \url{https://www.numdam.org/item/SE_1957-1958__1__A3_0/}
        \item \url{https://thenumb.at/Functions-are-Vectors/}
        \item Tate vector spaces
        \item Analytic sheaves, \url{https://mathoverflow.net/questions/484408/literature-on-fr\%c3\%a9chet-quasi-coherent-sheaves}
        \item \url{https://mathscinet.ams.org/mathscinet/article?mr=1257171}
        \item Vidav--Palmer theorem
        \item In the Hilbert space $\ell^{2}(\N;\C)$, the operator $(x_{n})_{n\in\N}\mapsto(x_{n+1})_{n\in\N}$ admits $(x_{n})_{n\in\N}\mapsto(0,x_{0},x_{1},\ldots)$ as its adjoint.
        \item \url{https://arxiv.org/abs/2110.06300}
    \end{enumerate}
    Lie algebras:
    \begin{enumerate}
        \item \url{https://arxiv.org/abs/2310.20300}
        \item Pre-Lie algebras
        \item Post-Lie algebras
        \item \url{https://arxiv.org/abs/2504.05929}
    \end{enumerate}
    Representation Theory:
    \begin{enumerate}
        \item \url{https://www.ams.org/journals/proc/1961-012-01/S0002-9939-1961-0123197-5/}
    \end{enumerate}
    Modular Representation Theory:
    \begin{enumerate}
        \item \url{https://en.wikipedia.org/wiki/Deligne\%E2\%80\%93Lusztig\_theory}
        \item \url{https://math.stackexchange.com/questions/167979/representation-of-cyclic-group-over-finite-field}
        \item \url{https://math.stackexchange.com/questions/153429/irreducible-representations-of-a-cyclic-group-over-a-field-of-prime-order}
    \end{enumerate}
    Homotopy theory:
    \begin{enumerate}
        \item \url{https://mathoverflow.net/questions/495229}
        \item \url{https://ncatlab.org/nlab/show/Moore+path+category}, \url{https://mathoverflow.net/questions/486905/has-the-path-category-of-a-topological-space-been-studied/487212#487212}
        \item \url{https://ncatlab.org/nlab/show/group+actions+on+spheres}, \url{https://www.maths.ed.ac.uk/~v1ranick/papers/wall7.pdf}, \url{https://math.stackexchange.com/questions/1575798/which-groups-act-freely-on-sn}, \url{https://arxiv.org/abs/math/0212280}.
        \item Pascal's triangle via homology of $n$-tori, \url{https://topospaces.subwiki.org/wiki/Homology_of_torus}
        \item Conditions on morphisms of spaces $f\colon X\to Y$ such that $f^{*}\colon[Y,K]\to[X,K]$ or $f_{*}\colon[K,X]\to[K,Y]$ are injective/surjective (so, epi/monomorphisms in $\Ho(\Top)$) or other conditions.
    \end{enumerate}
    Algebraic Geometry:
    \begin{enumerate}
        \item Galois points, \url{https://bdtd.ibict.br/vufind/Record/USP_c5e6638812a74657c40fcd402a894514}
        \item \url{https://arxiv.org/abs/2407.09256}
    \end{enumerate}
    Differential Geometry:
    \begin{enumerate}
        \item \url{https://en.wikipedia.org/wiki/Spherical\_3-manifold}
        \item functor of points approach to differential geometry
    \end{enumerate}
    Number Theory:
    \begin{enumerate}
        \item \url{https://math.stackexchange.com/questions/10233/uses-of-quadratic-reciprocity-theorem/10719#10719}
        \item \url{https://mathoverflow.net/questions/120067/what-do-theta-functions-have-to-do-with-quadratic-reciprocity}
    \end{enumerate}
    Classical Mechanics:
    \begin{enumerate}
        \item Koopman--von Neumann formalism
        \item Relativistic Lagrangian and Hamiltonian mechanics
    \end{enumerate}
    Quantum Mechanics:
    \begin{enumerate}
        \item \url{https://ncatlab.org/nlab/show/geometrical+formulation+of+quantum+mechanics}
    \end{enumerate}
    Quantum Field Theory:
    \begin{enumerate}
        \item \url{https://arxiv.org/abs/2309.15913} and \url{https://arxiv.org/abs/2311.09284}
        \item The current ongoing work on higher gauge theory, specially Christian Saemann's
        \item The recent work about determining the value of the strong coupling constant in the long-distance range, some pointers and keywords for this are available at \href{https://www.scientificamerican.com/article/physicists-finally-know-how-the-strong-force-gets-its-strength/}{this scientific american article}.
    \end{enumerate}
    Combinatorics:
    \begin{enumerate}
        \item Catalan numbers, \url{https://mathstrek.blog/2012/02/19/power-series-and-generating-functions-ii-formal-power-series/}
    \end{enumerate}
    Other:
    \begin{enumerate}
        \item \url{https://arxiv.org/abs/2202.00084}
        \item Are sedenions and higher useful for anything?
        \item \url{https://mathstodon.xyz/@pschwahn/113388126188923908}
        \item Tambara functors, \url{https://arxiv.org/abs/2410.23052}
        \item 2-vector spaces
        \item 2-term chain complexes. They form a 2-category and middle-four exchange holds, the proof using the fact that we have
            \[
                h_{1}\circ\alpha+\beta\circ g_{2}%
                =%
                k_{1}\circ\alpha+\beta\circ f_{2},%
            \]%
            which uses the chain homotopy identities
            \begin{align*}
                d_{V}\circ\alpha  &= g_{2}-f_{2},\\
                -\beta\circ d_{V} &= h_{1}-k_{1}.
            \end{align*}
            Can we modify this to work for usual chain complexes, seeking an answer to \url{https://mathoverflow.net/questions/424268}? What seems to make things go wrong in that case is that the chain homotopy identities are replaced with
            \begin{align*}
                \alpha_{n+1}\circ d^{V}_{n}+d^{W}_{n-1}\circ\alpha_{n} &= g_{n}-f_{n},\\
                \beta_{n+1}\circ d^{V}_{n}+d^{W}_{n-1}\circ\beta_{n}   &= k_{n}-h_{n}.
            \end{align*}
        \item \url{https://arxiv.org/abs/1402.2600}
        \item \url{https://grossack.site/blog}
        \item Classifying space of $\Q_{p}$
        \item \url{https://www.valth.eu/proc.htm}
        \item Construction of $\R$ via slopes:
            \begin{enumerate}
                \item \url{http://maths.mq.edu.au/~street/EffR.pdf}
                \item \url{https://arxiv.org/abs/math/0301015}
                \item \href{https://twitter.com/ColmezPierre/status/1809308351643165181?t=KAESWH44ufE7486diGcxrA}{Pierre Colmez's comment \say{Et si on remplace $\Z$ par $\Q$, on obtient les adèles.}}
                \item I wonder if one could apply an analogue of this construction to the sphere spectrum and obtain a kind of spectral version of the real numbers, as in e.g.\ following the spirit of \href{MO 443018}{https://mathoverflow.net/questions/443018}.
            \end{enumerate}
        \item \url{https://arxiv.org/abs/2406.04936}
        \item \url{https://mathoverflow.net/a/471510}
        \item \url{https://mathoverflow.net/questions/279478/the-category-theory-of-span-enriched-categories-2-segal-spaces/448523#448523}
        \item The works of David Kern, \url{https://dskern.github.io/writings}
        \item \url{https://qchu.wordpress.com/}
        \item \url{https://aroundtoposes.com/}
        \item \url{https://ncatlab.org/nlab/show/essentially+surjective+and+full+functor}
        \item \url{https://mathoverflow.net/questions/415363/objects-whose-representable-presheaf-is-a-fibration}
        \item \url{https://mathoverflow.net/questions/460146/universal-property-of-isbell-duality}
        \item \url{http://www.tac.mta.ca/tac/volumes/36/12/36-12abs.html} ( Isbell conjugacy and the reflexive completion )
        \item \url{https://ncatlab.org/nlab/show/enrichment+versus+internalisation}
        \item The works of Philip Saville, \url{https://philipsaville.co.uk/}
        \item \url{https://golem.ph.utexas.edu/category/2024/02/from\_cartesian\_to\_symmetric\_mo.html}
        \item \url{https://mathoverflow.net/q/463855} (One-object lax transformations)
        \item \url{https://ncatlab.org/nlab/show/analytic+completion+of+a+ring}
        \item \url{https://en.wikipedia.org/wiki/Quaternionic\_analysis}
        \item \url{https://arxiv.org/abs/2401.15051} (The Norm Functor over Schemes)
        \item \url{https://mathoverflow.net/questions/407291/} (Adjunctions with respect to profunctors)
        \item \url{https://mathoverflow.net/a/462726} ($\Prof$ is free completion of $\Cats$ under right extensions)
        \item there's some cool stuff in \url{https://arxiv.org/abs/2312.00990} (Polynomial Functors: A Mathematical Theory of Interaction), e.g.\ on cofunctors.
        \item \url{https://ncatlab.org/nlab/show/adjoint+lifting+theorem}
        \item \url{https://ncatlab.org/nlab/show/Gabriel\%E2\%80\%93Ulmer+duality}
    \end{enumerate}
    General TODO:
    \begin{enumerate}
        \item \url{https://arxiv.org/abs/2108.11952}
        \item \url{https://mathoverflow.net/questions/483243/is-there-a-theory-of-completions-of-semirings-similar-to-i-adic-completions-of}
        \item \url{https://mathoverflow.net/questions/9218/probabilistic-proofs-of-analytic-facts}
        \item \url{https://x.com/cihanpoststhms}
        \item Special graded rings, \url{https://mathoverflow.net/questions/403448/in-search-of-lost-graded-rings}
            \begin{enumerate}
                \item \url{https://arxiv.org/abs/1209.5122}
            \end{enumerate}
        \item Counterexamples in category theory
        \item \url{https://math.stackexchange.com/questions/279347/counterexample-math-books}
        \item Browse MO questions/answers for interesting ideas/topics
        %\item Revisit my questions/answers on MO and MSE
        \item Change Longrightarrow to Rightarrow where appropriate
        \item Try to minimize the amount of footnotes throughout the project. There should be no long footnotes.
    \end{enumerate}
\end{remark}
\begin{appendices}
\input{ABSOLUTEPATH/chapters2.tex}
\end{appendices}
\end{document}
