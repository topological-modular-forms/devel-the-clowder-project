\input{preamble}

% OK, start here.
%
\usepackage{fontspec}
\let\hyperwhite\relax
\let\hyperred\relax
\newcommand{\hyperwhite}{\hypersetup{citecolor=white,filecolor=white,linkcolor=white,urlcolor=white}}
\newcommand{\hyperred}{%
\hypersetup{%
    citecolor=TitlingRed,%
    filecolor=TitlingRed,%
    linkcolor=TitlingRed,%
     urlcolor=TitlingRed%
}}
\let\ChapterRef\relax
\newcommand{\ChapterRef}[2]{#1}
\setcounter{tocdepth}{2}
%▓▓▓▓▓▓▓▓▓▓▓▓▓▓▓▓▓▓▓▓▓▓▓▓▓▓▓▓▓▓▓▓▓
%▓▓ ╔╦╗╦╔╦╗╦  ╔═╗  ╔═╗╔═╗╔╗╔╔╦╗ ▓▓
%▓▓  ║ ║ ║ ║  ║╣   ╠╣ ║ ║║║║ ║  ▓▓
%▓▓  ╩ ╩ ╩ ╩═╝╚═╝  ╚  ╚═╝╝╚╝ ╩  ▓▓
%▓▓▓▓▓▓▓▓▓▓▓▓▓▓▓▓▓▓▓▓▓▓▓▓▓▓▓▓▓▓▓▓▓
%\usepackage{titlesec}
%▓▓▓▓▓▓▓▓▓▓▓▓▓▓▓▓▓▓▓▓▓▓▓▓▓▓▓▓▓▓▓▓▓▓▓▓▓▓▓▓▓▓▓▓▓▓▓▓▓▓▓▓▓▓▓
%▓▓ ╔╦╗╔═╗╔╗ ╦  ╔═╗  ╔═╗╔═╗  ╔═╗╔═╗╔╗╔╔╦╗╔═╗╔╗╔╔╦╗╔═╗ ▓▓
%▓▓  ║ ╠═╣╠╩╗║  ║╣   ║ ║╠╣   ║  ║ ║║║║ ║ ║╣ ║║║ ║ ╚═╗ ▓▓
%▓▓  ╩ ╩ ╩╚═╝╩═╝╚═╝  ╚═╝╚    ╚═╝╚═╝╝╚╝ ╩ ╚═╝╝╚╝ ╩ ╚═╝ ▓▓
%▓▓▓▓▓▓▓▓▓▓▓▓▓▓▓▓▓▓▓▓▓▓▓▓▓▓▓▓▓▓▓▓▓▓▓▓▓▓▓▓▓▓▓▓▓▓▓▓▓▓▓▓▓▓▓
\newcommand{\ChapterTableOfContents}{%
    \begingroup
    \addfontfeature{Numbers={Lining,Monospaced}}
    \hypersetup{hidelinks}\tableofcontents%
    \endgroup
}%

\let\DotFill\relax
\makeatletter
\newcommand \DotFill {\leavevmode \cleaders \hb@xt@ .33em{\hss .\hss }\hfill \kern \z@}
\makeatother

\definecolor{ToCGrey}{rgb}{0.4,0.4,0.4}
\definecolor{mainColor}{rgb}{0.82745098,0.18431373,0.18431373}
\usepackage{titletoc}
\titlecontents{part}
[0.0em]
{\addvspace{1pc}\color{TitlingRed}\large\bfseries\text{Part }}
{\bfseries\textcolor{TitlingRed}{\contentslabel{0.0em}}\hspace*{1.35em}}
{}
{\textcolor{TitlingRed}{{\hfill\bfseries\contentspage\nobreak}}}
[]
\titlecontents{section}
[0.0em]
{\addvspace{1pc}}
{\color{black}\bfseries\textcolor{TitlingRed}{\contentslabel{0.0em}}\hspace*{1.65em}}
{}
{\textcolor{black}{\textbf{\DotFill}{\bfseries\contentspage\nobreak}}}
[]
\titlecontents{subsection}
[0.0em]
{}
{\hspace*{1.65em}\color{ToCGrey}{\contentslabel{0.0em}}\hspace*{2.5em}}
{}
{{\textcolor{ToCGrey}\DotFill}\textcolor{ToCGrey}{\contentspage}\nobreak}
[]
\usepackage{marginnote}
\renewcommand*{\marginfont}{\normalfont}
\usepackage{inconsolata}
\setmonofont{inconsolata}%
\let\ChapterRef\relax
\newcommand{\ChapterRef}[2]{#1}
\AtBeginEnvironment{subappendices}{%%
    \section*{\huge Appendices}%
}%

\begin{document}

\title{Posets}

\maketitle

\phantomsection
\label{section-phantom}

This chapter contains some material about posets. Notably, we discuss and explore:
\begin{enumerate}
    \item Thing
\end{enumerate}
TODO copy pasted from email:
\begin{enumerate}
    \item \url{https://en.wikipedia.org/wiki/Total_order#Chains}
    \item \url{https://ncatlab.org/nlab/show/relation#binary_relations_as_a_coalgebra}
    \item \url{https://ncatlab.org/nlab/show/relation#endorel}
    \item \url{https://en.wikipedia.org/wiki/Preorder}
    \item \url{https://en.wikipedia.org/wiki/Binary_relation#Preorder_R}
    \item \url{https://www.google.com/search?client=firefox-b-d&q=left+residual+relation}
    \item \url{https://en.wikipedia.org/wiki/Composition_of_relations#Quotients}
    \item \url{https://en.wikipedia.org/wiki/Composition_of_relations#Schr\%C3\%B6der_rules}
    \item \url{https://proofwiki.org/wiki/Schr\%C3\%B6der_Rule}
    \item \url{https://groupprops.subwiki.org/wiki/Left_residual_operator_for_composition}
    \item \url{https://ncatlab.org/nlab/show/total+relation}
    \item \url{https://ncatlab.org/nlab/show/extensional+relation}
    \item \url{https://ncatlab.org/nlab/show/well-founded+relation}
    \item \url{https://ncatlab.org/nlab/show/apartness+relation}
    \item \url{https://ncatlab.org/nlab/show/partial+equivalence+relation}
    \item \url{https://ncatlab.org/nlab/show/order}
    \item \url{https://ncatlab.org/nlab/show/congruence}
    \item \url{https://ncatlab.org/nlab/show/allegory}
    \item \url{https://ncatlab.org/nlab/show/Rel}
    \item \url{https://en.wikipedia.org/wiki/Demonic_composition}
    \item \url{https://en.wikipedia.org/wiki/Converse_relation}
    \item \url{https://en.wikipedia.org/wiki/Binary_relation#Difunctional}
    \item \url{https://en.wikipedia.org/wiki/Algebraic_logic#Calculus_of_relations}
    \item \url{https://en.wikipedia.org/wiki/Binary_relation}
    \item \url{https://ncatlab.org/nlab/show/internal+relation}
\end{enumerate}
TODO:
\begin{enumerate}
    \item \url{https://mathoverflow.net/q/481011}
    \item Does the tensor product of suplattices induce something interesting in $\SimplexCategory$?
    \item Monotone map from $A\times B$ to $X\times Y$ is the same thing as a pair $(f,g)$ with $f\colon A\to X$ and $g\colon B\to Y$ monotone maps
    \item Example: divisibility poset.
        \begin{enumerate}
            \item Minimal elements are prime numbers
            \item It has meets given by $\gcd$. If we take all natural numbers then it becomes a suplattice with join given by $\lcm$ and meet by $\gcd$.
            \item Correct the following since $\lcm$ need not be in $D_{n}$: Tensor product of suplattices of $D_{n}\otimes D_{m}$ is given by functions satisfying $f(\lcm(k,\ell))=\gcd(f(k),f(\ell))$, and those are precisely the functions such that if $k\mid\ell$, then $f(\ell)\mid f(k)$.
                \begin{enumerate}
                    \item Tensor product of suplattices of $[n]\otimes[m]$ is given by functions satisfying $f(\max(k,\ell))=\min(f(k),f(\ell))$, and those are precisely the functions such that if $k\leq\ell$, then $f(\ell)\leq f(k)$.
                \end{enumerate}
        \end{enumerate}
    \item Example of tensor product of suplattices for linear total orders
    \item Add \url{https://ncatlab.org/nlab/show/conservative+cocompletion} for posets
    \item Prove that tensors and cotensors of elements of posets by $\true$ and $\false$ exist
    \item \url{https://jdh.hamkins.org/the-lattice-of-sets-of-natural-numbers-is-rich/}
\end{enumerate}

\ChapterTableOfContents

\section{Posets}\label{section-posets}
\subsection{Foundations}\label{subsection-posets-foundations}
\section{Limits and Colimits in Posets}\label{section-limits-and-colimits-in-posets}
\subsection{Joins}\label{subsection-joins-in-posets}
\subsection{Meets}\label{subsection-meets-in-posets}
\subsection{Suprema}\label{subsection-suprema-in-posets}
\subsection{Infima}\label{subsection-infima-in-posets}
\section{Totally Ordered Sets}\label{section-totally-ordered-sets}
\subsection{Maxima and Minima}\label{subsection-maxima-and-minima}
\begin{definition}{Maxima and Minima}{maxima-and-minima}%
    Let $(X,\leq_{X})$ be a totally ordered set.
    \begin{enumerate}
        \item\label{maxima-and-minima-minima}The \index[posets]{minimum}\textbf{minimum operation} is the function $\min\colon X\times X\to X$ defined by
            \[
                \min(x,y)%
                \defeq%
                \begin{cases}%
                    x &\text{if $x\leq_{X}y$,}\\
                    y &\text{otherwise}%
                \end{cases}%
            \]%
            for each $x,y\in X$.
        \item\label{maxima-and-minima-maxima}The \index[posets]{maximum}\textbf{maximum operation} is the function $\max\colon X\times X\to X$ defined by
            \[
                \max(x,y)%
                \defeq%
                \begin{cases}%
                    x &\text{if $y\leq_{X}x$,}\\
                    y &\text{otherwise}%
                \end{cases}%
            \]%
            for each $x,y\in X$.
    \end{enumerate}
\end{definition}
\begin{proposition}{Properties of Maxima and Minima}{properties-of-maxima-and-minima}%
    Let $(X,\leq_{X})$ be a totally ordered set.
    \begin{enumerate}
        \item\label{properties-of-maxima-and-minima-unitality}\SloganFont{Unitality. }
        \item\label{properties-of-maxima-and-minima-associativity}\SloganFont{Associativity. }
            % https://proofwiki.org/wiki/Max_Operation_is_Associativehttps://proofwiki.org/wiki/Max_of_Subfamily_of_Operands_Less_or_Equal_to_Max
        \item\label{properties-of-maxima-and-minima-commutativity}\SloganFont{Commutativity. }
            % https://proofwiki.org/wiki/Max_Operation_is_Commutative
        \item\label{properties-of-maxima-and-minima-idempotency}\SloganFont{Idempotency. }
            % https://proofwiki.org/wiki/Max_Operation_is_Idempotent
        \item\label{properties-of-maxima-and-minima-distributivity-of-maxima-over-minima}\SloganFont{Distributivity of Maxima Over Minima. }The diagrams
            \[
                \cdots%
            \]%
            commute, i.e.\ we have
            \begin{align*}
                \max(x,\min(y,z)) &= \min(\max(x,y),\max(x,z)),\\
                \max(\min(x,y),z) &= \min(\max(x,z),\max(y,z))%
            \end{align*}
            for each $x,y,z\in X$.
        \item\label{properties-of-maxima-and-minima-distributivity-of-minima-over-maxima}\SloganFont{Distributivity of Minima Over Maxima. }The diagrams
            \[
                \cdots%
            \]%
            commute, i.e.\ we have
            \begin{align*}
                \min(x,\max(y,z)) &= \max(\min(x,y),\min(x,z)),\\
                \min(\max(x,y),z) &= \max(\min(x,z),\min(y,z))%
            \end{align*}
            for each $x,y,z\in X$.
            % https://proofwiki.org/wiki/Max_and_Min_Operations_are_Distributive_over_Each_Other
            %
            %MORE:
            % https://proofwiki.org/wiki/Mapping_on_Integers_is_Endomorphism_of_Max_or_Min_Operation_iff_Increasing
            % https://proofwiki.org/wiki/Condition_for_Equivalence_Relation_for_Max_Operation_on_Natural_Numbers_to_be_Congruence
            % https://proofwiki.org/wiki/Mapping_on_Integers_is_Homomorphism_between_Max_or_Min_Operation_iff_Decreasing
            % For fields: https://proofwiki.org/wiki/Max_is_Half_of_Sum_Plus_Absolute_Difference
            % https://proofwiki.org/wiki/Max_Operation_on_Continuous_Real_Functions_is_Continuous
            % https://proofwiki.org/wiki/Max_Operation_on_Natural_Numbers_forms_Monoid
            % https://proofwiki.org/wiki/Max_Operation_on_Toset_forms_Semigroup
            % https://proofwiki.org/wiki/Max_Operation_on_Woset_is_Monoid
            % https://proofwiki.org/wiki/Max_Operation_Preserves_Total_Ordering
            % https://proofwiki.org/wiki/Max_Operation_Representation_on_Real_Numbers
            % https://proofwiki.org/wiki/Max_Semigroup_on_Toset_forms_Semilattice
        %\item\label{properties-of-maxima-and-minima-}\SloganFont{. }
    \end{enumerate}
\end{proposition}
\begin{Proof}{Proof of \cref{properties-of-maxima-and-minima}}%
\end{Proof}
\begin{appendices}
\input{ABSOLUTEPATH/chapters2.tex}
\end{appendices}
\end{document}
