\input{preamble}

% OK, start here.
%
\usepackage{fontspec}
\let\hyperwhite\relax
\let\hyperred\relax
\newcommand{\hyperwhite}{\hypersetup{citecolor=white,filecolor=white,linkcolor=white,urlcolor=white}}
\newcommand{\hyperred}{%
\hypersetup{%
    citecolor=TitlingRed,%
    filecolor=TitlingRed,%
    linkcolor=TitlingRed,%
     urlcolor=TitlingRed%
}}
\let\ChapterRef\relax
\newcommand{\ChapterRef}[2]{#1}
\setcounter{tocdepth}{2}
%▓▓▓▓▓▓▓▓▓▓▓▓▓▓▓▓▓▓▓▓▓▓▓▓▓▓▓▓▓▓▓▓▓
%▓▓ ╔╦╗╦╔╦╗╦  ╔═╗  ╔═╗╔═╗╔╗╔╔╦╗ ▓▓
%▓▓  ║ ║ ║ ║  ║╣   ╠╣ ║ ║║║║ ║  ▓▓
%▓▓  ╩ ╩ ╩ ╩═╝╚═╝  ╚  ╚═╝╝╚╝ ╩  ▓▓
%▓▓▓▓▓▓▓▓▓▓▓▓▓▓▓▓▓▓▓▓▓▓▓▓▓▓▓▓▓▓▓▓▓
%\usepackage{titlesec}
%▓▓▓▓▓▓▓▓▓▓▓▓▓▓▓▓▓▓▓▓▓▓▓▓▓▓▓▓▓▓▓▓▓▓▓▓▓▓▓▓▓▓▓▓▓▓▓▓▓▓▓▓▓▓▓
%▓▓ ╔╦╗╔═╗╔╗ ╦  ╔═╗  ╔═╗╔═╗  ╔═╗╔═╗╔╗╔╔╦╗╔═╗╔╗╔╔╦╗╔═╗ ▓▓
%▓▓  ║ ╠═╣╠╩╗║  ║╣   ║ ║╠╣   ║  ║ ║║║║ ║ ║╣ ║║║ ║ ╚═╗ ▓▓
%▓▓  ╩ ╩ ╩╚═╝╩═╝╚═╝  ╚═╝╚    ╚═╝╚═╝╝╚╝ ╩ ╚═╝╝╚╝ ╩ ╚═╝ ▓▓
%▓▓▓▓▓▓▓▓▓▓▓▓▓▓▓▓▓▓▓▓▓▓▓▓▓▓▓▓▓▓▓▓▓▓▓▓▓▓▓▓▓▓▓▓▓▓▓▓▓▓▓▓▓▓▓
\newcommand{\ChapterTableOfContents}{%
    \begingroup
    \addfontfeature{Numbers={Lining,Monospaced}}
    \hypersetup{hidelinks}\tableofcontents%
    \endgroup
}%

\let\DotFill\relax
\makeatletter
\newcommand \DotFill {\leavevmode \cleaders \hb@xt@ .33em{\hss .\hss }\hfill \kern \z@}
\makeatother

\definecolor{ToCGrey}{rgb}{0.4,0.4,0.4}
\definecolor{mainColor}{rgb}{0.82745098,0.18431373,0.18431373}
\usepackage{titletoc}
\titlecontents{part}
[0.0em]
{\addvspace{1pc}\color{TitlingRed}\large\bfseries\text{Part }}
{\bfseries\textcolor{TitlingRed}{\contentslabel{0.0em}}\hspace*{1.35em}}
{}
{\textcolor{TitlingRed}{{\hfill\bfseries\contentspage\nobreak}}}
[]
\titlecontents{section}
[0.0em]
{\addvspace{1pc}}
{\color{black}\bfseries\textcolor{TitlingRed}{\contentslabel{0.0em}}\hspace*{1.65em}}
{}
{\textcolor{black}{\textbf{\DotFill}{\bfseries\contentspage\nobreak}}}
[]
\titlecontents{subsection}
[0.0em]
{}
{\hspace*{1.65em}\color{ToCGrey}{\contentslabel{0.0em}}\hspace*{2.5em}}
{}
{{\textcolor{ToCGrey}\DotFill}\textcolor{ToCGrey}{\contentspage}\nobreak}
[]
\usepackage{marginnote}
\renewcommand*{\marginfont}{\normalfont}
\usepackage{inconsolata}
\setmonofont{inconsolata}%
\let\ChapterRef\relax
\newcommand{\ChapterRef}[2]{#1}
\AtBeginEnvironment{subappendices}{%%
    \section*{\huge Appendices}%
}%

\begin{document}

\title{Conventions}

\maketitle

\phantomsection
\label{section-phantom}

This chapter contains information on a few conventions we employ throughout Clowder.

\ChapterTableOfContents

\section{Global Conventions}\label{section-global-conventions}
\subsection{On Equality}\label{subsection-on-equality}
\begin{convention}{Notation for Equality}{notation-for-equality}%
    Outside of the chapters relating to type theory, we employ the following notational choice:
    \begin{enumerate}
        \item\label{notation-for-equality-defeq}Definitions use the symbol $\defeq$, as in e.g.:
            \[
                U\boxtimes_{X\times Y}V%
                \defeq%
                \{%
                    (u,v)\in X\times Y%
                    \ \middle|\ %
                    \text{$u\in U$ and $v\in V$}%
                \}.%
            \]%
        \item\label{notation-for-equality-doteq}When we unwind definitions inside proofs, we will use the symbol $\eqdef$, to mean that a manipulation in which we have replaced a definition has taken place. For example, we might write
            \[%
                \rmInt(f_{!}(S))%
                \eqdef%
                \rmInt(\{x\in X\ \middle|\ f(x)\in S\}),%
            \]%
            where we have replaced the definition of $f_{!}(S)$, namely
            \[%
                f_{!}(S)%
                \defeq%
                \{x\in X\ \middle|\ f(x)\in S\}.%
            \]%
        \item\label{notation-for-equality-equals}Finally, when two things are equal but not through a direct replacement of a definition, we use the usual symbol $=$ for equality:
            \[
                \{%
                    x\in\R%
                    \ \middle|\ %
                    x^{2}-5x+6=0%
                \}%
                =%
                \{2,3\}.%
            \]%
    \end{enumerate}
    Note that our usage of these conventions won't always align with the differentiation between \textit{propositional equalities} and \textit{judgemental equalities} in type theory. Rather, these serve a pedagogical role, as often a logical step using $\doteq$ will be \say{obvious} while a step using $=$ might be a bit involved.
\end{convention}
\subsection{Additional Notation for Functions}\label{section-additional-notation-for-functions}
\begin{notation}{Additional Notation for Functions}{additional-notation-for-functions}%
    Throughout this work, we will sometimes denote a function $f\colon X\to Y$ by $\llbracket x\mapsto f(x)\rrbracket$, writing
    \[
        f%
        \eqdef%
        \llbracket x\mapsto f(x)\rrbracket.%
    \]%
    \begin{enumerate}
        \item\label{additional-notation-for-functions-1}For example, given a function%
            \[
                \Phi%
                \colon%
                \Hom_{\Sets}(X,Y)%
                \to%
                K%
            \]%
            taking values on a set of functions such as $\Hom_{\Sets}(X,Y)$, we will sometimes write
            \[
                \Phi(f)%
                \eqdef%
                \Phi(\llbracket x\mapsto f(x)\rrbracket).%
            \]%
        \item\label{additional-notation-for-functions-2}This notational choice is based on the lambda notation
            \[
                f%
                \eqdef%
                (\lambda x\ldotp f(x)),%
            \]%
            but uses a \say{$\mathord{\mapsto}$} symbol for better spacing and double brackets instead of either:
            \begin{enumerate}
                \item\label{additional-notation-for-functions-2-a}Square brackets $[x\mapsto f(x)]$;
                \item\label{additional-notation-for-functions-2-b}Parentheses $(x\mapsto f(x))$.
            \end{enumerate}
            In doing this, we hope to improve readability when dealing with e.g.:
            \begin{enumerate}
                \item\label{additional-notation-for-functions-2-c}Equivalence classes, cf.:
                    \begin{enumerate}
                        \item\label{additional-notation-for-functions-2-c-i}$\llbracket[x]\mapsto f([x])\rrbracket$%
                        \item\label{additional-notation-for-functions-2-c-ii}$[[x]\mapsto f([x])]$%
                        \item\label{additional-notation-for-functions-2-c-iii}$(\lambda[x].\ f([x]))$%
                    \end{enumerate}
                \item\label{additional-notation-for-functions-2-d}Function evaluations, cf.:
                    \begin{enumerate}
                        \item\label{additional-notation-for-functions-2-d-i}$\Phi(\llbracket x\mapsto f(x)\rrbracket)$%
                        \item\label{additional-notation-for-functions-2-d-ii}$\Phi((x\mapsto f(x)))$%
                        \item\label{additional-notation-for-functions-2-d-iii}$\Phi((\lambda x.\ f(x)))$%
                    \end{enumerate}
            \end{enumerate}
        \item\label{additional-notation-for-functions-3}We will also sometimes write $-$, $-_{1}$, $-_{2}$, etc.\ for the arguments of a function. Some examples include:
            \begin{enumerate}
                \item\label{additional-notation-for-functions-3-a}Writing $f(-_{1})$ for a function $f\colon A\to B$.
                \item\label{additional-notation-for-functions-3-b}Writing $f(-_{1},-_{2})$ for a function $f\colon A\times B\to C$.
                \item\label{additional-notation-for-functions-3-c}Given a function $f\colon A\times B\to C$, writing
                    \[
                        f(a,-)%
                        \colon%
                        B%
                        \to%
                        C%
                    \]%
                    for the function $\llbracket b\mapsto f(a,b)\rrbracket$.
                \item\label{additional-notation-for-functions-3-d}Denoting a composition of the form%
                    \[
                        A\times B%
                        \xlongrightarrow{\phi\times\id_{B}}%
                        A'\times B%
                        \xlongrightarrow{f}%
                        C%
                    \]%
                    by $f(\phi(-_{1}),-_{2})$.
            \end{enumerate}
        \item\label{additional-notation-for-functions-4}Finally, given a function $f\colon A\to B$, we will sometimes write
            \[
                \ev_{a}(f)%
                \defeq%
                f(a)%
            \]%
            for the value of $f$ at some $a\in A$.
    \end{enumerate}
    For an example of the above notations being used in practice, see the proof of the adjunction
    \begin{webcompile}
        \AdjunctionShort#A\times -#{\Hom_{\Sets}(A,-)}#\Sets#\Sets,#
    \end{webcompile}
    stated in \ChapterRef{\ChapterConstructionsWithSets, \cref{constructions-with-sets:properties-of-products-of-sets-adjointness-1} of \cref{constructions-with-sets:properties-of-products-of-sets}}{\cref{properties-of-products-of-sets-adjointness-1} of \cref{properties-of-products-of-sets}}.
\end{notation}
\section{Local Conventions}\label{section-local-conventions}
\begin{appendices}
\input{ABSOLUTEPATH/chapters2.tex}
\end{appendices}
\end{document}
